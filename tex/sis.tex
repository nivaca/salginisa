
% this tex file was auto produced from TEI on 2021-11-18T15:48:22.803726-05:00
\documentclass[letterpaper, 12pt]{book}

\usepackage{fontspec}

\usepackage[english, spanish, spanish.mexico,latin]{babel}

\defaultfontfeatures{Ligatures=TeX, Scale=MatchLowercase}
\setmainfont[Ligatures=Common, Numbers=OldStyle]{Libertinus Serif}
\setmonofont[Scale=MatchLowercase]{DejaVu Sans Mono}


\usepackage[babel=true, verbose=false, tracking=true, expansion=true, protrusion=true, final, draft=false]{microtype}
\SetTracking{ encoding = *, shape = sc }{ 30 }
\SetTracking[context = notracking, ]{encoding = *}{0}

\usepackage[autostyle=false, style=british]{csquotes}
\usepackage{latexcolors}

\usepackage{geometry}
\geometry{left=4cm, right=4cm, top=3cm, bottom=3cm}


\righthyphenmin=62%
\lefthyphenmin=62%
\clubpenalty=9996
\widowpenalty=9999
\brokenpenalty=4991
\predisplaypenalty=10000
\postdisplaypenalty=1549
\displaywidowpenalty=1602
\flushbottom
\raggedbottom
% \midsloppy
\parindent=14pt
\frenchspacing


\usepackage{fancyhdr}
\pagestyle{fancy}
\setlength{\headheight}{15pt}


% a critical mark
\usepackage{amssymb}

% git package
\usepackage{gitinfo2}

% indices 
\usepackage{imakeidx}  % before reledmac


% reledmac settings -------------------------------------------
\usepackage[final]{reledmac}

\Xinplaceoflemmaseparator{0pt} % Don't add space after nolemma notes.
\Xlemmadisablefontselection[A] % In fontium lemmata, don't copy font formatting.
\Xarrangement{paragraph}
\linenummargin{outer}
\sidenotemargin{inner}
\lineation{page}

\Xendbeforepagenumber{}
\Xendafterpagenumber{.}
\Xendlineprefixsingle{}
\Xendlineprefixmore{}

\Xnumberonlyfirstinline[]
\Xnumberonlyfirstintwolines[]
\Xbeforenotes{\baselineskip}

% This should prevent overfull vboxes
\AtBeginDocument{\Xmaxhnotes{0.5\textheight}}
\AtBeginDocument{\maxhnotesX{0.5\textheight}}

\Xprenotes{\baselineskip}

\let\Afootnoterule=\relax
\let\Bfootnoterule=\relax
% ---------------------------------------------------------


\makeindex[name=persons, title={Index nominum}, columns=2]
\makeindex[name=works, title={Index operum}, columns=2]


% custom macros
\newcommand*{\name}[1]{#1}
\newcommand*{\lemmaQuote}[1]{\emph{#1}}
\newcommand*{\worktitle}[1]{\textit{#1}}
\newcommand*{\supplied}[1]{⟨#1⟩}
\newcommand*{\suppliedInVacuo}[1]{$\ulcorner$#1$\urcorner$} % Text added where witnes(es) preserve a space
\newcommand*{\secluded}[1]{{[}#1{]}}
\newcommand*{\metatext}[1]{\{#1\}}
\newcommand*{\hand}[1]{\textsuperscript{#1}}
\newcommand*{\del}[1]{\textlbrackdbl{}#1\textrbrackdbl{}}
\newcommand*{\no}[1]{\emph{#1}\quad}
\newcommand*{\added}[1]{$\backslash{}$#1$/$}
\newcommand*{\corruption}[1]{\textdagger#1\textdagger}
\newcommand*{\fenestra}[1]{$\ulcorner$#1$\urcorner$}
\newcommand*{\lacuna}{\supplied{\textasteriskcentered\textasteriskcentered\textasteriskcentered}}
\newcommand*{\missingContent}[1]{$\stackrel{\mbox{\normalfont\small\kern-2pt #1}}{\dots{}}$}



% bibliography settings -----------------------------------------┐
\usepackage[
  style=authoryear-comp,
  giveninits=true,
  uniquename=init,
  mincrossrefs=20,
]{biblatex}
\addbibresource{bib/SalgadoInIsagogem.bib}
% ---------------------------------------------------------------┘




  \usepackage{draftwatermark}
  \SetWatermarkText{DRAFT}
  \SetWatermarkFontSize{3.5cm}
  \SetWatermarkColor[gray]{0.9}


% ════════════════════════════════════════════════════════╗
\begin{document}

% Cover page ──────────────────────────────────────────┐

\pagestyle{empty}

\begin{center}
\LARGE
\MakeUppercase{Gregorio Agustín Salgado, OSA}

\vfill

\Large
\MakeUppercase{In Isagogem Porphyrii, liber vere aureaus, in quo continentur quinque voces necessariae}

\bigskip

\normalsize
Bogotá, D.C., Biblioteca Nacional de Colombia, MS RM150

\vfill

\small
\textsc{editado por}

\bigskip

\Large

\MakeUppercase{Wainer Andrés Durán Guerrero}

\vfill

\normalsize
Universidad de los Andes \\

Bogotá, D.C. \\

Noviembre de 2021 \\

\scriptsize

(v. 0.0.1-dev)

\end{center}
% ────────────────────────────────────────────────────┘


\cleardoublepage
    
    
% TOC ──────────────────────────────────────────┐
\pagestyle{fancy}
\fancyhead{}
\fancyfoot[C]{\thepage}
\fancyhead[L]{Index generalis}  
\renewcommand*{\contentsname}{Index generalis}
\tableofcontents
% ──────────────────────────────────────────────┘


% headings, footers ────────────────────────────┐
\fancyhead{}
\fancyfoot[C]{\thepage}
\fancyhead[L]{Salgado: In Isagogem Porphyrii…}  
% ──────────────────────────────────────────────┘

\cleardoublepage


\label{}
\begin{otherlanguage*}{latin}
\beginnumbering
    
    \addcontentsline{toc}{part}{In Isagogem Porphyrii, liber vere aureaus, in quo continentur quinque voces necessariae}
    \pstart
    \eledchapter*{In Isagogem Porphyrii, liber vere aureaus, in quo continentur quinque voces necessariae}
    \pend
  
        \addcontentsline{toc}{chapter}{Praefatio Salgadi}
        \pstart
        \eledchapter*{\supplied{Praefatio Salgadi}}
        \pend
      
        \addcontentsline{toc}{section}{Praefatio}
        \pstart
        \eledsection*{Praefatio}
        \pend
      
\pstart
 \edtext{ Huius operis Auctor fuit \name{\textsc{Malchus}\index[persons]{Porphyrius}} cognomento \name{\textsc{Porphyrius}\index[persons]{Porphyrius}} natione Phoenico, Tyrius Patria, seu natione Iudaeus, volente \name{\textsc{Baronio}\index[persons]{Caesar Baronius}} tomo 2 \worktitle{Annalium}. Doctrina, non tam Aristotelicus, quam Platonicus. Praeceptores habuit \name{\textsc{Plotinum}\index[persons]{Plotinus}} et \name{\textsc{Longinum Criticum}\index[persons]{Longinus}}, condiscipulum \name{\textsc{Originem}\index[persons]{Origenes}}; Auditorem inter alios \name{\textsc{Chrysaorium}\index[persons]{Chrysaorius}} Romanum Patritium, cuius rogatu hoc opus fecit. Fuit Magicarum artium superstitionibus deditus, ac Christiani Religionis desertor, et oppugnator ut Magnus Pater \name{\textsc{Augustinus}\index[persons]{Augustinus Hipponensis}} refert 19 \worktitle{De Civitate Dei} capitulo 23. Inscriptio operis est Εισαγωγή, id est `introductio'. Institutum est agere de quinque vocibus quae communi voce `praedicabilia' vocantur scilicet de Genere, Differentia, Specie, Proprio, et Accidente. }{\lemma{}\Afootnote[nosep]{ \textsc{Commentarii Collegii Conimbricensis e Societate Iesu}\index[persons]{}, \worktitle{In universam dialecticam Aristotelis Stagirita} (Lugduni: sumpt. Iacobi Cardon et Petri Cavellat, 1622), p. 35-36. }} Duabus partibus absolvitur quarum una praefationem, et altera quinque vocum enarrationem continet. Haec secunda bipartita: in prima parte singula universalia seorsum declarantur. In posteriori inter se conferuntur. Nunc ad singula ascendamus. 
\pend

        \addcontentsline{toc}{chapter}{Praefatio Porphyrii}
        \pstart
        \eledchapter*{\supplied{Praefatio Porphyrii}}
        \pend
      
        \addcontentsline{toc}{section}{Textus praefationis Porphyrii}
        \pstart
        \eledsection*{Textus praefationis Porphyrii}
        \pend
      
        \addcontentsline{toc}{section}{Textus Graecus}
        \pstart
        \eledsection*{Textus Graecus}
        \pend
      
\pstart
\noindent%
 \textnormal{|}\ledsidenote{BNC 93vb} \edtext{\enquote{ ΠΟΡΦΥΡΙΟΥ ΕΙΣΑΓΩΓΗ. Ὄντος ἀναγκαίου, \textsc{Χρυσαόριε}\index[persons]{Chrysaorius}, καὶ εἰς τὴν τῶν παρὰ Ἀριστοτέλει κατηγοριῶν διδασκαλίαν, τοῦ γνῶναι τί γένος καὶ τί διαφορα τί τε εἶδος καὶ τί ἴδιο, etc. }}{\lemma{}\Afootnote[nosep]{ \textsc{Commentarii Collegii Conimbricensis e Societate Iesu}\index[persons]{}, \worktitle{In universam dialecticam Aristotelis Stagirita} (Lugduni: sumpt. Iacobi Cardon et Petri Cavellat, 1622), p. 37. }} 
\pend

        \addcontentsline{toc}{section}{Textus Latinus}
        \pstart
        \eledsection*{Textus Latinus}
        \pend
      
\pstart
\noindent%
 \textnormal{|}\ledsidenote{BNC 93vc} \edtext{\enquote{ Praefatio ad \textsc{Chrisaorium}\index[persons]{Chrysaorius} Cum necessarium sit, \textsc{Chrisaori}\index[persons]{Chrysaorius}, et ad \textsc{Aristotelis}\index[persons]{Aristoteles} \worktitle{Praedicamentorum} doctrinam, quidnam Genus, quid Differentia, quid Species, quid Proprium, et quid Accidens sit cognoscere, etc. }}{\lemma{}\Afootnote[nosep]{ \textsc{Commentarii Collegii Conimbricensis e Societate Iesu}\index[persons]{}, \worktitle{In universam dialecticam Aristotelis Stagirita} (Lugduni: sumpt. Iacobi Cardon et Petri Cavellat, 1622), p. 37. }} 
\pend

        \addcontentsline{toc}{section}{Summa textus}
        \pstart
        \eledsection*{Summa textus}
        \pend
      
\pstart
\noindent%
 \textnormal{|}\ledsidenote{BNC 93va} Primo ponit necessitatem huius tractatus ad notitiam Praedicamentorum acquirendam. Deinde \textnormal{|}\ledsidenote{BNC 94ra}  manifestat materiam de quo acturus est, scilicet, de quinque vocibus. Pertractat questiones faciles, sublimes omittit, quia indigent maiori inquisitione. Nos vero cum communi Dialectiorum persolvemus, quae agitari solent, sive faciles, sive altiores siquidem cum praesuppositis possunt \textsc{Tyrones}\index[persons]{Porphyrius} comprehendere, quae olim difficilia putabantur. 
\pend

        \addcontentsline{toc}{section}{Annotationes circa praefationem}
        \pstart
        \eledsection*{Annotationes circa praefationem}
        \pend
      
\pstart
 \textnormal{|}\ledsidenote{BNC 94rb} Quod cognitio Praedicabilis sit necessaria ad Praedicamentorum notitiam constat, quia praedicamentum est series et coordinatio praedicabilis ut omnes fatentur, secundum rectam subiectionem et praedicationem; igitur praedicabilia sunt partes praedicamentorum, ac proinde eorum notitia multum conducit ad horum cognitionem. 
\pend

\pstart
 Dicis, si notitia praedicabilis foret necessaria eam traderet \textsc{Aristoteles}\index[persons]{Aristoteles}; sed non constat, igitur non est necessaria. Dico non constare fuisse ad eo omissam, cum plura eius opera temporum oblivione interciderint, nec erat difficile praeceptoribus ex explicatis ab eo praesertim in \worktitle{Topicis} harum rerum doctrinam colligere, ut fecit Porphyrius. Cetera quae solent adduci ad annotandum textum, in tota quaestione explanantur. Unde Quaestio 3 de Universalibus. 
\pend

        \addcontentsline{toc}{section}{Quaestio 3. De universali secundum se}
        \pstart
        \eledsection*{Quaestio 3. De universali secundum se}
        \pend
      
\pstart
 Tria sunt consideranda circa universale secundum se, scilicet, ipsa ratio universalis, causa a qua fit, et actus eius qui est actualis praedicatio de inferioribus. Circa primum tria considerantur. Primum ipsa natura in communi considerata denominataque subiectum universale. Secundum est fundamentum universalitatis, scilicet, unitas abstracta a multis, et aptitudo ad essendum in multis. Tertium est intentio \textnormal{|}\ledsidenote{BNC 94rb} universalitatis, qua respicit inferiora. Nunc Articulus 1 explicat acceptionem `Universalis'. 
\pend

        \addcontentsline{toc}{section}{Articulus 1. Explicat acceptionem universalis}
        \pstart
        \eledsection*{Articulus 1. Explicat acceptionem `universalis'}
        \pend
      
\pstart
 Universale est duplex, scilicet, complexum et incomplexum. Complexum est propositio cuius veritas ex veritate multorum singularium dependet, verbi gratia `omnis homo est animal'. Hoc universale dicitur posterioristicum estque principium scientiae, ut ait \edtext{ \name{\textsc{Aristotelis}\index[persons]{Aristoteles}} I \worktitle{Posteriorum}, capitulo 7 }{\lemma{}\Afootnote[nosep]{ \textsc{Aristoteles}\index[persons]{Aristoteles}, \worktitle{Analytica Posteriora} 1.24; 1.30 (85b 24ss; 87b 37-39) }} et \edtext{ I \worktitle{Metaphysicae}, capitulo 1 }{\lemma{}\Afootnote[nosep]{ \textsc{Aristoteles}\index[persons]{Aristoteles}, \worktitle{Metaphysica} 1.1 (981b 10ss.) }}. De hoc non tractamus hic. Universale incomplexum est quod simplex dicens ordinem ad plura, vocaturque universale prioristicum, de quo hic agimus. 
\pend

\pstart
 Hoc universale incomplexum est multiplex, scilicet, universale in significando, universale in causando, universale in essendo seu praedicando. Universale in significando est illa vox, conceptus vel signum plura significans, sicut vox `homo' appellatur universale quia significat plures homines. Universale in causando est id quod plures conceptus effectus causat, sic Deus Opus Maximum est causa universalissima; sic Caelum causa universalis dicitur. Universale in essendo est quod respicit plura in quibus est et de quibus praedicatur, ut animal invenitur in omnibus animalibus et homo in omnibus hominibus. 
\pend

\pstart
 Universale in essendo sic definitur a  \name{\textsc{Philosopho}\index[persons]{Aristoteles}}, VII \worktitle{Metaphysicae}, texto 14 : \edtext{\enquote{est unum in multis, et de multis.}}{\lemma{}\Afootnote[nosep]{ \textsc{Aristoteles}\index[persons]{Aristoteles}, \worktitle{Metaphysica} 7. (Non invenimus.) }} \textnormal{|}\ledsidenote{BNC 94va}   et  libro I \worktitle{Peri-hermeneias}, capitulo 5 : \edtext{\enquote{ universale est id quod de pluribus praedicari aptum est. }}{\lemma{}\Afootnote[nosep]{ \textsc{Aristoteles}\index[persons]{Aristoteles}, \worktitle{Analytica Posteriora} 1.11 (77a 5-9) }}. Explico definitiones. Particula `unum' explicat subiectum et fundamentum, scilicet, rem unam et unitatem; unitas vero separata a multis, et illis communicabilis, nam quod aptum est inesse. Unitas enim et aptitudo ut sit in illis est fundamentum relationis universalis. Particula `in multis' explicat ipsam relationem in qua consistit universalitas formalis. Dicitur autem esse in multis per identitatem cum illis; dicitur vero esse de multis per praedicationem, et haec praedicabilitas est passio universalis, sicut risibilitas hominis. 
\pend

\pstart
 Prima enim definitio est essentialis, scilicet, `esse unum in multis'. Secunda vero descriptiva, scilicet, `et de multis'. Circa quod est notandum essentiam universalis non esse actu inesse pluribus, quod notat \textsc{Angelicus Praeceptor}\index[persons]{Thomas Aquinas} ibidem lectione 13, sed aptum inesse multis. Similiter passio universalis non est actu praedicari de multis, sed aptitudo ut praedicetur de multis. Unde illa verba significant aptitudinem, non vero actum. Ratio utriusque: nam si nullum animal existeret a parte rei, vera esset haec propositio: `homo est animal', `equus est animal', quia illa copula `est' abstrahit a tempore, significando conexionem praedicatorum et non existentiam: igitur essentia non est actu esse, sed posse esse. Pariter: in homine est passio risibilitas, hoc est, posse ridere, non actu ridere; igitur in universali passio est potentia ut praedicetur de multis, non vero actualis praedicatio. 
\pend

\pstart
 Ex dictis inferes 1. Discrimen inter universale metaphysicum et logicum. Cum enim Metaphysicus contempletur principaliter naturas, logicus vero ipsas intentiones. Natura seu subiectum abstractum a pluribus dicitur universale metaphysicum, natura contemplata principaliter, abstractio vero ut conditio. Universale vero logicum principaliter respicit ipsam intentionem universalitatis, naturam vero in \textnormal{|}\ledsidenote{BNC 94va} qua fundatur secunda intentio praesuppositive: agit enim de secundis intentionibus ut fundatis in primis. Idcirco Metaphysicus et Logicus agunt de eodem, sed diverso modo ut ait \edtext{ Divum \name{\textsc{Thomas}\index[persons]{Thomas Aquinas}} IV \worktitle{Metaphysicae}, lectione 4. }{\lemma{}\Afootnote[nosep]{ \textsc{Thomas Aquinas}\index[persons]{Thomas Aquinas}, \worktitle{In Metaphysicae Aristotelis}, 4.4, nn. 572-577 (Taurini-Roma: Marietti, pp. 160-161) }} 
\pend

\pstart
 Inferes 2. Universalitatem et particularitatem pertinere ad statum naturae, quae denominari potest universalis vel particularis. Triplex est statum naturae ex \edtext{ Divo \name{\textsc{Thoma}\index[persons]{Thomas Aquinas}} \worktitle{De ente et essentia}, capitulo 4. }{\lemma{}\Afootnote[nosep]{ \textsc{Thomas Aquinas}\index[persons]{Thomas Aquinas}, \worktitle{De ente et essentia}, 5.1-5 (Roma: Editori di San Tomasso, 1976, p. 378) }} Primus est naturae secundum se, in quo non considerantur nisi ea, quae constituunt ipsam naturam et quidditatem. Vocatur etiam status indifferentiae; natura enim secundum se indifferens est ad praedicata accidentalia. Vocatur etiam status solitudinis quia est sola ab omni extrinseco praedicato. Vocatur communis negative, quia non intelligitur natura multiplicata. 
\pend

\pstart
 Secundus status est singularitatis, secundum esse quod habet in singularibus. Tertius est status abstractionis, secundum esse quod habet enim abstractione intellectus, qui dicitur etiam status solitudinis non solitudine ab omni extrinseco praedicato, sed ab individuis. Isti duo non competunt naturae secundum se quia neuter eorum est praedicatum essentiale naturae. Si enim natura secundum se esset universalis numquam esset particularis vel econtra. Unde praedicata, quae non sunt essentialia naturae, illi denegantur in statu secundum se. 
\pend

        \addcontentsline{toc}{section}{Articulus 2. Utrum ea quae alia dicuntur existant in rerum natura}
        \pstart
        \eledsection*{Articulus 2. Utrum ea quae alia dicuntur existant in rerum natura}
        \pend
      
\pstart
 \textsc{Heraclitus}\index[persons]{Heraclitus} et \textsc{Cratylus}\index[persons]{Cratylus Atheniensis} negaverunt esse in rerum natura universalia, sic refert \edtext{ \name{\textsc{Aristoteles}\index[persons]{Aristoteles}} I \worktitle{Metaphysicae}, capitulis 6; et IV, capitulo 5 }{\lemma{}\Afootnote[nosep]{ \textsc{Aristoteles}\index[persons]{Aristoteles}, \worktitle{Metaphysica} 1.6; 1.5 (987a 31-34; 1010a 10-15) }}. Nostri vero longe non distant ab hoc delirio,  admittunt voces universalis correspondentes universalibus conceptibus negant tamen naturas universales. Ita \textnormal{|}\ledsidenote{BNC 95ra}   \edtext{ \name{\textsc{Ockhamus}\index[persons]{Gulielmus Occamus}} capitulo 14 \worktitle{Logicae} }{\lemma{}\Afootnote[nosep]{ \textsc{Guillelmus de Ockham}\index[persons]{Gulielmus Occamus}, \worktitle{Summa Logicae}, 1, 14-15 (St. Bonaventure, NY., 1974, pp. 47-54) }} et \edtext{ in I distinctione 2, quaestio 7; }{\lemma{}\Afootnote[nosep]{ \textsc{Guillelmus de Ockham}\index[persons]{Gulielmus Occamus}, \worktitle{Scriptum I Sententiarum}, 5.1-5 (ed. Brown, II, p. 248-248, 252) }} \textsc{Grabriel}\index[persons]{Gabriel Biel}  et alii. Divinus \textsc{Plato}\index[persons]{Plato} econtra concedit naturas universales esse quae a parte rei a singularibus separatas et omnium individuorum esse principia, sic testatur \edtext{ \name{\textsc{Aristoteles}\index[persons]{Aristoteles}} I \worktitle{Metaphysicae}, capitulo 6; et VI, capitulo 8. }{\lemma{}\Afootnote[nosep]{ \textsc{Aristoteles}\index[persons]{Aristoteles}, \worktitle{Metaphysica} 1.6; 7.2 (987a 29-30; 1028b 18-22) }} Haec opinio si in bonum sensum explicetur sit quia intelligitur de Ideis in mente divina existentibus est secura, sin autem erronea. Pro Ideis contendunt gravissimi Reverendo Praeceptore Magno Patre \edtext{ \name{\textsc{Augustino}\index[persons]{Augustinus Hipponensis}} libro \worktitle{83 quaestionibus},  quaestione 46 }{\lemma{}\Afootnote[nosep]{ \textsc{Augustinus Hipponensis}\index[persons]{Augustinus Hipponensis}, \worktitle{De diversis quaestionibus octoginta tribus} q. 46, 1-2 (PL 40, 29-31) }}, \textsc{Dionysius Areopagita}\index[persons]{Dionysius Areopagita}, \textsc{Nazianzenus}\index[persons]{Gregorius Nazianzenus}, \textsc{Seneca}\index[persons]{Lucius Annaeus Seneca}, \textsc{Plutarchus}\index[persons]{Lucius Mestrius Plutarchus}, \textsc{Eugubinus}\index[persons]{Augustinus Steuchus Eugubinus}, \textsc{Picus Mirandula}\index[persons]{Ioannes Picus de Mirandula} et alii, quos referunt \textsc{Conimbricensis}\index[persons]{} et \textsc{Complutensis}\index[persons]{}. Multa refert circa hoc punctum \textsc{Ioannes Caramuel}\index[persons]{Ioannes Caramuel Lobkowitz}. 
\pend

\pstart
 Dico 1. Contra auctores non solum dantur in rerum natura voces aut conceptus formalis communis, sed etiam conceptus obiectivi seu naturae universales. Ita \edtext{ Noster \name{\textsc{Fundatissimus Doctor}\index[persons]{Aegidius Romanus}}  in I distinctione 36, parte 2, quaestione 2; et distinctione 19, parte 1; et quodlibet 2, quaestio 6 }{\lemma{}\Afootnote[nosep]{ \textsc{Aegidius Romanus}\index[persons]{Aegidius Romanus}, \worktitle{In primum librum Sententiarum} d. 36, q. 2, a. 2 (Córdoba: apud Lazarum de Risquez et Antorium Rosellon, 1699, f. 672-675)  \textsc{Aegidius Romanus}\index[persons]{Aegidius Romanus}, \worktitle{In primum librum Sententiarum} d. 19, p. 1 (Córdoba: apud Lazarum de Risquez et Antorium Rosellon, 1699, f. 366-387)  \textsc{Aegidius Romanus}\index[persons]{Aegidius Romanus}, \worktitle{Quodlibeta} 2, q.6 (Lovaina: typis Hieronymi Nempaei, 1646, f. 61-64) }}, quae sequitur \textsc{Gavardus}\index[persons]{Fridericus Nicolaus Gavardi}  hic. Ita \edtext{ \name{\textsc{Angelicus Praeceptor}\index[persons]{Thomas Aquinas}} \worktitle{De ente et essentia}, capitulo 4 }{\lemma{}\Afootnote[nosep]{ \textsc{Thomas Aquinas}\index[persons]{Thomas Aquinas}, \worktitle{De ente et essentia}, 5 (Roma: Editori di San Tomasso, 1976, p. 378-379) }} et \edtext{ \worktitle{Opusculo} 55 et 56 }{\lemma{}\Afootnote[nosep]{ \textsc{Pseudo Thomas Aquinas}\index[persons]{} \worktitle{Opuscula omnia}, 54, 55 (Lyon: apud Haeredes Iacobi Iuntae, 1562, p. 372-376) }}; \edtext{ Beatus \name{\textsc{Albertus}\index[persons]{Albertus Magnus}} Tractatus 2 \worktitle{Logicae} }{\lemma{}\Afootnote[nosep]{ \textsc{Albertus Magnus}\index[persons]{Albertus Magnus}, \worktitle{Super Porphyrium de V universalibus}, 1 (ed. Santos Noya, 2004, p. 17-40) }}. Omnes Thomistae. Item \edtext{ \name{\textsc{Scotus}\index[persons]{Ioannes Duns Scotus}} passim in \worktitle{Logica} }{\lemma{}\Afootnote[nosep]{ \textsc{Ioannes Duns Scotus}\index[persons]{Ioannes Duns Scotus}, \worktitle{Quaestiones in Librum Porphyrii Isagoge}, q. 9 (ed. Andrews, Etzkorn, Gál et al, 1999, p. 43-44) }} et \edtext{ in II distinctione 3, quaestione 1 }{\lemma{}\Afootnote[nosep]{ \textsc{Ioannes Duns Scotus}\index[persons]{Ioannes Duns Scotus}, \worktitle{Ordinatio}, 1, q. 1 (ed. Balić, Opera omnia II, 1950, pp. 125-128) }}, quae sequitur \textsc{Merinero}\index[persons]{Ioannes de Merinero}. 
\pend

\pstart
 Probatur 1. Ex  \name{\textsc{Philosopho}\index[persons]{Aristoteles}} I \worktitle{Peri-hermeneias}, capitulo 5 : \edtext{\enquote{ rerum aliae sunt universales, aliae particulares }}{\lemma{}\Afootnote[nosep]{ \textsc{Aristoteles}\index[persons]{Aristoteles}, \worktitle{De interpretatione} 7 (17a 38-39) }}; et  libro 1 \worktitle{Posterioribus}, capitulo 2 : \edtext{\enquote{universalia notiora esse singularibus}}{\lemma{}\Afootnote[nosep]{ \textsc{Aristoteles}\index[persons]{Aristoteles}, \worktitle{Analytica Posteriora} 1.2 (72a 1-5) }}, illa intellectu, haec vero sensu cognosci. Idem asserit \edtext{II \worktitle{De anima}, capitulo 5}{\lemma{}\Afootnote[nosep]{ \textsc{Aristoteles}\index[persons]{Aristoteles}, \worktitle{De anima} 2.5 (417b 21-25) }} quibus locis non loquitur de vocibus, nam voces communis etiam sensu percipiuntur, igitur loquitur de rebus. 
\pend

\pstart
 Probatur ratione: hae praedicationes verae sunt et necessariae. Petrus est homo, homo est animal, igitur id quod in illis enuntiatur est res aliqua et non voces aut conceptus formales. Consequentia probatur: si voces aut conceptus formalis praedicarentur utraque esset falsa et impossibilis, nam esset sensus Petrus est vox homo, conceptus homo est conceptus animal; sed hoc non est \textnormal{|}\ledsidenote{BNC 95ra} discendum, igitur neque illud. 
\pend

\pstart
 Dent auctores conceptus universales significantem, verbi gratia `hominem' aut `animal', esse nomen collectivum non significans aliquid unum, sed adunatum, scilicet omnes homines. Contra 1: quando universale ponitur ex parte praedicati verbi gratia Petrus est homo, erit sensus Petrus est omnes homines, sed hoc est falsum: igitur et solutio. \supplied{Contra} 2: quando terminus communis est subiectus sine distributione, verbi gratia homo currit, sensus erit homines omnes currere, sed hoc est falsum: igitur. \supplied{Contra} 3: haec propositio `essentia divina non generat esset falsa, nam esset sensus, omne quod est essentia divina non generat, sed Pater est res, quae est essentia divina: igitur non generat', sed hoc est haereticum: igitur. Vide alias solutiones et responsiones in \textsc{Ioannes Caramuel}\index[persons]{Ioannes Caramuel Lobkowitz}. 
\pend

\pstart
 Probatur 2: sciae tractant de rebus sed de singularibus: igitur de universalibus. Probatur maior: quando Physica definit hominem, `homo est animal rationale', non definit hanc vocem `homo'; cum haec vox non constat anima et corpore: igitur. Probatur minor: scia es de necessariis et perpetuis, sed singularia sunt contingentia et corruptibilia: igitur. 
\pend

\pstart
 Dico 2: hae naturae universales non sunt a parte rei a singularibus separatae. Est communis inter Philosophos, quondam probat contra \textsc{Platonem}\index[persons]{Plato} noster Princeps \edtext{VII \worktitle{Metaphysicae}, texto 51 }{\lemma{}\Afootnote[nosep]{ \textsc{Aristoteles}\index[persons]{Aristoteles}, \worktitle{Metaphysica} 7.2 (1028b 18-23) }}. Ita \edtext{ \name{\textsc{Magnus Pater}\index[persons]{Augustinus Hipponensis}}  libro \worktitle{83 quaestionibus},  questione 46 }{\lemma{}\Afootnote[nosep]{ \textsc{Augustinus Hipponensis}\index[persons]{Augustinus Hipponensis}, \worktitle{De diversis quaestionibus octoginta tribus} q. 46, 1-2 (PL 40, 29-31) }}, quae sequitur \textsc{Fundatissimo Doctore}\index[persons]{Aegidius Romanus}  loco citato et \textsc{Angelico Doctore}\index[persons]{} loco citato \worktitle{Metaphysicae}, lectio 4 et \worktitle{Opusculo} 55 et 56. 
\pend

\pstart
 Probatur ratione: quidquid praedicatur de aliquo secundum essentiam necessario debet existere in ipso, sed naturae universales, verbi gratia natura hominis; natura animalis, essentialiter praedicantur de singularibus verbi gratia de Petro, Ioanne, etc; igitur naturae quae denominantur universales debent existere in singularibus. 
\pend

\pstart
 Probatur 2: si naturae universales existerent a parte rei \textnormal{|}\ledsidenote{BNC 95va}  separatae, eadem natura esset simul universalis et singularis; sed hoc implicat; igitur et illud. Probatur sequela: esset universalis ut supponitur, sed etiam esset singularis; igitur. Probatur minor: talis natura vere existeret a parte rei; igitur vel quod creatum vel increatum. Si hoc esset Deus; igitur non esset eiusdem rationis cum individuis. Si primum; igitur existeret ubi alicuius actionis realis, sed quaelibet actio realis terminatur ad rem singularem; igitur talis natura vere esset singularis. Probatur minor: illius proprium est fieri, cuius proprium est esse seu subsistere, ut docet \edtext{ Divum \name{\textsc{Thomam}\index[persons]{Thomas Aquinas}} I parte, quaestione 65, articulo 4 }{\lemma{}\Afootnote[nosep]{ \textsc{Thomas Aquinas}\index[persons]{Thomas Aquinas}, \worktitle{Summa Theologiae}, I.65.4 (Roma: Commissio Leonina, 1889, pp. 152-153) }}. Sed in rebus creatis esse seu subsistere est proprium suppositi singularis, non vero naturae communis, ut docet idem \edtext{ \name{\textsc{Angelicum Doctorem}\index[persons]{Thomas Aquinas}} quaestione 9 \worktitle{De potentia}, articulo 5, ad 13 }{\lemma{}\Afootnote[nosep]{ \textsc{Thomas Aquinas}\index[persons]{Thomas Aquinas}, \worktitle{Quaestiones disputatae de potentia Dei}, 9.5.13 (Taurini: Typographia Pontificia Petri Marietti, 1895, p. 276) }}, igitur id quod fit est singulare. 
\pend

\pstart
 Obiectio 1. Per prima Sententia: Conceptus obiectivus universalis vel est aliquid unum vel plura. Si plura: igitur universale solum erit nomen collectivum. Si unum: in quantum unum non est aliquod reale, sed in quantum unum est obiectum illius conceptus universalis: igitur inquantum obiectum non est aliquid reale. Maior probatur: illud est ens reale quod est capax existentiae; sed inquantum unum non existit extra intellectum, nec potest existere: igitur non est quid reale. Confirmatur: si darentur naturae universales sequerentur plura inconvenientia: igitur non dantur. Probatur antecedens. Sequerentur 1: omnia illa individua sunt unum universale: igitur sunt idem inter se. Probatur consequentia: quaecunque uni et eidem sunt eadem inter se; sed Petrus et Paulus sunt uni et eidem, scilicet, homo idem: igitur sunt idem realiter inter se. 
\pend

\pstart
 Sequerentur 2: vel ipsa natura universalis tota descendit ad quodlibet individuum vel una pars ad unum et altera ad alterum. Sed hoc ultimum non potest dici, nam sic Petrus non esset integer homo, sed hominis pars: igitur primum. Sed tunc sequerentur in creatis mysterium Trinitatis: igitur non est dicendum. Probatur sequela: natura \textnormal{|}\ledsidenote{BNC 95vb} humana una existens ut aliquid reale tota communicatur uni individuo et tota alteri: igitur. 
\pend

\pstart
 Sequerentur 3: homo in universali vel est corruptibilis vel non? Si non est corruptibilis, igitur corruptibilitas non est propria hominis passio. Si est corruptibilis, igitur et generabilis. Igitur ad illum potest terminari generatio et existentia, quod est ponere ideas platonicas. 
\pend

\pstart
 Respondet distinctionem. Ille conceptus obiectivus est unus unitate reali, nego; unitate rationis et praecissionis, concedo. Ad probationem distinctionem maiorem: in quantum unum non est aliquid reale formaliter, concedo; materialiter, nego. Sicut enim non existat natura universalis a parte rei sub conceptu unitatis, existit tamen sub alio statu et hoc sufficit ut dicatur ens reale et capax existentiae secundum se et non in omni statu. Unde patet solutio ad probationem. 
\pend

\pstart
 Ad confirmationem nego antecedens. Ad primam inconveniens nego sequelam. Ad probationem nego per nunc axioma vel si debet exponi suo loco exponemus. Ad secundam distinctionem tota natura communicatur uni individuo communicatione integrali, nego; communicatione praedicabili, concedo. Communicatur totum integrale partibus integrantibus et communicatur totum praedicabile partibus subiectivis. Unde in divisione integrali, quaelibet pars non adaequat totum. In universali vero quaelibet pars subiectiva adaequat totum quantum ad naturam communicatam, non vero quantum ad extensionem et multiplicabilitatem. In mysterio enim Sanctae Trinitatis non fit communicatio naturae per eidem multiplicabilitatem, sed per identitatem naturae cum personis. 
\pend

\pstart
 Ad tertiam distinctionem: homo in communi est corruptibilis radicaliter, concedo; exercite, nego. Naturae competit corruptibilitas sed exercetur in individuis, sicut \textnormal{|}\ledsidenote{BNC 96rb}  risibilitas convenit ei. Sed exercitium ridendi non competit naturae in communi, sed in individuo. In omnibus passionibus propriis hoc fit, siquidem connexio et convenientia earum est respectu naturae communis, exercitium vero respectu individuorum. 
\pend

\pstart
 Obiectio 2. Ex \edtext{ \name{\textsc{Aristotele}\index[persons]{Aristoteles}} capitulo \worktitle{De substantia} definiente secundas Substantias, scilicet universales: \enquote{illas significare qualequid} }{\lemma{}\Afootnote[nosep]{ \textsc{Aristoteles}\index[persons]{Aristoteles}, \worktitle{Categoriae} 5 (2a 14-26) }}, sed significare vocibus convenit: igitur. Item  \edtext{ I \worktitle{De anima}, texto 8: \enquote{animal autem universale aut nihil est, aut posterius} }{\lemma{}\Afootnote[nosep]{ \textsc{Aristoteles}\index[persons]{Aristoteles}, \worktitle{De anima} 1.1 (402b 7-9) }}; similiter autem, et si quod communi aliud praedicatur. Igitur universalitas solus vocibus competit. Respondeo \textsc{Philosophum}\index[persons]{Aristoteles} in primo loco loqui de nominibus significantibus secundas substantias, non vero de ipsis ut patet. Secundum locum sic glozat \edtext{ \name{\textsc{Angelicum Doctorem}\index[persons]{Thomas Aquinas}} ibidem lectio 1 et libro 3 lectio 12: \enquote{Animal universale nihil est, ab speciebus separatum ut fingebat Plato: est tamen in eiusdem ut quid posterius sit, ab intellectu fundatum, et ex earum convenientia sumptum.} }{\lemma{}\Afootnote[nosep]{ \textsc{Collegium Complutense S. Cyrilli Discalceatorum Fratrum Ordinis Beatae Mariae de Monte Carmeli}\index[persons]{}, \worktitle{ Artium cursus sive Disputationes in Aristotelis Dialecticam et Philosophiam naturalem iuxta angelici doctoris D. Thomae doctrinam et scholam }, d. 3, q. 1, ad 9 (Madrid: apud Ioannem de Orduña, 1624, p. 193) }} 
\pend

\pstart
 Obiectio 3. Quidquid existit a parte rei est singulare: igitur a parte rei nulla datur res universalis. Patet consequentia quia universale opponitur singulari: igitur si a parte rei omnia singularia sunt, nulla res est universalis. Respondeo distinguendo consequens: non datur separata ab singularibus, concedo: identificata cum illis nego. Sicut naturae universales sunt a parte rei idem cum singularibus, ita existant per existentiam singularium: sic enim dantur a parte rei naturae universales singularizatae. 
\pend

\pstart
 Obiectio 4. Pro secunda sententia: dantur sciae; igitur naturae universales separatae. Probatur consequentia: veritates sciarum sunt aeternae et perpetuae, sed hae non possunt esse in singularibus: igitur. Probatur minor: singularia sunt corruptibilia et contingentia: igitur. Confirmatur: dantur causae exemplares ab quibus singularia participant suam essentiam, sicut Petrus participatione hominis est homo et participatione animalis est animal: igitur dantur homo et animal separati ut istae participationes derivantur. 
\pend

\pstart
 \textnormal{|}\ledsidenote{BNC 96rb} Respondeo negando consequentia. Ad probationem distinguo maiorem: veritates sciarum sunt aeternae et perpetuae quoad existentiam, nego: quoad conexionem praedicatorum, concedo. Hoc manet explicatum. Ad confirmationem distinguo antecedens: dantur causae exemplares quae quidem sunt Ideae divinae, concedo; quae sunt naturae creatae a singularibus separatae, nego. 
\pend

\pstart
 Obiectio 5. Unumquodque a suo simili praedicatur, sed multa praeducuntur a causa dissimili ut herbae a Sole: igitur extra res debent dari causae universales. Item si non darentur tales naturae universales separatae, intellectus concipendo illas falleretur, siquidem conciperet, quod non est in re, sed hoc non est dicendum: igitur. Respondeo distinguendo maiorem: si producatur a causa univoca, concedo; a causa aequivoca ut est Sol, nego. Ad secundum nego maiorem: intellectus enim vere concipit quod est in re et non concipiat omne illud, quod est in re ob sui imperfectionem. Articulus 3. Utrum ante mente, etc. 
\pend

        \addcontentsline{toc}{section}{Articulus 3. Utrum ante mentis operationem detur unitas formalis in natura?}
        \pstart
        \eledsection*{Articulus 3. Utrum ante mentis operationem detur unitas formalis in natura?}
        \pend
      
\pstart
 Primo sciendum est naturam essentialiter singularem non posse esse universalem. Quare natura divina, cuius essentialiter singularis et actus purus non est universalis. Item ens per accidens ut acervus et quod componitur ex entibus completis aut diversorum praedicamentorum non potest esse materia universalis. Nam universale est unum simplex, sed talia entia non sunt unum simplex: igitur non sunt materia universalis. Eadem ratione excluduntur analoga et aequivoca, nam sicut defectu unitatis excluduntur a linea praedicamentali, ita etiam a materia universalium. 
\pend

\pstart
 Secundo sciendum est ex \edtext{ \name{\textsc{Aristotele}\index[persons]{Aristoteles}} V \worktitle{Metaphysicae}, capitulo 6 et libro 10, capitulo 1. \enquote{Unum esse ens indivisum in se, \textnormal{|}\ledsidenote{BNC 96vb}  et divisum a quolibet alio} }{\lemma{}\Afootnote[nosep]{ \textsc{Franciscus Suarez}\index[persons]{}, \worktitle{Disputationes Metaphysicae}, 4.1.14 (París: apud Ludovicum Vives, 1861, p. 119) }}, igitur unitas nihil est aliud quem in divisio dicens per formali carentiam divisionis et ex parte entis aliquod positivum sic colligitur ex \edtext{ Doctore \name{\textsc{Thoma}\index[persons]{Thomas Aquinas}} I parte, quaestione 11, articulo 2 }{\lemma{}\Afootnote[nosep]{ \textsc{Thomas Aquinas}\index[persons]{Thomas Aquinas}, \worktitle{Summa Theologiae}, I.11.2 (Roma: Commissio Leonina, 1888, p. 109-110) }}; et in \edtext{ I, distinctione 9, quaestione 4, articulo 1, ad }{\lemma{}\Afootnote[nosep]{ \textsc{Thomas Aquinas}\index[persons]{Thomas Aquinas}, \worktitle{Scriptum super libros Sententiarum I}, I.8.4.1 (ed. Mandonnet, 1929, pp. 219-220) }}; et \edtext{ distinctione 24, quaestione 1, articulo 3 }{\lemma{}\Afootnote[nosep]{ \textsc{Thomas Aquinas}\index[persons]{Thomas Aquinas}, \worktitle{Scriptum super libros Sententiarum I}, I.24.1.3 (ed. Mandonnet, 1929, pp. 581-583) }} Unde tot erunt unitates quot divisiones. 
\pend

\pstart
 Divisio est duplex secundum \edtext{ \name{\textsc{Caietanum}\index[persons]{Thomas de Vio Caietanu}} \worktitle{De ente et essentia} capitulo 4, quaestione 6 }{\lemma{}\Afootnote[nosep]{ \textsc{Thomas de Vio Caietanu}\index[persons]{Thomas de Vio Caietanu}, \worktitle{De ente et essentia}, 4.6 27-30 (Lyon: apud Gulielmum Rovillium, 1588, p. 245) }} alia materialis et alia formalis. Materialis causatur per principia materialia seu differentias individuales. Hac divisione dividitur unaquaeque species in sua individua. Divisio formalis causatur per principia formalia seu differentias essentiales. Hac differunt species cuiuscunque generis. Haec est duplex, scilicet generica et specifica. Generica causatur per principia magis communia. Specifica vero per minus communia. Verbi gratia Petrus differt a Paulo non per principia formalia generica aut specifica, uterque enim est animal rationale: igitur per principia materialia seu per differentias individuales. Ideoque inter eos tantum est divisio materialis. Ast Petrus differt ab Bucephalo non solum ratione materiae, sed ratione formale; scilicet rationalitate. Igitur inter eos non solum datur divisio materialis, verum etiam formalis specifica. Ceterum Petrus differt ab hac Cedro non solum ratione materiae, nec solum ratione formae specificae, verum etiam ratione formae genericae, scilicet, animalitate. Igitur differt tam materialiter quam formaliter specifice et generice. Similiter differt ab hoc Adamante ratione superiori, scilicet animatu seu vitalitate: sic de aliis discurrendo per praedicamenti substantiae lineam. 
\pend

\pstart
 Unde sequitur unitatem numericam esse maximam unitatem quia nullam admittit divisionem. Similiter formale minorem esse illa quia non excludit divisionem numericam, et \del{diversitate}\added{unitate} formali per principium formale magis communi plus admittere divisionis. Verbi gratia in Petro unitas, qua est hoc individuum est maxima \textnormal{|}\ledsidenote{BNC 96vb} siquidem vi eius manet indivisibilis. Unitas vero qua est unus homo praecedenti est minor, nam vi eius non excluditur quod sit hic vel ille homo. Unitas enim qua est animal minor est praecedentibus, cum vi illius non sit indivisibile in hoc animal, scilicet hominem vel illud animal, scilicet equus: sic de aliis gradibus semata proportione. 
\pend

\pstart
 Ratio huius est: unita dicit carentiam divisionis. Carentia tantum maior est quantum minus relinquitur in subiecto de forma, qua privat ut videnti patebit. Igitur illa carentia erit maxima, quae nihil relinquit in subiecto suae formae; sed unitas numerica nihil relinquit de divisione. Formalis vero admittit magis, magisque iuxta latitudinem explicatam. Igitur unitas numerica est maxima; formalis vero minor, minoraque secundum causatum per principium magis, magisque commune. 
\pend

\pstart
 Prima sententia tenet a parte rei non dari unitatem formalem. Ita aliqui moderni inter quos est \textsc{Torrejon}\index[persons]{Petrus Fernandez de Torrejon} et \textsc{Martinius}\index[persons]{Ioannes Martinez de Prado} . Secunda econtra docet a parte rei dari unitatem formale est communis in Schola Divi \textsc{Thomae}\index[persons]{Thomas Aquinas} et \textsc{Scoti}\index[persons]{Ioannes Duns Scotus}, ut refert \edtext{ Doctor \name{\textsc{Suarez}\index[persons]{Franciscus Suarez}} VII \worktitle{Metaphysicae} sectione 1, numero 2 }{\lemma{}\Afootnote[nosep]{ \textsc{Franciscus Suarez}\index[persons]{}, \worktitle{Disputationes Metaphysicae}, 7.1.13-14 (París: apud Ludovicum Vives, 1861, pp. 254-255) }}. Colligitur ex \edtext{ \name{\textsc{Aristotele}\index[persons]{Aristoteles}} V \worktitle{Metaphysicae}, texto 11 }{\lemma{}\Afootnote[nosep]{ \textsc{Aristoteles}\index[persons]{Aristoteles}, \worktitle{Metaphysica} 5.11 (1018b 32-35) }}. 
\pend

\pstart
 Dico a parte rei datur unitas formalis. Probatur 1. Unitas formalis est carentia divisionis formalis, sed Petrus a parte rei habet carentiam divisionis formalis. Igitur a parte rei habet unitatem formalem. Probatur minor. Sicut Petrus a parte rei non dividitur in plures materias seu individua, ita non dividitur in plures formas seu essentias; sed eo ipso Petrus habet carentiam divisionis formalis: igitur et carentiam divisionis formalis. 
\pend

\pstart
 Confirmatur: a parte rei non solum datur divisio materialis inter individua eiusdem speciei, sed etiam divisio formalis inter individua diversorum specierum. \textnormal{|}\ledsidenote{BNC 97ra}  Igitur in Petro praeter unitatem numeralem datur unitas . formalis. Probatur consequentia. Carentiae regulantur per formas vel habitus oppositos, sed a parte rei dantur divisiones materialis et formalis: igitur et unitates. Primus antecedens probatur. Ante omnem operationem intellectus Petrus distinguitur a Paulo materialiter et ab equo essentialiter: igitur dantur talis divisiones. 
\pend

\pstart
 Probatur 2. Petrus et Paulus a parte rei habent relationem similitudinis seu identitatis, sed relatio similitudinis fundatur in unitate ex \edtext{ \name{\textsc{Aristotele}\index[persons]{Aristoteles}} V \worktitle{Metaphysicae}, texto 20 }{\lemma{}\Afootnote[nosep]{ \textsc{Aristoteles}\index[persons]{Aristoteles}, \worktitle{Metaphysica} 5.18 (1022a 25-35) }}. Igitur a parte rei datur talis unitas formalis. Nec dicas relationem fundari in unitate numerica, siquidem in ipsa sunt dissimiles. Confirmatur. Si non daretur talis unitas formalis, multa concedenda a parte rei, non salvarentur: igitur. Probatur antecedens. A parte rei conceditur una species praestantissima in unoquoque genere ceterarum mensura: a parte rei \enquote{unum uni est contrarium} et alia huiusmodi; sed talia non possunt intelligi sine unitate formali: igitur. 
\pend

\pstart
 Obiectio 1. \edtext{ \name{\textsc{Aristoteles}\index[persons]{Aristoteles}} IV \worktitle{Metaphysicae}, texto 7 }{\lemma{}\Afootnote[nosep]{ \textsc{Aristoteles}\index[persons]{Aristoteles}, \worktitle{Metaphysicae} 4. }}, agens de uno solum numerat unitatem numericam, specificam et genericam, sed istae duae sunt per intellectum. Igitur a parte rei solum datur unitas numerica. Respondeo \textsc{Philosophum}\index[persons]{Aristoteles} sub unitate generica et specifica comprehendisse unitatem formalem, nec \emph{imagine [?]}; siquidem unitas formalis est fundamentum ex parte naturae respectu unitates specificae et genericae. Quapropter dicitur fundamentaliter unitas generica vel specifica. 
\pend

\pstart
 Obiectio 2. Sicut non datur nisi duplex ens, scilicet reale et rationis, ita non datur nisi duplex unitas realis, scilicet et rationis. Sed unitas realis est numerica. Igitur formalis est solum per intellectum. Probatur minor. Omnia quae a parte rei sunt in singularibus sunt singularizatas: \textnormal{|}\ledsidenote{BNC 97rb} igitur quamvis Petrus sit homo, vivens, sensibilis a parte rei; nihil tamen istorum erit commune sed singulare. Igitur omina erunt unum numero: igitur solum datur unitas numerica. 
\pend

\pstart
 Respondeo negando minorem. Entitas realis Petri non solum continet principia materialia, sed etiam formalia. Similiter unitas non solum continet unitatem numericam, sed etiam formalem et a parte rei non distinguantur hae unitates, sicut nec differentiae. Ad probationem distinguo maiorem. Omnia quae a parte rei sunt in individuo sunt singularizata et virtualiter distincta, concedo; sine distinctione nego. Re ipsa quod sint singularizata sequitur unitatem formalem esse coniunctam cum numerica et non dari separatam, sicut et a parte rei natura sit singularizata non repugnat esse universalis modo supra explicato, repugnat vero esse separatam contra \textsc{Platonem}\index[persons]{Plato}. 
\pend

\pstart
 Obiectio 3. Esse formale non est esse quod convenit rebus a parte rei, sed prout sunt obiective in intellectu. Igitur similiter unitas. Igitur inquirere an duo individua a parte rei sint unum formaliter est. Inquirere an Petrus existens Romae dormiat vel vigilet Matriti, sic \textsc{Torrejon}\index[persons]{Petrus Fernandez de Torrejon}. Respondeo negando antecedens. Esse enim formale cuiuscunque rei competit illi in quocunque statu sit, siquidem est inseparabile ab ipsa cum nihil sit aliud quam esse intrinsecum et quiditativum. Discrimine tamen quando est obiective in intellectu non habet esse simpliciter ut a parte rei, sed solum intentionaliter. Item in intellectu est abstractum, a parte rei contractum. Articulus 4. An haec unitas, etc. 
\pend

        \addcontentsline{toc}{section}{Articulus 4. An haec unitas sit communis}
        \pstart
        \eledsection*{Articulus 4. An haec unitas sit communis}
        \pend
      
\pstart
 Suppono 1. \edtext{ Ex \name{\textsc{Caietano}\index[persons]{Thomas de Vio Caietanu}}  loco citato }{\lemma{}\Afootnote[nosep]{}} duplicem esse unitatem, scilicet commune positive et commune nega \textnormal{|}\ledsidenote{BNC 97va}   negtive. Communis negative dicitur quando non est appropiata alicui supposito, clarius negative communis est non singularis. Positive communis est quando natura indivisa manens secundum illam unitatem reperitur in multis. 
\pend

\pstart
 Suppono 2. Unitatem postove communem subdividi in numericam, specificam et genericam. Natura dicitur habere unitatem communem numericam positive quando manens eadem numero et indivisa reperitur in pluribus suppositis, ut natura divina et nulla alia. Dicitur denique habere unitatem communem positive genericam vel specificam quando indivisa manens secundum illum gradum genericum vel specificum reperitur in pluribus realiter distinctis. Remanet igitur difficultas de generica et specifica. His positis. 
\pend

\pstart
 Prima sententia asserit naturam a parte rei habere unitatem positive communem. Imponitur haec opinio \textsc{Scoto}\index[persons]{Ioannes Duns Scotus}, ut probat Illustrissimus \textsc{Merinero}\index[persons]{} hic. Hanc sequuntur \textsc{Antonius Andreas}\index[persons]{Antonius Andreas}, \textsc{Monlorius}\index[persons]{Ioannes Baptista Monlorius}  et alii. Secunda sententia negat naturae talem unitatem. Ita \edtext{ \name{\textsc{Angelicus Doctor}\index[persons]{Thomas Aquinas}} \worktitle{Opusculo} 55 et I parte, quaestione 85, articulo 1, ad 2 }{\lemma{}\Afootnote[nosep]{ \textsc{Thomas Aquinas}\index[persons]{Thomas Aquinas}, \worktitle{Summa Theologiae}, I.85.1.2 (Roma: Commissio Leonina, 1889, p. 331) }}. Quae sequuntur \textsc{Caietanus}\index[persons]{Thomas de Vio Caietanu}, \textsc{Capreolus}\index[persons]{Ioannes Capreolus}, \textsc{Soncinas}\index[persons]{Paulus Barbus Soncinatus}, \textsc{Soto}\index[persons]{Dominicus de Soto}, \textsc{Iabellus}\index[persons]{Chrysostomus Iavelli Canapicii}, \textsc{Toletus}\index[persons]{Franciscus Toletus} , \textsc{Rubius}\index[persons]{Antonius Ruvius Rodensis} et communiter auctores. Cum enim unitas formalis dicat aliquid positivum, scilicet ipsam entitatem rei et aliquid negativum, sit, carentiam divisionis, ut dictum est, oportet secundum hanc considerationem resolvere difficultatem. 
\pend

\pstart
 Dico 1. Non datur unitas formalis positive communis individuis a parte rei sumpta unitas quantum ad realem entitatem, sed tot dantur unitates formales quot individua. Ita \edtext{ \name{\textsc{Fundatissimus Doctor}\index[persons]{Aegidius Romanus}} in I, distinctione 36, parte 2, quaestione 2 et distinctione 19 }{\lemma{}\Afootnote[nosep]{ \textsc{Aegidius Romanus}\index[persons]{Aegidius Romanus}, \worktitle{In primum librum Sententiarum} d. 36, q. 2, a. 2; d. 19 (Córdoba: apud Lazarum de Risquez et Antorium Rosellon, 1699, f. 672-675; 366-387) }} et aliis locis citatis ab \textsc{Gavardo}\index[persons]{Fridericus Nicolaus Gavardi}. Ita Auctores secundae sententiae locis citatis a \textsc{Complutensibus}\index[persons]{} hic. 
\pend

\pstart
 \textnormal{|}\ledsidenote{BNC 97vb} Probatur prima pars conclusionis. Si a parte rei daretur unitas formalis positive communis, talis non competeret naturae secundum statum contractionis: igitur conveniret naturae secundum se. Modo sic: natura secundum se habet hanc unitatem formalem positive communem ut propriam passionem: igitur repugnat contrahi. Patet consequentia: repugnat naturam simul esse contractam et communem positive; sed si contraheretur esset contracta et simul communis positive: igitur repugnat. Probatur quod sit communis. Quod competit naturae secundum se, in omni statu competit. Siquidem competit ex principiis intrinsecis et ut propria passio; sed parte naturae secundum se competit unitas communis positive: igitur illi competit in omni statu. 
\pend

\pstart
 Confirmatur primo. Natura humana a parte rei est multiplicata in individuis: igitur non est communis illis. Antecedens probatur. Nam non distinguitur realiter a singularitate: igitur sicut tot sunt singularitates realiter distinctae quot individua tot etiam erunt naturae: igitur a parte rei natura est multiplicata. Confirmatur secundo. Omnis proprietas realis praedicata de natura, praedicatur etiam de individuis; sed talis unitas communis positive esset propria passio naturae. Igitur praedicata de natura, praedicaretur etiam de individuis. Igitur potest praedicari Petrus est communis positive pluribus; sed hoc est absurdum: igitur illud. 
\pend

\pstart
 Probatur secunda pars. A parte rei datur unitas formalis in natura; sed talis unitas non est communis positive. Igitur formalis constituendae quot individua: Confirmatur. unitas formalis quoad entitatem realem est ipsa natura, sed natura a parte rei multiplicatur. Igitur unitas formalis multiplicatur. Confirmatur amplius. Unitas formalis est passio naturae, sed passiones multiplicantur multiplicata natura. Igitur talis unitas formalis multiplicatur a parte rei, multiplicata natura. 
\pend

\pstart
 \textnormal{|}\ledsidenote{BNC 98ra}  Dico 2. Natura prout contracta non est communis negative. At sumpta secundum se fundamentaliter, scilicet seu virtualiter est communis negative. Sed nota hanc propositionem `natura secundum se est communis negative' non differre in re ab hac `natura secundum se non est singularis'. Exemplificatur. Homo secundum se nec est albus, nec niger, sed utrumque accidit. Similiter natura secundum se nec est singularis, nec universalis. Igitur idem est dicere `natura secundum se non est singularis' ac dicere `natura secundum se est communis negative'. Sed a parte rei verum est dicere `natura secundum se non est singularis'. Igitur verum est dicere `natura secundum se est communis negativa.' 
\pend

\pstart
 Dico 3. A parte rei non datur in natura negatio seu carentia divisionis, quae sit una et eadem in omnibus individuis, sed tot sunt negationes, quot individua. Probatur. Negationes et privationes se habent ad modum accidentium, sed repugnat unum accidens realiter indivisum in pluribus subiectis. Igitur repugnat negatio divisionis una in omnibus individuis. 
\pend

\pstart
 Consectarium primum. Sicut a parte rei datur unitas formalis et numerica in unoquoque individuo, sic dantur plures unitates formales pro numero graduum essentialium et non sint realiter distinctae. Unde de illis sicut de gradibus debemus philosophare. Probatur. Eo modo quo Petrus a parte rei constituitur in esse hominis per rationale, constituitur in esse alis per sensibile, et in esse viventis per animatum et sic deinceps. Igitur sicut caret divisione formali specifica per rationale, ita caret divisione formali generica pro sensibile et sic de reliquis. Igitur non solus habet unitatem formalem specificam, sed etiam genericam. Igitur sicut dantur in Petro unitates numerica et specifica, sic dantur plures unitates formales. 
\pend

\pstart
 \textnormal{|}\ledsidenote{BNC 98rb} Consectarium secundum. Datur unitas conformitatis Thomistarum priscorum: est conformis \edtext{ Divo \name{\textsc{Thoma}\index[persons]{Thomas Aquinas}} \worktitle{Opusculo} 48, tractatus 1, capitulo 4 }{\lemma{}\Afootnote[nosep]{ \worktitle{Summa totius Logicae Aristotelis} 1.4 (Parma: Fiaccadori, 1864, pp. 56-57) }}. Appellant Thomistae unitatem conformitatis seu similitudinis illam, quam habent individua eidem speciei, ad quam non requiritur alia unitas, sed quod natura unius individui sit conformis naturae alterius, veluti duo homines in facie conformes et similes dicuntur habere unam faciem, non quia sit in eis una facies vere et realiter; sed quia habent similes facies. Haec unitas vocatur hodie fundamentalis. Siquidem super illam convenientiam conformitatem aut similitudinem fundatur intellectus ut tribuat naturae abstractae universalitatem. 
\pend

\pstart
 Obiectio 1. A parte rei Petrus et Paulus sunt unum in natura humana. Igitur a parte rei habet natura unitatem illis communem. Probatur consequentia. Quando duo realiter distincta sunt unum in aliqua ratione, talis ratio debet esse una in illis; sed illud reperitur a parte rei: igitur et hoc. Respondeo distinguendo antecedens. Petrus et Paulus a parte rei sunt unum fundamentaliter, concedo. Formaliter no. Unitas fundamentalis est convenientia et similitudo, cumque a parte rei Petrus et Paulus sint similes in natura humana, ideo sunt unum fundamentaliter. 
\pend

\pstart
 Instabis: quot modis dicitur unum oppositorum, tot dicitur et aliud; sed a parte rei datur distinctio formalis inter hominem et equum verbi gratia igitur datur unitas formalis. Probatur distinguendo consequens. Datur unitas formalis una in quolibet, concedo. Una in omnibus subdistinguo fundamentaliter, concedo: formaliter nego. Sufficit enim unitas formalis in quolibet individuo; ut verificetur illud dictum. Et si amplius requiritur sufficiat unitas formalis communis fundamentaliter. 
\pend

\pstart
 \textnormal{|}\ledsidenote{BNC 98va}  Obiectio 2. A parte rei inter Petrus et Paulus datur convenientia et similitudo in natura humana, sed talis similitudo est relatio: igitur inter illos datur relatio. Sed ex  \edtext{ \name{\textsc{Aristotele}\index[persons]{Aristoteles}} capitulo \worktitle{de Relatione} }{\lemma{}\Afootnote[nosep]{ \textsc{Aristoteles}\index[persons]{Aristoteles}, \worktitle{Categoriae} 7 (8a 27-35) }}: relatio fundatur in unitate, igitur a parte rei datur unitas formalis communis pluribus. Respondeo distinguendo maiorem: a parte rei datur convenientia et similitudo qua fundamentum relationis, concedo; qua relatio similitudinis, nego. Consequentia: igitur convenientia fundatur in convenientia. Distinguo consequens: convenientia qua relatio fundatur in convenientia qua fundamento, concedo. In convenientia qua relatione, nego. Et distinguo minorem argumenti. Relatio fundatur in unitate communi formaliter, concedo; in unitate contracta, nego. 
\pend

\pstart
 Obiectio 3. Natura quae est in Petro et Paulo definitur unica definitione, sed quod definitur est unum ex \edtext{ \name{\textsc{Aristotele}\index[persons]{Aristoteles}} VII \worktitle{Metaphysicae}, texto 13; et IV, texto 10 }{\lemma{}\Afootnote[nosep]{ \textsc{Aristoteles}\index[persons]{Aristoteles}, \worktitle{Metaphysica} 4.13 (1018b 8-16) }} et \edtext{ IV, texto 10 }{\lemma{}\Afootnote[nosep]{ \textsc{Aristoteles}\index[persons]{Aristoteles}, \worktitle{Metaphysica} 5.4 (1014b 25-37) }}, igitur habet unitatem illis communem. Respondeo distinguendo consequens. Habet unitatem illis communem in statu contractionis nego: in statu abstractionis, concedo. 
\pend

\pstart
 Instabis: non ab intellectu habet natura quod sit definibilis. Igitur a parte rei est definibilis: igitur una. Respondeo distinguendo antecedens. Non ab intellectu habet quod sit definibilis proxime nego: remote, concedo. Et solutio applicatur consequenti. Ut enim natura sit remote definibilis sufficit unitas communis fundamentalis. 
\pend

\pstart
 Obiectio 4. Quantumcumque natura dividatur per differentias numericas manet formaliter indivisa: igitur una formalis. Sed sic communicatur pluribus: igitur ut una communicatur pluribus. Respondeo distinguendo antecedens. Manet formaliter indivisa unica indivisione nego: pluribus, concedo. Solutio constat ex tertia conclusione. Et nota, quodlibet unitas \textnormal{|}\ledsidenote{BNC 98vb} formalis a parte rei multiplicetur in individuis, ut dictum est, tamen unitas formalis unius non distinguitur formaliter ab unitate formali alterius, sed tantum materialiter et numerica. Sic et natura, quae in se quid formale est sit realiter divisa per individua, non est terminus formaliter divisa et si manet formaliter indivisa non vero indivisa realiter. Unde natura habet a parte rei plures unitates formalis non formaliter, sed materialiter divisas. 
\pend

\pstart
 Obiectio 5. Natura in statu solitudinis habet quidquid requiritur ut sit libera a singularitate: naturam non est contracta, non est determinata, non est incommunicabilis secundum se. Igitur secundum se est communicabitis, non singularis. Igitur unitas quam habet est communis multis. 
\pend

\pstart
 Respondeo negando consequentiam. Posita reduplicatione secundum se non valet a negativa ad affirmativam variatio praedicato penes finitum et infinitum; nec valet a negatione unus absolute ad positionem alterius. Sic enim materiale arguitur: natura secundum se no est singularis, non est determinata, non est incommunicabilis. Igitur secundum se est communicabilis, universalis, etc. Quia secundum se non est unum nec alterum, sed utrique capax. Unde si ponamus has propositiones: homo secundum se non est communicabilis, homo secundum se est communicabilis, haec est falsa et prima vera, et ab destructione unius valet ad positionem alterius, dummodo servetur reduplicatio seu appellatio illa secundum se. Nusquam tamen debet affirmari determinate de illa natura secundum se quod est communicabilis vel quod secundum se est incommunicabilis, sed semper negativa est vera, quia secundum se et essentialiter nec est unus, nec est alterum. 
\pend

\pstart
 Instabis: \edtext{ Divus \name{\textsc{Thomas}\index[persons]{Thomas Aquinas}} IV \worktitle{De ente et essentia} \textnormal{|}\ledsidenote{BNC 99ra}  dicit quod natura secundum se est id quod praedicatur de individuis, sed quod praedicatur de individuis est universale. }{\lemma{}\Afootnote[nosep]{ \textsc{Thomas Aquinas}\index[persons]{Thomas Aquinas}, \worktitle{De ente et essentia}, 3 (Roma: Commissio Leonina, 1976, pp.374-375) }} Igitur natura secundum se est universalis. Respondeo explicando \textsc{Angelicum Praeceptorem}\index[persons]{Thomas Aquinas} cum \textsc{Caietano}\index[persons]{Thomas de Vio Caietanu} ibi \worktitle{contra tertiam dubiam}. In natura aliud est res, quae praedicatur aliud conditio seu status ut praedicetur. Res, quae praedicatur est natura secundum se, nam eius quidditas communicatur inferioribus. Conditio autem seu status ut communicetur est superioritas et universalitas, et haec non convenit naturae secundum se. 
\pend

\pstart
 Obiectio 6. Natura secundum se est vera entitas: igitur habet veram et formalem entitatem. Respondeo distinguendo antecedens. Est vera entitas in aliquo statu determinata, concedo. Secundum sua praedicata essentialia, concedo. Unitas invenitur eo modo quo invenitur natura. Si enim non consideratur natura determinata in aliquo statu, sed solum secundum sua praedicata essentialita, sic non consideratur ut habens unitatem determinate et positive, sed solum negative, id est, non est diversa penes illa principia formalia. 
\pend

\pstart
 Instabis: unitas positiva est ipsa entitas cum negatione divisionis, sed talis natura habet entitatem cum negatione divisionis: igitur habet unitatem positivam. Respondeo distinguendo maiorem. Est ipsa entitas determinata, concedo. Est ipsa entitas secundum se nego. Cum enim natura secundum se relinquat indifferentiam ut sit multiplicata vel una absolute, ideo non manet perfecte una, sed negative, ut dictum est. 
\pend

\pstart
 Obiectio 7. Quando aliquid per se alicui competit eius oppositum nec per se, nec per accidens ei competit; sed communitas negativa et singularitas sunt oppositae. Igitur si communitas negativa competit naturae secundum se; singularitas non competit naturae nec per accidens, scilicet in statu contractionis. Respondeo negando \textnormal{|}\ledsidenote{BNC 99rb} minorem. Cum enim communitas negativa sit non includere ex intrinsecis singularitatem, eius oppositum est singularitas intrinseca, quae quidem nec per se, nec per accidens competit naturae secundum se. 
\pend

\pstart
 Obiectio 8. Quidquid per se in primo vel secundo modo perseitatis praedicatur de superiori, praedicatur etiam de inferioribus sub illo contentis. Sed communitas negativa praedicatur de natura secundum se, saltem secundo modo perseitatis. Igitur praedicatur etiam de inferioribus; sed hoc implicat: igitur et illud. Probatur minor. Implicat singulare excludere singularitatem, sed talia inferiora sunt singularia et per communitatem negativam excluderent singularitatem: igitur implicat. 
\pend

\pstart
 Respondeo distinguendo maiorem. Quidquid per se primo vel secundo modo perseitatis praedicatur de superiori afficiendo ut supponat personaliter, concedo. Afficiendo ut supponat simpliciter nego. Et solutio applicetur minori. Quando enim terminus supponit suppositione simplici solum substituitur per immediato significato et non pro remotis, nec potest sub illo termino ascendere vel descendere ut dictum fuit in  \worktitle{Logica parva}  . Communitas negativa afficit naturam, ita ut supponat simpliciter et non personaliter. 
\pend

\pstart
 Instabis: id quod praedicatur per se de alio facit ut illud supponat personaliter, sed communitas negativa praedicatur per se de natura: igitur facit, illam supponere personaliter. Respondeo distinguendo maiorem. Quod praedicatur per se absolute de alio, concedo. Quod praedicatur reduplicative, nego. Naturae enim per se competit communitas negativa quantum est ex se, et sic competit eum addito. 
\pend

\pstart
 \textnormal{|}\ledsidenote{BNC 99va}  Obiectio 9. Unitas formalis est qua aliquid dicitur unum formaliter seu unum in specie; sed unum individuum sine consortio alterius non dicitur unum in specie; nec Petrus est unus in specie, nisi addatur Petrus cum Paulo vel cum Francisco; igitur unitas formalis non multiplicatur in quolibet individuo. Sed facit plura individua esse unum inter se. Confirmatur ex Divo \name{\textsc{Thoma}\index[persons]{Thomas Aquinas}} III parte, quaestione 2, articulo 5, ad 2: quod natura humana \edtext{\enquote{non habet rationem speciei secundum quod est in uno individuo, sed secundum quod est abstracta ab omni individuo vel secundum quod est in omnibus. }}{\lemma{}\Afootnote[nosep]{ \textsc{Thomas Aquinas}\index[persons]{Thomas Aquinas}, \worktitle{Summa Theologiae}, III, q. 2, a. 5, ad 2 (Roma: Commissio Leonina, 1903, p. 34) }}. Igitur sentit formalem unitatem non inveniri in unoquoque. 
\pend

\pstart
 Respondeo distinguendo minorem. Unum individuum sine consortio alterius non dicitur unum, unitate in quolibet, nego; unitate in omnibus, concedo. Vel aliter, non dicitur unum quantum ad convenientiam et similitudinem, concedo; quantum ad indivisionem specificam, nego. Natura quae est in uno individuo etiam prout est in illo habet negatione divisionis specifice per principia formalia. Sine consortio alterius non habet convenientiam et similitudinem. Unde requiretur consortium alterius ut unitas sit una in communibus non tamen ut sit una in quolibet. 
\pend

\pstart
 Bene dicit Divum \textsc{Thomam}\index[persons]{Thomas Aquinas} quod natura humana habet rationem \secluded{rationem} speciei, secundum quod est in omnibus individuis. Eo modo quo universale dicitur esse in multis vel fundamentaliter quia abstrahitur una natura ab multis, in quibus invenitur una per convenientiam et similitudinem vel quia est in multis per aptitudinem ad essendum in multis. Primo modo invenitur a parte rei; secundo solum per intellectum. 
\pend

\pstart
 \textnormal{|}\ledsidenote{BNC 99vb} Obiectio 10. Similitudo est accidens, sed individua naturae humanae sunt similia; igitur conveniunt in aliquo accidente et non in substantia. Respondeo distinguendo maiorem. Similitudo in qualitate, concedo; in substantia, nego. Similitudo et dissimilitudo est proprietas qualitatis, abusive tamen dicitur in aliis praedicamentis. Quapropter in substantia proprie dicitur identitas et diversitas; in quantitate proprie aequalitas vel inaequalitas. Vide quod dicimus in Praedicamento Qualitatis. Utramque docet \edtext{ \name{\textsc{Magnum Patrem}\index[persons]{Augustinus Hipponensis}}  libro 16, \worktitle{Contra Faustum}, capitulo 15 }{\lemma{}\Afootnote[nosep]{ \textsc{Augustinus Hipponensis}\index[persons]{Augustinus Hipponensis}, \worktitle{Contra Faustum} 16.15 (ed. Zycha, CCEL 25, 1891, p. 455-457) }}. 
\pend

\pstart
 Instabis: si inter homines ratione naturae daretur similitudo substantialis, igitur homines perfecte unum sunt in essentia, sicut Deus Pater cum Filio. Respondeo negando consequentiam. Nam ut diximus natura divina una indivisa manens est in Tribus suppositis. Alia quaecunque natura est divisa et partita per differentias individuales. Hinc \edtext{ Magnus Pater \name{\textsc{Augustinus}\index[persons]{Augustinus Hipponensis}} \worktitle{Sermo 38} de verbis Domini inquit \enquote{Homo et Homo duo homines sunt, ibi Pater et Filius unus Deus} }{\lemma{}\Afootnote[nosep]{ Augustinus Hipponensis, \worktitle{Sermo 117}, c. 4, n. 14 (PL 38, 670) }}. Quia in illis non dantur duae naturae divinae similes, sed una tantum in utroque reperitur. Qui plura de hoc propositio desiderat legat \edtext{ \name{\textsc{Magnum Patrem}\index[persons]{Augustinus Hipponensis}} in \worktitle{Enarrationes in Psalmos} 68, prima parte }{\lemma{}\Afootnote[nosep]{}}. Saniter in \edtext{ \worktitle{Sermo 31} de verbis Apostoli capitulo 5 et 6 }{\lemma{}\Afootnote[nosep]{}}. Item in \edtext{ \worktitle{Enchiridion Psalmo}  49 }{\lemma{}\Afootnote[nosep]{}}. Similiter libro \edtext{ \worktitle{Contra Arrianorum}  capitulo 26 }{\lemma{}\Afootnote[nosep]{}}, et \edtext{  libro 3 \worktitle{Contra Maximinum Arrianum} capitulo 1 et 15 }{\lemma{}\Afootnote[nosep]{}}, et \edtext{ liber 4 \worktitle{Trinitate} capitulo 9 }{\lemma{}\Afootnote[nosep]{}}, \edtext{ et liber 2 \worktitle{De anima et eius origine} capitulo 4 et 5 }{\lemma{}\Afootnote[nosep]{}}, et libro \edtext{ \worktitle{Contra Arriano} capitulo 34 usque ad 36 }{\lemma{}\Afootnote[nosep]{}}. Denique \edtext{ tractatus 37 \worktitle{In Iohannem} }{\lemma{}\Afootnote[nosep]{}}. Cetera quae ad propositum expectant sequentibus argumentis manifestabuntur. 
\pend

        \addcontentsline{toc}{section}{Quaestio 4. De universalitate seu de forma, a qua denominatur universale}
        \pstart
        \eledsection*{Quaestio 4. De universalitate seu de forma, a qua denominatur universale}
        \pend
      
        \addcontentsline{toc}{section}{Articulus 1. An universale formale sit ante intellectus operationem?}
        \pstart
        \eledsection*{Articulus 1. An universale formale sit ante intellectus operationem?}
        \pend
      
\pstart
\noindent%
 Articulus 1. An universale formale sit ante intellectus operationem? 
\pend

\pstart
  Primo ex dictis colliguntur duae conditiones ut natura sit formaliter universalis. Prima ut sit una formaliter unitate communi positive pluribus, quae conditio continetur in definitione: universale est unum. Unitas enim formalis cuiuslibet individui requiretur ut fundamentum universalitatis, non est enim sufficiens ut constituat universali formaliter, quia cum sit idem realiter cum numerica, est diversa quolibet singulari. 
\pend

\pstart
  Secunda conditio ut sit natura apta esse in multis et haec continetur in residuo definitionis in multis. Haec autem aptitudo non potest esse realis ut in sequenti articulo videbimus. His positis. Prima sententia asserit a parte rei, dari universale pro formali. Ita \textsc{Ioannes Duns Scotus}\index[persons]{Ioannes Duns Scotus}, quamvis Illustrissimus \textsc{Merinero}\index[persons]{} probet contrarium. Sententia negat a parte rei universale formale est communis inter Doctores. 
\pend

\pstart
  Sit conclusio: a parte rei non datur universale formale. Ita \edtext{ \name{\textsc{Angelicus Doctor}\index[persons]{Thomas Aquinas}} \worktitle{Opusculis} 55 et 56; et capitulo 4 \worktitle{De ente et essentia} }{\lemma{}\Afootnote[nosep]{}}; ubi etiam \textsc{Caietanus}\index[persons]{Thomas de Vio Caietanu} \edtext{ \name{\textsc{Suarez}\index[persons]{Franciscus Suarez}} \worktitle{Disputatione 6 Metaphysicae} sectione 5 }{\lemma{}\Afootnote[nosep]{}}; \edtext{ \name{\textsc{Rubius}\index[persons]{Antonius Ruvius Rodensis}} quaestione 4 }{\lemma{}\Afootnote[nosep]{}} et alii, ab his  citati. Sic probatur ratione. Natura in statu contractionis non potest esse universalis, sed neque etiam secundum se: igitur solus in statu abstractionis. Consequentia patet. Non est alius status naturae considerandus ut constat numero 126 et 127. Antecedens vero quoad primam partem est evidens. Probatur quoad posteriorem: Ut natura sit universalis debet esse una communis \textnormal{|}\ledsidenote{BNC 100rb} positive, sed naturae secundum se non competit unitas positive communis: igitur secundum se non est universalis. Minor manet probata. Confirmatur: ut natura sit universalis formaliter debet esse una simpliciter et non multiplicata; sed a parte rei non datur natura, quae non sit multiplicata: igitur a parte rei non datur universale formale. 
\pend

\pstart
  Roboratur: haec praedicatio `homo est universalis' est accidentalis; sed praedicatum illius non est substantia, igitur accidens; non reale: igitur rationis. Probatur maior: si non esset accidentalis, esset essentialis; non potes esse essentialis: igitur accidentalis. Probatur minor: omne praedicatum essentiale competit inferioribus illius naturae: sed universalitas non competit inferioribus naturae. Siquidem non dicimus `Petrus est universalis', `Paulus est universalis, igitur non est praedicatum essentiale'. Minor primi syllogismi probatur: quod reperitur in omnibus praedicamentis non est substantia, sed universalitas reperitur in omnibus praedicamentis: igitur non est substantia. Minor constat: nam quantitas est universalis, qualitas est universalis, et sic de aliis: igitur reperitur in omnibus praedicamentis. Quod non sit accidens \del{non} reale etiam constat: quod non pertinet ad aliquid praedicamentum determinate non est ens reale; sed universalitas non pertinet ad aliquod praedicamentum determinate, igitur non est ens reale: ergo est ens rationis, hoc ipso repugnat a parte rei: igitur a parte rei non datur. 
\pend

\pstart
  Obiectio 1. \edtext{ \name{\textsc{Aristoteles}\index[persons]{Aristoteles}} 1 \worktitle{Perihermeneias} capitulo 4 ait: \enquote{rerum aliae sunt universales, et aliae particulares}; }{\lemma{}\Afootnote[nosep]{}} sed a parte rei sunt particulares: igitur etiam universales. Confirmatur: universale et particulare sunt correlativa, sed a parte rei datur particulare: igitur et universale. 
\pend

\pstart
  Respondeo \textsc{Aristotelem}\index[persons]{Aristoteles} locutum fuisse de rebus non prout sunt in se, sed prout significantur per nomina et sic sunt in intellectu. Quod si urgeant Phy \textnormal{|}\ledsidenote{BNC 100va}    \textsc{Philosophum}\index[persons]{Aristoteles} loqui de rebus secundum se, distinguendum est consequens: igitur a parte rei res sunt universales formaliter nego consequentiam, fundamentaliter concedo consequentiam. Ad confirmationem distinguo maiorem. Universale et particulare secundo intentionale sunt correlativa, concedo; universale et particulare primo intentionale, nego. De hoc postea videbitis. 
\pend

\pstart
  Obiectio 2. Obiectum praecedit potentiam etiam secundum id quo est obiectum; sed obiectum intellectus est universale: igitur praecedit intellum: igitur ante intellectus operationem datur universale. Respondeo distinguendo minorem. obiectum intellectus est universale fundamentaliter, concedo; formaliter, nego. Cum enim non constituatur obiectum et universale ab eadem forma, bene potest esse formaliter obiectum et universale solum fundamentaliter, sed de hoc postea redibit sermo. 
\pend

\pstart
  Obiectio 3. Universale qua tale non potest constitui per quid singulare, sed quaecunque intentio ab intellectu producta est singularis: igitur per illam non potest constitui in esse talis. Maior est certa quia universale et particulare opponuntur. Minor probatur. Quidquid fit a causa singulari, est singulare; sed quaecunque intentio fit a causa singulari, scilicet ab intellectu: igitur est singularis. 
\pend

\pstart
  Respondeo distinguendo minorem. Quaecunque intentio est singularis ut quod, concedo; ut quo, nego. Sicut enim species impressa vel expressa rei et in se sit quid singulare est tamen in repraesentando formam universalis. Similiter et intentio in se sit singularis ut quod potest tamen ut quo esse forma constitutiva universalis. 
\pend

\pstart
  Instabis: ergo sicut universale in repraesentando non est simpliciter universale, sic universale in essendo non est proprie universale. Respondeo negando paritatem. Est disparitas species impressa et quaecunque alia res denominanda universale in repraesentando realiter est singularis quare ut sic denominetur exigit \textnormal{|}\ledsidenote{BNC 100vb} singularitatem, unde est singularis simpliciter et universalis secundum quid. Res vero quae denominatur universalis in essendo non solum non est singularis, sed etiam exigit abstractionem a singularibus. Quare materia propria huius universalis non est singularis et sic potest denominari simpliciter universalis, quin forma sit universalis ut quod, sed sufficit, quod sit ut quo universalis. Sicut albedo et non sit alba ut quod, constituit tamen album ut quod , quia est alba ut quo. Reliqua argumenta aut manent soluta ex dicitis aut solvenda ex dicendis. Articulus 2. Utrum a parte rei detur aptitudo etc. 
\pend

        \addcontentsline{toc}{section}{Articulus 2. Utrum aptitudo et indifferentia ad multa detur ante intellectus operationem?}
        \pstart
        \eledsection*{Articulus 2. Utrum aptitudo et indifferentia ad multa detur ante intellectus operationem?}
        \pend
      
\pstart
\noindent%
 Articulus 2. Utrum aptitudo et indifferentia ad multa detur ante intellectus operationem? 
\pend

\pstart
  Suppono 1: Articulus non procedere de potentia activa vel receptiva cum non agat de universali in causando; unde nomine potentiae intelligitur capacitas seu non repugnantia ut multiplicetur in multis, et de illis praedicetur. Haec aptitudo duplex est. Prima formalis, quae nihil est aliud quam respectus ad plura. Secunda est fundamentalis, quae ipsa non repugnantia ut sit in multis. Prima est ipsa forma universalis, secunda est fundamentum huius formae. 
\pend

\pstart
  Suppono 2. Nam posse dupliciter sumi. Primo ut est possibilis et antecedit actualem existentiam. Secundo ut existens realiter extra causas; et hoc dupliciter vel praecisive, id est, natura existens, non tamen considerata existentia vel reduplicative ut existens. Igitur est certus naturam existentem reduplicative non habere talem aptitudinem; nam sic est singularis. Unde solum restat difficultas de \textnormal{|}\ledsidenote{BNC 101ra}   ipsa ut possibili vel praecisa ab existentia. 
\pend

\pstart
  Prima sententia asserit a parte rei dari in natura talem indifferentiam. Ita \textsc{Scotus}\index[persons]{Ioannes Duns Scotus} et eius acutissima schola. Secunda negat, ita \edtext{ \name{\textsc{Angelicus Doctor}\index[persons]{Thomas Aquinas}} \worktitle{Opusculo 56} }{\lemma{}\Afootnote[nosep]{}},  \edtext{ \name{\textsc{Caietanus}\index[persons]{Thomas de Vio Caietanu}} \worktitle{De ente et essentia} capitulo 4, quaestione 7 }{\lemma{}\Afootnote[nosep]{}} et communiter omnes Thomistae. 
\pend

\pstart
  Sit nostra conclusio: ante intellectus operationem non datur in natura quomodolibet \emph{considerata [?]} aptitudo ut sit in multis. Probatur ratione fundamentali. Aptitudo ad essendum in multis non est sola communitas negativa, sed est capacitas seu non repugnantia ut natura una existens multiplicetur in multis: igitur in statu in quo natura non est una nec plures vel in quo est contracta ad individuationem non habet capacitatem unitatis ad multiplicitatem. Sed in statu singularitatis habet repugnantiam ad istam multiplicitatem. In statu vero solitudinis non habet unitatem multiplicabilem in plura, sed solum praedicata quidditativa, inter quae non est numeranda talis capacitas: igitur in neutro statu habet capacitatem seu aptitudinem. 
\pend

\pstart
  Confirmatur: talis aptitudo est indifferentia ad essendum in pluribus; sed talis indifferentia non datur a parte rei in natura sive \emph{considerata [?]} ut possibili, sive ut existente praecisa existentia: igitur in ea prout sic non datur aptitudo universalis. Probatur minor. In natura ut possibili ad summum potest intelligi indifferentia ad unam singularitatem seu existentiam vage sumpta, id est, hanc vel illam, quando vero contrahitur per unam singularitatem vel existentiam iam amittit illam indifferentiam ad essendum in hoc vel illo: igitur non datur in natura quomodolibet \emph{considerata [?]} indifferentia sufficiens ad universalitatem. 
\pend

\pstart
  \textnormal{|}\ledsidenote{BNC 101rb} Roboratur: natura secundum se et in singularibus non est capax intentionis et respectus ad plura: igitur non est capax istius fundamenti proximi. Sed tale fundamentum est ipsa aptitudo unitatis ad multa: igitur in natura a parte rei non datur talis aptitudo. 
\pend

\pstart
  Obiectio 1. Natura secundum se non est singularis: igitur secundum se est communis. Non est incapax ex principiis essentialibus ad essendum in multis: igitur est capax. Hoc argumentum manet solutum numero 187. Posita reduplicatione secundum se non licet inferre a termino finito ad infinitum. Natura secundum se non est singularis; igitur secundum se est non-singularis; igitur multo minus non licet inferre; igitur est universalis. 
\pend

\pstart
  Obiectio 2: Naturae secundum se competunt proprietates universalis: igitur ipsa universalitas. Antecedens probatur. Esse obiectum intellectus, esse perpetuae veritatis, esse definibile unica definitione sunt proprietates universalis, haec conpetunt naturae ante operationem intellectus: igitur illi competunt proprietates universalis. 
\pend

\pstart
  Respondeo distinguendo antecedens. Naturae secundum se competunt fundamentaliter proprietates universalis, concedo; formaliter, nego. Similiter ad probationem. Sed haec competunt naturae ante operationem intellectus fundamentaliter, concedo; formaliter, nego. Quomodo natura sit definibilis a parte rei dixi numero 183 et 184. Similiter dicetur ad reliquas proprietates. 
\pend

\pstart
  Obiectio 3. Non repugnat natura ex principiis malibus determinata et ex principis formalibus indeterminata: igitur apta ad essendum in multis. Probatur assumptum. Non repugnat materia prima quandiu est sub una forma habere aptitudinem ad omnes formas ut illis informetur, et natura angelica est apta ad essendum in multis, quamvis illi repugnat illa multiplicitas. Igitur natura non repugnat determinata et indeterminata ex diversis principiis. 
\pend

\pstart
  \textnormal{|}\ledsidenote{BNC 101va}   Respondeo negando antecedens et paritate probationis. Nam materia prima ex sua propria natura habet capacitatem ad plures formas et eius aptitudo et potentia ad illas est eius essentia. Quare ubicunque salvatur eius entitas, salvatur ista aptitudo, quia convenit illi secundum se. At vero aptitudo ad essendum in multis et praedicandum sequitur naturam ratione status et unitatis abstractae, non vero secundum se. Nec sequitur paritas de natura angelica ut postea videbimus. 
\pend

\pstart
  Obiectio 4. Potentia sumit suam specificationem ab actu, si actus est realis etiam potentia. Sed actus aptitudinis universalis est realis: igitur realis est ipsa aptitudo. Probatur minor. Existere in pluribus singularibus est quid reale, sed existere in pluribus est actus illius aptitudinis: igitur actus illius aptitudinis est realis. 
\pend

\pstart
  Respondeo negando minorem. Ad probationem etiam negatur minor. Existentia enim plurium individuorum non est actus huius aptitudinis, sed praedicatio seu contractio unius naturae ad plura individua, quae quid rationis est. Existentia enim plurium individuorum est fundamentum ut abstrahatur unitas universalis et reddatur apta ad istam praedicationem. Particulare enim primo intentionale non est correlativum universalis, sed fundamentum. 
\pend

\pstart
  Obiectio 5. Quia ista aptitudo est fundamentum universalitatis non est realis, sed secunda intentio fundatur in prima: igitur ista aptitudo non est \emph{prima [?]} intentio. 
\pend

\pstart
  Respondeo distinguendo causalem. Quia est fundamentum proximum, concedo; remotus, nego. Et distinguo consequens: non est secunda intentio formaliter, concedo; fundamentaliter proxime, nego. Secunda intentio dicit pro fundamento remoto primam intentionem, pro fundamento vero proximo aliquid rationis, non quidem quod sit aliqua secunda intentio, sed quod sit aliqua negatio seu denominatio extrinseca. 
\pend

\pstart
  \textnormal{|}\ledsidenote{BNC 101vb} Quia ista aptitudo fit ab intellectu: igitur prius est \secluded{est} cognitum obiectum, quam resultet illa aptitudo: igitur universalitas nec fundamentalis requiretur ad obiectum intellectus. Respondeo negando consequentiam: Universalitas enim ut iam dixi non requiritur ad obiectum intellectus cum habeat distinctam rationem. 
\pend

\pstart
  Quia id quod fit ab intellectu est singulare, sed illa intentio fit ab intellectu: igitur singularis est. Respondeo distinguendo maiorem. Quod fit ab intellectu efficientia effectus, concedo. Efficientia obiecti, subdistinguo, est singulare ut quod, concedo; ut quo, nego. Solutio patet in \worktitle{Logicae proemialibus} de obiecto formali, subdistinctio constat iam numero 234. 
\pend

\pstart
  Obiectio 6. Natura ut possibilis non petit hanc numero singularitatem seu existentiam: igitur indifferens est ad omnes singularitates. Respondeo distinguendo consequens. Est indifferens ut sit in illis realiter una positive manens, nego; partita et divisa, concedo. Articulus 3. Universalitas, etc. 
\pend

        \addcontentsline{toc}{section}{Articulus 3. An Universalitas consistat essentialiter in relatione?}
        \pstart
        \eledsection*{Articulus 3. An Universalitas consistat essentialiter in relatione?}
        \pend
      
\pstart
  Notata iam differentia universalis logici a metaphysico numero 125, supere examinare in quonam consistat universale logicum. Communis sensus Auctorum est universale consistere in respectu ad inferiora, sed est triplex respectus, scilicet aptitudinis ad essendum in inferioribus; aptitudinis ad praedicandum, quae est ipsa praedicabilitas; et relatio ipsius praedicationis actualis, qua non solum dicitur praedicabile, sed praedicatum. Iuxta hos respectus tres sunt opiniones. Prima docet universale consistere in prima aptitudine. Secunda in secunda et in tertia, tertia. Nunc sic his relictis ut clare procedamus. 
\pend

\pstart
  \textnormal{|}\ledsidenote{BNC 102ra}   Sit prima conclusio: in universali solum tria inveniuntur pertinentia ad eius integram rationem, scilicet universale materiale, fundamentale et formale. Explico conclusionem: universale materiale est ipsa natura seu subiectum, quod denominatur universale. Fundamentale est ipsa unitas praecisa ab inferioribus cum aptitudine ad essendum in illis. Formale vero est ipsa relatio, qua comparatur et respicit plura. Ratio est: universalitas est secunda intentio, quae est relatio rationis; sed in omni relatione haec tria concurrunt et sufficiunt, scilicet subiectum, fundamentum et forma relationis: igitur haec tria concurrunt et sufficiunt ad universalitatem. 
\pend

\pstart
  Sit secunda conclusio: Universalitas fundamentalis constituit naturam universalem metaphysicae et consistit in absoluta unitate praecisionis et aptitudine non relative sumpta, sed per non repugnantia. Probatur ratione. Universale metaphysicum non dicitur tale per additionem alicuius modi ad universalitatem ut sic; igitur consistit in illa unitate absoluta. Antecedens probatur. Si consisteret in aliquo modo addito, talis modus vel esse realis vel rationis. Si rationis contra: modus rationis non pertinet ad metaphysicum sed ad logicum. Sed talis modus esset rationis: igitur non pertinet ad metaphysicum: igitur non constituitur per talem modum. Si realis contra: modus realis pertinet ad naturam et non ad universalitatem, sed talis est realis: quae igitur etiam invenitur in singularibus. 
\pend

\pstart
  Nulla scientia considerat naturas in singularibus, sed metaphysica est scientia, igitur non considerat naturas in singularibus; sed sola separatio seu praecisio reddit fundamentum universalitatis, igitur; natura separata et abstracta est tale universale. Probatur minor. Natura tamquam unama unitate praecisionis et separationis est universalis metaphysicae, sed natura abstracta et separata est una unitate praecisionis: igitur sic est universalis. Item natura sic abstracta est apta ad essendum in multis: igitur \textnormal{|}\ledsidenote{BNC 102rb} nihil \emph{dicit [?]} ut sit universalis. Probatur antecedens. Natura dicens convenientiam cum illis a quibus fuit abstracta dicit aptitudinem et non repugnantiam ut in illis; sed natura abstracta dicit talem convenientiam: igitur dicit aptitudinem. 
\pend

\pstart
  Quod autem non dicat relationem probatur: ubi non est terminus ad quem non intelligitur relatio, sed posita natura abstracta nondum intelligitur terminus ad quem: igitur non intelligitur relatio. Probatur minor. Natura quaae dicit terminum a quo, non dicit terminum ad quem; sed natura abstracta solum dicit terminum a quo fuit praecisa: igitur non dicit terminum ad quem: igitur universalitas fundamentalis est universale metaphysicum. 
\pend

\pstart
  Sit tertia conclusio: Haec universalitas fundamentalis datur tam in abstractione positiva quam negattiva. Probatur ratione. De abstractione negativa non est dubium. Probat de positiva: abstractio positiva fit cum cognitione positiva tam termini ab quo, quae relinquitur, quam naturae, quae accipitur. Sed haec operatur eumdem effectus ac negativa, scilicet non respicere terminum ad quem sed solum terminum a quo: igitur non ponit relationem, sed tantum praecisionem. Hoc constituit universale metaphysicum. Igitur haec universalitas datur in utraque abstractione. Nota differentiam inter una et alteram abstractionem. Utraque considerat naturam nudam, negativa tamen non considerat conditiones individuales. At positiva illas considerat nihilominus utraque facit eumdem effectum. 
\pend

\pstart
  Sit quarta conclusio: universale logicum non consistit in comparatione attributionis sive praedicationis, sed simplicis relationis seu ordinis sine inclusione actuali in inferioribus. Haec conclusio constabit cum clarius eam tractemus. Sufficit tamen ostendere eius fundamentum. 
\pend

\pstart
  Universalitas logica sive secunda intentio est re \textnormal{|}\ledsidenote{BNC 102va}    relatio rationis; sed ante relationem actualis praedicationis invenitur ipsa relatio unius ad multa; igitur ante actualem praedicationem invenitur \emph{inveni [?]} universale logicum. Probatur minor: Ante actualem praedicationem invenitur natura sine singularibus ut praedicabilis de multis; igitur ante relationem actualis, etc. Antecedens patet. Potentia qua potest praedicari antecedit actum praedicationis; igitur ante actualem praedicationem, etc. 
\pend

\pstart
  Obiectio. Tunc universale iudicatur ad multa, cum concipitur convenire multis; sed convenire multis fit per actualem praedicationem; igitur universale fit per actualem praedicationem. Respondeo negando maiorem: Potius enim concipitur universale, cum iudicatur aptitudo ordinata ad multa ut potens praedicari de illis, quod fit ante acutalem praedicationem. Indiviso enim non exigitur, nam ista inclusio est esse in multis et cum illis identificari. Sed universale prius est aptum ad essendum et identificandum, igitur ante actualem identificationem datur universalitas. Demum inclusio actualis est contractio actualis, siquidem non includitur in singularibus nisi modo quo est in illis, sed in illis est contractum. Igitur includitur ut contractum et determinatum; igitur non ut indifferens; igitur ut inclusum non est universale. 
\pend

\pstart
  Obiectio iterum. Inferiora non sunt nisi per inclusionem superioris in illis; igitur superior non est nisi per inclusionem in inferioribus. Respondeo distinguendo antecedens: Non sunt identice, concedo; formaliter, nego. Sicut enim superius ante inclusionem est potentia includi, sic inferiora ante inclusionem sunt potentia includere. Et sicut aliquid actu est superius quando actu ei convenit aptitudo ut includatur et sit in multis, non quando actu includitur; ita actu sunt inferiora sub formalitate inferiorum quando eis actu convenit aptitudo ut includant et contrahant superius, non solum quando actu includunt. 
\pend

\pstart
  \textnormal{|}\ledsidenote{BNC 102vb} Sit quinta conclusio: Praedicabilitas est passio universalitatis. Est contra Illustrissimus \textsc{Merinero}\index[persons]{}. Ratio est: fundamentum et radix praedicabilitatis est identitas extremorum; igitur aptitudo ad essendum fundamentum ad aptitudinem ad praedicandum. Antecedens patet: Quando non est idem cum alio non vere praedicatur: igitur radix praedicationis est identitas extremorum: igitur relatio ad essendum est prior relatione ad praedicandum: igitur tantum est passio. 
\pend

\pstart
  Obiectio. \emph{Aptitudo ad essendum est non repugnantia non positiva relatio [?]}; igitur in genere positivae relationis prior est praedicabilitas quam positiva aptitudo ad essendum in multis. 
\pend

\pstart
  Respondeo distinguendo antecedens. Aptitudo ad essendum fundamentaliter, concedo; formaliter, nego. Prius enim concipitur natura, natura \supplied{est} contrahibilis et identificabilis fundamentaliter per non repugnantiam. At natura comparata est formaliter contrahibilis per relationem. Cumque tunc sit etiam praedicabilis, antecedit aptitudinem praedicandi. 
\pend

        \addcontentsline{toc}{section}{Articulus 4. Utrum universale sit substantia vel accidens corporum vel incorporeum?}
        \pstart
        \eledsection*{Articulus 4. Utrum universale sit substantia vel accidens corporum vel incorporeum?}
        \pend
      
\pstart
  Breviter resolvam titulum. Suppono universale nomen esse concretum dicens pro materiali naturam realem quae denominat universalis, pro formali vero relationem rationis, scilicet secundam intentionem, a qua natura denominaterur formaliter universalis. Verbi gratia species quae est secundum praedicabile pro materiali dicit naturam humanam vel leoninam vel quamcunque aliam. Pro formali vero dicit relationem specificam, quam specieitatem potes vocare, sicut in genere relatio dicit genereitas et in differentia diffe \textnormal{|}\ledsidenote{BNC 103ra}    differentieitas. Vide doctrinam oculis in hoc concreto albo; dicit pro materiali corpus in quo est albedo, pro formali vero albedinem a qua corpus denominatur album. Hoc posito: 
\pend

\pstart
  Dico primo: Universale materialiter sumptum aliquando est substantia, aliquando accidens; ut homo, animal et albedo, color. Aliquando vero etiam est accidens rationis ut genus, species. Ast universale formaliter sumptum semper est accidens rationis, quod est illa secunda intentio sufficienter et copiose iam explicata. 
\pend

\pstart
  Dico secundo: Universale formaliter sumptum proprie neque est corporeum, nec incorporeum quia corporeum et incorporeum sunt differentiae discriminates ens reale. Sed universale formaliter sumptum non est ens reale: igitur non competit proprie corporeitas vel incorporeitas. Dicit tamen incorporeum negative. Ast materialiter sumptum si singularia a quibus abstrahiteruntur sunt incorporea. Incorporeum evadit sic natura angelica. Si est universalis, cum eius singularia essent incorporea, fieret incorporeum universale. Si tamen singularia sunt corporea etiam universale est corporeum, quomodo natura humana abstracta ab singularibus corporeis dicit universale corporeum. 
\pend

\pstart
  Ut igitur universalem regulam cognoscendi, an universalia sint entia realia, corporea vel incorporea, substantialia vel accidentalia statuam. Praedicamentum videndum est, ad quod spectare inveniunt et sic iudicandum. Totum hoc tam verum est quod non indiget probatione, ac proinde caret obiiectionibus. Sic manet explanatum quod pertinet ad substantiam universalis. Nunc videndum est quo actu intellectus fit universale. Pro quo. 
\pend

        \addcontentsline{toc}{section}{Quaestio 5. De causa universalis}
        \pstart
        \eledsection*{Quaestio 5. De causa universalis}
        \pend
      
        \addcontentsline{toc}{section}{Articulus 1. Explicat modus procedenti intellectus}
        \pstart
        \eledsection*{Articulus 1. Explicat modus procedenti intellectus}
        \pend
      
\pstart
  Nihil est in intellectu quin prius fuerit in sensibus: ergo prior operatio sensus operatione intellectus. Sensatio enim fit in sensibus externis per hoc quod obiectum sensibile mittit speciem intentionalem se repraesentantem, qua informatus sensus producit speciem expressam obiecti per cuius receptione fit sensatio. 
\pend

\pstart
  Item ex sensatione sensuum externorum imprimunt aliae species intentionales sensibus internis et sic fiunt proportionabili modo sensationes interiores. Species vero expressae ab sensibus internis productae appellari solent phantasmata. 
\pend

\pstart
  Ex dictis sequitur primo phantasmata omnia seu species expressas repraesentare obiectum singulare, verbi gratia Petrus cum hoc numero colore, sapore, sono, calore aut aliis qualitatibus sensibilibus secundum sensationum diversitatem. Secundo, obiectum sic positum non posse percipi ab intellectu cum nondum sit universale. Quare necesse est ut prius denudetur ab his conditionibus individualibus, separetur quae ab omni materia singulari et fiat proportionatum obiectis intellectus. Ad hoc enim ponitur intellectus agens, ut enim melius percipiatur. 
\pend

\pstart
  Nota in nobis esse duplicem intellectum alterum agentem, alterum possibilem ex \edtext{ Divo \name{\textsc{Thoma}\index[persons]{Thomas Aquinas}} 1 parte, quaestione 79, articulo 2 et 3 }{\lemma{}\Afootnote[nosep]{}}. Dicitur intellectus agens non ab agendo et producendo cognitionem, sed ab \textnormal{|}\ledsidenote{BNC 103va}   agendo species quae recipiuntur in intellectu possibili. Ideo enim dicitur possibilis quia patitur seu recipit istas species. Unde officium intellectus agentis est denudare phantasmata ab conditionibus individualibus et sic species quae directe repraesentabat Petrum, iam directe repraesentet naturam humanam. 
\pend

\pstart
  Haec igitur denudatio seu separatio naturae a conditionibus individualibus dicitur abstractio intellectus agentis, nam abstrahit et separat naturam humana a singularibus. Non tamen dicitur intellectiva, quia cum intellectio producatur a potentia et specie, et ista potentia faciat species, non supponit illas; unde faciendo illas non intelligit. Intellectus agens dicitur a \edtext{ Divo \name{\textsc{Thoma}\index[persons]{Thomas Aquinas}} \enquote{lumen} 1 parte, quaestione 54, articulo 4, ad 2 }{\lemma{}\Afootnote[nosep]{}}. Sicut enim ut visus videat colorem requiritur species coloris et lumen, et utrumque concurrit ad visionem. Ita phantasmata requiruntur ut species coloris, et intellectus agens ut lumen, ut sic intellectui possibili repraesentetur species. 
\pend

\pstart
  Praeterea intellectus possibilis praedicata specie informatus producit in se ipso speciem expressa, seu verbum mentis, quod est veluti parere conceptum. Hinc oritur duplex cognitio, una qua intellectus cognoscit naturam praecise secundum sua praedicata essentialia non cognoscendo singularia, sed ab illis praescindendo et haec dicitur `abstractio praecisiva' pertinetur ad primam operationem. Altera qua cognoscit naturam in ordine ad singularia ut ab illis participabilis, unde haec cognitio supponit priorem, dicitur enim comparatio. 
\pend

\pstart
  \textnormal{|}\ledsidenote{BNC 103vb} Nota demum discrimen inter actum comparativum et absolutum. Actus comparativus sumitur dupliciter. Primo ut pertinet ad primam operationem, quo unum non attribuit alteri vel deducitur ex alio, sed cognoscitur unum ad aliud ordinatum. Secundo ut pertinet ad secundam et tertiam operationem quo comparat unum alteri conponendo et dividendo vel per illationem discurrendo. 
\pend

\pstart
  Unde per secundam operationem attribuitur aliquid subiecto ut praedicatum, et sic per hanc potest natura praedicati de inferioribus et universalitas potest etiam praedicari de natura quam denominat tanquam forma. At in prima operatione non datur praedicatio, sed tantum respectus naturae ad suum terminum quod quae ordinatur. Dicitur hic comparativus per respectum ad terminum ad quem, non vero per respectum ad terminum a quo. Undde solum terminus ad quem est comparativus. 
\pend

\pstart
  Actus absolutus non comparat unum ad aliud, sed rem ipsam cognoscit secundum se. Hic dividitur in negativum et positivum. Negativus accipit unum omni alio suppresso vel omisso. Positivus accipit separando ab alio cum cognitione utriusque. Unde oritur duplex abstractio, scilicet negativa et positiva. Negativa cognoscit naturam omissis conditionibus individuantibus, negativa causa illas se habendo. Positiva separat naturam a conditionibus cognoscendo naturam et conditiones. Relinquitur in summa dari abstractionem intellectus agentis, abstractionem intellectus possibilis negativam et positivam, et comparationem ipsius intellecto simplicem et compositivam. Nunc inquiramus. 
\pend

        \addcontentsline{toc}{section}{Articulus 2. Per abstractionem intellectus agentis fit aliquod universale?}
        \pstart
        \eledsection*{Articulus 2. Per abstractionem intellectus agentis fit aliquod universale?}
        \pend
      
\pstart
  Prima sententia asserit in abstractione intellectus agentis fieri universale logicus. Ita \edtext{ \name{\textsc{Zumel}\index[persons]{Franciscus Zumel}}  1 parte, quaestione 13 articulo 7, conclusione 7 }{\lemma{}\Afootnote[nosep]{}}, \edtext{ \name{\textsc{Soto}\index[persons]{Dominicus de Soto}} quaestione 2 \worktitle{Universalis} }{\lemma{}\Afootnote[nosep]{}}, \edtext{ \name{\textsc{Capraeolus}\index[persons]{Ioannes Capreolus}} in 3, distinctine 5, quaestione 3 }{\lemma{}\Afootnote[nosep]{}}, \edtext{ \name{\textsc{Flandria}\index[persons]{Dominicus de Flandria}}  7 \worktitle{Metaphysicae} quaestione 3, articulo 2 }{\lemma{}\Afootnote[nosep]{}}, \edtext{ \name{\textsc{Albertus}\index[persons]{Albertus Magnus}}  1 \worktitle{De anima} commento 8 }{\lemma{}\Afootnote[nosep]{}}, \edtext{ \name{\textsc{Masius}\index[persons]{Didacus Masius}}  sectione 2, quaestione 5, conclusione 1 }{\lemma{}\Afootnote[nosep]{}}. 
\pend

\pstart
  Secunda sententia non solum negat fieri universale logicum in tali abstractione, sed etiam nec universale metaphysicum. Ita Illustrissimus \edtext{ \name{\textsc{Merinero}\index[persons]{}} distinctione 2 \worktitle{universalis}, quaestione 1, numero 29 }{\lemma{}\Afootnote[nosep]{}} refert de aliquibus. 
\pend

\pstart
  Sit prima conclusio: per abstractionem intellectus agentis non fit universale logicum. Ita omnes Thomistae. Probatur ratione. Nullus ens rationis resultat per abstractionem intellectus agentis; sed omnis universalitas est ens rationis, igitur nulla universalitas resultat per abstractionem intellectus agentis. Probatur maior. Ens rationis fieri non potest a potentia non congnoscitiva, sed intellectus agens non est potentia cognoscitiva. Igitur nullus ens rationis resultat per abstractionem dictam. 
\pend

\pstart
  Sit secunda conclusio: per abstractionem intellectus agentis repraesentatur universale metaphysicum, non cognoscitur. Quod non cognoscatur \emph{patente [?]}, quia talis potentia non est cognoscitiva ut constat numero 274. Est expressa mens \edtext{ \name{\textsc{Angelico Praeceptore}\index[persons]{Thomas Aquinas}} I parte, quaestione 14, articulo 11, ad 1; et quaestione 79 articulo 3; et quaestione 85, articulo 1, ad 3 et 4; et II \worktitle{Contra gentes} capitulo 77, et \worktitle{Opusculo} 55 }{\lemma{}\Afootnote[nosep]{}}. Et \edtext{ \name{\textsc{Caietanus}\index[persons]{Thomas de Vio Caietanu}} IV \worktitle{De ente et essentia}, quaestione 6, sectione istis praelibatis. }{\lemma{}\Afootnote[nosep]{}} 
\pend

\pstart
  Probatur ratione. In intellectu agente repraesentatur \textnormal{|}\ledsidenote{BNC 104rb} natura separata ab individuis ut una; igitur non repraesentatur natura secundum se, nec singularis; sed natura sic sumpta est universale metaphyscicum et fundamentum universalis logici. Igitur in tali abstractione repraesentatur universale metaphysicae et fundamentum universalis logici. Prima consequentia patet. Natura secundum se nec est una nec plures, sed ibi repraesentatur natura ut una; igitur non repraesentatur natura secundum se. Probatur altera pars consequentiae. Ibi repraesentatur natura ut una, sed non unitate singulari ut supponimus: igitur, universali, sed non unitate universale logica; ut patet in prima conlusione: igitur metaphysica. 
\pend

\pstart
  Probatur ibi naturam esse unam, ratione cuius efformata est tota ratio. Cognitio id attingit cognoscendo, quod species repraesentando; sed intellectus possibilis informatus illa specie cognoscit naturam ut unam. Igitur species facta in agente intellectu repraesentat naturam unam. Probatur minor. Sit intellectus possibilis informatus intellecta specie non cognosceret naturam ut unam ex vi talis cognitionis. Natura non redderetur una et cognita, sed natura redditur una ut cognita: igitur cognoscit naturam ut unam. 
\pend

\pstart
  Obiectio 1. Contra prima conclusionem. Esse universale in actu est esse intelligibile in actu, sed per abstractionem intellectus agentis res fiunt intelligibiles actu: igitur universales. Confirmatur: natura tunc est universalis quando habet conditiones, quibus universale platonicum esset universale si daretur. Sed posita natura abstracta et separata per intellectum agentem habet conditiones universalis platonici: igitur tum est universalis. 
\pend

\pstart
  \emph{Respondeo [?]}. Esse universale logicum in actu est esse intelligibile in actu formaliter, nego; esse universale logicum fundamentaliter, concedo et similiter distinguitur consequens. Optime salvatur esse obiectum formaliter et fundamentaliter universale. Esse obiectum intellectus et actu intelligibile \textnormal{|}\ledsidenote{BNC 104va}   habet natura a sua intrinseca entitate et a perfectione et immaterialitate, quam habet vel quia est actus separatus vel quia est abstracta ab aliqua materia saltem singulari. Ceterum esse formaliter universale habet natura a secunda intentione. 
\pend

\pstart
  Ad confirmationem interrogo vel talis natura diceret separata relationem realem vel non? Si diceret: datur disparitas quare a parte rei esset universalis et non per abstractionem agentem; siquidem ex eo praecise, quod natura separetur non refertur ad inferiora. Si non diceret esset universalis metaphysice: et sic bene concludit argumentum per talem abstractionem esse universale metaphysicae repressum. 
\pend

\pstart
  Obiectio 2. Universale est quid communi et indifferens, sed esse denudatum ab conditionibus individuantibus est quod commune et indifferens fieri. Igitur esse denudantum est esse universale, atque per abstractionem intellectus agentis denudatur natura a conditionibus: igitur fit universales 
\pend

\pstart
  Respondeo distinguendo maiorem. Universale logicum est quod commune et indifferens in repraesentando, nego. In essendo subdistinguo: est quod commune et indifferens absolute, nego; relative, concedo. Solutio applicatur etiam minori et probationi. In abstractione agentis natura est una et repraesentata, non ut cognita 
\pend

\pstart
  Obiectio 3. Natura contracta est singularis, igitur abstracta est universalis. Respondeo distinguendo consequens. Abstracta est universalis logice, nego; metaphysice subdistinguo: est universalis metaphysice in repraesentando, concedo; in essendo, nego. 
\pend

\pstart
  Contra secundam conclusionem obiectio 1. Natura abstracta per intellectum agentem non est una: igitur non est universalis metaphysice. Probatur antecedens. Ideo est una, quia intellectus possibilis illam attingit ut unam, sed haec ratio non probat: igitur non est una. Probatur minor. Quando intellectus cognoscit naturam secundum se, cognoscit illam ut unam, sed ex hoc natura illa non habet \textnormal{|}\ledsidenote{BNC 104vb} unitatem universalis: igitur nec natura abstracta. 
\pend

\pstart
  Respondeo negando antecedens. Ad probationem nego minorem et ad eius probationem distinguo: maiorem cognoscit naturam secundum se ut unam ex parte rei cognitae. Nego maiorem. Ex parte modi concipiendim, concedo. Natura secundum se solum significatur ex parte rei cognitae. Si vero cognoscatur per modum unius iam habet ex modo concipiendi unitatem praecisionis et universalitatis. 
\pend

\pstart
  Obiectio 2. Universalitas fit per cognitionem, sed in intellectu agente non datur cognitio: igitur nec universale. Respondeo distinguendo maiorem. Universalitas cognita actu fit per cognitionem, concedo. Universalitas actu repraesentata et virtute cognita, nego. Non asserimus universale metaphysicum cognosci, sed per repraesentari per hanc abstractionem. 
\pend

\pstart
  Instabis: in illa specie non repraesentatur universale formaliter: igitur fundamentaliter. Ergo vel fundamentaliter remote vel proxime. Non remote aliter nihil aliud esset quam ipsa singularia. Non proxima quia hoc ipso est ens rationis. 
\pend

\pstart
  Respondeo concedendo antecedens et distinguendo consequens. Igitur fundamentaliter proxima ut cognitum, nego; ut repraesentatum, concedo. Ad probationem distinguo: in illa specie est ens rationis formaliter, nego, fundamentaliter subdistinguo: ens rationis fundamentaliter cognoscitive, nego; repraesentative, concedo. Fundamentum universalitatis est ipsa natura cognita, cumque ipsa natura repraesentetur in intellectu agente. Ideo datur fundamentum ut repraesentatum. 
\pend

\pstart
  Urgebis: ubi est impossibilis actus est impotis potentia, sed in intellectu agente est impossibilis actus universalis: igitur et ipsa aptitudo, igitur et universale. Probatur minor. Actus universalis est praedicatio de multis, sed in intellectu agente est impossibilis praedicatio: igitur et actus. 
\pend

\pstart
  \textnormal{|}\ledsidenote{BNC 105ra}   Respondeo distinguendo minorem. Ubi est impossibilis actus  ab intrinseco est impossibilis potentia, concedo. Ubi est impossibilis actus ab extrinseco defectu applicationis, nego. Sicut ignis hic existens habet potentiam conburendi lignum Romae, sed non applicationem sine qua non potest conburere. Sic illa natura abstracta habet capacitatem saltem in repraesentando, deest tamen applicatio, scilicet cognitio comparativa. 
\pend

\pstart
  Iterum urgebis: intellectus agens non format unam naturam ex pluribus individuis: igitur non repraesentat naturam unam et indifferentem, sed solum secundum se. Probatur antecedens. Intellectus agens non comparat unum alteri, sed solum abstrahit negative. Sed haec comparatio requeritur ut natura sit una et apta esse in pluribus: igitur intellectus agens non repraesentat naturam unam, igitur neque universalem. 
\pend

\pstart
  Respondeo distinguendo antecedens. Non format unam naturam ex pluribus individuis collative, concedo; praecissive, nego. Intellectus agens repraesentat naturam tali modo, scilicet praecisionis et sic non repraesentatur natura solum secundum sua praedicata, sed etiam una unitate praecisionis. Ad probationem distinguo secundam partem maioris. Solum abstrahit negative \emph{hic [?]} negative reduplicando supra individua, quae ommittit, concedo; reduplicando supra naturam, nego. Ad minorem dico: haec comparatio requiretur ut natura sit una unitate formali logica, concedo; unitate repraesentativa, nego et currit solutio ad consequentiam. Articulus 3. 
\pend

        \addcontentsline{toc}{section}{Articulus 3. Per abstractionem intellectus posibilispossibilis fit aliquod universale?}
        \pstart
        \eledsection*{Articulus 3. Per abstractionem intellectus possibilis fit aliquod universale?}
        \pend
      
\pstart
  Prima sententia docet universale fieri formaliter per abstractionem intellectus possibilis non discriminando universale \textnormal{|}\ledsidenote{BNC 105rb} logicum ab metaphysico. Ita \edtext{ \name{\textsc{Durandus}\index[persons]{Durandus a Sancto Porciano}} in I, distinctione 3, parte 2, quaestione 5, ad 2 et in II, distinctio 3, quaestione 7, numero 7 et 12 }{\lemma{}\Afootnote[nosep]{}},  \edtext{ \name{\textsc{Toletus}\index[persons]{Franciscus Toletus}}  quaestione 2 }{\lemma{}\Afootnote[nosep]{}} et alii quos citat  \name{\textsc{Ioannes a Sancto Thoma}\index[persons]{}} hic  et  \name{\textsc{Complutenses}\index[persons]{}}.  Sententia etiam procedit indistincte asserendo universale fieri per actum comparativum. 
\pend

\pstart
  Sit prima conclusio. Per abstractionem intellectus possibilis non fit universale logicum. Est communis inter Auctores forma constituens universale logicum est ens rationis, sed in hac abstractione non dat ens rationis: igitur neque universale logicum. Probatur minor. In actu in quo cognoscitur natura secundum praedicata, quae habet a parte rei non fit ens rationis, sed per talem abstractionem praecisivam solum cognoscitur natura secundum praedicata, quae habet a parte rei: igitur in tali abstractione non fit ens rationis. Maior est evidens. Nam tunc fit ens rationis quando cognoscitur in natura aliquod, quod non habet a parte rei. Minor etiam est certa. Sequidem per talem abstractionem non cognoscitur in natura aliquod quod a parte rei non habeat. Et probatur. Ex eo quod intellectus cognoscat naturam sine singularibus, non cognoscit illam aliter ac est in se. Sed sic cognoscit naturam per hanc abstractionem: igitur non cognoscit naturam aliter. Probatur maior. Intellectus seu visus videns colorem in pomo mere negative se habens causa saporem non videt illud aliter, nec format ens rationis: igitur intellectus cognoscens naturam, omissis singularibus non cognoscit illam aliter. Igitur solum cognoscit naturam secundum praedicata, quem habet a parte rei 
\pend

\pstart
  Sit secunda conclusio. Haec abstractio est actus absolutus et per illam fit universale metaphysicum. Probatur prima pars. Ille actus est absolutus, \emph{qui [?]} respicit seu attingit obiectum sine respectu et ordine ad terminum ad quem, sed actus praecisionis non attin \textnormal{|}\ledsidenote{BNC 105va}    attingit obiectum cum ordine ad aliquod ut ad terminum ad quem: igitur talis abstractio est actus absolutus. Probatur minor. Repugnat respicere simul aliquod ut terminum a quo et ut terminum ad quem, sed actus praescindens et abstrahens respicit id, a quo abstrahit ut terminum ab quo: igitur non potest respicere illud tamquam terminum ad quem, sed comparatio est ad aliquod ut terminum ad quem. Igitur actus abstrahens non est comparativus. Igitur abstractio possibilis est actus absolutus. 
\pend

\pstart
  Probatur secunda pars. Non solus abstractio positiva, sed etiam negativa relinquit inferiora et ab illis separat naturam, positiva per positivam cognitionem, negativa omittendo illa, ut dictum est. Igitur utraque solum pertinet ad universale metaphysicum. Patet consequentia. Utraque abstractio non continet ad universale logicum: igitur ad metaphysicum. Antecedens probatur. Ubi non datur comparatio et ordo ad aliquid ut terminum ad quem non datur universale logicum, sed utraque abstractio est actus non comparativus, sed absolutus. Nam utraque negotiatur eumdem effectum, scilicet separationem naturae ab inferioribus et per consequens respicere illa ut terminum ab quo. Igitur per utramque non fit universale logicum. Igitur solum fit universale logicum metaphysicum. 
\pend

\pstart
  Obiectio 4. Contra primam conclusionem. Tunc fit universale logicum cum cognoscitur natura una aptaque ad essendum in pluribus. Sed per hanc abstractionem cognoscitur natura una apta ad essendum in pluribus. Igitur tunc fit universale logicum. 
\pend

\pstart
  Respondeo distinguendo maiorem. Tunc fit universale logicum cum cognoscitur una natura et apta esse in pluribus absolute, nego; comparative, concedo. Sed per abstractationem cognoscitur natura una et apta absolute, \textnormal{|}\ledsidenote{BNC 105va} concedo; comparative subdistinguo fundamentaliter, concedo; formaliter, nego. Verum est naturam abstractam unam esse, siquidem uno actu cognoscitur; et aptam, siquidem est conformis, sed totum hoc est absolute cum nondum respiciat terminum ad quem, sed tantum terminum ab quo. Quare sic concepta natura est universalis metaphysice, non logice, ut manet probatum. 
\pend

\pstart
  Obiectio 2. Natura abstracta per illum potentialem non est fundamentum universalitatis: igitur iam est universale. Probatur antecedens. Natura separata a singularibus manet una et indivisa, et apta ut sit in pluribus. Sed hoc habet ab intellectu agente: igitur ab intellectu possibili non habet fundamentum, sed ipsam universalitatem. 
\pend

\pstart
  Respondeo negando antecedens. Ad probationem distinguo maiorem. Manet una et indivisa ut repraesentata, concedo; ut cognita, nego. Universalitas enim cum sit secunda intentio indiget denominatione extrinseca cogniti; talem enim denominationem habet natura per abstractionem intellectus possibilis et sic abstracta est fundammentum universalitatis, non \emph{vero [?]} universalis. 
\pend

\pstart
  Obiectio 1. Contra secundam conclusionem. Per talem abstractionem cognoscitur natura secundum quod convenit ei in sua essentia, sed natura sic cognita est natura sola: Igitur non est universalis, nec metaphysice. Respondeo distinguendo maiorem. Secundum quod convenit ei in sua essentia ex parte rei cognitae, concedo; ex parte modi concipiendi, nego. Non est possibile concipere naturam, quae ex parte modi concipiendi nec universalis sit, neque particularis, quam possibile sit hoc ex parte rei conceptae. Et sic per hanc abstractionem natura est una positive non \emph{vero [?]} comparative. 
\pend

\pstart
  Instabis: natura ab intellectu cognoscitur \textnormal{|}\ledsidenote{BNC 106ra}   abstracta ab individuis: igitur ipsa praecisio non dat universalitatem. Probatur consequentia. Universalitas convenit naturae ut cognitae et ut praecisae: igitur prius datur cognitio et praecisio quam universalitas. 
\pend

\pstart
  Respondeo distinguendo consequens. Ipsa praecisio non dat universalitatem logicam, concedo; metaphysicam, nego. Ad probationem distinguo consequens. Prius datur cognitio quam universalitas logica, concedo; quam universalitas metaphysica, nego. Verum est universalitatem per relatione seu logicam fieri per actum comparativum. At vero fundamentum eius, quod est unitas praecisionis habetur eo ipso quod natura cognoscatur praecisa ab inferioribus et sic universalitas metaphysica non praesupponit naturam cognitam, sed ipsa praecisio cognitam reddit. Articulus 4. 
\pend

        \addcontentsline{toc}{section}{Articulus 4. An universale logicum fiat per actum comparativum?}
        \pstart
        \eledsection*{Articulus 4. An universale logicum fiat per actum comparativum?}
        \pend
      
\pstart
  Opinio communis inter Thomistas affirmat. Ita  \edtext{ \name{\textsc{Arauxo}\index[persons]{Franciscus de Arauxo}}  3 \worktitle{Methaphisicae} quaestione 3, articulo 5 }{\lemma{}\Afootnote[nosep]{}}; \edtext{ Soncinas  libro 5 quaestione 27 }{\lemma{}\Afootnote[nosep]{}}; \edtext{ \name{\textsc{Sanchez}\index[persons]{Ioannes Sanchez Sedeno}}  libro 2, \worktitle{Logicae} quaestione 16 }{\lemma{}\Afootnote[nosep]{}}; \edtext{ \name{\textsc{Rubius}\index[persons]{Antonius Ruvius Rodensis}}  hic quaestione 5 }{\lemma{}\Afootnote[nosep]{}} et alii quos refert et sequitur \edtext{ \name{\textsc{Gallego}\index[persons]{Baranbas Gallego de Vera}}  \worktitle{Controversia} 10 }{\lemma{}\Afootnote[nosep]{}}. Secunda sententia negat, affirmat fieri per comparationem compositam, eam docuerunt antiqui Auctores tamen his temporibus displicet. 
\pend

\pstart
  Sit conclusio prima. Universale logicum fit per simplicem comparationem. Ita \edtext{ \name{\textsc{Angelicus Praeceptor}\index[persons]{Thomas Aquinas}} 1 parte, quaestione 28, articulo 1 et in 1 distinctione 2, quaestione 1, articulo 3 et quaestione 7 \worktitle{De potentia} articulo 11 et \worktitle{Opusculo} 48, capitulo 1. }{\lemma{}\Afootnote[nosep]{}} 
\pend

\pstart
  Probatur ratione. Universale logicum est relativum secundum esse et universalitas relatio rationis: igitur fit per \textnormal{|}\ledsidenote{BNC 106rb} comparationem simplicem. Probatur consequentia. Ens rationis fit cum aliquid apprehenditur ubi non est: igitur relatio rationis fit cum apprehenditur ordo in ea, inter quae talis ordo non est. Sed hoc est comparare unum alteri: igitur universale ex ratione communi relationis fit per comparationem. Ceterus ratione suae entitatis, scilicet intentionis simplicis debet excludere indicium et discursum: igitur fit per comparationem simplicem. 
\pend

\pstart
  Secunda conclusio: haec comparatio non est per modum \emph{compositis [?]} et iudiciis, nec per modum inclutionis in inferioribus. Haec conclusio est explicatio primae conclusionis. 
\pend

\pstart
  Probatur ratione fundamentali. Ut detur relatio rationis requiritur comparatio, sed non compositiva et attribuens praedicatum subiecto vel inclutionem faciens \added{in} illo: igitur relatio tantum ordinativa ad terminum. Maior constat siquidem in hoc distinguitur ab actu absoluto. Probatur minor. Comparatio compositis et attributionis vel attribuit naturam cognitam inferioribus vel attribuit relationem ipsi naturae cognitae. Primum esse non potest primo: relatio praedicati ad subiectum est longe diversa a relatione universalis ad inferiora. Sed comparatio compositiva efficit relationem praedicati ad subiectum: igitur non efficit relationem universalis ad inferiora. Secundo: praedicabilitas prior est praedicatione, sed per actualem praedicationem fit praedicatum: igitur talis praedicationis praesupponit praedicabile. Igitur sicut relatio praedicabilis antecedit relationem praedicati sicut potentia actum, sic relatio universalis antecedit comparationem compositivam. 
\pend

\pstart
  Probatur non consistere in inclusione. Nam talis inclusio vel est actualis vel aptitudinalis. Si primum est inclusio per praedicationem et contractionem, \textnormal{|}\ledsidenote{BNC 106va}   sed actualis praedicatio et inclusio supponit praedicabile et contrahibile: igitur talis inclusio praesupponit relationem praedicabilitatis et contrahibilitatis. Si est inclusio aptitudinalis habeo intentum: igitur solum est aptum includi, non vero inclusum actum in inferioribus et sic aptitudo ut includatur est aptitudo ut contra habetur et praedicetur. Sed inclusio fit per contractionem: ergo ante inclusionem datur universale logicum 
\pend

\pstart
  Si vero dicatur \emph{secundis [?]}, scilicet comparatione tribuere ipsam relatione naturae cognitae; contra quia tunc non concideratur ratio formalissima relationis. Probatur quando concipitur relatio in ordine ad subiectum et non ad terminum, seu in ratione in et non in ratione ad non concipitur formaliter ut relatio. Sed quando relatio attribuitur naturae cognitae concipitur in ratione in et non in ratione ad: igitur tunc non concipitur ratio formalissima relationis. Probatur maior. Quando concipitur relatio ratione communi omni accidenti, non concipitur formaliter ut relatio, sed quando concipitur in ordine ad naturam ut subiectum, concipitur ratione communi omni accidenti: igitur tunc non concipitur formaliter ut relatio. Minor constat. Omni accidenti competit ordo ad subiectum et ratio in, etiam accidenti absoluto: igitur quando sic concipitur relatio, relatio concipitur ratione communi omni accidenti. Ergo quando attribuitur relatio naturae cum non concipiatur ut ad terminum, sed ad subiectum, non constituitur formaliter, sed potius praesupponitur constituta: igitur relatio universalis non fit comparatione compositiva. 
\pend

\pstart
  Obiectio 1. Posito fundamento ponitur relatio, sed ante actum comparativum ponitur fundamentum: igitur et relatio. Probatur minor. Ante actum comparativum natura est una et apta esse in pluribus, sed hoc est fundamentum: igitur ante actum comparativum ponitur fundamentum. 
\pend

\pstart
  \textnormal{|}\ledsidenote{BNC 106vb} Confirmatur: ante actum comparativum natura denominatur universalis: igitur datur universale. Probatur consequentia. Denominatio formae fit ab ipsa forma, sed ante actum comparativum denominatur natura universalis: igitur a forma, scilicet ab universalitate. Antecedens patet. Natura abstracta non denominatur singularis cum sit denudata a conditionibus singularibus; nec est secundum se quia secundum se nec est una, nec plures, nec contracta, nec abstracta: igitur est universalis. 
\pend

\pstart
  Respondeo distinguendo maiorem. Posito fundamento ponitur relatio habens esse per resultantiam, concedo; habens esse per cognitionem, nego. Cum in comparatio relatio non cognoscatur nisi per actum comparativum, ideo universalitas fit actu comparativo. Relatio quae resultat posito fundamento proximo ponitur. At relatio quae habet esse per ipsum cognosci, ponitur cum cognoscitur. 
\pend

\pstart
  Ad confirmationem distinguo antecedens. Ante actum comparativum natura denominatur universalis metaphysice, conditio; logice subdistinguo fundamentaliter, concedo; formaliter, nego: sic distinctionem etiam communes. Ad probationem eadem solutione distinguo minorem. Ad probationem distinguo consequens. Igitur est universalis metaphysice, concedo; logice subdistinguo formaliter, nego; fundamentaliter, concedo. Sicut enim contractio efficit singulare metaphysicum, sic abstractio universale metaphysicum non logicum. 
\pend

\pstart
  Obiectio 2. Universale logicum consistit in relatione naturae ad inferiora, sed relatio nequit cognosci sine termino: igitur ipsa inferiora cognoscantur. Sed inferiora constituuntur per inclusionem sui superioris: igitur universale logicum requirit indivisionem in suis inferioribus. 
\pend

\pstart
 \textnormal{|}\ledsidenote{BNC 107ra}  Confirmatur: actus comparativus vel intelligit naturam esse in multtis attribuendo et praedicando eam de illis actu vel solum intelligit eam ei aptum ad essendum in illis. Si primus est comparatio praedicationis. Si secundum: igitur ille actus qui sufficit ad ponendam aptitudinem sufficit ad ponendam relationem aptitudinis. Sed abstractio sufficit ad ponendam aptitudinem: igitur et relationem. 
\pend

\pstart
 Respondeo distinguendo consequens. Ipsa inferiora cognoscuntur per inclusionem, nego; per exclusionem subdistinguo formaliter, nego; praesuppositive, concedo. Et distinguo minorem. Inferiora constituuntur per inclusionem exercite, nego; signata, concedo. Sicut inferiora in re constituuntur includendo naturam superiorem formaliter tamen non constituuntur in ratione inferioris per inclusionem, sed per subisectionem ad naturam superiorem. Sicut enim superius et identificetur cum inferiori cognoscitur in ratione superioris sine illo, sic inferius et identificetur cum superiori cognoscitur tamen ab illo non per inclusionem, sed per subisectionem. Quare conceptus inferioris prout terminat respectum superioris non explicat inclusionem et illam habeat, sed solum subisectionem. 
\pend

\pstart
 Instabis: igitur non potest dari formaliter intentio generis sine intentione speciei; nec species sine intentione individui. Respondeo distinguendo antecedens. Sine specie ut subiscibili, concedo; ut praedicabili, nego. Similiter non potest dari species quin cognoscatur individuum ut subiscibile, concedo; ut praedicabile, nego. Species ut subescibilis est terminus generis et individuum speciei et sic non possunt esse formaliter genus et species sine suis terminis. 
\pend

\pstart
 Ad confirmationem assero secundum dilemmatis. Ad replicam distinguo consequens. Actus quae sufficit ad ponendam aptitudinem absolutam et negativam, sufficit ad ponendam relationem, nego; aptitudinem positivam et \textnormal{|}\ledsidenote{BNC 107rb} respectivam, concedo. Aptitudo est duplex, scilicet positiva et negativa. Negativa solum est non repugnantia ad essendum in pluribus et ista aptitudo cognoscitur non actu comparativo, sed absoluto abstractionis. Positiva dicit ordinem ad ea ad quae habet aptitudinem et ista solum cognoscitur actu comparativo. 
\pend

\pstart
 Ceterus actus comparativus, quo natura attribuitur multis non constituit universale, sed quo natura concipitur apta esse in multis, non aptitudine negative, se positive. Neque ad hoc requiritur, quod relatio cognoscatur in actu signato et tamquam res cognita ut quod. Sed sufficit quod actu cognoscatur aliquod extremum comparatum ad aliud et ad \emph{instar [?]} relativi; sic enim in ipso exercitio comparatio et relatio cognoscitur. Antequam de passionibus Universalibus agamus. Opus est, ut distinctionem inter unitates numericam et formalem ac inter gradus metaphysicos cognoscamus. Unde Questio 5 de etc. 
\pend

        \addcontentsline{toc}{section}{Quaestio 6. De distinctionibus omnibus}
        \pstart
        \eledsection*{Quaestio 6. De distinctionibus omnibus}
        \pend
      
\pstart
 Distinctio in sua generalissima acceptione primo dividitur in adaequata et in inadaequata. Verbi gratia in hac distinctione `Deus' literae `De', adaequate distinguitur a literis `us' et a literis `eu' inadaequate. Adaequate distinguitur, quae eo genere quo distinguitur nihil habent commune; sicut in exemplo posito nulla litera est communis his partibus `De' et `us'. Ea vero distinguuntur inadaequate, quae eo genere quo distinguuntur aliquod retinent commune, ut patet in his dictitionibus `De', `eu', `us', in quibus prima et ultima distinguuntur adaequate ut dictum est. Media ab alterutra extrema inadaequate habet enim est commune \textnormal{|}\ledsidenote{BNC 107va}  cum prima, et est commune cum secunda. Quae adaequate differunt dicuntur diversa, quae inadaequate vero differentia vocantur. Distinctio \del{in}adaequata dicitur condistinctio et quae ea distinguuntur dicuntur condistincta. Sic adaequata proprie explicatur. 
\pend

\pstart
 Inadaequata alia est includentis et inclusi, et alia excedentis, et excessi. Particula `De' a litera `E' distinguitur primo modo, siquidem particula includit literam. Er particula `De' a particula `us' secundo modo. Utraque habet literam `E' communem, sed prima excedit secundam, nam habet literam `D', qua caret secunda et haec excedit primam, quia habet literam `U', qua caret prima. Ita enim et animal se habent sicut includens et inclusum, homo et equus sicut excedens et excessum rationalitas et sensibilitas sicut condistincta. Articulus 1. 
\pend

        \addcontentsline{toc}{section}{Articulus 1. De distinctione reali et formali}
        \pstart
        \eledsection*{Articulus 1. De distinctione reali et formali}
        \pend
      
\pstart
 Distinctio enim est remotio et negatio unius de alis. Identitas vero est affirmatio unius de alio. Distinctio realis est duplex, scilicet positiva et negativa. Positiva versatur inter duas entitates positivas ut inter Petrum et Paulum, Album et Nigrum. Negativa inter duas negativas, ut inter caecitatem et tenebras. His additur mixta, quae versatur inter entitates, quarum una est positiva et alia negativa, ut inter aerem et tenebras, mentem et ignorantiam. 
\pend

\pstart
 Distinctio pure positiva est duplex, scilicet Realis et modalis. Illa versatur inter duas res et haec inter duos modos. Quibus additur mixta, ut inter rem et modum. Distinctio realis est physica. Pro haec distinctione sunt aliqua signa. Primum separatio \textnormal{|}\ledsidenote{BNC 107vb}  realis, idem enim non potest separari ab seipso. Secundum ratio causae; quia idem non potest seipsum producere. Tertium relatio realis; quia idem non potest ad seipsum referri 
\pend

\pstart
 Distinctio realis alia est accidentalis et alia essentialis. Accidentalis causatur per principia accidentalia, sic distinguetur homo albus a nigro; ad hanc reducetur distinctio numerica, quia \emph{licet [?]} sit substantialis est tamen extra essentiam. Distinctio essentialis causatur per principia essentialia, hac distinguuntur res diversi generis vel speciei. Ad hanc reducitur distinctio numerica Angelorum in sententia asserentiunt, non posse esse plures Angelos eidem speciei. 
\pend

\pstart
 Distinctio formalis versatur inter formalitates, et quia umbra est realis tot habet species quod illa. Distinctio formalis positiva est inter positivas formalitates, negativa inter negativas, mixtas vero inter positivam et negativam. Formalitates dicuntur conceptus abiectivi pertinentes ad entitatem realem. In conceptu rationalitatis non clauditur animalitas, nec haeccitas et hae duae sunt carentiae formales. Realiter enim rationalitas nec sine animalitate, nec sine haecceitate reperitur. 
\pend

\pstart
 Distinctio formalis positiva subdividitur in eam quae est similis reali et in eam quae es similis modali. Prima est inter formalitates independentes quarum alterutra sine altera concipi potest, ut animalitas et rationalitas in nostra sententia. Secunda est inter modos formales, quales sunt passiones entis, nam bonitas et veritas distinguuntur distinctione formali modali. Ens enim et veritas aut ens et bonitas distinctione formali mixta, ut res ab modo. 
\pend

\pstart
 \textnormal{|}\ledsidenote{BNC 108ra}   Distinctio formalis reali similis iterum est duplex: nam aliae formalitates sunt substantialis formaliter et alii accidentales. In homine enim animalitas et rationalitas sunt similis materiae et formae physicis; et risibilitas non distinguitur realiter ab essentia et est quodam accidens formales. Distinctio formalis, quae est similis physicae est inter materiam formalem et formam formalem, hoc est inter genus et differentia. Nam genus in nostra sententia est materia quaedam intellectualis contrahibilis et differentia est forma intellectualis contrahens et perficiens. Et sicut datur materia prima realis nullam formam realem habens, sic datur materia prima formalis, quae nullam formam intellectualem habet et talis est genus supremum, quod non componitur genere et differentia. Articulus 2, etc. 
\pend

        \addcontentsline{toc}{section}{Articulus 2. De distinctione fundamentali, et potentiali.}
        \pstart
        \eledsection*{Articulus 2. De distinctione fundamentali, et potentiali.}
        \pend
      
\pstart
 Loquor in omni Schola. Distinctio fundamentalis est quae probet fundamentum, ut vel Scotista rimetur distinctionem formalem in re ante mentis operationem existentem vel Thomista formalitates ante mentis operationem non distinctas mentis acie praecidat, dividat et distinguat. Haec secatur in proximam et remotam. 
\pend

\pstart
 Distinctio fundamentalis remota est ipsa realis physica, quae invenitur inter materiam et formam. Haec secundum se considerata est realis. Αt prout probet fundamentum Scotistae ut suspicetur esse in rebus etiam quamdam distinctionem formalem seu ex natura rei physicae et reali similem; et Thomistae ut suae mentis activitate formale quamdam illi similem faciat, fundamentalis nominatur, remote tamen. Distinctio fundamentalis proxima est ipsa entitatis capacitas, qua una entitas simplex potest distingui in duas vel plures formalitates. \textnormal{|}\ledsidenote{BNC 108rb} vel formaliter ab ipsa natura (secundum \textsc{Scotum}\index[persons]{Ioannes Duns Scotus}) vel ab intellectu efficiente (secundum Divum \textsc{Thomam}\index[persons]{Thomas Aquinas}). 
\pend

\pstart
 Potentia dicitur in ordine ad actum. Unde distinctio potentialis convenit rebus quae distingui possunt. Potentia est duplex, scilicet logica seu praecisiva et physica seu negativa. Illa dicit potentiam praecisam ab actu; et haec haec dicit potentiam carentem actu. Prima Deo adscribitur a theologis dicentibus cum habere potentias intelligendi et volendi. Secunda Deo negatur, quia istae Dei potentiae numquam canuerunt suis propriis actibus. 
\pend

\pstart
 Distinctio potentialis primo est realis. Reperietur igitur inter duas formalitates, quae licet de facto realiter sint identificatae sunt tamen distinguibiles realiter. Talis invenitur in partibus continui, quia de facto illae identificantur et facta distinctione distinguuntur de facto realiter. Est conformis haec distinctio \textsc{Aristoteli}\index[persons]{Aristoteles}. 
\pend

\pstart
 Secundo distinctio potentialis formalis, si agamus in sensu physico et negativo, hoc est si agamus de potentia, quae caret actu non admittitur ab \textsc{Scoto}\index[persons]{Ioannes Duns Scotus}; siquidem realitas, quae habet duas formalitates distinctas formaliter. Numquam fuit in potentia ad hanc distinctionem et formalitates habendas, quia illis numquam caruit. Admittitur tamen a Divo \textsc{Thoma}\index[persons]{Thomas Aquinas}. Homo enim habet distinctionem potentialem formalem, non enim habet ante excisionem genus et differentiam. At potentia ab intellectu in has formalitates incidi et si ille de facto has formalitates excidat tunc erunt duae de facto. Sic similiter \emph{analogiza [?]} de distinctione potentiali modali, quae reperietur inter rem et modum. Saniter distinctio potentialis extrinseca est potentia ut res exprimatur penes diversa connotata. Omitto alias distinctiones potentiales, quia facili negotio possunt scire. Articulus 3. 
\pend

        \addcontentsline{toc}{section}{Articulus 3. De distinctione virtuali tam thomistica, quam scotistica}
        \pstart
        \eledsection*{Articulus 3. De distinctione virtuali tam thomistica, quam scotistica}
        \pend
      
\pstart
 Suppono formalitates secundum se nec distingui, nec identificari. Animalitas secundum se est vita sensibilis, secundum se nec est idem cum rationalitate, nec distinguitur ab rationalitate. Ceterum secundum se non repugnat illi talis distinctio, nec talis identificatio. Constat, quia si animalitas essentialiter exigeret esse idem cum rationalitate, illa exigentia vel esset de conceptu animalitatis vel non? Si non: igitur non est exigentia essentialis. Si vero est de conceptu animalitatis: igitur male definitur animalitas est vita sensibilis, quia deminute praeceditur. Sed deberet sic: animalitas est vita sensibilis exigens identificari cum rationalitate et hoc damnatur ad universis, quia prout sic non posset reperiri in bruto; nam animalitas in bruto non petit identificari cum rationalitate. 
\pend

\pstart
 Distinctionem ex natura rei \textsc{Scoti}\index[persons]{Ioannes Duns Scotus} ipsam esse distinctionem virtualem tenet \edtext{ \name{\textsc{Suarez}\index[persons]{Franciscus Suarez}} \worktitle{De trinitate} capitulo 4, numero 2 et 5 et disputatione 7 \worktitle{Methaphysicae} numero 13 }{\lemma{}\Afootnote[nosep]{}}; \edtext{ \name{\textsc{Aureolus}\index[persons]{Petrus Aureolus}} in 1, distinctione 8, parte 3, articulo 4 }{\lemma{}\Afootnote[nosep]{}}; et alii, quorum meminit noster \edtext{ \name{\textsc{Gregorius}\index[persons]{Gregorio Ariminensis}} in 1, distinctione 8, articulo 1 }{\lemma{}\Afootnote[nosep]{}}. Idem sentit \edtext{ \name{\textsc{Caietanus}\index[persons]{Thomas de Vio Caietanu}} 3 parte, quaestione 3, articulo 3 }{\lemma{}\Afootnote[nosep]{}}. Et colligitur ex verbis \edtext{ \name{\textsc{Scoti}\index[persons]{Ioannes Duns Scotus}} in 1, distinctione 2, quaestione 7 }{\lemma{}\Afootnote[nosep]{}}. Ego a nolo conciliare has distinctiones. \textsc{Ioannes Caramuel}\index[persons]{Ioannes Caramuel Lobkowitz} penitus negat distinctionem virtualem et eam vocat distinctionem moralem. 
\pend

\pstart
 Distinctio igitur virtualis est ipsa eminentia rei virtute continens plura. Hance tenet nostra Schola et \textsc{Angelicus Praeceptor}\index[persons]{Thomas Aquinas} quae sequuntur omnes Thomistae. Ut autem haec innotescat examinemus distinctionem \textnormal{|}\ledsidenote{BNC 108vb} rationis, quae ab hac pendet. Articulus 4 de distinctionis rationis ratiocinantis et ratiocinatae seu cum fundamento et sine fundamento. Quod idem est. 
\pend

        \addcontentsline{toc}{section}{Articulus 4. De distinctione rationis}
        \pstart
        \eledsection*{Articulus 4. De distinctione rationis}
        \pend
      
\pstart
\noindent%
 Articulus 4. De distinctione rationis 
\pend

\pstart
 Non est tenuis difficultas ad tollendam aequivocationem inter haec nomina. Distinctio pluralitas divisio separatio diversitas differentia. Aliquando sumuntur per eodem et non sub eadem formalitate. Nam distinctio seu pluralitas seu multitudo opponitur unitati vel identitati, divisio vero vel separatio unioni seu continuati. Diversitas et differentia dicunt distinctionem, at differentia cum inclusione convenientiae. Diversitas vero cum exclusione illius. Unde ea sunt diversa, quae in nullo conveniunt et ea differunt quae in aliquo conveniunt. 
\pend

\pstart
 Certum est unitatem et distinctionem sequi ordinem entis. Cum igitur solus dentur duo genera entium, scilicet realis et rationis. Ita ut implicet aliud ens medium sic datur duplex distinctio prima realis; secunda rationis. Realis est remotio seu carentia identitatis, quae datur a parte rei sine fictione intellectus. Rationis vero, quae beneficio intellectus fit et in re non datur. Explicata reali videam illam rationis. 
\pend

\pstart
 Sed cur dicatur rationis es duabus. \textsc{Duns Scotus}\index[persons]{Ioannes Duns Scotus} in 1, distinctione 3, quaestione 7, \emph{scholio [?]}, sed hic restat ulterior differentiam. asserit suam distinctionem ex natura rei posse rationis nominari, quia interest duabus rationibus seu formalitatibus; nam sicut vox conceptus aequivoce sumitur iam per conceptu formali seu cognitione, iam per obiecto seu cognito non secus vox Ratio; haec sunt Verba \enquote{potest ab vocari \textnormal{|}\ledsidenote{BNC 109ra}  differentia rationis, non quod ratio sumatur per differentia formali (seu ficta) ab intellectu, sed ut modo accipitur per quidditate (seu conceptu obiectivo) secundum quod quidditas est obiectum intellectus.} 
\pend

\pstart
 Sed iuxta principia \textsc{Aristotelis}\index[persons]{Aristoteles} et Divi \textsc{Thomae}\index[persons]{Thomas Aquinas} distinctio rationis dicitur prout fit ab ipsa ratione et in re non datur. Quae ut melius percipiatur dividitur in distinctionem rationis ratiocinantis et rationis ratiocinantae. Prima fingitur ab intellectu sine fundamento in re et ita solum est distinctio quo ad modum significandi et intelligendi sicut si aliquod apprehendatur praedicari de se et distingui a se. Secunda formatur ab intellectu cum fundamento in re, ut quando distinguimus in una entitate plures gradus eiusdem et in Deo plura attributa. 
\pend

\pstart
 Quidam vero explodunt distinctionem rationis ratiocinantis et solum alterum admittunt. Videatur \edtext{ \name{\textsc{Suarez}\index[persons]{Franciscus Suarez}} in Metaphysica disputationem 7, sectio 1 }{\lemma{}\Afootnote[nosep]{}}; \edtext{ \name{\textsc{Vazquez}\index[persons]{Gabriel Vazquez}}  1 parte, disputatio 117, c. 3 }{\lemma{}\Afootnote[nosep]{}}, alii vero econtra solum cognoscunt distinctionem rationis ratiocinantis, eo quod non inveniant a parte rei fundamentum. Qui vero admittunt distinctionem rationis ratiocinatae variantur in assignado fundamentum. 
\pend

\pstart
 Quidam existimant in re, quae per rationem distinguitur nec actualem, nec virtualem distinctionem suppponi, sed sufficere quod res sit cognoscibilis ad instar distinctiorum cum connotatione ad illa. Ita \textsc{Vazquez}\index[persons]{Gabriel Vazquez} sicut declarat \edtext{ \name{\textsc{Torrejon}\index[persons]{Petrus Fernandez de Torrejon}}  tractato 2, disputatione 1, quaestione 1 }{\lemma{}\Afootnote[nosep]{}}. Alii vero asserunt tale fundamentum esse in re, quod nihil est aliud, quam eminentia quemdam uniens diversas perfectiones seu formalitates, et vocatur distinctio virtualis, quia eadem forma virtute facit, quidquid facerent diversae. Sic \edtext{ \name{\textsc{Caietanus}\index[persons]{Thomas de Vio Caietanu}}  1 parte, quaestione 39, articulo 1 }{\lemma{}\Afootnote[nosep]{}}, cui \textnormal{|}\ledsidenote{BNC 109rb} debet adiungi imperfectus modus intelligendi, non valens totam illam eminentiam unico actu attingere, sed diversis conceptibus. Sic communiter Thomistae. His suppositis aliquas conclusiones proponam in Sententia Divi \textsc{Thomae}\index[persons]{Thomas Aquinas}. 
\pend

\pstart
 Prima distinctio rationis rationcinantis est illa, quae inter extrema, quae distinguit nullam identitatem tollit ex parte obiecti neque materialiter seu entitative, neque formaliter. Unde non supponit distinctionem virtualem et tota est in modo significandi et concipiendi. Distinctio vero rationis ratiocinatae relinquit identitatem ex parte obiecti materialem, sed non virtualem. Ita \edtext{ \name{\textsc{Angelicus Praeceptor}\index[persons]{Thomas Aquinas}} 1 parte, quaestione 13, articulo 4 et quaestione 7 \worktitle{De potentia} articulo 6 et 1 \worktitle{Contra gentiles} capitulo 36 }{\lemma{}\Afootnote[nosep]{}} et \edtext{ \name{\textsc{Caietanus}\index[persons]{Thomas de Vio Caietanu}} \worktitle{De ente et essentia} capitulo 3 }{\lemma{}\Afootnote[nosep]{}}. 
\pend

\pstart
 Secunda, ad fundamentum distinctionis rationis rationis ratiocinatae non suffit distinctio rerum, quarum instar fit distinctio. Sed requiritur in ipso obiecto fundamentum distinctionis. Ratio est efficax: nam in ipso obiecto, quod distinguitur vel est aliqua proportio seu fundamentum ut an instas distinctorum distinguatur vel non? Si primum: igitur non sumitur tota ratio distinguendi ex parte distinctorum, sed ex parte ipsius obiecti proportionem habentis. Si secundus: igitur sine ullo fundamento rem istam concipimus in ordine ad distincta: igitur sine fundamento distinguimus. Sed talis distinctio est rationis ratiocinantis: igitur implicat. 
\pend

\pstart
 Tertia, fundamentum distinctionis rationis ratiocinatae est virtualis distinctio ex parte obiecti: ex parte ab nostri intellectus est imperfectio ipsius non adaequate concipientis omnes illas rationes obiecti, sed diversis conceptibus illas attingentis et comparantis. Ita \edtext{ \name{\textsc{Angelicus Doctor}\index[persons]{Thomas Aquinas}} locis citatis et in 1, distinctione 2, quaestione 1, articulo 2 }{\lemma{}\Afootnote[nosep]{}} et  \edtext{ \name{\textsc{Caietanus}\index[persons]{Thomas de Vio Caietanu}} 1 parte, quaestione 39, articulo 1 }{\lemma{}\Afootnote[nosep]{}}. Ratione: res in quanto supe \textnormal{|}\ledsidenote{BNC 109va}   superior et eminentior plures perfectiones unit, quam inferior, quare in una unitate simpliciori inveniuntur rationes, quae in inferioribus diversas entitates constituerent: igitur si intellectus lumine suo non manifestat illas rationes. Sed uno conceptu attingit unam et alio aliam proculdubio distinguit illas in esse obiecti, cum in re non sint distinctae, sed unum. Fundamentum huius distinctionis consisti in eminentia seu unitate rei continente plures rationes simul cum intellectu inadaequate attingente illam et pluribus conceptibus dividente et abstrahente unam rationem ab alia. Sic distinguitur divina attributa ut affirmat Theologi, sic etiam gradus metaphysici ut dicimus. 
\pend

        \addcontentsline{toc}{section}{Articulus 5. De distinctione penes diversa connotata seu extrinseca}
        \pstart
        \eledsection*{Articulus 5. De distinctione penes diversa connotata seu extrinseca}
        \pend
      
\pstart
 Haec distinctio in pluribus admittenda. Est enim relatio ex una formalitate prout respicit unem terminum ad eamdem prout respicit alium. Sic scientia Dei prout respicit possibilia distinguitur a semet ipsa prout existentia respicit: quia respicere possibilia non est respicere existentia. Unde scia simplicis intelligentiae et visionis est adem entitative, sed habent connotata, quae realiter distinguuntur. Quanta igitur sit haec distinctio, quanta fuerit distinctio inter connotata, ideo dicitur extrinseca: sic igitur cum in sententia \edtext{ Divi \name{\textsc{Thomae}\index[persons]{Thomas Aquinas}} \enquote{essentia distinguatur realiter ab existentia} }{\lemma{}\Afootnote[nosep]{}}. Ideo distinctio actualis inter illas scientias Dei extrinsecus est realis, sic debes discurrere in omnibus. 
\pend

\pstart
 Ut bene intelligatur multiplicabo exempla. \enquote{Idem cubiculum potest et solet vocari Refectorium, cenaculum, Oratorium, Museum, Schola, \textnormal{|}\ledsidenote{BNC 109vb} Gymnasium, Penetrale, Dormitorium, Asylum, etc. Vocabitur Refectorium si in illo reficiantur homines et comedant. Cenaculum si cenent, Οratorium si orent, Museum in illo privatim studeas. Schola si alios doceas. Gymnasium si in illo disputetur, si sit intenrius Penetrale. Dormitorium si in eo dormiatur. Asylum si sit sacrum ac inviolabile ad illum confugium, quae alibi non sunt securi.} 
\pend

\pstart
 \enquote{Una et eadem pecunia vocatur Symbolum, \emph{Naulus [?]}, Minerval, Stipes, Stipendium, Salarium, Honorarium, etc. Symbolum si sit permensa, Naulus si pro navi, Minerval si solvatur Magistro, Stipes si sit capitalis, Stipendium si detur militi: Salarium si famulo: Honorarium si gratuiti laboris dicatur compensatio.} Sic se explicant \textsc{Connotatistae}\index[persons]{} ab quibus aliquando Thomistae usurpant talium. Nota nunc miram meditatione animae. 
\pend

\pstart
 \enquote{Anima indivisibilis est, unica est, simplex est, et quia frangi non potest tot nominibus dilucidatur, ut sola queat \emph{flavissas dictionarias [?]} implere. Anima vocabulo universali dicitur; et quia principaliter eam Petrus verbi gratia crescit, sentit, et intelligit; Vegetativa, Locomotiva, Sensitiva, et rationalis merito nominatur.}  
\pend

\pstart
 Qua Vegetativa plura exercit munia et adipiscitur multa vocabula. Adauget substantiam materialem et nominatur Nutritiva. Auget molem et est Augmentativa. Et quia ad conservationem speciei et novas hominum productiones concurrit Generativa nominatur. Qua Nutritiva utitur alimentis et quia illa calore et siccitate exsugit vocatur adtractiva et quia eadem frigore et sicci \textnormal{|}\ledsidenote{BNC 110ra}   siccitate detinet Retentiva, et quia calore et humore illa concoquit Digestiva, et quia interlectis partibus consentaneis, superfluas frigore et humore egerit Expulsiva. 
\pend

\pstart
 Qua Sensitiva plura exerce. Prout operatur exterius Sensus dicitur; prout interius potentia. Sed quia multa et multifariam sentit et plura et plurifariam potest operationes eius intentionales meditabor. 
\pend

\pstart
 Percipit calores universos in oculis, et est Visus. Sonos et numeros vocales duorum corporum collisione perductos in aure sentiscit et vocatur Auditus. Odores universos distinguit in naribus et dicitur Olfactus. Tandem in nervis carnosis aut in carne nervosa primas qualitates adsequitur et nominatur Tactus. Haec omnia extrinsecus. 
\pend

\pstart
 Intrinsecus distinguit inter propria sensata ut inter colorem et sonum. Percipit communia sensata, scilicet motum, quietem, magnitudinem, figuram, numerum et tunc est centrum et radix quinque sensum (hinc oriuntur ut lineae, propagantur ut ramis) et dicitur `Sensus communis'. Sensibilia separat et separata considerat et graece `Phantasia', latineque `Imaginativa' vocatur. Sensibilium rerum imagines et species retinet et quodammodo coagit et dicitur Cogitativa. Eadem ordinat et disponit et dicitur Aestimativa. Etiam cognoscit insensibilia, nam sicut ovis naturali instinctu horret lupum, gallina vulpem, felis canem, sic multa naturaliter percipit, cuius nec ratione assequetur, nec causas agnoscit, et tunc vocatur `Instinctus'. Et quia cognita recogitat et recognoscit \textnormal{|}\ledsidenote{BNC 110rb} Reminiscentia aut Memoria nuncupatur. Optat, ambit, anhelat et dicitur `Appetitus sensiblis'. 
\pend

\pstart
 Ostrea in mari et multa Conchilia sentiunt et non monentur loco: sydera in aethere moventur, et non sentium; sed utrumque anima et ideo `Virtus' `motiva' vocatur. 
\pend

\pstart
 Prout Intellectiva est divinissima, et `intelectus' dicitur, quia species producit et imprimit. `Intellectus agens', quia easdem recipit et patitur. `Intellectus passibilis' et quia easdem potest exprimere `Possibilis'. Si sequatur rationes ductum a Magno Pater \textsc{Augustino}\index[persons]{Augustinus Hipponensis} `Superior' nuncupatur, si concupiscentias carnis `Inferior'. Res simplicia cognoscit `Apprehensiva'; dum unum enunciat de altero `Iudicativa'; dum unum deducit de altero `Discursiva'. Quia assentitur propter dicentis testimonium, Fides; quia propter nullas rationes `Dubium negativum'; quia propter leves `Dubium positivum'; quia propter praedicabiles `Opinio'; quia propter certas et manifestas `Scientia'. 
\pend

\pstart
 Sed generali vocabulo Scientiae contenta non quiescit. Prout literas scit colligere `Orthographia'; prout a parte et congrue loquitur `Grammatica'; prout floride et ornate eloquendo delectas auditores `Rhetorica'; prout secundas intentiones fabricatur `Logica' seu `Dialectica'; prout numeros computat `Arithmetica'; prout sonoros ad proportiones et consonantias ducti `Musica'; prout punctis lineas, lineis superficies, superficiebus claudit corpora `Geometria'; prout naturaliter contemplatur `Physica'; prout supernaturalia `Hyperphysica'; prout ab omnibus rebus conceptus abstractos et transcendentales speculatur a \textnormal{|}\ledsidenote{BNC 110va}  nonnullis `sapientia', ab aliis `Metaphysica' vocatur; prout Caelorum magnitudines, motus, loca et distantias metitur `Astronomia'; prout eorumdem influxus, directiones, Sexus, domos, exaltationes, depressiones et tandem praedictiones mentitur `Astrologia'; prout humani corporis valetudinem curat, morbos pellit et sanitatem restituit `Medicina'; prout mores regendos et moderados suscipitur `Ethica'; prout Reipublicae conservandae et propugnandae studet `Politica'; prout domesticos instituit `Oeconomica'; prout iura pronunciat aut iuxta \del{pronuntia} pronunciata iudicat `Iures-prudentia'; et tandem prout omnium sciarum finem veritatem aeternam meditatur eminenti vocabulo `Theologia' appellatur. 
\pend

\pstart
 Qua Volitiva Rationalis appetitus vocatur, quia vult nominatur `Voluntas' et quia non vult `Noluntas'. Amat, odio habet, concupiscit, abominatur, dolet, tristatur, delectatur, et dicitur `Concupiscibilis'. Sperat, desperat, timut, audet, indignatur et tunc vocatur `Irascibilis'. 
\pend

\pstart
 Prout se submittit superiore dicitur `Obedientia'; prout aequali `Modestia'; prout inferiori, si id faciat politice `Urbanitas'; et si sincere `Humilitas'; prout est munifica et liberalis `Largitas'; prout concupiscentae cohibet externos motus `Pudicitia'; prout etiam malas etiam cogitationes `Castitas'; prout elegantiam et suavitatem morum servat `Benignitas'; prout cibi et potius ponit luxui modum `Temperantia'; prout proximi imperfectiones tolerat \textnormal{|}\ledsidenote{BNC 110vb} `Patietia'; prout satisfacit diligenter suis obligationibus `Devotio' seu `Pietas'; prout nihil temere aggreditur `Prudentia'; prout mori mavult quam viam veritatis deserere `Fortitudo'; et prout est perpetua quaedam et sincera voluntas unicuique tribuendi quod suum est `Iustitia' nominatur. 
\pend

\pstart
 Ab actibus oppositis sortitur oppositas denominationes et ideo `Contemptus', `Inobedientia', `Petulantia', `Inurbanitas', `Superbia', `Avaritia', `Luxuria', `Invidia', `Gula', `Ira' `Acedia', `Temeritas', et `Iniustitia'. Sic in hac Sententia haec omnia distinguuntur penes extrinseca connotata, quae alias virtualiter aut ex natura rei. Haec meditatio locum habet in Sententia \textsc{Angelici Praeceptoris}\index[persons]{Thomas Aquinas} loquendo radicaliter. Siquidem anima est omnium illorum radix et omnia exercet mediis suis potentialis et habitibus. Articulus 6. De distinctione, etc. 
\pend

        \addcontentsline{toc}{section}{Articulus 6. De distinctione confusiva}
        \pstart
        \eledsection*{Articulus 6. De distinctione confusiva}
        \pend
      
\pstart
 Haec distinctio exemplum sumit a sensu ab oculo verbi gratia: cernis longe dissitum non nihil et nescis an sit lapis vel arbos vel animal; et quia confuse videtur esse corpus affirmas quale sit non determinas. Accedis et motus percipis et quia lapides non monentur corpus illud non esse lapideum affirmas; sed nondum percipis an moveatur ab intrinseco, an ab extrinseco, et dubitas an sit arbos vento agitata vel animans quaedam progrediens. Iam motus progressivum percipis non esse arborem dicis, sed adhuc dubitas, qualis sit illa animans, an belluina, an humana? Accedis et vides esse hominem, nec statim quis sit cognos \textnormal{|}\ledsidenote{BNC 111ra}   cognoscis, donec accedens propriissime esse Franciscum cognoscis et affirmas. Ergo iuxta oculum Franciscus confuse videbatur prius et minus confuse posterius, quosque tandem distinctissime videri potuit: igitur primo intuitu non totum Franciscum in Francisco, sed aliquid omni corpori commune oculis percepisti. Secundo aliquid particularius et communi non omni corpori, sed omni viventi solummodo. Tertio percepisti motum progressivum, qui solum animalibus convenit. Quarto figuram et humanam membrorum proportionem. Quinto accidentia, quae simul in solo reperiuntur Francisco et non in alio. In primo intuitu vidit in Francisco colores visibiles, quos habet qua corpus. Secundo quos habet qua vivens. Tertio quos habet, qua animal. Quarto quos habet qua homo. Quinto tandem quos habet qua Franciscus: sic de ceteris. 
\pend

\pstart
 Cur igitur sicut de oculo de intellectu non possemus philosophari? Ergo intellectus sicut et sensus unam et eamdem indivisibilem rem totum, iam obscure, iam minus obscure, iam tandem clare intelligit. Unde potest intellectus elicere conceptum Petri primo ita clarum, ut nulli nisi Petro conveniat; secundo minus clare, ita ut non solum Petro, sed etiam respondeat Francisco. Tertio adhuc minus clare, ita ut solum hominibus sed etiam brutis omnibus conveniens sit. Quarto adhuc obscurius, ita ut conveniat hominibus brutis, et plantis; non autem elementis metallis, lapidibus et aliis inanimis corporibus. Quinto multo adhuc obscurius, ita ut omni substantiae materiali conveniat. Sic procedunt qui negant praecisiones obiectivas et universalia per actus confusos exponunt. Articulus 7. 
\pend

        \addcontentsline{toc}{section}{Articulus 7. De praescissionibus}
        \pstart
        \eledsection*{Articulus 7. De praescissionibus}
        \pend
      
\pstart
 Praescinditur una formalitas ab altera, quando cognoscitur non cognita alia. Ut bene rem intelligas exemplo loquor. Aliud est rem taceri, aliud negari. Si hunc librum Petrus et Paulus legerint et ego dicam cum lectum a Petro, lectum a Paulo taceam non mentiar; sed lingua Petro asseram et lectum fuisse a Paulo negem errabo. 
\pend

\pstart
 Sicut igitur inter realitates potest lingua alterum dicere et alterum tacere (hoc est, nec affirmare, nec negare), sic mens inter formalitates. Possum concipere naturam divinam quin concipiam personalitates et tunc verbum mentale proferam quo naturam divinam dicam, quo Personalitates non dicam; quo illam exprimam istas tacebo; quo illam praescindam et abstraham a personalitatibus non cognitatis. 
\pend

\pstart
 Potest intellectus primo cognoscere unam rem non cognita alia et potest secundo cognoscere unam formalitatem non cognita alia. Et hoc est abstrahere praecisive. Praeter hanc abstractionem praecisivam datur etiam abstractio negativa, quae rem ita abstrahit ab uno contradictionis termino ut alterum contradictorium affirmet. Exemplificatur: ut cognoscam Deum sufficit quaevis creatura quovis universalis et imperfecta, in illa inveniam Deum imperfectiones negative abstrahendo: ex canis est substantia, est ens, est vivens. Huc>usque perfectiones video: est materialis parvus irrationalis, istae imperfectiones sunt. Igitur retineo omnes perfectiones et has imperfectiones negative abstraho, hoc est, nego, et contradictorias perfectiones affirmo, et Deum invenio et assero, illum esse \textnormal{|}\ledsidenote{BNC 111va}  Substantiam viventem spiritualem infinitam et intellectualem. 
\pend

\pstart
 Additur his abstractio incisiva. Abstraho praecisive quando cognosco unam formalitatem et non cognosco alteram. Abstraho negative quando unam formalitatem capio et captae formalitati nego alteram. Abstraho incisive cum considero animal et considero rationale utrumque simul cognosco, utrumque apprehendo et tamen alterum ab altero distinguo et mente incido et me incidisse et distinxisse percipio reflexa cognitione. 
\pend

\pstart
 Datur etiam abstractio iudicativa seu affirmativa. Vides plures homines Petrum, Paulum, Franciscum, etc, eos concipis et tamen de uno affirmas quod sit albus, de ceteris nil affirmas aut negas. De singulis affirmas diversa, scilicet de Petro quod sit Belga; de Paulo, quod sit Hispanus; de Francisco quod sit Indus, etc. Sic igitur potest accidere intellectui multas formalitates meditanti. Nam si in Petro consideret Haecceitatem, Rationalitatem et animalitatem non cognoscet unam formalitatem sine altera, quia supponitur universas cognoscere, sed tamen illas mente incisiva distinguet et abstractione iudicativa separabit dicendo de singulis diversa praedicata sit. Primo intentionaliter: Animalitas est principius sentiendi, Rationalitas est principius ratiocinandi, etc. Secundo intentionaliter: Animalitas est natura generica, Rationalitas est differentia, Humanitas est formalitas specifica, Petreitas est differentia individualis. 
\pend

\pstart
 Similiter datur abstractio discursiva. Haec datur quando de una formalitate infertur, quod non infertur de altera, quod bifariam potest contingere, scilicet \textnormal{|}\ledsidenote{BNC 111vb} si nihil de altera aut si aliud ex una, et aliud ex alia inferatur. Verbi gratia considero in Petro rationalitatem et animalitatem inferoque. Igitur Petrus est volitivus et dico me vel hac ipsa sola consequentia animalitatem a rationalitate excidisse sequidem volendi perfectio non ab animalitatis gradu sed a rationalitate progreditur. Ecce hic praecidi discursive unam consequentiam ex rationalitate et nullam ex animalitate diducendo. 
\pend

\pstart
 Quod sub eadem consideratione inferam. Igitur Petrus vivit et sentit, tunc animalitatem a rationalitate distinguo quia ex illa consequentias aliquas infero, et nullas existentia. Insuper ipsam animalitatem discursive intercido, quoniam duas consequentias infero et illa igitur vivit, non oritur a sensibilitate et illa: igitur sentit, non oritur a vitalitate. Igitur his consequentiis intellectus discurrens animal in vivens et sensibili dividit. 
\pend

\pstart
 Accepi harum abstractionum definitiones. `Praecisio apprehensiva' cognoscit unam formalitatem non cognita alia. `Praecisio iudicativa' aliquid affirmat de altera formalitate et de altera nihil affirmat aut negat. `Praecisio discursiva' aliquid infert de altera formalitate de altera vero nihil infert. At vero `incisio apprehensiva' utramque formalitatem concipit, sed duobus actibus, quorum singuli singulas tantum formalitates intelligunt. `Incisio iudicativa' de singulis diversa enunciat. `Incisio discursiva' ex singulis diversa infert. In mira doctrina de Praecisionibus. Alias subdivisiones, et intelligentias videbis nego. 
\pend

        \addcontentsline{toc}{section}{Articulus 8. De identificatione}
        \pstart
        \eledsection*{Articulus 8. De identificatione}
        \pend
      
\pstart
 Contrariorum analoga est et similis ratio. Ideo postquam de distinctione egimus de identificatione disputabimus quando quidem distinctio opponitur Identificatione. 
\pend

\pstart
 Res quaelibet dicit sua praedicata intrinseca per quae definitur. Unde quidquid est extra definitionem, accidit definito. Animalitas definitur est `Vita sensibilis'. Unde non est de eius essentia, quod sit aliud ab arbore, nec quod sit idem cum rationalitate. Igitur non est de eius essentia, quod distinguatur realiter ab arbore aut quod identificatur realiter cum rationalitate. Patet quia si ad eius essentiam pertineret illa distinctio aut illa identificatio non posset sine illis quidditative definiri. 
\pend

\pstart
 Ideae rerum possibilium secundum se sunt conceptus essentiales, nec ordinem nec de ordinationem dicentes. Homo `A' possibilis secundum se ne maior, nec minor est Petro, non est Hispanus, nec Romanus, non est coniugatus nec caelebs, non est amicus nec hostis, non est alienigena nec consanguineus; sed est hoc animal rationale. Unde haec omnia non sunt de conceptu huius animalis rationalis, ideoque illi accidunt. 
\pend

\pstart
 Si afficiatur magna mole corporea, erit magnus; parvis si parva. Si nascatur Madriti erit Hispanus; Romanus si Romae. Si ducat uxorem, erit coniugatus; si non caelebs. Si nos amet, erit amicus; si oderit erit hosties. Si originem trahat a nostris maioribus, erit consanguineus; alienigena si ab exteris. Igitur similiter distinguetur realiter, si realem habeat distinctionem; separabitur \textnormal{|}\ledsidenote{BNC 112rb} si habeat separationem; unietur si habeat unitionem. Identificabitur alteri, si habeat identificationem. Probatur consequentia. Sicut esse Hispanum, Indum, amicum, hostem, coniugatum, caelibem, etc., sunt extra conceptum essentialem huius hominis, sic esse realiter distinctum ab hac arbore: igitur. 
\pend

\pstart
 Obiectio. Identificatio est carentia distinctionis, sed omnis res secundum se esse distincta ab \secluded{ab} aliis. Igitur est idem cum aliis. Respondeo distinguendo maiorem. In sensu formali, nego; in sensu illativo, concedo. Albedo non est carentia nigredinis, alias Angeli forent albi, quia nigredine carent. Calor non est carentia frigoris ob eamdem rationem. Sic identificatio non est carentia distinctionis, sed nexus adaequate et perfectissime connectens duas formalitates. Est enim identificatio relatio connectens duo extrema positive per se. Distinctio etiam est relatio positiva connectens extrema non per se. Unde conveniunt in una formalitate et differunt per terminos contradictorios. Sic enim distinctio est relatio, quae non est identificatio. Ceterum omnia secundum se dicunt solum sua quidditativa praedicata. 
\pend

\pstart
 Identitas enim est unio perfectissima. Ita \edtext{ \name{\textsc{Lalemandetius}\index[persons]{}} disputatione 3 \worktitle{Metaphysicae} parte 1, columna 70 }{\lemma{}\Afootnote[nosep]{}}. Explicatur. Unio est duplex: Positiva et negativa. Prima reperitur inter duo extrema realia ut inter materiam et formam;, animam et corpus. Secunda inter rem et seipsam, nam Petrus non dividitur aut separatur a seipso. Igitur Petrus unitur negative sibi ipsi. Hinc oritur duplex identitas, scilicet positiva et negativa. Prima reperitur inter formalitates ratione (aut ex natura rei) distinctas. Secunda inter unam formalitatem et seipsam. Essentia enim divina identificatur Relationi positive et sibi ipsi negative. Haec unio est indistincta ab extremis. Nam non militant rationes, quae \textnormal{|}\ledsidenote{BNC 112va}  circa unionem materiae cum forma, vide quae ibi dicemus. 
\pend

\pstart
 Quaeres quanta sit haec unio quae identitas dicitur? Duplicem opinionem reperio: Prima tenet esse infinitam, sic pure Reales. Secunda docet identitatem esse unionem formalem qua uniuntur duo extrema, quantum unibilia ipsa sunt, hoc est, tam intime, tam perfecte, tam adaequate uniuntur, ut intimius et perfectius non possint. Sic Peripatetici, cum eorum Principe, et haec placet. 
\pend

\pstart
 Nunc videmus de formalitatibus identificatis. In primis dubitatur, an gradus metaphysici identificentur? Varia opinantur Auctores Platonici seu Reales, realiter distinguunt istos gradus. Scotistae formaliter ex natura rei ante mentis operationem. Thomistae virtualiter. Expressivi secundum clarum et obscurum conceptus. Connotatistae secundum diversos terminos, et alii aliter. Unde Scotistae et Thomistae admittunt realem identificationem. Sed hoc postea uberius tractabitur. 
\pend

\pstart
 Dubitatur 2. An existentia et essentia identificentur? Thomistae respondent 1. In Deo essentiam et exsistentiam identifacari, imo omnes catholici. Nam est de fide, definitum olim in Conciliis et tandem contra \textsc{Gilbertus Porretanus}\index[persons]{Gilbertus Porretanus} in Rhemensi. Sed quomodo? Positive fatentur omnes fere Theologi; negative Pater \textsc{Horstius}\index[persons]{Iacobus Merlonus Horstius} sed de hoc in nostra \worktitle{Metaphysica}. Dicunt 2. In creatis essentiam distingui realiter ab essentia: ibidem. 
\pend

\pstart
 Dubitatur 3. An materia et forma realiter identificentur? Certum est, et receptum ab omnibus, realiter distingui. Igitur non identificantur realiter. Vide rationes huius asserti in nostra \worktitle{Physica} libro 1 et vide alia in  \edtext{ Illustrissimo \name{\textsc{Caramuel}\index[persons]{Ioannes Caramuel Lobkowitz}} Disputatione 11, articulo 3, libro 3 }{\lemma{}\Afootnote[nosep]{}}. Insuper dubitatur de quantitate, de qualitate, de Relatione, an identificentur materiae, subiecto et \textnormal{|}\ledsidenote{BNC 112vb} fundamento? Habes in Logica resolutionem negativam, dum ibi affirmamus quantitatem realiter distingui ab materia. Similiter qualitatem et Relationem distingui ab fundamento. Igitur non identificantur. Et haec sufficiant de identificatione et distinctione. Nunc videamus quomodo distinguantur gradus metaphysici, per quod haec fuerunt tradita. Articulus 9. 
\pend

        \addcontentsline{toc}{section}{Articulus 9. Utrum Gradus metaphysici in qualibet re distinguantur sola distinctione rationis?}
        \pstart
        \eledsection*{Articulus 9. Utrum Gradus metaphysici in qualibet re distinguantur sola distinctione rationis?}
        \pend
      
\pstart
 Gradus metaphysicos intelligimus praedicata superiora et inferiora, quae de aliquo praedicantur. Ideo dicuntur gradus, quia cum unum sit universalius et superius altero in cognoscendis illis \emph{quasci [?]}  ascendimus et descendimus. Sic de Petro praedicatur homo, supra illud animal, supra illud corpus, etc. Quaerimus qua distinctione distinguantur? Ex hac manet resolutum quomodo distinguantur naturae universales et unitates formales et numerica. Et omissa distinctione reali Platonica, controversia est inter Thomistas et Scotistas. 
\pend

\pstart
 Unica conclusio: nulla datur distinctio ex natura rei formalis actu extra intellectum inter gradus metaphysicos, sed solus datur distinctio virtualis et fundamentalis. Quae actualis redditur per intellectum. Ita \edtext{ \name{\textsc{Fundatissimus Doctor}\index[persons]{Aegidius Romanus}} \worktitle{De esse et essentia} quaestione 8, 10 et 11. Et \worktitle{De gradibus formarum} parte 2 et \worktitle{De sacramento altaris} propositione 27 et \emph{Quodlibet [?]} 4, quaestione 8 et in 1 distinctione 7, parte 2, quaestione 2 et saepe alibi }{\lemma{}\Afootnote[nosep]{}}, quae sequuntur omnes nostri Doctores ut refert nostro \edtext{ \name{\textsc{Gavardus}\index[persons]{Fridericus Nicolaus Gavardi}} \textnormal{|}\ledsidenote{BNC 113ra}  Quaestio 3 \worktitle{De Universalibus}, articulo 6, sectione 2 }{\lemma{}\Afootnote[nosep]{}}. \edtext{ Item Angelicus Praeceptor Divus \name{\textsc{Thomas}\index[persons]{Thomas Aquinas}} 1 parte, quaestione 7, articulo 3 et in 2 distinctione 3, quaestione 2, articulo 1. Et 7 \worktitle{Metaphysicae} lectio 12 et 13 et \worktitle{De ente et essentia} capitulo 3 et \worktitle{Opusculo} 42 capitulo 9 et 1 parte, quaestione 76, articulo 3 et \worktitle{Quaestione de Spiritualibus creaturis} articulo 1, ad 9 }{\lemma{}\Afootnote[nosep]{}}. Haec est communis et characteristica inter Thomistas, ut referunt \edtext{ \name{\textsc{Complutenses}\index[persons]{}} Disputatione 3, quaestione 4 }{\lemma{}\Afootnote[nosep]{}}. 
\pend

\pstart
 Probatur prima pars conclusionis 1. Divisio distinctionis in realem et rationis datur per differentias, quarum una includit negationem alterius, ratione cuius opponuntur contradictorie. Igitur impossibilis est aliqua distinctio media inter illas. Consequentia patet. Quia inter contradictorie opposita nullum potest fingi medium. Antecedens probatur. Distinctio realis seu ante intellectum est distinctio et non ab intellectu. Distinctio rationis est distinctio ab intellectu. Igitur distinctio realis et rationis aequivalent distinctioni ab intellectu et distinctioni non ab intellectu, quae contradictorie opponuntur. Modo sic: distinctio media inter realem et rationis est impossibilis, sed distinctio formalis ex natura rei est media inter realem et rationis: igitur est impossibilis. 
\pend

\pstart
 Confirmatur vulgari tamen efficaci argumento. Distinctio et unio consequuntur rationem entis, sed implicat dari ens, quod non sit reale vel rationes. Igitur unio et distinctio, quae nec sit realis, nec rationis sed media. 
\pend

\pstart
 Probatur 2. Omnis distinctio tollit aliquam identitatem. Est enim distinctio relatio, quae non est identitas ut dictum est, sed distinctio ex natura rei ante operationem intellectus est distinctio. Igitur ante operationem intellectus tollit aliquam identitatem. Modo sic: talis distinctio tollit ante operationem intellectus tollit aliquam identitatem, sed ante operationem intellectus non datur alia identitas nisi in entitate et realitate: igitur hanc tollit. Igitur est realis. Patet consequentia: distinctio quae tollit \textnormal{|}\ledsidenote{BNC 113rb} identitatem realem est realis:, sed distinctio ex natura rei tollit identitatem realem: igitur realis. 
\pend

\pstart
 Dices non tollere identitatem realem, sed tollere identitatem conceptus, seu formalis rationis, id est, quod unum in re non sit de conceptu alterius seu de constitutione illius. Contra ad hoc non requiritur distinctio actualis, sed sufficit virtualis et fundamentalis, quia sit, probet fundamentum ut unum sub distinctione ab altero concipiatur et sine illo repraesentetur, et sic ibi relucet distinctio, ubi actu est absoluta identitas. 
\pend

\pstart
 Contra 2. Igitur talis distinctio ex natura rei non est actualis. Probatur consequentia. Talis distinctio consistit in hoc quod \emph{licet [?]} unum realiter sit alterum, tamen realiter unum non est de conceptu alterius. Igitur tantum est negatio habitudinis et conexionis intresecae unius cum alio. Sed haec negatio non sufficit constituire extrema distincta actualiter et distinctionem ipsam constituere realem: igitur. Maior est certa (alias consequentia) quia unum esse de conceptu formali alterius nil est aliud quam habere conexionem seu habitudinem essentialem et intrinsecam cum illo: igitur negatio istius identitatis solum est negatio istius habitudinis et conexionis. Probatur maior. Posita illa negatione habitudinis essentialis tantum extrema non manent ita distincta, ut possint fundare inter se relationem huius distinctionis. Sed talis distinctio est negatio habitudinis: igitur. Probatur maior. Ipse \textsc{Scotus}\index[persons]{Ioannes Duns Scotus} ponit hanc distinctionem inter divina attributa, sed inter haec non sunt reales relationes quia tantum quatuor notionales cognoscuntur ab Theologis et in homine, ipse non refertur ad se in quantum animal, sed distinctio est relatio. Igitur talis distinctio \emph{esto [?]} dicat negationem rationis, habitudinis et conexionis. Non tamen ipsam relationem, sed hoc erat necessarium, ut diceretur realis \textnormal{|}\ledsidenote{BNC 113va}  et actualis: igitur. 
\pend

\pstart
 Probatur haec minor. Tunc aliquid actu datur in re, quando actu existit in re, si autem no existit, solum est in potentia et virtualiter. Sed ut sit actu in re non sufficit quod detur quantum est ex una parte si ex animalia non detur, sed requiritur quod absolute et simpliciter detur: igitur quod unum non sit idem cum alio vi conceptus formalis non sufficit, ut dicatur simpliciter esse actu non idem seu distinctum in re. Sed solum secundum quid et negative, id est, quantum est ex vi constitutionis et conceptus formalis non est idem. Sicut dicitur quod natura secundum se non est singularis in re, sed non infertur quod actu non sit singularis. Vide alias rationis in Thomistis praecipive in Theologia. 
\pend

\pstart
 Probatur secunda pars conclusionis 1. Gradus metaphysici distinguuntur inter se, sed non realiter: igitur distinctione rationis ratiocinatae. Sed haec distinctio supponit distinctionem virtualem, ut dictum est numero 1185. Igitur tales gradus distinguuntur virtualiter ante intellectus operationem. 
\pend

\pstart
 Probatur 2. Distinctio virtualis est ipsa eminentia rei plures perfectiones coadunans, sed gradus metaphysici sunt ipsa entitas rei: igitur. Probatur minor. Eadem entitas in homine vegetat, ut facit forma plantae; et sentit, ut facit forma animalis et sic de ceteris. Igitur tales gradus sunt ipsa entitas cum eminentia plurium perfectionum. Igitur virtualiter distinguuntur. 
\pend

\pstart
 Probatur 3. Individuum non addit supra naturam aliquid distinctum ab illa. Igitur natura non distinguitur actualiter a tali differentia a parte rei. Igitur nec unitates formalis, scilicet et numerica. Antecedens probatur. Individuum addit supra naturam singularitatem, sed singularitas non distinguitur actualiter ab natura: igitur. Probatur minor. Si singularitas distingueretur actualiter ante operationem intellectus, natura a parte rei \textnormal{|}\ledsidenote{BNC 113vb} habent unitatem positive communem pluribus. Deinde sequeretur universale platonicum: igitur. Probatur prima sequela. Si natura distingueretur a parte rei haberet in se propriam unitatem realem, per queam constitueretur una et valens distingui; sed talis unitas esset una in omnibus individuis: igitur. Probatur maior. Prius intelligitur res in se indivisa et una quam ab alia divisa, sed talis natura esset divisa a singularitate: igitur esset in se indivisa et una. Probatur illa minor. Per unionem cum individuis non amitteret suam unitatem: igitur esset communis illis. Antecedens probatur. Sicut materia prima realiter distinguitur a forma per suam unitatem, ita natura distingueretur a singularitate per suam unitatem; sed unitas materiae non amittitur per unionem ad plures formas: igitur. Probatur secunda sequela. Et natura sit unita cum singularibus, tamen per illam unitatem vere esset entitas in se actu universalis; sed saltem de potentia absoluta posset esse sine singularibus. Igitur daretur universale separatum, sed hoc implicat: igitur et illuc. 
\pend

\pstart
 Consectarium primum in unoquoque individuo negationes divisionis materialis et divisionis formalis distinguuntur realiter negative. Ita \edtext{ \name{\textsc{Caietanus}\index[persons]{Thomas de Vio Caietanu}} \worktitle{De ente et essentia} capitulo 4, quaestione 6 }{\lemma{}\Afootnote[nosep]{}}. Probatur. Quae distinguuntur essentialiter distinguuntur specie, sed tales negationes distinguuntur essentialiter: igitur distinguuntur specie, quae distinguuntur specie distinguuntur numero; sed tales negationes distinguuntur specie, igitur numero. Igitur in unoquoque individuo alia est negatio, quae sequitur unitatem materialem et alia quae sequitur unitatem formalem. Igitur realiter negative distinguuntur. 
\pend

\pstart
 Consectarium secundum quando actualiter fit ab intellectu distinctio inter gradus ex conceptibus formalibus resultant diversi conceptus obiectivi non in esse rei, sed \textnormal{|}\ledsidenote{BNC 114ra}  in esse obiecti et repraesentati. Explicatur apprehensio non causat aliquid in genere rei in obiecto apprehenso, sed tantum in genere cogniti. In hoc conveniunt distinctio rationis ratiocinatae, et ratiocinantis at differunt quia prima ita distinguit extrema, quod non omnia, quae manifestantur respectu unius conceptus manifestantur respectu alterius. Sed in uno relucet aliqua ratio vel formalitas, quae in alio non relucet, quia obiectum respectu talis luminis et cognitionis non es ex omni parte seu formalitate manifestabile, sed ubi est diversa manifestatio; et diversum manifestabile resultat diversa ratio obiecti in esse obiecti seu in esse cognoscibilis, et manifestabilis, non in esse rei. Igitur respondet diversum obiectum conceptibus istius distinctionis. Ceterum secunda non respicit diversam manifestationem in obiecto, ita quod unum extremum manifestetur uni conceptui et alterum alii, sed idem extremum seu obiectum manifestatur utrique quantum ad intrinsecam rationem obiecti. Nam si diversa formalitas repraesentaretur uni et alteri non haberent identitatem secundum formales rationes, sed in illis different ideoque in esse obiecti. 
\pend

\pstart
 Hac ratione \enquote{moti \textsc{Suarez}\index[persons]{Franciscus Suarez} et \textsc{Vazquez}\index[persons]{Gabriel Vazquez} dicunt quod distinctio rationis ratiocinantis, non est distinctio ex parte conceptus obiectivi, sed repetitio conceptus formalis circa idem omnino obiectum. Sed falluntur, quia bis concipere seu cognoscere idem obiectum non astruit hanc distinctionem. Quia tunc oculus bis videndo Petrus faceret hanc distinctionem. Ergo ut intellectus illam faciat debet facere aliquam comparationem vel respectum apprehendat in ipso obiecto, quo accipiat ipsum ut dico, non secundam diversas rationes in obiecto intrinsecas et ex parte eius fundatas, sed ex comparatione extrinseca resultantes; sicut \textnormal{|}\ledsidenote{BNC 114rb} quando praedicatur idem de seipso ut `Petrus est Petrus' fit quaedam comparatio eiusdem ad se ipsum respectu cuius ut duplicatus concipitur Petrus; et ita fit distinctio non in ipso obiecto intrinsece et secunudm propia considerato: sed extrinsece secundum comparationem vel respectum convictum.} 
\pend

\pstart
 \enquote{At distinctio rationis ratiocinatae et versetur circa eadem rem, quia tamen non adaequate repraesentat omnes eius rationes quilibet conceptus respiciens distinctam rationem in ipso obiecto format distinctam rationem obiecti quasi intrinsecam, quia ex intrinsece obiecto convenientibus accipit unum et relinquit aliud. Igitur non esse obiecti et quasi intrinsece fit ista distinctio, quia in rationibus in obiecto inventis fundatur.} Ita explicat  \edtext{ Divus \name{\textsc{Thomas}\index[persons]{Thomas Aquinas}} in 1, distinctione 2, quaestione 1, articulo 2 et distinctione 22, quaestione 1, articulo 3 }{\lemma{}\Afootnote[nosep]{}}. 
\pend

\pstart
 Dices cum \edtext{ Divo \name{\textsc{Thomas}\index[persons]{Thomas Aquinas}} \worktitle{Opusculo} 9 \enquote{quod distinctio inter attributa divina non est ex parte ipsius Dei, sed in conceptionibus intellectus in quibus est diversa ista ratio significata subiective} }{\lemma{}\Afootnote[nosep]{}} Dico, in hoc loco et similibus \textsc{Sanctus Doctor}\index[persons]{Thomas Aquinas} docere distinctionem istam actualiter solum haberi per conceptiones intellectus et non ante ex parte Dei, quod est verissimum. 
\pend

\pstart
 Obiectio 1. Ante operationem intellectus verificantur duo contradictoria quae respectu eidem verificari non possunt, sed de distinctis; sed a parte rei verificatur oppositio: igitur et distinctio. Antecedens probatur. Est verum dicere a parte rei, animal non est propria et specifica differentia hominis, rationale sic: essentia divina communicatur paternitas non communicatur. Filius procedit per intellectum, non per voluntatem: igitur. 
\pend

\pstart
 Respondeo distinguendo maiorem. Sed de distinctis actualiter, nego; virtualiter, concedo. Contradictoria possunt verificari in re ratione diversarum rationem virtualium, quae fundamentaliter et non actualiter sunt diversa. \emph{De [?]} verificantur illa contradictoria sub diversis rationibus praesertim quando sumuntur sub aliqua reduplicatione vel appellatione \textnormal{|}\ledsidenote{BNC 114va}  saltem virtuali vel quia explicarit aliquem actum materialem aut formalitatem, ut quod essentia communicatur non communicata personalitate, quod filius procedit per intellectum et non per universalitatem. Explicatur enim propius actus notionalis procedendi, communicandi vel non communicandi. Unde cadunt illa condradictoria super naturam et persona sub proprio conceptu naturae vel personae 
\pend

\pstart
 Unde ad illam affirmationem et negationem sufficit distinctio rationis nec requiritur in re distinctio actualis sed sufficit fundamentalis et eminentialis, id est res tantae eminentiae ut aequivaleat pluribus rebus quibus conveniunt oppositae conditiones. Unde quando ponitur propositio negativa \del{non} praedicatum non negatur de subiecto secundum omnem rationem etiam realiter et identice, sed solum in sensu formali. Verbi gratia `Pater generat, essentia non generat'; ista negativa verificatur in sensu formali, id est essentia prout significatur tali modo, id est, in abstracto. Non vero in sensu identico, at quia propositiones si absolute perferantur in sensu formali procedunt, ideo absolute negatur essentiam generare. Ceterum negativa in sensu identico est falsa, ut si dicatur nulla res, quae est essentia generat, Deus non generat falsae sunt, quia essentia divina est res quae generat. 
\pend

\pstart
 Instabis: oppositio facit distinctione. Igitur illa positas inter animal, et rationale, et haec ponitur. Respondeo distinguendo antecedens. Oppositio inducens limitationem separationis, concedo; non inducens, nego. Si oppositio fiat in eminentiori forma facit distinctionem virtualem, non actualem. Animal enim non est natura specifica hominis sumendo animal sub abstractione ab homine et in quantum est illi commune. Rationale pertinet ad essentiam hominis in quantum distinguitur ab animali. Igitur ista non sunt contradictoria; siquidem non sunt eiusdem de eodem. 
\pend

\pstart
 Obiectio 2. Ante operationem intellectus verificantur diversae definitionem. Nam aliter definitur animal, aliter homo, aliter persona \textnormal{|}\ledsidenote{BNC 114vb} aliter natura. Sed quae diversas habent definitiones habent diversa constitutiva: igitur distinguuntur. Respondeo distinguendo maiorem. Verificantur diversae definitiones in re, nego; supposita apprehensione, concedo. Definitur aliquid in quantum apprehenditur. Igitur non repugnat quod si de eodem apprehenduntur et precindutur diversae rationes formales, etiam diverso modo, definiantur et explicentur. 
\pend

\pstart
 Obiectio 3. Actus formalis circa ista obiecta sunt distincti. Igitur supponunt conceptus obiectivos esse distinctos, alias daretur circulus, scilicet quod conceptus formalis distinguuntur per obiectivos, et obiectivi per formales. Respondeo distinguendo consequens. Supponunt conceptus obiectivos esse distinctos virtualiter, concedo; actualiter, nego. Unde non datur circulus, siquidem conceptus formales distinguuntur ex diversis virtualiter, \emph{quaequidemquae quidem [?]} actualiter distinguuntur per intellectum. 
\pend

\pstart
 Obiectio 4. Individuum addit supra naturam aliquid reale. Igitur id distinguitur actualiter a natura ante operationem intellectus. Consequentia patet. Non potest intelligi additio absque distinctione, sed additio est ante intellectum: igitur et distinctio. Respondeo negando maiorem. Sicut enim differentia numerica sumatur ab principio realiter distinctio, ipsa tamen non distinguitur a natura, sed est ipsa natura facta haec per ordinem ad principium individuationis. Unde si additio sumatur proprie per differentia numerica nil realiter additur naturae. Dicitur tamen additio per similitudinem ad augmentum quantitatis, quod fit per additionem. 
\pend

\pstart
 Obiectio 5. Animalitas separatur a rationalitate, quia est in equo in quo non est rationalitas: igitur distinguuntur ex natura rei. Item Homo per animalitatem convenit cum equo; per rationalitatem disconvenit; per animalitatem fundat relationem similitudinis; per rationalitatem vero fundat relationem dissimilitudinis; sed idem formaliter independenter ab intellectu non potest \textnormal{|}\ledsidenote{BNC 115ra}  esse principium convenientiae et disconvenientiae, similitudinis et dissimilitudinis: igitur. 
\pend

\pstart
 Respondeo ad primum negando consequentiam. Animalitas enim equi ab \secluded{a} rationalitate distinguitur realiter, non tantum ex natura rei. Ad secundum negando minorem. Albedo enim eadem est principium convenientiae cum alia albedine et similitudinis, et etiam disconvenientiae et dissimilitudinis cum nigredine. Eadem enim quantitas bicubita est aequalis respectu alterius bicubitae, et inaequalis respectu tricubitae: sic de relationibus adductis. 
\pend

\pstart
 Obiectio 6. Totum metaphysicum et gradus illius correspondet toti physico et eius partibus. Sed totum physicum constat partibus realiter distinctis ut ex corpore et anima. Igitur pariter totum metaphysicum conponitur ex partibus vel ex natura rei distinctis. Maior constat quia compositum metaphysicum non est pura compositio rationis, aliter conveniret Deo. Igitur correspondet illi aliqua compositio. Consequentia patet primum. Quia totum physicum et metaphysicum in re sunt idem: igitur admittunt eamdem distinctionem in suis componentibus. Secundum. Quia \edtext{ Divus \name{\textsc{Thomas}\index[persons]{Thomas Aquinas}} \worktitle{De ente et essentia capitulo 3} }{\lemma{}\Afootnote[nosep]{}} dicit hominem componi ex corpore et anima, sicut ex duabus rebus tertia res. Sed corpus non solum dicit materiam, sed etiam gradus corporeitatis. Igitur unus gradus distinguitur realiter ab alio. Confirmatur. Quia si non distinguerentur isti gradus, possent praedicari in abstracto. Unde esset verum dicere animalitas est rationalitas, etc. Sed hoc implicat ut infra dicemus: igitur. 
\pend

\pstart
 Respondeo distinguendo maiorem. In rebus habentibus materiam, concedo; in aliis nego. Totum metaphysicum non correspondet compositioni ex materia et forma praecise, cum inveniatur compositum metaphysicum etiam in hominis, quae non habent materiam et formam, ut in Angelis et accidentibus; sic enim cadit supra totum physicum sumens ipsum totum ut magis vel minus determinatum. \textnormal{|}\ledsidenote{BNC 115rb} Et non consequentiam totum metaphysicum non componitur ex partibus sed ex praedicatis et conceptibus, unde ista compositio est rationis cum fundamento in re, concedo; sine fundamento, nego. Nec datur in Deo, quia cum sit actus purus non habet aliquid minus determinatum et actuabile; et aliquid magis determinatum et actuans seu contrahens. Et distinguendo primam probationem consequentiae. In re sunt idem et eiusdem compositionis, nego; et diversae, concedo. Ad secundam. Dico \textsc{Sanctus Doctor}\index[persons]{Thomas Aquinas} loqui de corpore, id est de materia et non de corpore informato forma: unde homo conponitur ex corpore, id est ex materia disposita et organizata et sic loquitur de compositione physica. Et distinguendo minorem: corpus informatum, concedo; corpus sumptum pro materia disposita, nego. Patet ex dictis. 
\pend

\pstart
 Ad confirmationem nego maiorem. Sicut rationalitas et animalitas non distinguantur a parte rei, tamen propter modum significandi non verificantur enim sensu formali. Identice tamen verificantur quod rationalitas est res quae est animalitas et est converso, sed de his amplius dicemus sequenti Quaestione. 
\pend

        \addcontentsline{toc}{section}{Articulus 10. Resolvit fortissimum argumentum contra praefatam conclusionem}
        \pstart
        \eledsection*{Articulus 10. Resolvit fortissimum argumentum contra praefatam conclusionem}
        \pend
      
\pstart
 Argumentum quod militat contra distinctionem rationis tantum formatur ex axiomate \edtext{ \name{\textsc{Aristotelis}\index[persons]{Aristoteles}} 3 \worktitle{Physica}, texto 21 per haec verba: \enquote{quaecunque uni et eidem sunt eadem sibi invicem sunt eadem} }{\lemma{}\Afootnote[nosep]{}}. Sed gradus sunt uni et eidem eisdem: igitur sibi invicem sunt isdem. Item sed Paternitas et Filiatio sunt idem uni tertio, scilicet essentiae. Igitur idem sunt inter se. Videamus varias et diversas Auctorum solutiones, ut oppugnamus. 
\pend

\pstart
 Prima solutio. Quae sunt eadem uni tertio re et ratione, concedo; re tantum, nego. Est expressa \textnormal{|}\ledsidenote{BNC 115va}  \textsc{Angelico Praeceptore}\index[persons]{Thomas Aquinas} 1 parte, quaestione 28, articulo 3, ad primum \enquote{ad primum igitur dicendum quod secundum Philosophum argumentum illud tenet in iis quae sunt ideam re et ratione, sicut tunica et indumentum; non autem in his quae differunt ratione.} Hanc sequuntur communiter Thomistae. 
\pend

\pstart
 Sed contra 1. Hoc axioma est principium cui nititur tota ars syllogistica, sed sic expositum non est tale principium: igitur. Probatur minor. Nullus syllogismus perfectus est, in quo medium sit re, et ratione idem cum extremitatibus, nisi solum in synonymis. Verbi gratia Marcus est Tullius, sed Tullius est Cicero: igitur Marcus est Cicero vel ut loquitur Divo \textsc{Thoma}\index[persons]{Thomas Aquinas}. Haec vestis est indumentum, sed indumentum est tunica: igitur haec vestis est tunica. Igitur tale axioma non est principium syllogizandi. Antecedens constat posita distinctione inter universalia seu gradus metaphysicos: igitur tantum deferuit synonymis. 
\pend

\pstart
 Respondit Illustrissimus \textsc{Ioannes a Sancto Thoma}\index[persons]{Ioannes a Sancto Thoma} 1 parte, quaestione 27, disputatione 42, articulo 3, numero 32; \emph{hic [?]} re et ratione debere construi cum \emph{ti [?]} uni, non vero cum \emph{hic [?]} sunt. Unde sensus est, quando illud tertium est unum et idem secundum rem et secundum rationem unde non est virtualiter multiplex, tunc ex identitate cum tertio, sequitur identitas eorum inter se. Contra quia non tollit instantiam: nam in nullo syllogismo perfecto medium seu tertium est idem re et ratione, sine virtuali multiplicitate. Igitur aut axioma non deferuit syllogismis nisi synonimis et tunc non est doctrinale, nec principium syllogizandi aut ruit solutio. 
\pend

\pstart
 Contra 2. \edtext{ \name{\textsc{Divus Thomas}\index[persons]{Thomas Aquinas}} 2 \worktitle{Contra Gentiles} capitulo 9 }{\lemma{}\Afootnote[nosep]{}} probat potentiam Dei et eius operationem esse idem secundum rem, quia quae sunt eadem uni tertio sunt eadem inter se, sed potentia est idem cum essentia et similiter operatio: igitur potentia et operatio sunt idem. Modo sic: potentia et operatio non sunt nomina synonyma ex ipso \textsc{Angelico Doctore}\index[persons]{Thomas Aquinas}: sed nihilominus ex axiomate probat identitatem inter se: igitur sibi ipsi est \emph{canonicus [?]}. Probo. \textnormal{|}\ledsidenote{BNC 115vb} Axioma tenet, quando tertium est idem re et ratione, sed essentia non est tertium idem ratione: igitur male probat ex axiomate aut sibi contradicit. 
\pend

\pstart
 Secunda solutio. Quae sunt eadem uni tertio sunt idem inter se, in illo tertio, concedo; inter se, nego. Ita \edtext{ \name{\textsc{Scotus}\index[persons]{Ioannes Duns Scotus}} in 1, distinctione 2, quaestione 4 }{\lemma{}\Afootnote[nosep]{}} quam approbat \textsc{Lulius}\index[persons]{Raimundus Lullus} , \textsc{Capreolus}\index[persons]{Ioannes Capreolus}, \textsc{Ferrariensis}\index[persons]{Franciscus Silvester Ferrariensis} , \edtext{ \name{\textsc{Caietanus}\index[persons]{Thomas de Vio Caietanu}} 1 parte, quaestione 28, articulo 3 }{\lemma{}\Afootnote[nosep]{}}. Eam tenet etiam \edtext{ \name{\textsc{Angelicus Doctor}\index[persons]{Thomas Aquinas}} in 1, distinctione 33, quaestione 1, articulo 1, ad. 2 }{\lemma{}\Afootnote[nosep]{}}. Contra. Ex hoc axiomate intendo identificationem extremorum inter se, eo quod fuerint identificata in tertio, sed ex hac solutione solum sequitur identificatio extremorum in tertio: igitur non est principium syllogizandi. Contra 2. Quia sic expositum est nugatorium; nam quod est extrema esse eadem inter se in illo tertio? Dicunt esse eadem uni tertio: igitur idem est dicere quae sunt eadem uni tertio, sunt eadem inter se in illo tertio ac dicere: quae sunt eadem uni tertio, sunt eadem uni tertio: igitur est nugatorium. 
\pend

\pstart
 Tertia solutio. Quae sunt eadem uni tertio praedicatione sunt idem inter se praedicatione, hoc est quae dicuntur de uno tertio invicem de se ipsis praedicantur. Ita \edtext{ \name{\textsc{Vazquez}\index[persons]{Gabriel Vazquez}} 1 parte, quaestione 28, articulo 3, disputatione 122, capitulo 2, contra 1 }{\lemma{}\Afootnote[nosep]{}}. \textsc{Philosophus}\index[persons]{Aristoteles} tradit textum in \worktitle{Physica} 3, sed in Physicis non docet praedicationes, sed habitudines reales: igitur non fuit loquitur in sensu expositio contra 2. Igitur iste syllogismus est bonus: natura divina est Pater, natura divina est Filius: igitur Pater est Filius et Filius est Pater. Patet quia quae dicuntur de aliquo tertio invicem de se ipsis praedicantur, sed Pater et Filius dicuntur de uno tertio, scilicet de divinitate: igitur. Contra 3. Igitur hoc axioma in sensu identico seu de identitate reali extremorum intellectum est falsum ex doctrina hoc sequitur, sed argumentatio utitur illo in sensu hoc. Igitur ruit principium syllogizandi. 
\pend

\pstart
 \textnormal{|}\ledsidenote{BNC 116ra}  Quarta solutio. Quae sunt eadem uni tertio universali atquae communi sunt eadem inter se, nego. Quae singulari sub distinctione si fuerit communicabile, nego; si fuerit non communicabile, concedo. Contra. Igitur nullus syllogismus perfectus regulatur ab hoc axiomate. Probatur consequentia: nullus est syllogismus in quo medium non sit universale et communicabile, excepto expositorio, sed haec solutio excludit medium universale et communicabile. Igitur non est principium regulativum: igitur ruit expositio \edtext{ \worktitle{Duacensis} 1 parte, quaestione 28, articulo 3  }{\lemma{}\Afootnote[nosep]{}}. 
\pend

\pstart
 Quinta solutio. Quae sunt eadem uni tertio intentionaliter sunt realiter vel intentionaliter eadem inter se. Contra: igitur iste syllogismus est bonus: quae sunt eadem uni tertio intentionaliter, sunt realiter vel intentionaliter eadem inter se, sed Petrus et Paulus quando intelligunt lapidem hunc singularem et realiter incommunicabilem sunt intentionaliter lapis. Igitur realiter aut intentionaliter sunt idem inter se. Consequens est falsum. Igitur aliqua praemissa est falsa, sed non minor: igitur maior: igitur ruit expositio. 
\pend

\pstart
 Sexta solutio. \edtext{ \name{\textsc{Durandus}\index[persons]{Durandus a Sancto Porciano}}  convictus hoc axiomate sic respondet in 1 distinctione, 2 quaestione }{\lemma{}\Afootnote[nosep]{}}. Quae sunt eadem uni tertio sunt idem inter se, sed Paternitas et filiatio sunt idem naturae divinae: igitur sunt idem inter se. Concedo maiorem et non minore. Admittit relationes distinctas realiter a natura divina, teste Illustrissimo \textsc{Caramuel}\index[persons]{Ioannes Caramuel Lobkowitz}. Contra si distinctio est realis est eadem cum senentia \textsc{Gilberti Porretani}\index[persons]{Gilbertus Porretanus}, sed haec est damnata a Concilio Rhemensis igitur et alia. Vide Theologos. 
\pend

\pstart
 Septima solutio. Quaequaecunque sunt eadem uni tertio sunt idem inter se in humanis, concedo; in divinis, nego; ita \textsc{Suarez}\index[persons]{Franciscus Suarez} et alii. Contra 1: igitur in humanis iste est bonus syllogismus: quae sunt eadem uni tertio sunt idem inter se, sed Petrus et Paulus sunt idem uni tertio, scilicet homini: igitur sunt idem inter se. Consequens est falsum. Igitur aliqua \textnormal{|}\ledsidenote{BNC 116rb} praemissa: non minor; igitur maior. Sed est axioma: igitur in humanis etiam est falsum axioma. Contra 2. Sicut de creaturis sic de Creatore debemus philosophare: igitur hoc axioma debet vel utrobique concedi vel utrobique negari. Si neges antecedens, imo et analogiam: igitur tollitur theologiae veritas. Probo assumptum. Asseris Dei Verbum esse Filium: negat haereticum, tu probas: filiatio seu generatio est origo viventis a viventi in similitudinem naturae; sed processio Verbi divini est huiusmodi: igitur. Respondet ille distinctione maiorem: in humanis, concedo; in divinis, nego et negat consequentiam. Sed est tua solutio: igitur tua solutio evertit totam theologiam contra haereticos. 
\pend

\pstart
 Proposui omnes solutiones praecipuas huius argumenti, nunc vero restat respondere. Meo videri est mera aequivocatio et insufficiens intelligentia solutionum. Ut recte procedam et iudicium feram de his. 
\pend

\pstart
 Dico 1. Philosophum loqui de synonimis. Probatur. Quia exemplum quod adducit est in terminis synonymis, scilicet indumentum, vestis, tunica: igitur axioma secundum Philosophum debet intelligi si vel exemplum non quadrat. Dico 2. Axioma contra dialecticos non postulat tertium unum et idem re, et ratione, sed potius ratione diversum, sic enim est principium syllogizandi. His positis. 
\pend

\pstart
 Solutio. \textsc{Angelicus Praeceptor}\index[persons]{Thomas Aquinas} est optima. In primo testimonio loquitur iuxta mentem Philosophi, ideo dicit secundum Philosophum ponendo idem exemplum, sicut tunica et indumentum, etc. Non vero infecundo sensu, natura in hoc sensu negat dictum \enquote{non vero in his quae differunt ratione} et ponit exemplum in actione et passione quae identificantur motui, quiquidem non est unus re et ratione, sed unus virtute multiplex. In loco \worktitle{Contra Gentiles} sumit axioma in Secundo Sensu, unde recte \textnormal{|}\ledsidenote{BNC 116va}  probat potentiam et operationem esse idem in Deo, quia cum essentia sit virtute multiplex est potentia et est operatio; et utraque est ipsa essentia: igitur identificantur. Sic igitur \textsc{Angelicus Praeceptor}\index[persons]{Thomas Aquinas} non est sibi contrarius. 
\pend

\pstart
 Ad prima replicam distinguo maiorem. Hoc axioma, si sumatur per medio quod est unum et idem re et ratione es principium syllogizandi, nego; si sumatur per medio quod est unum re et virtute multiplex, concedo. De primo procedit replica, non vero de secundo. Et ad instantiam contra Illustrissimo \textsc{Ioannem a Sancto Thoma}\index[persons]{Ioannes a Sancto Thoma}. Concedo antecedens et distinguo consequens. Axioma primo modo sumptum non deferuit nisi synonymis, concedo; secundo modo, nego. Unde iam dixi axioma esse principium syllogizandi, si tertium sit idem rem et non ratione. Quare non valet actio et passio sunt idem motui, sed actio crucifigentium \emph{raptum [?]} fuit mala: igitur et passio. Quia motus cum sit virtute multiplex identificat actione secundum unam formalitatem et passionem secundum aliam. Unde patet solutio ad secunda replicam contra Divum \textsc{Thomam}\index[persons]{Thomas Aquinas}. 
\pend

\pstart
 Nunc ad argumentum quae sunt eadem uni tertio sunt idem inter se, sed Petrus et Paulus sunt idem uni tertio, scilicet homini: igitur sunt idem inter se. Distinguo maiorem. Quae sunt eadem uni tertio, quod est unum re et ratione, sunt idem inter se, concedo; uni tertio, quod est virtute multiplex, nego. Isto homo realiter identificetur cum Petreitate, cum tamen virtualiter distinguantur; non identificatur Pauleitas cum Petreitate, esto identificetur cum humanitate. Sic in divinis. 
\pend

\pstart
 Urgebis: sed isto medium sit virtute multiplex tamen extrema manent identificata inter se. Igitur ruit expositio. Probatur minor. In hoc syllogismo: Omnis homo est animal, sed Petrus est homo: igitur Petrus est animal; Petrus et animal sunt extrema identificata in consequentia, quia sunt identificata in praemissis. Sed nihilominus medium, scilicet homo est virtute multiplex: igitur esto medium, etc. 
\pend

\pstart
 \textnormal{|}\ledsidenote{BNC 116vb} Respondeo distinguendo minorem. Manent identificata inter se ratione huius principii, nego; ratione alterius, concedo. Ad probationem nego causalem. Licet sit verum quod extrema identificata in praemissis, sunt identificata inter se in consequentia, non tamen, quia sunt identificata in praemissis ut patet in hoc syllogismo: essentia divina est Pater, sed Filius est essentia divina: igitur Pater est Filius in quo extrema identificantur cum tertio et non inter se: igitur non ideo in primo syllogismo extrema sunt identificata inter se in consequentia, quia fuirunt identificata in praemissis cum medio. Sed propter aliud principium, scilicet dici de omni: nam quidquid praedicatur de superiori, praedicatur de inferiori contento sub illo; sed de homine praedicatur animali: igitur de Petro patet, quia Petrus est inferior contentus sub homine. Quapropter hoc axioma est principium syllogizandi supposito altero principio. 
\pend

\pstart
 Sic igitur ut concluderet hic syllogismus: Essentia divina est Pater; sed Filius est essentia divina: igitur Pater est Filius; deberet iuxta regulas distribui essentia, ita ut maior deberet intelligi hoc modo: quidquid est essentia divina est Pater; et tunc es falsa, quia etiam essentia divina est Filius. Ideo hic syllogismus non concludit. Quae sunt eadem uni tertio sunt idem inter se, sed Pater et Filius sunt idem in essentia: igitur inter se; quia non supponit dici de omni. Econtra hic syllogismus concludit: Omnis homo est animal, sed Petrus est homo: igitur Petrus est animal; quia extrema identificantur in praemissis per dici de omni. Unde recte identificantur inter se in conclusione. 
\pend

\pstart
 Ceterum axioma sequitur regulas medii. Unde in expositorio aliter debet intelligi, quam enim non expositoriis. Vide in \worktitle{Logica parva} de principiis syllogizandi. Iuxta hanc intelligentia non \emph{suam [?]} \textnormal{|}\ledsidenote{BNC 117ra}  mihi contrarius, si alicubi negaverim axioma vel aliter exposuerim; nam loquebar modo sub uno sensu et modo sub alio. Nec obstat si dicas superfluere hoc axioma, supposito alterio non obstat nam involuit primum et explicat ipsum. 
\pend

\pstart
 Hoc est certissimum et ut liqueat iterum resumo difficultatem. Hic syllogismus est bonus: Omnis homo est animal, Petrus est homo: igitur Petrus est animal. Hic malus: Essentia divina est Pater, Filius est essentia divina: igitur Pater est Filius. Ut quae regulatur axiomate hoc modo: quae sunt eadem uni tertio, sunt idem inter se; sed Petrus et animal identificantur uni tertio, scilicet homini: igitur identificantur inter se: concludit. Item quae sunt eadem uni tertio sunt idem inter se, sed Pater et Filius sunt idem uni tertio, scilicet essentiae: igitur sunt idem inter se: non concludit. Quare? Quia primus nititur principio dici de omni —Omnis homo est animal— non vero secundus, nam si regularetur maior esset falsa, scilicet quidquid est essentia divina est Pater et etiam consequens. Infero: igitur tunc axioma est principium syllogizandi quando includit principium dici de omni. 
\pend

\pstart
 Iam facile respondeo ad difficultatem distinguendo quae sunt eadem uni tertio per dici de omni sunt eadem inter se, concedo; aliter, nego. Sed Pater et Filius sunt idem uni tertio per dici de omni, nego; aliter, concedo. Et nota quod non loquor de syllogismo expositorio, nam hic regulatur aliis principiis. Exemplifico et probo meum assumptum. Ponamus tria corpora contigua, scilicet A, B, C, et dico quae sunt contigua uni tertio sunt contigua inter se. Sed A et C sunt contigua uni tertio, scilicet B: igitur sunt contigua inter se. Consequens est falsum; minor est vera: igitur axioma est falsum. Est utique, quia non supponit dici de omni: nam si supponeret, \textnormal{|}\ledsidenote{BNC 117rb} diceret B est A, sed C est B: igitur C est A; quod totum est falsum: igitur debet axioma dicere tale principium. 
\pend

\pstart
 Demum \emph{do [?]} alteram intelligentiam \textsc{Angelici Praeceptoris}\index[persons]{Thomas Aquinas} supponimus medium seu tertium debere esse virtute multiplex, alias deserviret synonymis. Adhuc tenet solutio \textsc{Angelicus Praeceptor}\index[persons]{Thomas Aquinas} hoc modo: quae sunt eadem uni tertio re et ratione, hoc est non quid tertium sit unum re et ratione; sed quod cum sit virtute multiplex, tamen respectu extremorum sit idem re et ratione id est sumatur secundum eadem formalitatem respectu illorum ex pluribus quas eminentia sua \emph{praecontinet [?]}, ut evadant identificata inter se. Nunc ad primam minorem. Sed Pater et Filius sunt idem uni tertio, concedo; ergo inter se, nego: quia essentia divina quae est virtute multiplex, non est eadem re et ratione in utroque, nam in Patre est cum formalitate Paternitatis; in Filio cum formalitate filiationis; igitur non identificantur inter se. Ad secundam. Petrus et animal identificantur, quia homo esto sit virtute multiplex, tamen ut medium et idem re et ratione, id est secundum eadem formalitatem in extremis, alias variaretur appellas variata formalitate. Igitur re varietur debet esse eadem formalitas; quod autem sit eadem formalitas, dicitur a Sancto \textsc{Thoma}\index[persons]{Thomas Aquinas} 
\pend

\pstart
 Ergo tantum deservit synonymis, nego; quia in synonymis medium est idem re et ratione, hoc est non est multiplex virtute in aliis vero medium est idem re et ratione, hoc est, est virtute multiplex, sed est idem cum extremitatibus secundum eamdem formalitatem ut teneat appellatio. Haec est facilis via et iuxta examinata poteris aut oppugnare aut propugnare solutiones propositas, excepta \textsc{Durandi}\index[persons]{Durandus a Sancto Porciano} in sensu accepto ab \textsc{Caramuel}\index[persons]{Ioannes Caramuel Lobkowitz} .Et haec sufficiant de axiomate. Quaestio. 
\pend

        \addcontentsline{toc}{section}{Quaestio 7. De passionibus universalis}
        \pstart
        \eledsection*{Quaestio 7. De passionibus universalis}
        \pend
      
\pstart
\noindent%
 \textnormal{|}\ledsidenote{BNC 117va}  Quaestio 7. De passionibus universalis 
\pend

\pstart
 Habet universale suas passiones a Logica examinandas, sicut quodlibet accidens. Duae sunt a dialecticis assignatae, scilicet esse perpetuum et esse praedicabile. Perpetuitas convenit ratione naturae denominandae. Praedicabilitas convenit ratione formae; secunda intentio est radix praedicationis specialiter agemus de utroque. 
\pend

        \addcontentsline{toc}{section}{Articulus 1. Quomodo Universale sit perpetuum et an pereant universalia deficientibus singularibus?}
        \pstart
        \eledsection*{Articulus 1. Quomodo Universale sit perpetuum et an pereant universalia deficientibus singularibus?}
        \pend
      
\pstart
 Esse sumitur trifariam, scilicet esse essentiae, esse existentiae et esse quod significat veritatem propositionis. De esse tertio modo sumpto in articulo sequenti. De esse secundo modo est de fide Universalia non esse perpetua. Unde solum est difficultas de esse existentiae. 
\pend

\pstart
 Prima sententia asserit, Universalia esse perpetua, quia habent ab aeterno extra Deum realitatem essentiae, quae dicitur entitas diminuta. Ita \edtext{ \name{\textsc{Scotus}\index[persons]{Ioannes Duns Scotus}} in 1 distinctione 35, quaestione unica }{\lemma{}\Afootnote[nosep]{}}, \textsc{Mayronis}\index[persons]{}  et ceteri Scotistae \edtext{ \name{\textsc{Niger}\index[persons]{Petrus Georgius Niger}}  2 parte \worktitle{Clypei} quaestione 2  }{\lemma{}\Afootnote[nosep]{}}. Secunda tenet contrarium. 
\pend

\pstart
 Dico 1. Universalia non esse perpetua secundum aliquam realitatem positive. Unde destructis singularibus in nihilum redeunt. Ita \edtext{ Divus \name{\textsc{Thomas}\index[persons]{Thomas Aquinas}} 1 parte, quaestione 40, articulo 3, ad 3 et \worktitle{De potentia} quaestione 3, articulo 5, ad 2 }{\lemma{}\Afootnote[nosep]{}} quem sequuntur omnes eius Discipuli. 
\pend

\pstart
 Probatur ratione. Repugnat aliquid habere extra Deum esse essentiae et non habere esse existentiae, sed universalia ab aeterno nihil sunt secundum esse \textnormal{|}\ledsidenote{BNC 117vb} existentiae: igitur nihil sunt secundum esse essentiae. Probatur maior. Quod est extra Deum habet illud esse a Deo: igitur si illud esse est reale habebit esse per veram et realem actionem, emanationemve; sed est chimerica actio emanatiove realis non terminata ad ponendam singulare extra causas, in quo constat existentia: igitur implicat aliquid habere esse extra Deum et non esse existens et singulare. 
\pend

\pstart
 Dico 2. Universalia esse perpetua secundum esse essentiae negative. Ita \edtext{ Divus \name{\textsc{Thomas}\index[persons]{Thomas Aquinas}} 1 parte, quaestione 16, articulo 7, ad 2 et quaestione 1 \worktitle{De veritate} articulo 5, ad 13 et 14 }{\lemma{}\Afootnote[nosep]{}}. Ita communiter eius angelica Schola. Probatur ratione. Quod est negative incorruptibile, est negative perpetuum; sed universalia sunt negative incorruptibilia: igitur negative perpetua. Probatur minor. Quod de se non habet principium a quo corrumpatur est negative incorruptibile; sed universalia de se non habent principium, a quo corrumpantur: igitur sunt negative incorruptibilia. Probatur minor. generatio et corruptio sunt in singularibus: igitur. 
\pend

\pstart
 Confirmatur: sicut universalia nullum sibi determinat locum, ita sibi nullum determinat tempus; sed eo ipso dicuntur ubique et immenso negative: igitur hoc ipso dicuntur negative aeterna et semper. Minorem probat Sanctus \textsc{Thomas}\index[persons]{Thomas Aquinas} sic: sicut materia prima dicitur una non positive sed negative per remotionem, scilicet omnium formarum: igitur eodem modo universalia dicuntur negative perpetua. Siquidem ex se non habent, per quod determinentur ad hoc vel illud tempus. 
\pend

\pstart
 Ex quo patet solutio circa secundum punctum difficultatis:. Sicut enim unitas negativa materiae non obiective quominus plurificetur \textnormal{|}\ledsidenote{BNC 118ra}   iuxta numerum compositorum et habeat toto unitates positive quot formas: ita pariter quod universalia \emph{sint [?]} negative perpetia non obiective quominus vere generentur ac corrumpantur ad generationem et corruptionem singularium, quod est per accidens. 
\pend

\pstart
  Quamvis enim universalia pereant peremptis singularibus possunt tamen tripliciter manere. Primo in virtute suarum causarum. Secundo in exemplaribus seu ideis divinis. Tertio in intellectu in creato vel contingentis vel hominis vi specierum illa repraesentantivam. 
\pend

\pstart
  Obiectio 1. Quod continetur in praedicamentis est aliquid extra Deum, sed universalia antequam existant continentur in praedicamentis: igitur antequam existant sunt aliquid extra Deum; Non realitate existe: igitur essentiae. Probatur minor. Antequam rosa existat vere est in praedicamento substantiae: igitur universalia antequam, etc. 
\pend

\pstart
  Respondeo distinguendo minorem. Universalia antequam existat continentur in praedicamentis actu, nego; potestate subdistinguo; per aliquam potentiam passivam realem ex parte ipsorum, nego; per potentiam activam causae in cuius virtute praeexistunt, concedo. Et alia solutione sunt ab aeterno in praedicamento formaliter, nego; repraesentative, concedo. Cum enim universalia antequam existant nil sint a \emph{parte rei [?]} non possunt actu et realiter contineri in praedicamentis, nisi potentia et repraesentative, ratione divinorum exemplarium. 
\pend

\pstart
  Obiectio 2. Quae sunt unum et idem habent eadem entitatem, sed natura universalis exstitura et existes est una et eadem: igitur habet eadem entitatem, sed non quantum ad existam: igitur quamtum ad essentiam. Illa minor probatur. Eadem est rosa in humi producibilis et in vere producta: igitur natura universalis exstitura, etc. 
\pend

\pstart
  \textnormal{|}\ledsidenote{BNC 118rb} Respondeo distinguendo maiorem. Quae sunt unum et idem positive habent eamdem unitatem, conditio; quae sunt unum et idem negative, nego. Reale et animal in homine quia sunt unum positive habent eadem entitatem. At materia prima quia est una negative in omnibus generabilibus et corruptibilibus non habet eamdem entitatem in omnibus, sed aliam in homine, aliam in equo, etc. Cum enim natura universalis sit negative una, ideo non est \emph{adest [?]} antequam sit et postquam est: quia nil et aliquid non possunt esse unum positive. 
\pend

\pstart
  Obiectio 3. Ab aeterno verum est dicere, homo est animal: igitur ab aeterno dantur homo et animal. Respondeo concedendo antecedens et negando consequentiam. Ab aeterno datur entitas in qua fundatur veritas aeterna illius propositionis, sed non est entitas creata hominis et alis, sed entitas divinorum exemplarium, indistinctorum ab ipso Deo. Articulus 2, etc. 
\pend

        \addcontentsline{toc}{section}{Articulus 2. An et quomodo propositiones universales sint aeternae veritatis?}
        \pstart
        \eledsection*{Articulus 2. An et quomodo propositiones universales sint aeternae veritatis?}
        \pend
      
\pstart
  Suppono 1. Duplicem esse veritatem, scilicet transcendentalem et formalem. Veritas transcendentalis est conformitas rei cum intellectu. Unde per prius est veritas in intellectu, quam in rebus et sic ab ipso derivatur ad illas; quod econtra contingit bonitati. Veritas formalis reperitur in intellectu componente et dividente, datur \emph{quo [?]} in propositionibus; est enim conformitas intellectus cum rebus. Hoc docet \edtext{ \name{\textsc{Angelicus Doctor}\index[persons]{Thomas Aquinas}} 1 parte, quaestione 16, articulo. 1 et 2 }{\lemma{}\Afootnote[nosep]{}}. 
\pend

\pstart
  In hac autem veritate duo nobis obiciuntur cognoscenda. Primum est formale veritatis. Secundum eius fundamentum. Formale est enuntiare seu cognoscere connexionem praedicati et subiecti. Ergo si ab aeterno datur intellectus cognoscens hanc \textnormal{|}\ledsidenote{BNC 118va}   conexionem, dabitur proculdubio veritas formalis propositionis. Fundamentum in quo fundatur formale veritatis seu ipsa veritas est convexio seu unitas rerum inter quas fit convexio, quae est ipsa identitas praedicati et subiecti. 
\pend

\pstart
  Suppono 2. Hanc identitatem duplicem esse, scilicet positivam et negativam. Positiva datur inter aliqua habentia eamdem realitatem et entitatem; sicut animal rationale Petri existentis habet realitatem et entitatem naturae humanae. Relinquimus iam hac. Identitatem positivam, seu veritatem fundamentalem non fuisse ab aeterno. 
\pend

\pstart
  Identitas negativa est negatio diversitatis inter aliqua extrema seu non diversitas. Ergo tunc erit haec identitas quando abfuerit diversitas negata: igitur non ab aeterno abest diversitas inter aliquid subiectum et praedicatum ut accedit propositionibus necessarriis, ab aeterno erit negatio diversitatis inter tale subiectum et praedicatum, et consequenter identitas negativa seu veritas fundamentalis negativa. Relictis variis dicendi modis circa hanc materiam. 
\pend

\pstart
  Sit prima conclusio. Ab aeterno non datur propositiones universales quoad veritatem fundamentale positivam. Unde hoc modo tales propositiones necessariae non sunt aeternae veritatis. Non est cur hic sit immorandum. Siquidem haec conclusio manet probata antecedenti articulo. 
\pend

\pstart
  Sit secunda conclusio. Ab aeterno dantur propositiones universales quoad veritatem fundamentale negativam. Est clara ex dictis. Probatur breviter. Veritas fundamentalis negativa est identitas negativa; sed ab aeterno datur identitas negativa \textnormal{|}\ledsidenote{BNC 118vb} inter subiectum et praedicatum propositionum necessariarum: igitur et veritas fundamentalis negativa. Probatur minor. Identitas negativa est negatio diversitatis, sed ab aeterno fuit negatio diversitatis inter tale suiectum et praedicatum: igitur et identitas negativa. Probatur minor. Inter praedicatum et subiectum talium propositionum nec fuit, nec est, nec erit, nec potest esse diversitas: igitur ab aeterno fuit negatio diversitatis 
\pend

\pstart
  Sit tertia conclusio. Propositiones universales necessariae fuerunt ab aeterno actu verae, veritate propria et formali. Ita \edtext{ \name{\textsc{Angelicus Doctor}\index[persons]{Thomas Aquinas}} 1 parte, quaestione 10, articulo 3, ad 3 }{\lemma{}\Afootnote[nosep]{}}; \edtext{ quae sequuntur \name{\textsc{Bannez}\index[persons]{}}, ibidem }{\lemma{}\Afootnote[nosep]{}} \edtext{ \name{\textsc{Capreolus}\index[persons]{Ioannes Capreolus}} in 1, distinctione 8, quaestione 1, conclusio 1 }{\lemma{}\Afootnote[nosep]{}}; \edtext{ \name{\textsc{Soncinas}\index[persons]{Paulus Barbus Soncinatus}} 9 \worktitle{Metaphysicae} quaestione 5 }{\lemma{}\Afootnote[nosep]{}}; \edtext{ \name{\textsc{Niger}\index[persons]{Petrus Georgius Niger}} 1 parte \worktitle{Clypei} quaestione 48 }{\lemma{}\Afootnote[nosep]{}}; \edtext{ \name{\textsc{Soto}\index[persons]{Dominicus de Soto}} 2 libro \worktitle{Summulae} capitulo 2 et 10 et omnes Theologiae }{\lemma{}\Afootnote[nosep]{}}. 
\pend

\pstart
  Probatur ratione. Veritas formalis non stat in rebus, sed in intellectu cognoscente conexionem praedicati et subiecti. Sed ab aeterno intellectus divinus actu cognoscit conexionem harum propositionum: igitur ipsae ab aeterno fuerunt verum veritate formali. Minor probatur. Quod est enuntiabile necessarium ab aeterno fuit versatum ab intellectu divino; sed propositiones istae sunt enuntiabilia necessaria: igitur circa illas intellectus divinus ab aeterno non potuit non versari. 
\pend

\pstart
  Confirmatur: propositiones necessariae sunt determinatae, ad distinctionem propositionum contingentium: igitur vel determinatae ut verae vel ut falsae. Si ut verae relucet conclusio: si ut falsae, igitur verum fuit ab aeterno tales propositiones esse falsas. 
\pend

\pstart
  Roboratur: ex duobus contradictoriis in materia naturalis, altera est determinate et actu vera; sed istae propositiones `homo est animal', `homo non est animal', sunt \textnormal{|}\ledsidenote{BNC 119ra}   contradictoriae. Etiam si homo non existat: igitur altera earum est determinate et actu vera, etiam si homo non existat ab aeterno. 
\pend

\pstart
  Obiectio 1. Ex eo quod res rest vel non est propositio dicitur vera vel falsa, sed res neque secundum essentiam nec secundum existentiam fuerunt ab aeterno. Igitur propositiones de illis non fuerunt ab aeterno nec verae nec falsae. 
\pend

\pstart
  Respondeo distinguendo maiorem. Ex eo quod res est vel non est positive vel negative, concedo; positive praecise, nego. Et similiter distinguo minorem. Res ab aeterno non fuerunt positive, concedo; negative nego. \textsc{Aristoteles}\index[persons]{Aristoteles} intelligitur de una vel altera identitate seu veritate, non praecise de positiva. Talis enim veritas \supplied{ne}gativa sufficiens fuit, ut Deus ab aeterno cognosceret conexionem praedicatorum, posito quod ipse ab aeterno non produxit praedicatum et subiectum. 
\pend

\pstart
  Obiectio 2. Si veritas harum propositionum esset ab aeterno actu maxime quia connexio inter praedicatum et subiectum esset actu ab aeterno, sed ab aeterno non datur talis connexio, cum nondum sunt homo et animal: igitur ab aeterno dari nequit veritas aeterna. 
\pend

\pstart
  \emph{Respondet distinctionem maiorem [?]}: maxime quia connexio inter praedicatum et subiectum esset actu ab aeterno positive nego: negative \emph{conditio [?]}. Et similiter distinguenda est minor. Non enim requiretur identitas positiva, sed sufficit negativa ad fundandam veritatem formalem, ut dictum est. 
\pend

\pstart
  Obiectio 3. Verbum in propositione debet esse indicativi modi et temporis praesentis; sed quod significat cum tempore praesenti non potest significare veritatem ab aeterno: igitur propositiones non habent veritatem actu ab aeterno 
\pend

\pstart
  \textnormal{|}\ledsidenote{BNC 119rb} Respondeo distinguendo maiorem. Verbum si absolvatur a tempore debet esse temporis praesentis, nego; si non absolvatur, concedo. In propositionibus necessariis verbum est absolvitur a tempore, hoc est abstrahit ab omni differentia temporis et significat solus conexione praedicatorum ut explicuimus in summulis. Quod si urgeas, verbum in his propositionibus vere significare aliquam praesentem durationem. Respondeo aliquam durationem praesentem, praesentialitate aeternitatis, concedo;: praesentialitate temporis, nego. 
\pend

\pstart
  Obiectio 4. Haec propositio `homo est vivus' est essentialis et tamen non est aeternae veritatis: igitur non omnis propositio necessaria est aeternae veritatis. Probatur minor primo. Ad vitam hominis requiritur unio animae ac corporis, sed talis unio non fuit ab aeterno: igitur nec vita. Secundo. Homo aliquando est mortuus: igitur non semper est vivus. Igitur talis propositio non est aeternae veritatis. 
\pend

\pstart
  Tertio haec propositio, `Christus est homo', est essentialis, sed non habet veritatem aeternam: igitur non omnis propositio necessaria, etc. Probatur minor. \emph{Christus [?]} in triduo mortis non fuit homo: ut docent Scholastici cum \edtext{ \emph{\textsc{Magistro}\index[persons]{Petrus Lombardus}  [?]} in 3, distinctione 22 }{\lemma{}\Afootnote[nosep]{}} et Theologi cum suo Principe \edtext{ Divo \name{\textsc{Thoma}\index[persons]{Thomas Aquinas}} 3 parte, quaesitone 50, articulo 4 }{\lemma{}\Afootnote[nosep]{}}; igitur in triduo haec propositio erat falsa. Igitur non est aeternae veritatis. 
\pend

\pstart
  Respondeo negando inorem. Ad primam probationem distinguo minorem. Talis unio non fuit ab aeterno positive, concedo; negative, nego. Ad veritatem illius propositionis non requiritur conexio positiva inter animam et corpus, sed sat est conexio negativa inter subiectum et praedicatum. Ad secundam distinguo antecedens. Homo aliquando est mortuus proprie, nego; abusive et aequivoce, concedo. Quando moritur homo, desinit esse homo: igitur numquam homo est mortuus. 
\pend

\pstart
  \textnormal{|}\ledsidenote{BNC 119va}   Ad tertium negatur minor. Ad probationem negatur antecedens in rigore loquendo. In triduo mortis non fuit \textsc{Christus}\index[persons]{Iesus Nazarenus}, qua \textsc{Christus}\index[persons]{Iesus Nazarenus} significat hunc hominem, quia eius divina anima dissoluta fuit ab eius sacrosancto corpore. Quamquam verbum divinum permanserit unitum animae et corpori seorsum. Unde haec propositio `Christus est homo' semper est vera. Theologi enim non loquuntur in hoc logico rigore. Articulus 3. 
\pend

\pstart
  Ante resolutionem difficultatis necesse est aliam difficultatem resolvere, scilicet an amittatur universale posita praedicatione de multis et inclusione in multis? Affirmat \edtext{ \name{\textsc{Scotus}\index[persons]{Ioannes Duns Scotus}} quaestione 9 \worktitle{in Porphyrio} }{\lemma{}\Afootnote[nosep]{}} ut testantur \edtext{ \name{\textsc{Conimbricensibus}\index[persons]{}}  quaestione 6 \worktitle{Universalius}, numero 2 }{\lemma{}\Afootnote[nosep]{}}. 
\pend

\pstart
  Dico tamen posita praedicatione et inclusione manere universale metaphysicum, non logicum. Probatur. Proprium est naturae universalis esse praedicabilem de multis: igitur ipsa praedicatio de multis non repugnat universali. Igitur quando praedicatur ipsa praedicatio non tollit universalitatem. 
\pend

\pstart
  Obiectio 1. Non minus quantitas decem palmorum est divisibilis in decem palmos. Qualia universalis est contrahibilis et praedicabilis, sed posita divisione in quantitate non manet ultra divisibilis: igitur posita contractione et praedicatione non manet natura contrahibilis et praedicabilis. Igitur nec universalis manet. 
\pend

\pstart
  Respondeo aliud esse quantitatem actu divisam in facto esse non posse amplius dividi, \textnormal{|}\ledsidenote{BNC 119vb} et aliud ipsum actum dividendi divisibilitatem tollere. Primus est verum, falsum secundum. Divisio enim super divisibilitatem cadit illi non repugnat, licet facta divisione divisibilis non maneat. Sicut corruptio non repugnat corruptibili, sed potius convenit et tamen posita corruptione in facto esse, iam non corruptibile, sed corruptum dicitur. 
\pend

\pstart
  Sic praedicatio convenit praedicabili et divisio divisibili, sed posita praedicatione et divisione. Iam natura non est praedicabilis et divisibilis, sed praedicata et divisa. Igitur in hac praedicatione `Petrus est homo', hic homo manet universalis metaphysice, quia non supponit singulariter, sed communiter alias sensus foret Petrus est hic homo. Igitur numquam de Petro praedicaretur aliquod commune. 
\pend

\pstart
  Obiectio 2. Remota indifferentia tota universalitas ruit, sed quando natura praedicatur removetur indifferentia: igitur et universalitas. Probatur minos: implicat natura applicata et indifferens; sed per praedicationem natura applicatur isti individuo, de quo praedicatur. Igitur non manet indifferens. 
\pend

\pstart
  Respondeo distinguendo minorem. Removetur indifferentia intrinsece, nego; extrinsece, concedo. Eadem solutione proceditur ad probationes. Quando natura praedicatur intrinsece est universalis et indifferens; extrinsece autem applicatur individuo determinato. 
\pend

\pstart
  Obiectio 3. Natura non habet universalitatem nisi in suppositione simplici. Nam quando dico `homo est species', hic homo supponit simpliciter; sed quando actu natura praedicatur non habet suppositionem simplicem, sed personalem. Igitur quando praedicatur non est universalis. 
\pend

\pstart
  \textnormal{|}\ledsidenote{BNC 120ra}   Respondeo negando antecedens. Etiam quando terminus suppositione personali retinet universalitatem. Nam quamdiu universalitas non tollitur universalitas metaphysica manet. Suppono simplex tunc datur quando praedicatur de re aliqua secunda intentio, quae non potest inferioribus communicari. Cum enim quando praedicatur natura fiat applicatio naturae repraesentatae et non quoad secundam intentionem universalitatis, ideo non fit suppositione simplici. 
\pend

\pstart
  Sit conclusio: nam esse in multis non consistit in ipsa relatione universalis ad multa, sed esse in multis potentialiter est ipsa aptitudo; actualiter vero est identificatio et contractio ad illa. 
\pend

\pstart
  Probatur ratione. Potentia ad essendum in multis antecedit ipsum esse in multis, sicut potentia actum. Sed ipsa potentia est aptitudo universalis: igitur aptitudo praecedit esse in multis. Sed relatio est respectus ipsius aptitudinis: igitur si aptitudo antecedit esse in multis etiam relatio. Igitur esse in multis non est ipsa relatio universalis. 
\pend

\pstart
  Praeterea: sicut universale est potens et aptum esse in multis, sic dicitur esse in multis: igitur si erat aptitudo esse in multis per \secluded{per} identificationem et essentialem multiplicationem vel accidentalem, ipsa identificatio multiplicata est esse in multis. Igitur eo modo est explicandum esse in multis ac posse esse in multis. 
\pend

\pstart
  Dubitas quomodo differt contractio seu inclusio, seu esse in multis, a praedicati de multis? Dico esse in multis esse identificationem cum multis, quae quidem fit per simplicem comparationem qua concipitur natura ut conveniens inferiori et cum illo identificari. \textnormal{|}\ledsidenote{BNC 120rb} at praedicatio fit per applicationem unius ad alterum quod fit operatione conponente et praedicante. Unde praedicabilitas est passio universalis, ex eo enim quod universale est in multis per identificationem sequitur praedicari de illis. Articulus 4, etc. 
\pend

        \addcontentsline{toc}{section}{Articulus 4. Quid et quotuplex sit praedicatio?}
        \pstart
        \eledsection*{Articulus 4. Quid et quotuplex sit praedicatio?}
        \pend
      
\pstart
  Suppono 1. Praedicationem pertinere ad secundam intellectus operationem. Est enim attributio unius ad alterum negando vel affirmando. Ita \edtext{ Divus \name{\textsc{Thomas}\index[persons]{Thomas Aquinas}} 4 \worktitle{De ente et essentia} }{\lemma{}\Afootnote[nosep]{}}. Ubi explicatur ratio formalis praedicationis quae est ipsa attributio seu coniunctio extremorum per modum attributionis. Nam coniunctio relatio est. Fundamentum autem huius relationis atributionisve est identitas seu convenientia extremorum si praedicatio est affirmativa vel disconvenientia si est negativa. Sic etiam \edtext{ 1 parte, quaestione 13, articulo 12 et 1 \worktitle{Perihermeneias} lectione 3 }{\lemma{}\Afootnote[nosep]{}}. 
\pend

\pstart
  Suppono 2. Verificationem propositionis ita fundari in identitate et convenientia extremorum, quod non solum illa est attendenda ex parte rei significatae, sed etiam ex parte modi significandi. Modus significandi relucet in hoc quod significetur aliquod in abstracto vel in concreto per modum totius vel partis \edtext{ 1 parte, quaestione 39, articulo 4 et 5 }{\lemma{}\Afootnote[nosep]{}}. 
\pend

\pstart
  Colligitur ex dictis quod abstractum respectu concreti dicitur tale abstractione formali non totali. Abstractio formalis est qua forma abstrahit a materia, actus a potentia; essentiale et proprium ab extraneo et alieno. Totalis est qua \textnormal{|}\ledsidenote{BNC 120va}   superius abstrahit ab inferiori et commune ab individuis, quia se habet ut totum respectu inferioris includendo illud virtualiter et implicite. Abstractum enim non se habet ut totum sed ut pars et totum forma respectu concreti. 
\pend

\pstart
  Suppono 3. Praedicatum superius esse totum respectu inferioris. Ita explicat \edtext{ \name{\textsc{Divus Thomas}\index[persons]{Thomas Aquinas}} loco citato, et 3 \worktitle{De ente et essentia} }{\lemma{}\Afootnote[nosep]{}}. Genus enim habet rationem partis et ratione totius, nam aliud explicat, et aliud implicat. \enquote{similiter specificum et differentiale.} Verbi gratia animal explicat naturam sensitivam, id est exprimit id, quo acti constituit sensitivum; includit vero omne illud a quo ratio sensitivi potest provenire. Unde ratione expliciti genus ut animal solum est pars et non totum; ratione vero impliciti cum significet omnem formam, a qua ratio animalis provenit ideo secundum illum implicitum est totum 
\pend

\pstart
  Unde genus et universale non debet abstrahere formam abstractio est formali, sed totali: igitur genus posita abstractione adhuc implicite consernit ulteriorem perfectionem inventam in tali forma. Nota tamen esse per modum totius et partis no esse, esse totum et partem. Nam pars etiam potes se haberi per modum totius respectu alterius partis, in qua includitur, sicut animalitas respectu humanitatis: significari enim per modum totius consistit in eo, quod significetur aliquid respectu alterius ut includens illud: et per modum partis quando aliquid significat alterum ut excludens illud, licet in re concingatur cum illo. 
\pend

\pstart
  Suppono 4. Propositionem generali divisione dividi in identicam, disparatam et \textnormal{|}\ledsidenote{BNC 120vb} mediam. Praedicatio identica alia materialiter, alia formaliter. Propositio formaliter identica est quando idem praedicatur de se ipso vi significationis, ut homo est homo. Identica materialiter: quando idem praedicatur de se ipso posita eadem realitate et entitate, significatibus diversis. Verbi gratia actio est passio. Et similiter quando duo accidentia praedicatur ratione convenientiae in eodem subiecto ut album est dulce; et unum attributum de alio in Deo, ut misericordia est Iustitia. Sunt enim identicae hae propositiones materialiter. 
\pend

\pstart
  Praedicatio disparata est illa quae constat ex terminis disparate se habentibus qui non habent habitudinem convenientiae inter se. Nota tamen aliud esse propositionem esse disparatam vel identicam;: et aliud praedicationem esse disparatam vel identicam. Propositio respicit veritatem et falsitatem, praedicatio vero habitudinem terminorum. Unde potest esse praedicatio disparata et propositio vera; et inter terminos identicos potest esse propositio falsa. Verbi gratia `homo non est lapis' est propositio vera et praedicatio disparata. `Homo non est homo' est propositio falsa et praedicatio identica. 
\pend

\pstart
  Praedicatio media est in qua subiectum et praedicatum habent sufficientem conexionem, sed non omnimodam identitatem. Haec subdividitur in directam et indirectam. Directa est in qua id quod habet se per modum formae est praedicatum. Indirecta in qua quod habet se per modum subiecti est praedicatum. Unde hae propositiones `homo est animal', `homo est rationalis', `homo est risibilis' sunt directae et earum convertentes indirectae. 
\pend

\pstart
  Nota tamen nomine formae et subiecti non solum nos intelligere materiam et formam physicas, sed omne actuans et actuabile vel superius et \textnormal{|}\ledsidenote{BNC 121ra}   inferius. Forma dicitur superius respectu inferioris, non quia gradus superior sit actualior, sed quia contrahibilis. Si vero sunt praedicata aequalia ut homo et rationale quale est forma. Tunc quod acuat se habet ut forma et quod actuatur ut subiectum. 
\pend

\pstart
  Praedicatio directa item dividitur in essentialem et accidentalem. Essentialis est quando praedicatum est de intrinseca ratione subiecti. Accidentalis quando est extrinseca illam. Nota tamen discrimen inter identitatem et convexionem extremorum. Identitas pertinet ad coniunctionem seu convenientiam et adaequationem, quae ad omnem praedicationem requiritur. Connconnexio ad habitudinem qua extrema ordinantur vel essentialiter et intrinsece vel accidentaliter et extrinsece. 
\pend

\pstart
  Quaeris quando gradus superior individuatus praedicatur, verbi gratia `homo est hoc animal' est praedicatio direcat? Dico ut talis propositio sit directa debet poni ex parte subiecti genus individuatum, verbi gratia hoc animal est homo. Hoc ipso namque quod genus individuatur se habet ut recipiens ipsam rationem \secluded{rationem} speciei. Nam supponit pro animali, quod est individuum hominis: igitur respicit hominem ut superius. Igitur ut praedicatum formale. Unde est directa `hoc animal est homo'; indirecta `homo est hoc animal'. 
\pend

\pstart
  Cum enim praedicationes communiter fiant in concretis et abstractis aliqua de eis est notandum quare. Suppono 5. Concretum et abstractum eamdem rem significate. Ratio clamitat: haec praedicatio `homo est albus' est accidentalis et contingens: igitur res significata per praedicatum est quod potest abesse et adesse subiecto. Sed hoc est sola albedo et non \textnormal{|}\ledsidenote{BNC 121rb} subiectum: igitur sola albedo et non subiectum est res significata. 
\pend

\pstart
  Differunt tamen ex modo significandi. Abstractum enim significat formam per modum per se stantis, id est ad modum substantiae sine ordine ad subiectum, sed per se ipsam, sicut substantia. At concretum significat eamdem formam per modum alteri adiacentis cum connotatione ad subiectum. Quapropter appellatur concretum connotativuum, nam connotat subiectum in obliquo. 
\pend

\pstart
  Ex hoc infert \edtext{ \name{\textsc{Angelicus Doctor}\index[persons]{Thomas Aquinas}} \worktitle{Opusculo 69}, lectione 2 }{\lemma{}\Afootnote[nosep]{}} triplex discrimen inter concretum et abstractum. Primum abstractum est tale per se ipsum ut albedo; et concretum est tale pro abstractis; ut albus est tale per albedinem. Secundum abstractum significat formam simplicem, puram et impermixtam. Albedo enim solum albedinem significat sine permixtione alterius qualitatis. Concretum significat formam sine praecissione aliarum qualitatum, album significat albedinem in subiecto in quo etiam reperitur quantitas, calor, etc. Hinc oritur ratio cur non praedicentur abstracta diversarum specierum ut albedo est dulcedo; si tamen praedicantur eorum concreta ut `album est dulce in lacte'. 
\pend

\pstart
  Tertium abstractum se habet ut pars concretum ut totum. Ita \edtext{ 1 parte, quaestione 3, articulo 3 et 3 parte, quaestione 2 }{\lemma{}\Afootnote[nosep]{}}. Unde oritur radix cur abstractum non praedicetur de concreto, ut postea videbimus. Ex dictis duo colliges. Primum, sola praedicatio directa est propria praedicatio. Quare de ea solum sumus acturi. Etiam ex ipse praedicatione in qua definitio praedicatur de definitio, ratione est quia cum significetur nomine complexo deficit a simplicite praedicabilium. Secundum, cum duo sunt genera nominum, scilicet nomina \textnormal{|}\ledsidenote{BNC 121va}   primae intentionis et secundae intentionis. Praedicationes utriusque generis sunt examinandae. Articulus 5. Quomodo, etc. 
\pend

        \addcontentsline{toc}{section}{Articulus 5. Quomodo praedicationes fiant in nominibus primae intentionis?}
        \pstart
        \eledsection*{Articulus 5. Quomodo praedicationes fiant in nominibus primae intentionis?}
        \pend
      
\pstart
  Nomina primae intentionis alia sunt concreta, alia abstracta. Unde possunt quatuor fieri combinationes. Quare primo dubito, an concreta praedicentur de concretis? Secundo, an abstracta de abstractis? Tertio, an concreta de abstractis? Quarto, an abstracta de concretis? 
\pend

\pstart
  Concreta enim et abstracta alia sunt substantialia ut homo, animal, humanitas, animalitas; alia accidentalia ut album, coloratum, albedo, color: de omnibus praecedimus. Relinquimus igitur propositiones negativas ut pertinentes ad praedicabilia. Unde solum restat de affirmativis propositionibus. 
\pend

        \addcontentsline{toc}{section}{Paragraphus 1. An concreta praedicentur de concretis?}
        \pstart
        \eledsection*{Paragraphus 1. An concreta praedicentur de concretis?}
        \pend
      
\pstart
  In primis noto quod concretum duo significat alterum explicite et alterum implicite. Verbi gratia animal explicite significat formam nempe animalitate; implicite vero omne id quod habet talem animalitatem. Similiter album explicat formam, scilicet albedinem et implicat omne subiectum habens albedinem. 
\pend

\pstart
  Sit conclusio: praedicationes in quibus concreta tam substantalia quam accidentalia praedicantur ad invicem sunt verae, erbi gratia homo est animal, album est coloratum. Intellige conclusionem etiam quando concreto substantiale \textnormal{|}\ledsidenote{BNC 121vb} praedicatur concretum accidentale ut `homo est albus', et non econtra nam tunc est indirecta de qua non loquimur. Est communis inter Auctores et continetur \edtext{ capitulo 3 \worktitle{De ente et essentia} }{\lemma{}\Afootnote[nosep]{}} et \edtext{ \name{\textsc{Caietanus}\index[persons]{Thomas de Vio Caietanu}} ibidem }{\lemma{}\Afootnote[nosep]{}}. 
\pend

\pstart
  Probatur ratione. Ut verificetur praedicatio satis est subiectum contineri implicite in praedicato, sed quando praedicantur de concretis concreta substantia continentur implicite in praedicatis: igitur verificantur praedicationes. Probatur maior. In his praedicationibus `homo est animal', `Petrus est realis', `Petrus est homo', `Petrus est risibilis', `homo est animalibus' praedicata involvunt substantia aut supposita ut videnti patet. Igitur quando praedicantur concreta de concretis praedicata continent involute substantia. 
\pend

\pstart
  Obiectis. Quoties unum non continet aliud non potes praedicari de illo, sed unum concretum non continet aliud: igitur non potest praedicari de illo. Probatur minor: homo addit supra animal rationalitate: igitur animal non continet hominem. Similiter album addit super coloratum suam differentiam: igitur coloratum non continet album. Confirmatur. Pars non potest praedicari de toto, sed animal est hominnis pars et coloratum albi. Igitur nec animal de homine, nec coloratum de albo possunt praedicari. 
\pend

\pstart
  Respondeo distinguendo minorem. Ad probationem concedo antecedens et distinguo consequens. Animal qua pars actualis non continet hominem, concedo; qua totum potentiale, nego. Ad confirmationem distinguo maiorem. Animal qua pars acutalis est hominis pars, concedo; qua totum potentiale, nego. Solutio constabit suo loco nempe de genere. Sic praeceditur de concretis et abstractis accidentalibus. Verbi gratia igitur coloratum non continet album, distinguo coloratum, qua totum potentiale, nego; qua pars actualis, concedo. Paragraphus 2. 
\pend

        \addcontentsline{toc}{section}{Paragraphus 2. An abstracta praedicentur de abstractis?}
        \pstart
        \eledsection*{Paragraphus 2. An abstracta praedicentur de abstractis?}
        \pend
      
\pstart
  Suppono 1. Abstracta qua talia non dicere ordinem ad sua inferiora, sed ab subiectum et suppositum a quibus abstracta sunt. Unde dicitur color alicuius colorati est color, non alicuius albedinis; nec dicitur animalitas alicuius humanitatis est animalitas, sed alicuius animalis. Unde sequitur abstractum significari ut partem seu quo aliquid est tale. Nam id quod abstrahitur abstractione formali et praecissiva comparatur ut actus et constitutivum. Vide numero 379. 
\pend

\pstart
  Suppono 2. Abstractum ratione abstractionis formalis de nullo posse praedicari: igitur ut praedicetur indiget abstractione universalive totali. Eminet ratio: ut aliquid praedicetur de alio debet se habere ut totum, ut contineat subiectum. Sed abstractum abstractione formali non se habet ut totum, sed ut pars: igitur vi talis abstractionis non potest de aliquo praedicari. Ergo ut praedicetur debet intervenire abstractio totalis. 
\pend

\pstart
  Suppono 3. Discrimen inter compositum accidentale et substantiale. Compositum substantiale est ens per se et simpliciter unum. Igitur eius partes sunt simpliciter partes et secundum quod solum tota. Compositum accidnentale est ens per accidens et unum secundum quid: igitur eius partes sunt secundum quid parte et simpliciter tota. Hoc posito nota hanc illationem. Pars compositi substantialis est simpliciter pars et secundum quid totum, sed abstractum substantiale est pars compositi substantialis. Igitur abstractum substantiale est simpliciter pars et secundum quid totum. Nota alteram priori similem. Pars compositi \textnormal{|}\ledsidenote{BNC 122rb} accidentalis est simpliciter totum et secundum quod pars; sed abstractum accidentale significatur ut pars illius: igitur ut pars secundum quid et ut totum simpliciter. 
\pend

\pstart
  Quare concreta accidentalia reductive ad sua abstracta dicuntur universali et praedicabilia et sic ponuntur in praedicamento. Et contra vero abstracta substantialia sunt in praedicamento reductive ad sua concreta. Ita docet \edtext{ \name{\textsc{Angelicus Doctor}\index[persons]{Thomas Aquinas}} \worktitle{De ente et essentia} capitulo 7 }{\lemma{}\Afootnote[nosep]{}}. 
\pend

\pstart
  Animadverte hoc ratiocinium. Non minus abstracta accidentalia sunt secundum quod partes, quam abstracta substantialia sunt simpliciter partes, sed esse secundum quod partes non obest quominus dicantur abstracto \emph{accipere [?]} totum respectu suorum inferiorum. Igitur esse simpliciter partes non obest quominus dicantur abstracto substantia totum secundum quod respectu suorum inferiorum ut de illis praedicentur. 
\pend

\pstart
  Suppono 4. Quod in hac comparatione abstractorum ad sua inferiora, sicut concretum substantiale consideratur implicite prout continet inferiora sicque est totum et praedicatur de illis vel praescindit ab eis et excludit illa sicque est pars eorum et non praedicatur de eis. Ita similiter concretum substantiale qua comparatur ad sua inferiora consideratur vel implicite, et sic competit ei ratio totum secundum quid vel praescindit ab eis et sic est pars eorum simpliciter. Hoc secundo modo repugnat quod praedicentur non secundo modo primo. His positis. 
\pend

\pstart
  Prima sententia negat. Ita \edtext{ \name{\textsc{Iohannes Duns Scotus}\index[persons]{Ioannes Duns Scotus}} in 1, distinctione 2, quaestione 7 et distinctione 5, quaestione 1 }{\lemma{}\Afootnote[nosep]{}}; \edtext{ \name{\textsc{Niphus}\index[persons]{Augustinus Niphus}}  3 \worktitle{Metaphysicae} quaestione 2 }{\lemma{}\Afootnote[nosep]{}}; \edtext{ \name{\textsc{Canarens}\index[persons]{Bartholomaeus Torres}}  1 parte, quaestio 28,  articulo 2, disputatione 3 }{\lemma{}\Afootnote[nosep]{}}; \edtext{ \name{\textsc{Rubius}\index[persons]{Antonius Ruvius Rodensis}} \worktitle{Tractatus de praedicata} numero 14 }{\lemma{}\Afootnote[nosep]{}} \textnormal{|}\ledsidenote{BNC 122vb}   Secunda sententia affirmat. Ita omnes Thomistae. Sed ut melius intelligatur variis conclusionibus praecedam. 
\pend

\pstart
  Sit prima conclusio: abstracta abstractione formali et praecissiva non praedicantur de abstractis. Ita  \edtext{ \name{\textsc{Soto}\index[persons]{Dominicus de Soto}} quaestione 3 \worktitle{Universalium} }{\lemma{}\Afootnote[nosep]{}}; \edtext{ \name{\textsc{Masius}\index[persons]{Didacus Masius}} ibide sectione 4, quaestione 6 }{\lemma{}\Afootnote[nosep]{}}; \edtext{ \name{\textsc{Sanchez}\index[persons]{Ioannes Sanchez Sedeno}} libro 3, quaestione 13 }{\lemma{}\Afootnote[nosep]{}}; \edtext{ \name{\textsc{Oña}\index[persons]{Petrus de Oña}}  quaestione 3, articulo 3 }{\lemma{}\Afootnote[nosep]{}}; \edtext{ \name{\textsc{Suarez}\index[persons]{Franciscus Suarez}} \worktitle{Disputatione 6 metaphysicae}, sectione 10 }{\lemma{}\Afootnote[nosep]{}}; \edtext{ \name{\textsc{Complutenses}\index[persons]{}} disputatione 2 quaestione 4 }{\lemma{}\Afootnote[nosep]{}}; \edtext{ \name{\textsc{Iohannes a Sancto Thoma}\index[persons]{Ioannes a Sancto Thoma}} quaestione 5, articulo 4 et 5 }{\lemma{}\Afootnote[nosep]{}}. 
\pend

\pstart
  Probatur ratione. Nulla pars praedicatur de toto, sed abstractum abstractione formali est pars: igitur prout sic non praedicatur. Ut verificetur praedicatio, praedicatum debet esse in subiecto et includere subiectum, sed abstractum abstractione formali no continet subiectum: igitur non valet praedicari. Probatur minor. Ut contineret subiectum deberet esse totum potentiale, salter secundum quid. Sed abstractum abstractione formali non est totum nec secundum quid: igitur abstractum hac abstractione non potest praedicari. 
\pend

\pstart
  Sit secunda conclusio: abstracta abstractione totali vere possunt praedicari de abstractis. Ita Auctores citati. Probatur ratione. Ad veram praedicationem requiritur praedicatum continere subiectum aliquomodo confuse et habere rationem totius. Sed non repugnat abstracta aliquomodo continere alia abstracta et respectu eorum habere rationem totius: igitur non repugnat praedicari. Probatur minore. \del{Sed non} abstracta dicuntur talia abstractione formali, sed cum abstractione formali non repugnat abstractio \del{abstracta} universalis vi cuius se habeant ut totum et contineant inferiora implicite. Igitur non repugnat abstracta aliquomodo continere inferiora et se habere ut totum. 
\pend

\pstart
  Roboratur: designata animalitate verum \textnormal{|}\ledsidenote{BNC 122vb} est dicere haec animalitas est animalitas. Igitur verum etiam es dicere `humanitas est animalitas'. Antecedens patet in his et similibus praedicationibus, haec humanitas est humanitas, humanitas \textsc{Christi}\index[persons]{Iesus Nazarenus} est humanitas. Probatur consequentia. Sicut in hac praedicatione humanitas est animalitas, subiectum includit aliquid quod non est de essentia praedicati (scilicet principium specificum) sic etiam in hac haec animalitas est animalitas subiectum includit aliquod, quod non est de essentia praedicati (scilicet principium individuationis). Igitur si de ratione animalitatis esset exclusio quorumcunque unde principio esset falsa. Eadem ratione, secunda erit falsa: sed potest animalitas praedicari quamvis abstractum substantiale et habere rationem totius respectu huius et illius animalitatis. Igitur potest etiam de humanitate et equinitate. 
\pend

\pstart
  Sit tertia conclusio: tales praedicationes non per se et simpliciter, sed reductive et secundum quod pertineret ad praedicabilia. Ita Auctores citati et probatur. Ut aliquid per se et simpliciter pertineat ad praedicabilia debet esse ens per se et completum, sed abstracta utraque abstractione non sunt simpliciter et complete ens. Igitur simpliciter et per se non possunt pertinere ad praedicabilia. Probatur minor. Id quod simpliciter est pars, non est simpliciter totum; sed abstractum abstractione formali est simpliciter pars: igitur tali abstractione non est simpliciter totum. De altera. Quod secundum quid tantum est totum non est simpliciter totum, sed abstractum abstractione totali secundum quid tantum est totum: igitur hac abstractione non est simpliciter totum. Ergo eo modo quo est totum pertinet ad praedicabilia, sed solum secundum quid est totum: igitur sic pertinet. 
\pend

\pstart
  \textnormal{|}\ledsidenote{BNC 123ra}   Antequam argumenta proponam resolvenda suppono ut certum in accidentibus praedicata superiora praedicari de inferioribus in abstracto, ut albedo est color, est qualitas, etc. Ratio es quia in accidentibus abstracta significant totam essentiam et rationem intrinsecam. Quapropter accidnetia in concreto non ponuntur in praedicamento ut album, musicum, sed solum in abstracto ut albedo, musica. \edtext{ \worktitle{Opusculo} 42, capitulo 19 et quaestione 3 \worktitle{De Veritate} articulo 8, ad 2 }{\lemma{}\Afootnote[nosep]{}}. 
\pend

\pstart
  Etiam suppono praedicationes ire falsas, in quibus abstractum praedicatum non est de essentia abstracti subiecti: ut humanitas est risibilitas. Ratio est clara: praedicatio in qua praedicatum non est essentia subiecti non est in quid, sed in quale: igitur est accidentalis. Nunc sic: ut praedicatio accidentalis sit vera, debet esse praedicatum adiective seu per modum alteri adiacentis, sed abstractum non potest esse adiective et per modum alteri adiacentis: igitur nec praedicatum in propositione accidentali. Igitur si sic praedicatur propositio est falsa. Eadem ratio militat contra propositionem in quae differentia in abstracto praedicatur de specie in abstracto; nam licet differentias sit de essentia speciei, non tamen praedicatur in quid, sed in quale quid: igitur debet significari adiective. 
\pend

\pstart
  Idem dicendum est quando abstractum generis praedicatur de abstracto differentiae et abstractum differentiae superioris de abstracto inferioris; ut rationalitas est animalitas, rationalitas est sensibilitas. Unde difficultas fuit soluta de abstracti substantialibus, quorum unum est de essentia alterius et praedicatur in quid: an scilicet sicut sunt verae haec praedicationes `homo est animal', `animal est corpus'; ita `hae humanitas est animalitas', `animalitas est \textnormal{|}\ledsidenote{BNC 123rb} corporeitas'. Nec procedit difficultas in sensu identico sed in sensu formali. Supponimus enim identice esse veras, quoties est idem habens. 
\pend

\pstart
  Obiectio 1. Ex \edtext{ Divo \name{\textsc{Thoma}\index[persons]{Thomas Aquinas}} 4 \worktitle{Contra Gentiles} capitulo 81 et \worktitle{De ente et essentia} capitulo 3 et 4 et \worktitle{Opusculo} 69, capitulo 2 \enquote{abstracta significantur ut partes: igitur unum non praedicatur de alio} }{\lemma{}\Afootnote[nosep]{}} Patet consequentia. Si praedicarentur iam animalitas esset genus et humanitas species, nam haberent rationem totius. Confirmatur ex  \edtext{ \name{\textsc{Avicenna}\index[persons]{Avicenna}} docente: \enquote{equinitas tantum est equinitas: igitur non est animalitas} }{\lemma{}\Afootnote[nosep]{}} 
\pend

\pstart
  Respondeo distinguendo antecedens. Abstracta significantur ut partes abstractione formali, concedo; abstracta abstractione totali, nego. Ad confirmatione distinguo essent genus et species directe, nego; reductive, concedo. Quamvis abstracta abstractione formali significentur ut partes cum praescisione suppositi. Haec tamen abstractio non obest quominus intelligantur abstracta abstractio: abstractione totali et prout sic habeant respectu inferiorum rationem totius ut explicui numero 400 et 402 et 403 et 404. Ad confirmationem dico sensu \textsc{Avicennae}\index[persons]{Avicenna} hunc esse: equinitas tantum est equinitas, id est de se nec est singularis nec universalis, nec subiscibilis, nec praedicabilis, non tamen negavit praedicata essentialia. 
\pend

\pstart
  Obiectio 2. Praedicatum in potentia et in confuso debet salter continere id, quod continet actu subiectum, ut propio sit vera. Sed abstractum generis nec potentia continet abstractum speciei: igitur non potest de illo praedicari. Probatur minor. Omne abstractum potius excludit id ab quo abstrahitur: igitur non includit 
\pend

\pstart
  \textnormal{|}\ledsidenote{BNC 123va}   Respondeo concedendo maiorem et distinguendo minorem. Abstractum generis abstractione formali nec potentia continet abstractum differo speciei, concedo; abstractione totali subdistinguo explicite, concedo; implicite nego. Ad probationem distinguo antecedens. Abstractum abstractione formali potius excludit, quam includit, concedo; abstractione totali, nego. Solutio patet ex dictis. 
\pend

\pstart
  Obiectio 3. Humanitas est id est quo homo est homo, sed humanitas est animalitas: igitur animalitas est id quo homo est homo. Consequens est falsum, maior est certa: igitur minor est falsa. 
\pend

\pstart
  Respondeo concedendo maiorem et distinguendo minorem. Humanitas est animalitas abstractione formali, nego; abstractione totali, concedo. Vel aliomodo et sic totam doctrinam comprehendentis per quo notate formas abstractas sumi vel specificative vel reduplicative. Primo modo explicant naturam, id est omnia sua formalia constitutiva sub abstractione. Secundo modo solum explicant et reduplicant officium abstractionis formalis. Unde primo modo praedicantur, non vero secundo. Hoc suppono negatur consequentia. Ratio est quia variatur appellatio. In maiori sumitur humanitas reduplicative. Unde facit hunc sensum: humanitas ut distinguitur ab animalitate est id quo homo est homo, id est ut respicit homine sub ratione constituentis illum, quare licet includat animalitatem, non tamen sub ratione illius costituit. At in minori sumitur specificative, ut constat. 
\pend

\pstart
  Obiectio 4. Sicut ab homine et equo abstrahitur animal, ita ab humanitate et equinitate abstrahitur animalitas; sed animal praedicatur de homine \textnormal{|}\ledsidenote{BNC 123vb} et equo: igitur etiam animalitas. 
\pend

\pstart
  Respondeo distinguendo maiorem. Animalitas abstrahitur ab humanitate et equinitate, abstractione formali, nego; abstractione totali secundum quid, concedo. Et distinguo consequens. Animalitas praedicatur abstractione formali, nego; abstractione totali secundum quid, concedo. Vide solutione et doctrinam numero 339. 
\pend

\pstart
  Obiectio 5. Partes definitionis bene praedicantur de re definita, sed animalitas et rationalitas sunt partes humanitatis: igitur bene praedicantur de illa. 
\pend

\pstart
   Respondeo distinguendo maiorem. Bene praedicantur abstractione formali, nego; nam tunc sunt partes, bene praedicantur abstractione totali, concedo. Nam tunc habent rationem totius et distinguo minorem: sunt partes definientes ut quo, concedo; ut quod, nego. 
\pend

\pstart
  Obiectio 6. Haec est vera praedicatio albedo est color, albedo est qualitas: igitur haec vera praedicatio humanitas est animalitas abstractione formali. 
\pend

\pstart
   Respondeo negando paritatem. Primo propter rationem datum nego 411. Secundo quia quando dico albedo est color est praedicatio concretorum non ultimorum seu metaphysicorum, de quo videndum est in summulis numero 26. Ut autem esset de abstractis deberet dicere albedineitas est coloreitas, de quo iudicandum est ut dictum est. 
\pend

\pstart
  Instabis: non minus albedo abstrahit a subiecto, quam humanitas a supposito; sed cum illa abstractione stat praedicatio: igitur cum hac. Respondeo negando causam ratione data. Sicut albedo abstrahit a subiecto se habet ut concretum non ultimum et similiter color; cumque unum concretum valeat praedicari de alio: ideo valet talis praedicatio: quod non accidit in substantiis. 
\pend

        \addcontentsline{toc}{section}{Paragraphus 3. An concreta de abstractis et abstracta de concretis praedicentur?}
        \pstart
        \eledsection*{Paragraphus 3. An concreta de abstractis et abstracta de concretis praedicentur?}
        \pend
      
\pstart
  Praesuppono abstractum non includere quod concretum homo enim praeter humanitatem includit personalitatem quam excludit humanitas. Album praeter albedinem includit subiectum, quod excludit albedo. Quapropter abstractum dicitur pars ut saepe diximus. 
\pend

\pstart
  Unde sequitur, si aliquid abstractum non se habet ut pars, nec eius concretum includit aliquid, quod non includit abstractum tunc praedicabuntur. Nam tunc habet omnimodam identitatem tamen distinguerentur quo ad modum significandi. 
\pend

\pstart
  Sit prima conclusio: abstracta et concreta transcendentia et divina ad invicem praedicantur. Probatur: qauando abstractum non se habet ut pars, nec concretum includit quod non includat abstractum, talia possunt praedicari ad invicem. Sed in transcendentibus hoc accidit ut constat et similiter in infinitis: igitur verae sunt haec praedicationes `ens est entitas', `veritas est verum', `Deus est deitas', `deitas est Deus'. Igitur solum distinguerentur quoad modum significandi, scilicet deitas est ratio qua Deuis est Deus: quod non habet Deus. Veritas es ratio qua verum sit verum, quod non habet verum. 
\pend

\pstart
  Sit secunda conclusio: abstracta et concreta non transcendentia et non infinita ad invicem non possunt praedicari. Ita Doctor \textsc{Thomas}\index[persons]{Thomas Aquinas} quae sequuntur  \edtext{ \name{\textsc{Capreolus}\index[persons]{Ioannes Capreolus}} in 1 distinctione 4, quaestione 2, conclusione 2 }{\lemma{}\Afootnote[nosep]{}} \edtext{ \name{\textsc{Soto}\index[persons]{Dominicus de Soto}} quaestione 5 }{\lemma{}\Afootnote[nosep]{}} et ceteri quos citant \textsc{Complutenses}\index[persons]{}. 
\pend

\pstart
  \textnormal{|}\ledsidenote{BNC 124rc} Probatur ratione. Inter praedicatum et subiectum debet esse identitas realis; sed quando concretum includit aliquid praeter abstractum non sunt idem realiter, sed distinguuntur reali distinctione includentis et inclusi: igitur quando concretum includit aliquid praeter abstractum non possunt praedicari ad invicem. Sed hoc contingit praedicatis non transcendentibus, nec infinitis: igitur talia non possunt praedicari. Igitur hae sunt falsae praedicationes `homo est humanitas', `humanitas est homo', `album est albedo', `albedo est album'. 
\pend

\pstart
  Confirmatur communi argumento: nulla pars praedicatur de toto, nullum totum praedicatur de parte. Sed quando concretum includit aliquid praeter abstractum, concretum est totum et abstractum est pars: igitur prout sic non praedicantur ad invicem. 
\pend

\pstart
  Obiectio 1. Hae sunt verae praedicationes `Deus est deitas', `deitas est Deus'. Similiter `bonum est bonitas', `bonitas est bonus'. Sed tales sunt concretorum de abstractis et econtra: igitur. Respondeo talia praedicata esse infinita et transcendentia perboneque argumentum nostram primam conclusionem. 
\pend

\pstart
  Obiectio 2. Hae vera praedicatio `quantitas est quanta', sed est praedicatio concreti de abstracto finito et non transcendenti: igitur ruit secunda conclusio. Respondeo hanc praedicationem esse exceptionem unde difficultas procedit hac excepta, sic satisfaciunt omnes Auctores. Ratio est quia quantitas est mensura substantiae et simul mensurabilis per aliam quantitatem ideoque prout est mensurativa substantiae dicitur quantitas et prout est mensurabilis seu mensurata per aliam quantitatem dicitur quanta. Unde respecta diversorum dicitur quantitas et quanta. Sed de hoc specialius in suo loco, scilicet de quantitate. 
\pend

        \addcontentsline{toc}{section}{Articulus 6. Quomodo fiant praedicationes in nominibus secundae intentionis tam respectu sui quam respectu primarum?}
        \pstart
        \eledsection*{Articulus 6. Quomodo fiant praedicationes in nominibus secundae intentionis tam respectu sui quam respectu primarum?}
        \pend
      
\pstart
  Sicut difficultas antecedens quatuor continebat dubia ita haec utraque comparatione. Unde dubitatur 1. An concretum secundae intentionis praedicetur de concreto eiusdem intentionis? Respondeo. Quando una secunda intentio in concreto sumpta non repugnat alteri, haec concreta possunt praedicari. Verbi gratia `genus est universale', `species est praedicabilis'. Non vero si repugnat. Verbi gratia genus est species vel differentia. 
\pend

\pstart
  Dubitatur 2. An concreta de abstractis, et econtra? Respondeo. Negative propter rationes adductas probantes non praedicari de concretis abstracta et econtra in nominibus primae intentionis numero 431, 432. Unde sicut est falsa haec praedicatio `album est albedo', `albedo est albus', ita haec `genus est generetias', `genereitas est genus'. 
\pend

\pstart
  Dubitatur 3. An abstracta de abstractis? Respondeo. Abstracta non ultima possunt praedicari de abstractis non ultimis si sint superiora. Probatur in accidentibus realibus bene praedicatur albedo est color, sed secundae intentiones cum sint accidentia rationis, assimilantur realibus: igitur similiter bene praedicatur genereitas est universalitas. Et sicut non valet de abstractis ultimis, verbi gratia albedineitas est coloreitas, ita similiter non valet in ultimis rationis. 
\pend

\pstart
  Comparatione autem ad primas intentiones dubitatur 4. An concretum secundae intentionis de concreto primae intentiones? Respondeo. Bene posse dummodo \textnormal{|}\ledsidenote{BNC 124vb} non repugnet. Probatur ratione. Accidens reale in concreto praedicatur de concreto substantiae facit veram praedicationem. Verbi gratia `homo est albus', sed sic philosophatur de secundis intentionibus: igitur bene praedicatur `homo est species', `animal est genus'. 
\pend

\pstart
  Dubitatur 5. An abstractum secundae intentionis de concreto primae? Respondeo negative. Abstractum accidentis realis non praedicatur de concreto substantiae, verbi gratia `homo est albedo': igitur nec abstractum secundae intentionis de concreto primae intentionis, verbi gratia `homo est specieitas'. 
\pend

\pstart
  Dubitatur 6. An abstractum secundae intentionis de abstracto primae. Respondeo negative. Falsa est praedicatio in qua abstractum accidentis realis praedicatur de abstracto substantiae, verbi gratia `humanitas est albedo'. Igitur falsa es in qua abstractum secundae intentionis praedicatur de abstracto primae, verbi gratia `animalitas est genereitas'. 
\pend

\pstart
  Dubitatur 7. An concretum secundae intentionis de abstracto primae? Respondeo negative. Probatur. Concretum accidentis realis non praedicatur de abstracto substantiae, verbi gratia `humanitas est album'. Igitur nec concretum secundae intentionis de abstracto primae, verbi gratia `humanitas est species'. 
\pend

\pstart
  In calce quaestionis. Nota 1. Duo requiri ut primae intentiones denominetur a secundis intentionibus. Primum est esse entia completa, secundum esse entia per se. Cumque hae duae conditiones solum inveniantur in concretis substantiae et in abstractis accidentium ut vidimus numero 101. Sequitur sola concreta substantiae et abstracta accidentium proprie et simpliciter denominari a secundis intentionibus. 
\pend

\pstart
  Nota 2. Causam cur accidentia realia praedicata de superiori praedicantur etiam de \textnormal{|}\ledsidenote{BNC 125ra}   \secluded{de} aliquo inferiori et praedicata de inferioribus praedicatur de superiori; et hoc non contingat secundis intentionibus? Satisfacio. Accidentia realia conveniunt rebus absolute; at rationis conveniunt secundum diversorum modum concipiendi. Unde quia Petrus concipitur ut indivisis et modus nego denominatur individus. Cumque homo non sic concipiatur, ideo non denominat cum individuum. Quaestio 7. Utrum etc. 
\pend

        \addcontentsline{toc}{section}{Articulus 7. In quonam differant Universale et commune?}
        \pstart
        \eledsection*{Articulus 7. In quonam differant Universale et commune?}
        \pend
      
\pstart
  Certum est universale non esse idem ac commune. Audito \edtext{ Divum \name{\textsc{Thomam}\index[persons]{Thomas Aquinas}} in 1 \worktitle{ad Annibaldum} distinctione 19, quaestione 4, articulo 2 ad secundum: \enquote{Universale non est idem quod commune differunt enim quia commune de se non determinat an id quid communicatur pluribus sit idem numero vel non, sed universale non determinat quia numquam idem numero est in pluribus.} Et 1 parte, quaestione 30, articulo 4, ad 3 \enquote{in Deo non est particulare et universale genus et species.} Et tamen quaestione 13, articulo 9 inter terminos communes ponit Deum. Item individuum vagum ut aliquos homo est terminus communis, et tamen non est universalis }{\lemma{}\Afootnote[nosep]{}}. 
\pend

\pstart
  Sit conclusio prima: ut aliquid sit commune sufficit quod ex parte modi concipiendi sit comunicabile pluribus quin plura sint existentia. vel possibilia. Quando non requiratur pluralitas individuorum actu existentiae est apta mens  \edtext{ Divi Doctori \name{\textsc{Thomae}\index[persons]{Thomas Aquinas}} 7 \worktitle{Metaphysicae} lectione 13. Unde etiam si sint impossibilia ut postea videbimus }{\lemma{}\Afootnote[nosep]{}}. 
\pend

\pstart
  Alium cognoscendi sufficere ut aliquid sit commune docet \edtext{ \name{\textsc{Angelicus Doctor}\index[persons]{Thomas Aquinas}} 1 parte, questione 83, articulo 9 }{\lemma{}\Afootnote[nosep]{}}. \textnormal{|}\ledsidenote{BNC 125rb} Quare in summo numero dixi terminum communi significare plura divisim, hoc est communicabiliter ad plura vel ratione rei significatae vel ex modo cognoscendi. Unde hoc nomen Deus nomen commune appellavi. 
\pend

\pstart
  Quaeris quod est communi ex parte rei et ex parte modi cognoscendi? Dico quando in rebus datur multiplicatio plurium convenientium similitudinaliter in aliqua ratione communi tunc datur commune ex parte rei significatae. Si autem non datur haec multiplicatio in re, tunc directe non datur fundamentum ad constituendum commune. Bene tamen indirecte, quia non repugnat concipere natura inadaequate et imperfecte, id est non apprehensa singularitate, sed sola natura \emph{instantiorum [?]} in quibus natura multiplicatur. 
\pend

\pstart
  Sit secunda conclusio: ut aliquid sit universale requiritur ut ita abstrahat a pluribus ut etiam possit contrahi ad illa per additionem differentiae. Unde etiam requiretur potentialitas in contrahibili respectu contrahentis. Unde essentia divina quia potest concipi sine singularitate est communis ex parte modi concipiendi. Ast quia non potest concipi cum potentialitate non est contrahibilis et universalis. 
\pend

\pstart
  Obiectio. Etiam actui puro repugnat conceptus communis: igitur. Probatur antecedens. Conceptus repraesentans naturam et non singularitatem est multiplicabilis ex parte modi: igitur est contrahibilis et in potentia ad singularitatem.. 
\pend

\pstart
  Respondeo negandoa ntecedens. Ad probationem nego etiam antecedens. Ut autem natura divina concipiatur sine singularitate non requiritur potentialitas, sed \textnormal{|}\ledsidenote{BNC 125va}   inadaequata cognitio illius, id est sine singularitate; sicut cognoscitur natura sine personis et unum attributum sine alio, non quia cognoscatur in potentia ad illud, sed ut includens et non exprimens illud. At ut concipiatur ut contrahibilis et multiplicabilis requiritur quod natura sit in potentia et hoc repugnat actui puro. 
\pend

\pstart
  Instabis: igitur analoga non sunt universalia. Probatur consequens. Analoga non possunt abstrahere perfecte a suis inferioribus, nam includuntur in suis differentiis, sed hoc obstat universali: igitur, etc. 
\pend

\pstart
  Respondeo distinguendo consequens. Analoga non sunt universalia directe et per se, concedo; reductive, nego. Non enim eodem modo continet analogum differentias, sicut Deus singularitatem. Nam continet differentias imperfecte et inconfuso, Deus autem in actu purissimo, 
\pend

\pstart
  Iterum instabis: iste terminus complexus Petrus et Paulus habent vim termini communis, sed non praescindit ab ipsis individuis: igitur hoc non requiritur ad rationem universalis. 
\pend

\pstart
  Respondeo. Habet vim termini communis et universalis respectu partium ex quibus componitur, concedo; ex omni parte, nego. Quia supponit pro quolibet eorum dicitur terminus universalis distributus, non termini habet omnes conditiones ad universalitatem. Et distinguo maiorem. Non praescindit complexive, nego; non praescindit universaliter, concedo. Quomodo quo totum complexum abstrahit a partibus determinatis et incomplexis. 
\pend

\pstart
  Ultimo instabis: implicat Deum verum esse commune etiam ex modo significandi: igitur. Probatur antecedens. Modus cognoscendi ut sit verus debet conformiori modo essendi, sed \textnormal{|}\ledsidenote{BNC 125va} repugnat Deum verum non esse unum et singularem: igitur. 
\pend

\pstart
  Confirmatur: quia Petrus est intrinsece singularis, non potest concipi ut communis; sed Deus multo magis intrinsece est singularis: igitur. Roboratur: licet possit apprehendi in Deo unum attributum sine alio. Non tamen cum opposito, sicut licet possit apprehendi sine misericordia, non tamen cum crudelitate: igitur licet possit considerari sine singularitate, non tamen cum communitate ad plura. 
\pend

\pstart
  Respondeo distinguendo antecedens. Repugnat Deum verum esse \emph{commune [?]} ex parte cognoscendi adaequate, concedo; cognoscendi inadaequate, nego. Ad probationem distinguo maiorem. Modus cognoscendi conformatur modo essendi quantum ad affirmationem et negationem ut sit verus falsusve, concedo; quantum ad modum repraesentationis, nego. Modus cognoscendi non conformatur quo ad repraesentationem modo essendi in rebus, sed modo quo obiicitur cognitioni. Quare si intellectus cognoscens modo communi, iudicaret Deum esse communem falleretur; at repraesentando et appraehendendo sine singularitate ad instar rei communis non fallitur. Sic \edtext{7 Divus \name{\textsc{Thomas}\index[persons]{Thomas Aquinas}} 1 parte, quaestione 13, articulo 9, ad 2 \enquote{nomina non sequuntur modum essendi, quae est in rebus, sed modum essendi secundum quod est in cognitione nostra. Et tamen secundum rei veritatem est incommunicabile illud nomen} }{\lemma{}\Afootnote[nosep]{}}. Ceterum modus cognoscendi conformatur modo essendi quando cognoscit rem ut est in se. Quare licet Deus repraesentetur modo communi, non affirmatur esse communicabilem. 
\pend

\pstart
  Ad confirmationem distinguo minorem. Deus est multo magis intrinsece singularis ex parte \textnormal{|}\ledsidenote{BNC 126ra}   rei cognitae, concedo; ex parte modi cognoscendi, nego. Petrus enim solam singularitatem naturae significat, ideoque non potest concipi ut communis. Deus non significat directe singularitatem, sed naturam, cui essentialiter competit singularitas et alia attributa, a quibus praescindit ex parte conceptus inadaequati. Quare  \edtext{ Divus \name{\textsc{Thomas}\index[persons]{Thomas Aquinas}} 1 parte, quaestione 13, articulo 2 ait: \enquote{Deus est nome\supplied{n} appellativum et non proprium} et articulo 8 \enquote{Deus est nomen naturae} }{\lemma{}\Afootnote[nosep]{}}. 
\pend

\pstart
  Ad alteram probationem nego consequentiam. Communitas ex parte modi cognoscendi non opponitur singularitati. Ex parte modi cognoscendi non potest in Deo sequi crudelitas. Nam si apprehenditur in Deo, debet ipsi attribui talis forma, quod semper est falsum. Articulus 8. 
\pend

        \addcontentsline{toc}{section}{Articulus 8. An universalia seu praedicabilia sint tantum quinque?}
        \pstart
        \eledsection*{Articulus 8. An universalia seu praedicabilia sint tantum quinque?}
        \pend
      
\pstart
  Sicut \textsc{Aristoteles}\index[persons]{Aristoteles} omnes modos essenti ad certas coordinationes revocavit, quas `cathegorias' vocavit seu `praedicamenta'; sic Porphyrius omnes praedicandi modos reducit ad certa capita, quae `universalia' seu `praedicabilia' vocantur. Unde duo inquiret quo, scilicet quodnam sit divisum huius divisionis? et an sit divisio adaequata? 
\pend

\pstart
  Sit prima conclusio: divisum huius divisionis est universale univoce praedicabile. Ratio est: membra dividentia sunt praedicabilia de multis: igitur divisum debet esse universale univoce praedicabile. Vide iam individuum exclusum per hic praedicabile de multis, quare hic non agimus de \textnormal{|}\ledsidenote{BNC 126rb} individuo, nec de praedicatione identica, quae artificialis non est. Excipitur etiam analogam, quod non univoce praedicatur. Cum enim praedicabilis tractentur ut conducentia et constituentia praedicamenta; et constituantur haec in ratione specie genere et differentia, quod non potest praestare analogum, ideo reiicitur tam respectu praedicabilium quam praedicamentorum. 
\pend

\pstart
  Sit secunda conclusio: hoc divisum est universale pro formali, id est ratione universalitatis et praedicabilitatis, non vero ratione naturae. Probatur universalitas et praedicabilitas accidit naturae: igitur diversus modus praedicandi non est divisio naturae qua talis, sed qua praedicabilis. Igitur haec divisio respicit per divisio ipsam rationem formalem universalitatis et praedicabilitatis, et non naturam. 
\pend

\pstart
  Confirmatur: divisum debet esse quod certum et determinatum, sed ipsa natura qua subiicitur universalitati non est quod certum et determinatum: igitur divisium non est ipsa natura. Probatur minore. Quod modo est substantia modo accidens, modo ens reale et modo ens rationis non est certum et determinatum. Sed natura qua subiicitur universalitati sic se habet: igitur non est quod certum. Sed ipsa intentio semper est accidens rationis: igitur est quod certum: igitur universale per formali est divisum. 
\pend

\pstart
  Sit tertia conclusio: divisio qua universale dividitur in quinque praedicabilia, scilicet genus, specie, differentia, proprium et accidens, est exacta et adaequata. Ita \edtext{ Divus \name{\textsc{Thomas}\index[persons]{Thomas Aquinas}} 1 \worktitle{Contra Gentiles} capitulo 32 }{\lemma{}\Afootnote[nosep]{}} et omnes Auctores. 
\pend

\pstart
  Probatur ratione: ratio formalissima praedicationis consistit in identitate et conexione extremorum: igitur illa erit adaequata divisio praedicabilitatis, quae est adaequata divisio identitatis extre \textnormal{|}\ledsidenote{BNC 126va}    extremorum. Sed talis divisio includit omnem identitatem extremorum: igitur est adaequata. Probatur minore. Omnis identitas seu conexio extremorum vel est essentialis vel accidentalis, intrinseca vel extrinseca, necessaria vel contingens. Si essentialis vel explicat totam integram quae essentia, id est omnia quae ad definitionem concurrunt vel partem. Si integram essentiam est species. Si partem vel partem potentialem et contrahibiliem vel actualem et contrahilem. Si potentialem et contrahibilem est genus et dicitur in quod incomplete, nam dicit partem quidditativam, id est primam et substantialem. Sed determinabile per alias non determinantem et qualificantem. Si partem actualem est differentia quae est praedicatum determinans et qualificans. 
\pend

\pstart
  Si non essentialis, sed accidentalis, id est non ingrediens constitutivam essentiam vel est praedicatum proveniens ex propriis principiis necessariam habens connexionem cum illis et est proprium vel convenit extraneis et communibus et sic accidens. Sed hae divisiones traduntur per membra contradictorie opposita: igitur recte et adaequate dividitur. 
\pend

\pstart
  Sit quarta conclusio: non solum universale logicum, sed etiam metaphysicum ut eius fundamentum dividitur hac divisione proportionali modo. Probatur praedicatio fundatur in identitate: igitur diversus modus praedicationis supponit diversum modum identitatis, sed talis identitas non solum debet accipi per ipsa comparatione et relatione ad terminum. Sed etiam per ipsa aptitudine ad essendum et identificandum cum inferioribus. Igitur supponendus es modus diversus abstractionis fundantis aptitudinem praedicatam in natura sic vel sic identificabili cum illis. 
\pend

\pstart
  \textnormal{|}\ledsidenote{BNC 126vb} Ad hanc conclusionem expectat fundamentum  \edtext{ \name{\textsc{Complutensium}\index[persons]{}},  quo probant sufficientiam divisionis quod tale est: \enquote{proprium et adaequatum subiectum universalitatis est natura simpliciter una (quae est universale metaphysicum). Igitur quot fuerint modi diversi naturae unius tot erunt differentiae divisive universalis} }{\lemma{}\Afootnote[nosep]{}}. Sed quinque tantum assignari possunt modi diversi naturae simpliciter unius, scilicet natura perfectibilis seu constitutibilis, quae est subiectum adaequatum generestatis; natura perfectiva seu constitutiva, quae est subiectum adaequatum differentialeitatis; natura perfecta seu constituta, quae est adaequatum subiectum specietatis; et natura perfluens ab essentia, quae est subiectum adaequatum proprieitatis; et demum natura extrinseca, seu contingens, quae est adaequatum subiectum accidentaleitatis. Igitur quinque sunt diversi modi universali metaphysicae seu subiecti et fundamenti Logici. 
\pend

\pstart
  Probatur tales modos esse diversos. Natura perfectibilis comparatur ad inferiora ut quid potentile in illis et per modum materiae. Natura perfectiva comparatur ut constitutivum essentiae eorum per modum actus seu formae. Natura perfecta comparatur ut tota eorum essentia per modum constituti ex potentia et actu. Natura perfluens comparatur ut quod necessario coniunctum cum eorum essentia. Et demum natura extrinseca seu contingens comparatur ut potens abesse et adesse eorum essentiae. Sed hae comparationes sunt diversae: igitur modi sunt diversi. 
\pend

\pstart
  Obiectio 1. Plure dantur praedicationes de pluribus non contentae sub hac divisione: igitur non est adaequata. Probatur antecedens. Ratio analoga praedicatur de multis analogatis, sed talis non est genus, nec species, etc., igitur. Respondeo distinguendo antecedens. Plures dantur praedicationes de pluribus u \textnormal{|}\ledsidenote{BNC 127ra}    univoce, nego; analogice, concedo. Analogum non est simpliciter universale, sed solum reductive. Ideo dixi numero 461: divisum esse universale univoce praedicabile. 
\pend

\pstart
  Instabis: quod actu praedicatur est praedicabile, sed individuum vagum ut aliquos homo persona, suppositum praedicatur de multis: igitur est praedicabile. Sed non est unum ex quinque assignatis: igitur mala est divisio. Confirmatur: etiam definitio praedicatur de pluribus, sed non pertinet ad unum ex assignatis modis: igitur. 
\pend

\pstart
  Respondeo distinguo minorem. Individuum vagum praedicatur de multis simplici praedicatione, nego; praedicatione non simplici, concedo. Individuum non solum dicit naturam, sed etiam modum essendi, scilicet individuationem seu personalitatem \edtext{ 1 parte, questio 30, articulo 4 }{\lemma{}\Afootnote[nosep]{}}. Unde praedicatur in quid, quale: in quid qua dicit naturam; in quale qua dicit individuationem, cumque haec sit accidentalis. Individuum vagum reducitur ad duo praedicabilia, scilicet ad essentiale et ad accidentale. 
\pend

\pstart
  Ad confirmationem distinguo antecedens. Definitio praedicatur de pluribus complexe, concedo; incomplexe nego. Definitio excluditur a ratione praedicabilis simpliciter, quia est terminus complexus; reducitur tamen ad suum definitionum, scilicet ad secundum praedicabile. 
\pend

\pstart
  Iterum instabis: existentia et mors et similia praedicata praedicantur de pluribus, sed non essentialiter, quia non pertinent ad eius quidditatem. Non ut proprium, quia non dimanat a principiis intrincesis. Non ut accidens, quia non absunt et adsunt sine corruptione subiecti. Igitur in nullo istorum praedicabilium comprehenduntur. 
\pend

\pstart
  Respondeo concedendo antecedens et minorem quoad primam et secundam partem et negatur quo ad tertiam. Ast \textnormal{|}\ledsidenote{BNC 127rb} probationem distinguo non absunt et adsunt sine corruptione physica subiecti, concedo; sine corruptione metaphysica, nego. Quando definitur accidens intilligitur de corruptione metaphysica, id est possunt accidentia abesse et adesse sine destructione conceptus quidditativi. Quare licet mors et existentia non adsunt vel absunt sine corruptione physica, bene tamen sine metaphysica; quid non accidit in propria passione naturam si convenientia eius ad subiectum destruitur, destruitur essentia ex consequenti. 
\pend

\pstart
  Obiectio 2. Tot modis dicitur unum contrariorum, quot et alterum ex \edtext{ \name{\textsc{Aristotele}\index[persons]{Aristoteles}} 1 \worktitle{Topicorum} capitulo 12 }{\lemma{}\Afootnote[nosep]{ \textsc{Aristoteles}\index[persons]{Aristoteles}, \worktitle{Topica} 1.12 (105a, 10--19). }}. Sed universale et singularia sunt causa natura: igitur. Sed singularia sunt numero infinito: igitur et universalia. 
\pend

\pstart
  Respondeo distinguendo maiorem. Tot modis dicitur unum contrarium econtra formaliter, concedo; materialiter, nego. Secundae intentiones quas important universale et singulare sunt oppositae et prout sic quod modis dicitur una, dicitur et altera, nam cuilibet praedicabili correspondet suum subiicibile; cumque sunt tantum quinque praedicabilia, tot sunt subiicibilia formaliter. At materialiter numero prope infinito sunt singularia, id est substantia, seu supposita. Dictum \textsc{Philosophi}\index[persons]{Aristoteles} non attenditur materialiter, sed formaliter quoad significationes formales. Verbi gratia si nunc existerent decem substantia alba et unum tantum nigrum, albedo et nigredo opponerentur? Opponerentur: igitur multiplicato subiecto in albedine ut non in nigredine, adhuc manet hic tot modis, quia verificatur de significatione formale. 
\pend

\pstart
  Obiectio 3. Tot sunt partes universalis, quot entis, ex \edtext{ \name{\textsc{Aristotele}\index[persons]{Aristoteles}} 4 \worktitle{Metaphysicae} capitulo 1 }{\lemma{}\Afootnote[nosep]{ \textsc{Aristoteles}\index[persons]{Aristoteles}, \worktitle{Metaphysica} 4.1 (984b, 23ss). (No hallamos la cita. En ese lugar Aristóteles no habla de los universales.) }}. Sed partes entis sunt decem, scilicet praedicamenta, igitur decem sunt universalis. 
\pend

\pstart
  \textnormal{|}\ledsidenote{BNC 127va}   Respondeo distinguendo antecedens. Tot sunt partes universale materialiter quot entis, concedo. Materia universalis est omnis res et sic bene dicit \textsc{Philosophus}\index[persons]{Aristoteles}. At partes dividentes universale sunt quinque rationibus adductis supra. 
\pend

\pstart
  Instabis: igitur universale Metaphysicum non dividitur in quinque modos naturae unius. Respondeo distinguendo consequens. Universale Metaphysicum Metaphysice sumptum, concedo; logice sumptum, id est prout est fundamentum relationis intentionis, nego: rationibus allatis supra. 
\pend

\pstart
  Obiectio 4. Praedicata humana praedicantur de Verbo divino, sed non praedicantur de Deo essentialiter. Nam in tempore ceperunt illi convenire, neque contingenter ut accidens. Nam unio est substantialis facitque Deum vere et univoce esse hominem; neque ut proprium. Nam talia praedicata non dimanat ex propriis Dei. Igitur datur modus praedicandi non pertines ad hanc divisionem. 
\pend

\pstart
  Respondeo concedendo maiorem et distinguendo minorem. Non praedicantur de Deo ut subsistenti in se essensialiter, concedo; de Deo ut subsistenti in natura humana, nego. Ad instantiam subsumptam: igitur Deus ab aeterno esset homo distinguitur. Deus ut in se subsistens, nego; ut subsistens in natura humana \emph{subdistincturum [?]} esset existentialiter, nego; esset essentialiter quoad convexionem praedicamentorum \emph{subdistincturum [?]} esse negative, concedo; possitive nego. Deo ut est in se subsistenti accidentaliter conpetunt praedicata humana: at supposita incarnatione essentialiter. 
\pend

\pstart
  Obiectio 5. Sufficiunt duo praedicabilia in quae alia conveniunt: igitur superfluunt quinque. Probatur antecedens. Sufficiunt praedicabile essentiale et \textnormal{|}\ledsidenote{BNC 127vc} accidentale: igitur. Respondeo distinguendo antecedens. Sufficiunt duo praedicabilia ut genera, concedo; ut specie ultimae, nego. 
\pend

\pstart
  Instabis: si quinque praedicabilia ponerentur ut species infimae, plura deberent poni quam quinque: igitur vel sunt duo vel non sunt quinque. Probatur assumptum. Sicut praedicatio in quod dividitur in praedicationem de pluribus specie differentibus, quae est genus, et in praedicationem de pluribus numero differentibus, quae est species. Ita praedicabile in quale essentialiter potest dividi in praedicatione de pluribus specie differentibus, quae est differentia generica; et in praedicationem de pluribus numero differentibus, quae est differentia specifica. Similiter genus potest dividi in supremum et subalternum. 
\pend

\pstart
  Respondeo negando antecedens. Ad probationem no sicuitatem. Ratio est, praedicari de pluribus specie et numero differentibus non astruit formaliter praedicabilia distincta, siquidem idem genus praedicatur de speciebus et individuis differentibus. Unde distinctio formalis inter genus et speciem est, quod genus praedicatur in quid incomplete; species autem in quid complete. Quare genus dicit partem quidditatis et non integram quidditatem, at species dicit totam et integram quidditatem. Differentia autem sive generica, sive specifica semper praedicat partem essentialem. Idcirco in modo praedicandi non differt, sed in re materiali, quam praedicat. Similiter genus sive supremum, sive subalternum semper praedicat partem quidditativam et non totum quidditatem: licet una pars quidditativa sit universalior et superior altera; et hoc accidentaliter se habet respectu modi praedicandi per modum partis in quod. 
\pend

\pstart
  \textnormal{|}\ledsidenote{BNC 128ra}   Instabis: non minus dividitur universale in essensiale et accidentale, quam in separabiliter et inseparabiliter, intrinsece et extrinsece. Igitur praedicta divisio non exaurit totum divisum. 
\pend

\pstart
  Respondeo negando antecedens. Dantur modi variantes per se rationem praedicabilitatis, nempe qui variant modum identificationis et conexionis extremorum sicut essentiale et accidentale, in quid, in quale, et dantur modi non variantes ac per consequentes, nec constituentes diversa praedicabilia. Unde debent redduci ad illos modos, scilicet intrinsece et extrinsece; et in aliquibus separabiliter et inseparabiliter. 
\pend

\pstart
  Instabis ultimo: ex \edtext{ \name{\textsc{Philosopho}\index[persons]{Aristoteles}} 1 \worktitle{Topicorum} capitulo 3 nominante solum quatuor praedicata, scilicet genus, definitionem, proprium, et accidens: igitur totidem erunt praedicabilia. }{\lemma{}\Afootnote[nosep]{}} Praeterea praedicabilia sunt in ordine ad praedicamenta, sed praedicamenta sunt decem: igitur et praedicabilia. 
\pend

\pstart
  Respondeo \textsc{Aristotelem}\index[persons]{Aristoteles} non numerasse modos praedicandi universaliter, cuius ratio est, quia ibi numeravit definitionem, quae excluditur a praedicabilibus ut constat numero 479 et 480. Praeterea enim praedicabilium no sequuntur modum essendi absolute et secundum modum essendi per participationem entis, sed secundum modum comparativum, quo inferiora conectuntur cum superioribus. 
\pend

\pstart
  Obiectio 6. Accidens saltem completum vel est genus vel species. Siquidem habet inferiora de quibus essentialiter praedicatur: igitur quantum praedicabile superfluit vel coincidet cum genere et specie. 
\pend

\pstart
  Respondeo distinguendo antecedens. Accidens est genus vel species respectu inferiorum quae constituit, concedo; \textnormal{|}\ledsidenote{BNC 128rb} respectu inferiorum quae denominat, nego. Accidens enim respectu inferiorum quae denominat solum habet rationem accidentis et prout sic est membrum dividens. At respectu suorum inferiorum habet genus et specie et cetera praedicabilia. 
\pend

\pstart
  Obiectio 7. Genus et differentia continentur in specie, sed de ratione bonae divisionis est, quod unum membrum non includatur in alio: igitur est mala divisio. Respondeo genus et differentia continentur in specie fundamentaliter, concedo; formaliter, nego. Genus et differentia sunt gradus metaphysicae constituentes speciem. At prout sic non sunt praedicabilia. 
\pend

\pstart
  Obiectio 8. Aliud datur praedicabile distinctum ab his: igitur. Probatur antecedens. Sicut ex genere et differentia resultat aliud praedicabile quod nec est genus, nec differentia, sed species. Ita ex proprio et differentia potest resultare aliud praedicabile, quod nec sit proprium, nec differentia: igitur datur aliud praedicabile. 
\pend

\pstart
  Respondeo negando antecedens. Ad probatione nego assumptum. Sicut ex genere et differentia componatur species, non tamen ex differentia et proprio. Ratio est quia genus cum se habeat ut potentia et differentia ut actus sunt partes capaces compositis metaphysicae. At differentia et proprium se habent ut actus et per omnes non potentes componeere unum totum. ceterum quando intellectus tribuit differentiam et proprium individuis facit praedicationem complexam praedicative. Verbi gratia `Petrus est rationalis risibilis', unde pertinet ad duo praedicabilia. 
\pend

\pstart
  Obiectio 9. Datur aliud praedicabile ab his distinctum: igitur. Probatur assumptum. Appetitiuum praedicatur de homine, sed non unum ex his includens: igitur. Probatur minor. Non ut differentia: igitur. Probatur antecedens amplius. \textnormal{|}\ledsidenote{BNC 128va}   Non ut proprium ad quod possunt pertinere: igitur ad nullum. Probatur antecedens. Proprium competit omni soli et semper, sed appetitiuum non solum hominibus competit: igitur non est proprium. 
\pend

\pstart
  Respondeo negando antecedens. Ad probationem nego minorem. Ad probationem nego secundum antecedens. Ad probationem distinguo minorem. Appetitiuum genericum non solum hominibus competit, concedo; appetitiuum specificum, nego. Proprium enim genericum convenit omni, soli et semper, generi ac per communes speciebus illius generis. Ast proprium specificum convenit omni, soli et semper, speciei ac per communes individuis illius speciei. Unde si appetitiuum sumatur generice non solum hominibus competit, sed etiam Angelis, at specifice solum illis competit. 
\pend

\pstart
  Obiectio 10. Rationale in ordine ad Petrum et hoc rationale est universale, sed non est differentia: igitur ad nullum praedicabile pertinet. Probatur minore. Respectu huius rationalis est tota essentia: igitur non est differentia. Respondeo negando minorem. Ad probationem nego antecedens. Hoc et illud rationale sunt individua, cumque respectu individuorum sit pars formalis et non tota essentia. Ideo sicut praedicatur de Petro, praedicatur de hoc vel illo. 
\pend

        \addcontentsline{toc}{section}{Articulus 9. An Universale sit univocum?}
        \pstart
        \eledsection*{Articulus 9. An Universale sit univocum?}
        \pend
      
\pstart
  Suppono 1. Rationem univocam convenire non solum secundum nomen, sed etiam secundum rem significatam. Verbi gratia animal dicitur genus respectu hominis et equi; nam homo et equus non solum conveniunt in ratione animalis nomine, sed etiam in re significata. 
\pend

\pstart
  Suppono 2. Quaestionem non procedere de universali primo intentionaliter, id est de natura supra \textnormal{|}\ledsidenote{BNC 128vb} quam cadit secunda intentio, sed ipso universali formali seu de ipsa universalitate. Ratio est clara: ad ens reale et rationis, ad substantiam et accidens non datur ratio univoca, sed universale pro materiali modo est substantia, est accidens, est ens reale, est ens rationis: igitur prout sic universale non potest esse univocum. 
\pend

\pstart
  Sit conclusio: universale est univocum. Ita omnes Thomistae et Scotistae. Probatur ratione: quod convenit pluribus secundum nomen et rationem est univocum ex \textsc{Aristotele}\index[persons]{Aristoteles}, sed universale convenit pluribus, scilicet praedicabilibus secundum nomen et rationem: igitur est univocum. Probatur minore. Universale est unum aptum esse in pluribus et praedicari de pluribus, sed quidlibet praedicabilium est aptum esse in pluribus et praedicari de pluribus: igitur universale convenit pluribus, etc. Parificatur: ideo animal est univocum ad hominem et brutum, quia essentia animalis convenit illis. Sed essentia universalis convenit quinque praedicabilibus: igitur universale est univocum ad illa. 
\pend

\pstart
  Obiectio 1. Si universale ut sic praescinderet ab omnibus universalibus praescinderet a se ipso, sed hoc repugnat: igitur non praescindit. Probatur sequela. Universale ut sic praescindit ab omnibus universalibus, sed ipsum es universale: igitur praescindit a se ipso. 
\pend

\pstart
  Respondeo negando sequelam maioris. Ad probationem concedo maiorem et distinguo minorem. Ipsum est universale qua comparatum, concedo; qua praecisum subdistinctionem est universale fundamentaliter, concedo; formaliter, nego. Cum enim universale constituantur per comparationem, usquam universale ut sic non comparetur, non censetur tale formaliter. Per abstractionem enim a praedicabilibus adhuc non comparatur: igitur quando praescindit ab omnibus non potest a se ipso praescindere cum nondum sit nisi fundamentaliter. 
\pend

\pstart
  \textnormal{|}\ledsidenote{BNC 129ra}   Instabis: igitur universale ut sic non universale in actu est. Respondeo distinguendo consequens. Non est universale in actu ut sic, nego; in specie, concedo. Universale ut abstractum a praedicabilibus est universale ut sic et fundamentaliter est genus. At comparatum ad illa iam est universale in specie. Unde qua universale in actu ut sic est in potentia ut sit universale in specie. 
\pend

\pstart
  Replicabis: Universale ut sic qua abstractum est universale, sed abstrahit ab omnibus universalibus: igitur abstrahit a se ipso. Respondeo distinguendo maiorem. Est universale ut si, concedo; est universale in specie, nego. Et distinguo minorem. Abstrahit ab omnibus universalibus ut sic, nego; ab omnibus universalibus in suis speciebus, concedo. Animal enim non abstrahit ab animalibus, sed ab speciebus. 
\pend

\pstart
  Obiectio 2. Genus est magis universale quam cetera praedicabilia: igitur universale non est univocum. Probatur antecedens. Genus se extendit ad plura inferiora: igitur est magis universale. Respondeo negando antecedens. Ad probationem concedo antecedens et nego consequentiam. Quod autem genus extendatur ad plura inferiora probat esse maius universale, non autem magis. Magis autem pertinet ad intentionem formae, maius autem ad extensionem. Verbi gratia quaternarius numerus est maior, binario et tamen non est magis numerus. 
\pend

\pstart
  Obiectio 3. Si universale esset univocum conveniret univoce, sed homo non: igitur non est univocum. Probatur minorem. Si conveniret univoce omnia praedicabilia essent univoca, sed hoc non: igitur. Probatur minore. Proprium et accidens non sunt univoca: igitur. Probatur antecedens. Denominativa non sunt univoca ex \textsc{Philosopho}\index[persons]{Aristoteles}, sed proprium et accidens sunt denominativa: igitur non sunt univoca 
\pend

\pstart
  Confirmatur: univocum debet esse \textnormal{|}\ledsidenote{BNC 129rb} ens per se, sed proprium et accidens praedicatione quarti et quinti praedicabilis non sunt entia per se: igitur qua talia non sunt univoca. Probatur minor. Concretum accidentale non est ens per se, sed proprium et accidens praedicantur in concreto: igitur non sunt entia per se qua talia. 
\pend

\pstart
  Roboratur: quamvis accidens uniatur subiecto non tamen identificatur: igitur solum potest abstrahi acccidens a subiecti abstractione formali. Sed ad universale potius requiritur abstractio totalis quam formalis: igitur accidens et proprium praedicatione quarti et quinti praedicabilis non sunt universalia. 
\pend

\pstart
  Respondeo negando minorem. Ad probationem nego minorem. Ad probationem nego antecedens et ad eius probationem distinguo maiorem. Denominativa non sunt univoca prout univocum dicit praedicari essentialiter de inferioribus, concedo; prout univocum dicit praedicari aequaliter de inferioribus, nego. Univocum bifariam constituatur. Primo stricte prout distinguitur non solum ab aequivocis et analogis, sed etiam a denominativis et in hoc sensu excluduntur denominativa a ratione univoci. Nam sic solum illa praedicabilia dicuntur univoca, quae sunt de essentia inferiorum. 
\pend

\pstart
  Secundo late prout distinguitur solum ab aequivocis et analogis. Nam sic significat unam rationem aequaliter repertam in inferioribus et in hoc sensu denominativa dicuntur univoca. Quando enim \textsc{Aristoteles}\index[persons]{Aristoteles} excludit denominativa ab univocis loquitur in primo, non in secundo sensu. 
\pend

\pstart
  Vel aliter et facilius. Respondeo distinguendo antecedens. Denominativa non sunt univoca sub eadem ratione, concedo; sub diversa, nego. Igitur denominativum vel constituatur ad formam denominantem vel in \textnormal{|}\ledsidenote{BNC 129va}   ordiordine ad inferiora, in quibus aequaliter invenitur. Primo modo est denominativum qua tale, secundo modo est univocum et universale. 
\pend

\pstart
  Ad confirmationem concedo maiorem et nego minorem. Ad probationem distinguo maiorem. Concretum accidentale non est ens per se adiiective sumptum, nego; substantive sumptum, concedo. Et distinguo minorem. Proprium et accidens praedicantur in concreto adiiective, concedo; substantive, nego. 
\pend

\pstart
  In praedicationibus enim concretorum accidentalium concretum ex parte subiecti se habet substantive, at concretum ex parte praedicati se habet adiiective. In praedicatione enim praedicatum se habet ut forma, subiectum vero ut materia: igitur quando praedicatur unum concretum de alio praedicatum se habet ut forma, id est sine connotatione subiecti. Sed prout sic accidens est ens per se: igitur quando se habet adiiective est ens per se. Quare concretum accidentale proprie non est ens per accidens. Nam semper concretum et abstractum significant eamdem formam, licet diverso modo. Vide numero 388 et sequenti. 
\pend

\pstart
  Ad roborationem distinguo antecedens. Non tamen identificatur quidditative et intrinsece, concedo; denominative, nego. Et distinguo consequens. Solum potest abstrahi abstractione formali, qua constituens, concedo; qua denominans, nego. Omnia inferiora conveniunt in accidente ut in denominante, non ut in constitute. 
\pend

        \addcontentsline{toc}{section}{Articulus 10. An universale sit genus ad praedicabilia?}
        \pstart
        \eledsection*{Articulus 10. An universale sit genus ad praedicabilia?}
        \pend
      
\pstart
  Prima sententia asserit universale non esse genus cum sit analogum. Ita \edtext{ Beatus \name{\textsc{Albertus Magnus}\index[persons]{Albertus Magnus}} \worktitle{Tractato} 2, capitulo 9 }{\lemma{}\Afootnote[nosep]{}}; \edtext{ \name{\textsc{Canterus}\index[persons]{Ioannes Cantero}} ibidem capitulo 4 }{\lemma{}\Afootnote[nosep]{}}; \edtext{ \name{\textsc{Oña}\index[persons]{Petrus de Oña}} quaestione 2, articulo 3 }{\lemma{}\Afootnote[nosep]{}}; \textsc{Amonius}\index[persons]{Ammonius Hermiae} ; \textsc{Simplicius}\index[persons]{Simplicius}; \edtext{ \name{\textsc{Fonseca}\index[persons]{Petrus Fonsecae}} 5 \worktitle{Methaphysicae} capitulo 28, quaestione 12, \textnormal{|}\ledsidenote{BNC 129vb} sectione 2 }{\lemma{}\Afootnote[nosep]{}}. Secunda sententia docet universale esse genus ad quinque praedicabilia. Sic \edtext{ \name{\textsc{Lovanienses}\index[persons]{}}  capitulo \worktitle{de Propio} }{\lemma{}\Afootnote[nosep]{}}; \edtext{ \name{\textsc{Ioannes Anglicus}\index[persons]{Ioannes Venator Anglicus}}  }{\lemma{}\Afootnote[nosep]{}}; \edtext{ \name{\textsc{Antonius Andreas}\index[persons]{Antonius Andreas}} }{\lemma{}\Afootnote[nosep]{}}; \edtext{ Illustrissimus \name{\textsc{Merinero}\index[persons]{}} disputatione 4, quaestione 2 }{\lemma{}\Afootnote[nosep]{}}, quae citat alios ex sua florentissima schola cum suo  \edtext{ Subtili Doctore \name{\textsc{Scoto}\index[persons]{Ioannes Duns Scotus}} quaestione 8 \worktitle{Universalibus} }{\lemma{}\Afootnote[nosep]{}}; \edtext{ \name{\textsc{Molina}\index[persons]{Ludovicus de Molina}} hic sectione 4 }{\lemma{}\Afootnote[nosep]{}};  \edtext{ \name{\textsc{Masius}\index[persons]{Didacus Masius}}  sectione 5, quaestione 2, conclusione 2 }{\lemma{}\Afootnote[nosep]{}}; \edtext{ \name{\textsc{Ayala}\index[persons]{Martinus Perez de Ayala}} quaestione 2 }{\lemma{}\Afootnote[nosep]{}}; \edtext{ \name{\textsc{Iabellus}\index[persons]{Chrysostomus Iavelli Canapicii}}  Tractatu 4, capitulo 5 et 7 }{\lemma{}\Afootnote[nosep]{}}; \edtext{ \name{\textsc{Sotus}\index[persons]{Dominicus de Soto}} quaestione 4 }{\lemma{}\Afootnote[nosep]{}}; \edtext{ \name{\textsc{Toletus}\index[persons]{Franciscus Toletus}}  quaestione 2 }{\lemma{}\Afootnote[nosep]{}}; \edtext{ \name{\textsc{Sanchez}\index[persons]{Ioannes Sanchez Sedeno}} libro 2, quaestione 8, articulo 3 }{\lemma{}\Afootnote[nosep]{}} et alii quos refert \edtext{ \name{\textsc{Gallego}\index[persons]{Baranbas Gallego de Vera}}  \worktitle{Controversia} 11 }{\lemma{}\Afootnote[nosep]{}}; \edtext{ \name{\textsc{Complutenses}\index[persons]{}} hic disputatio 4, quaestione ultima }{\lemma{}\Afootnote[nosep]{}}; \edtext{ \name{\textsc{Ioannes a Sancto Thoma}\index[persons]{Ioannes a Sancto Thoma}} quaestione 6, articulo 2 }{\lemma{}\Afootnote[nosep]{}}; \edtext{ \name{\textsc{Rubius}\index[persons]{Antonius Ruvius Rodensis}}   questione 7 est denique communit recepta. }{\lemma{}\Afootnote[nosep]{}} 
\pend

\pstart
  Sit nostra conclusio: Universale est genus ad quinque praedicabilia. Ita \edtext{ Nostro \name{\textsc{Angelico Doctor}\index[persons]{Thomas Aquinas}} 1 \worktitle{Contra Gentes} capitulo 32 et \worktitle{Opusculo} 56. Probatur breviter: \enquote{genus praedicatur in quod potentialiter de differentibus specie; sed universale praedicatur in quod potentialiter de differentibus, scilicet de praedicabilibus: igitur est genus} }{\lemma{}\Afootnote[nosep]{}}. Ideo animal est genus ad hominem et equum, etc., quia est indifferens et contrahibile per differentiam cuiusvis. Sed etiam universale est indifferens et contrahibile per differentiam cuiusvis praedicabilis: igitur est genus. 
\pend

\pstart
  Ut quaestionem resolutam magis explanem. Suppono 1. Controversiam facilem esse, sed confusam ratione intentionum reflexarum. Ut obscuritatem penitus \emph{teneat [?]} conclusio multifariam multisque linguis loquuntur Doctoribus citati. Alii se expediunt per materialiter et formaliter, ita ut genus primum praedicabile sit tale formaliter, etsi denominetur species comparatione ad universale hoc tamen esse materialiter. Alii primum vocant essentialiter, accidentaliter secundum. Alii secundum appellant \textnormal{|}\ledsidenote{BNC 130ra}   denominative et essentialiter primum. Omnes equidem optime, expedite et clare procedunt: ita ut quolibet modo maneret resolutio splendore splendidior et luce micantior. 
\pend

\pstart
  At ego altera via procedo non invia quamvis inusitata et hucusque non inventa, nec lecta, habita tamen proprio \emph{marle [?]}. Pro cuius intelligentia noto, nos in ordine secundarum intentionum non uti numero arithmetico ut aliquando dixi, at non dici, non indigere. Siquidem hic arithmetica suis innumeris numeris multum iubat. Datur prima intentio ut homo, datur secunda ut species et datur tertia ut quando genus primum praedicabile iterum denominatur species respectu universalis ut sic. In tota resolutio. Quod autem Auctores unius ordinis vocabant materialiter, et alterius ordinis accidentaliter, et ultimi denominative; hoc ipsum appello tertiam intentionem. Ergo omnes idem ferimus \emph{licet [?]} vocibus analogis. 
\pend

\pstart
  Habes iam primam, secundam et tertiam intentionem. Sed nota primam et retiam esse subiectam secundae. Loquatur exemplum genus est primum praedicabile, scilicet secunda intentio genereitatis. Haec intentio denominat animal, quod est prima intentio, accidentaliter genus et constituitur inferius primi praedicabilis. Similiter denominat universale ut sic genus ad praedicabilia accidentaliter et constituitur inferius primi praedicabiliter. Similiter species secundum praedicabile denominat hominem speciem et quodlibet praedicabile speciem. Er sicut animal primo intentionaliter essentialiter est animal et accidentaliter genus secundo intentionaliter, ita universale ut sic essentialiter est universale et \textnormal{|}\ledsidenote{BNC 130rb} accidentaliter genus tertio intentionaliter. Et sicut homo essentialiter est homo primo intentionaliter et accidentaliter est species secundo intentionaliter, sic quodlibet praedicabile essentialiter est tale secundo intentionaliter et accidentaliter est species tertio intentionaliter. Solum restat ut pariter syllogizes de ceteris et nunc applicare doctrinam obiiectionibus. 
\pend

\pstart
  Obiectio 1. Si universale esset genus una species praedicaretur de alia, sed hoc implicat: igitur et illud. Probatur sequela. Tunc primum praedicabile esset species: igitur vere praedicaretur genus est species universalis et eadem ratione species praedicaretur de differentia et de reliquis. 
\pend

\pstart
  Secundum inconveniens. Species praedicaretur de seipsa praedicatione formali, nam haec est praedicatio secundi praedicabilis, species est species universali; sed hoc implicat: igitur quod universali sit genus. 
\pend

\pstart
  Tertium. Concretum praedicaretur de suo abstracto, sed hoc non potest esse: igitur. Probatur sequela. Haec est vera praedicatio, specieitas est species universalis: igitur concretum praedicaretur de suo abstracto. 
\pend

\pstart
  Quartum. Genus contineretur sub sua specie, sed hoc repugnat: igitur. Probatur sequela. Universale eo ipso quod sit genus continetur sub genere primo praedicabili, sed genus primum praedicabile est species universalis: igitur continetur sub sua specie. 
\pend

\pstart
  Quintum. Dabitur genus quod sit individuum suae speciei, sed hoc implicat: igitur. Probatur sequela. Primum praedicabile est species infima: igitur eius inferiora sunt individua, sed universale ut sic qua genus est inferius: igitur ut genus est individuum suae speciei. 
\pend

\pstart
  Respondeo distinguendo maiorem. Una species \textnormal{|}\ledsidenote{BNC 130va}   tertio intentionaliter praedicaretur de alia, concedo; secundo intentionaliter, nego. Ad probationem distinguo genus esset species secundo intentionaliter, nego; tertio intentionaliter, concedo. Et similiter de reliquis praedicabilibus. Genus enim secundo intentionaliter essentialiter est genus et est species tertio intentionaliter accidentaliter. 
\pend

\pstart
  Ad secundum haec est praedicatio formalis. Species est species distinguitur species est species secundo intentionaliter, nego; tertio intentionaliter, concedo. Species secundum praedicabile essentialiter est speccies ut praedicabilis, hoc est secundo intentionaliter. At haec species ut subiicitur universali ut sic ut generi est species ut subiicibilis, hoc est tertio intentionaliter et sic accidentaliter est species. 
\pend

\pstart
  Ad tertium distinguo concretum praedicaretur de suo abstracto tertio intentionaliter, concedo; secundo intentionaliter, nego. Cum enim tale abstractum se haberet ut fundamentum tertiae intentionis respectu eius potius est materia quem forma; ac per consequens non implicat, imo est necessarem concretum tertio intentionale praedicari de abstracto secundo intentionale. 
\pend

\pstart
  Ad quartum distinguo maiorem. Genus contineretur sub sua specie secundo intentionaliter, nego; tertio intentionaliter, concedo. Et similiter distinguo probationem. Continetur sub genere secundo intentionaliter, nego; tertio intentionaliter, concedo. Genus universale enim secundo intentionaliter essentialiter est universale et prout sic non continetur sub genere. At tertio intentionaliter est genus accidentaliter, et prout sic continetur sub genere. 
\pend

\pstart
  Ad quintum distinguo maiorem. Dabitur \textnormal{|}\ledsidenote{BNC 130vb} genus quod sit individuum secundo intentionaliter, nego; tertio intentionaliter, concedo. Et similiter ad probationem. 
\pend

\pstart
  Instabis: igitur universale ut sic esset superius et inferius primo praedicabilis. Probatur assumptum. Primum praedicabile est species universalis ut sic: igitur universali ut sic est superius illo. Deinde universale ut sic est genus. Igitur prout sic est inferius primo praedicabili. 
\pend

\pstart
  Respondeo distinguendo consequens. Esset superius et inferius respectu eiusdem intentionis, nego; respectu diversarum intentionis, concedo. Universale ut sic secundo intentionaliter est superius, tertio intentionaliter cum sit genus est inferius. 
\pend

\pstart
  Replicabis: universale ut sic qua genus est superius genere, sed qua genus est inferius genere: igitur respectu eiusdem intentionis esset superius et inferius. Respondeo distinguendo maiorem. Qua genus est superius ut praedicabile, concedo; ut subiicibile, nego. Et distinguo minorem. Qua genus ut subiicibile est inferius genere, concedo; qua genus ut praedicabile, nego. Quando universale comparatur ad quinque praedicabilia, primum praedicabile est species; sed sic universale ut sic superius: igitur superius illo ut specie tertio intentionaliter. Universale ut sic constituatur ut praedicabile et ut subiicibile; prout praedicabile est superius, prout subiicibile inferius sicut species qua subiicibilis non est praedicabilis. 
\pend

\pstart
  Obiectio 2. Supra praedicamenta non datur aliud genus, cum sint genera suprema. Sed universale praedicatur de omnibus praedicamentis: igitur universale non est genus. Confirmatur: ideo ens non est genus quia transcendit omnes suas differentias, sed etiam universale transcendit omnes \textnormal{|}\ledsidenote{BNC 131ra}   suas differentias: igitur non est genus. Probatur minor. Differentia est unum ex quinque praedicabilibus: igitur quaelibet differentia est universalis. 
\pend

\pstart
  Respondeo distinguendo maiorem. Supra praedicamenta non datur aliud genus, quo univoce conveniant essentialiter primo intentionaliter, concedo; qua univoce conveniant secundo intentionaliter, nego. Cum enim praedicamenta sint primo diversa, non possunt habere genis in esse entitativo. Ast potest intellectus tribuere illis rationem universalitatis et sic habere genus, quod est forma secundo intentionalis. 
\pend

\pstart
  Ad confirmationem nego minorem. Ad probationem distinguo antecedens. Differentia est unum ex quinque praedicabilibus differentia adaequate sumpta, concedo; inadaequate, nego. Et distinguo consequens. Quaelibet differentia est universalis modaliter, concedo; essentialiter, nego. Differentiae quibus universale dividitur in quinque praedicabilia non sunt relationes universalitatis, sed modi differentialis, quibus relatio universalis contrahitur et sic formaliter non sunt universales. Siquidem non valet differentia constitutiva genereitatis essentialiter est universalitas. Quod autem unum ex praedicabilibus sit essentialiter differentia non arguit, quod universale includatur in differentiis, ed est in modis contrahentibus, sed in intentione differentiae, quae est species universalitatis ex tali modo contrahente resultans. 
\pend

\pstart
  Obiectio 3. Si universale esset genus singula praedicabilia essent species: igitur omnia pertinerent ad praedicabile speciei. Igitur tantum erunt duo praedicabilia. 
\pend

\pstart
  Respondeo distinguendo antecedens. Singula essent species secundae intentionaliter, nego; tertiae intentionaliter, concedo. Et distinguo consequens. Omnia pertinerent ad speciem secundae intentionaliter, nego; tertiae intentionaliter, concedo. Dicuntur species \textnormal{|}\ledsidenote{BNC 131rb} ratione universalis communis, quod participant et hoc est tertia intentionaliter, non tamen respectu inferiorum de quibus praeducatur. 
\pend

\pstart
  Instabis: eo ipso quod respectu universalis ut sic sint species debent habere inferiora, de quibus praedicentur: igitur etiam respectu inferiorum sunt species. 
\pend

\pstart
  Respondeo distinguendo communes. Respectu inferiorum de quibus praedicantur ut tota essentia, concedo; aliter, nego: Quodlibet praedicabile respectu diversorum est tale et species. Verbi gratia genus ut species habet inferiora omnia genera; at ut genus habet inferiora species de quibus in quod materialiter. 
\pend

\pstart
  Obiectio 4. Ex dictis haec praedicatio est vera `genus est species', sed hoc implicat: igitur. Probatur assumptum. Species opposita non praedicatur de alia, sed esse genus et speciem sunt species oppositae: igitur. 
\pend

\pstart
  Respondeo negando minorem. Ad probationem distinguo maiorem. Species opposita non praedicatur de alia essentialiter, concedo; accidentaliter, nego. Si genus et species considerentur in esse secunda intentionaliter sunt oppositae, at si constituentur tertia intentionaliter non sunt oppositae. Genus enim verbi gratia secundo intentionaliter et essentialiter non est species, sed solum accidentaliter et tertio intentionaliter; est genus et species respectu diversorum, ut dictum est. 
\pend

\pstart
  Obiectio 5. Ad substantiam et accidens non datur genus, sed tria prima praedicabilia pertinent ad substantiam rei et posteriora ad esse accidentale: igitur. Respondeo distinguendo maiorem. Ad substantiam et accidens non datur genus primaae intentionaliter, concedo; secundae intentionaliter, nego. Et distinguo minorem. Tria prima et materialiter, concedo; formaliter, nego. Non loquimur de universali materialiter. 
\pend

\pstart
  \textnormal{|}\ledsidenote{BNC 131va}   Obiectio 6. Ordo naturae postulat ut potentia praecedat actum, deinde pervenit ad speciem, qua media ad differentiam. Sed genus est pars potentialis, species tota essentia et actus differentia: igitur universalitas primus dicitur de genere, deinde de specie et de differentia ultimo: igitur inaequaliter participatur. Confirmatur: modus praedicandi sequitur modum essendi, sed secundum modum essendi praedictum ordinem postulat natura: igitur et intellectus in praedicando. 
\pend

\pstart
  Respondeo argumentum loqui in esse in primo intentionale et materialiter quod verum est. A secunda intentionaliter intellectus attribuit universalitatem omnibus aequaliter. Ad confirmationem distinguo maiorem. Modus praedicandi sequitur modum essendi absolute, nego; sequitur modum essendi ab intellectu \emph{constit [?]}, concedo. Modus essendi est duplex. Primus quae res dicunt ex propria entitate reali: Secundus quae habent ex \emph{constitutione [?]} intellectus. Posteriorem sequitur modus praedicandi, non priorem. Rationem assignat  \edtext{ Doctor \name{\textsc{Rubius}\index[persons]{}} hic: \enquote{modus praedicandi est proprietas non ab intrico rebus conveniens, sed ab intellectu attributa: igitur erit idem modus praedicandi; atque adeo ratio universalis univoca} }{\lemma{}\Afootnote[nosep]{}}. 
\pend

        \addcontentsline{toc}{section}{Articulus 11. An haec divisio sit immediata?}
        \pstart
        \eledsection*{Articulus 11. An haec divisio sit immediata?}
        \pend
      
\pstart
  Quando inter divisum et membra dividensia non mediant alia membra, in quae primus dividatur tale divisum, dicitur divisio immediata; si medient est divisio mediata. Verbi gratia vivens divisum in suas species athomas proprie dividitur sed non immediate, quia inter vivens et species athomas quae sunt membra dividentia mediant species subalternae viventis aut genera specierum athomarum. 
\pend

\pstart
  \textnormal{|}\ledsidenote{BNC 131vb} Sit conclusio: divisio universalis in quinque praedicabilia non est immediata. Est communis inter Auctores. Quare universale prius dividitur in quid et in quale. Deinde subdividuntur ista membra, scilicet universale in quid dividitur in genus, speciem et differentiam. Universale in quale in proprium et accidens. 
\pend

\pstart
  Probatur ratione. A convenientia essentiali multorum abstrahi potest conceptus eis communis, sed praedicabilia conveniunt essentialiter in aliqua ratione inferiori respectu universalis: igitur abstrahi potest ab ea conceptus communis, quae sit eis superior et universali inferio: igitur mediat aliqua ratio communis. Sed talis praedicatur de pluribus differentibus specie in quid: igitur erit genus intermedium: igitur universale non erit genus immediatum. Probatur minor. Genus, species et differentia conveniunt univoce in eo, quod praedicata intrinseca sunt: igitur ab eis potes abstrahi conceptus communis praedicati intrinseca. Similiter a proprio et accidenti potest abstrahi conceptus communis praedicati extrinseci, sed tam ipse, quam ille conceptus praedicatur in quid, illa de tribus et hic de duobus specie diversis: igitur hi conceptus sunt genera intermedia: igitur adaequate dividitur universale: in universale in quid et in universale in quale; et primum in genus, speciem et differentiam, et secundum in proprium et accidens. 
\pend

\pstart
  Sin altera conclusio: quinque praedicabilia sunt species athomae. Probatur quinque praedicabilia non sunt ulterius divisibilia per alios modos praedicandi de pluribus essentialiter. Sed hoc erat necessarium ut non essent athomae. Igitur sunt species athomae. Probatur maior. Sicut haec et illa genereitas, verbi gratia quantitas, substantia, etc., pro naturis \textnormal{|}\ledsidenote{BNC 132ra}   abstractis differant genere generalissimo; tamen secundo intentionaliter non differunt nisi solo numero. Similiter haec et illa specieitas; verbi gratia homo et equus secundo intentionaliter solum differunt numero, quamvis naturae ab eis denominatae specie inter se differant, quia hae intentiones sunt eiusdem rationis: igitur praedicabilia non dividuntur amplius in modos praedicanti essentialiter. 
\pend

\pstart
  Item quamvis genus dividatur in supremum et subalternum, et differentia in genericam, et specificam et similiter de aliis; haec membra non differunt ex parte modi praedicandi, et in ratione formali universalis, sed accidentaliter tantum secundum maiorem vel minorem multitudinem inferiorum de quibus praedicantur. Vide numero 484 et 485. 
\pend

\pstart
  Obiectio 1. Si universale non divideretur immediate inter universale et quinque praedicabilia mediaret illa duo genera media: igitur universalia iam essent septem. Probatur consequentia. Dantur illa duo universalia media. Sed etiam dantur quinque infima: igitur darentur septem. 
\pend

\pstart
  Respondeo negando consequentiam. Ad probationem distinguo maiorem. Darentur illa duo universalia media in specie athoma, nego; in specie \del{ab} media, concedo. et nego absolute consequentiam. Cum enim inferiora uniatur in superiori non numerantur cum superiori. Quare genus, species, differentia et universale essentialiter praedicabile non sunt quatuor universali, quia haec tria includuntur in quarto et quartum dividitur in illa tria praedicabilia, ac per consequens non numeratur cum eis. Intellige hoc clarissimo exemplo, dividitur animal in rationale et irrationale. Irrationale in equum, \textnormal{|}\ledsidenote{BNC 132rb} et leonem. Supponamus in rerum natura non esse nisi has animantes, scilicet hominem, equum, et leonem. Interrogo, quot sunt species animalis? Dico tres esse, scilicet hominem, equum et leonem; et bene. Dico quinque esse, scilicet animal rationale, animal irrationale, hominem, equum et leonem; pessime. Nam animal rationale includitur in homine, animal irrationale in equo et leone: igitur non numerantur quia sunt rationes superiores dividendae. Similiter in nostro casu. Nota in fine idem esse universale in quId ac essentialiter, necessario et intrinsece; universale in quale ac universale accidentale, contingenter et extrinsece. Sic definitio non nullas obscuritates. 
\pend

        \addcontentsline{toc}{section}{Articulus 12. An universale vel potius praedicabile sit adaequatum obiectum huius libri?}
        \pstart
        \eledsection*{Articulus 12. An universale vel potius praedicabile sit adaequatum obiectum huius libri?}
        \pend
      
\pstart
  Universale ut iam diximus est unum aptum esse in pluribus; praedicabile quod de pluribus praedicari aptum est. In pluribus esse est primum quod constituatur in natura, unde praedicari de pluribus est quod consequens et propria passio. Sic \edtext{ Doctor \name{\textsc{Thomas}\index[persons]{Thomas Aquinas}} \worktitle{Opusculo 48}, capitulo 1: \enquote{Ubi asserit intellectum tribuere rebus universalitatem tamquam fundamentum ceterarum denominationum} }{\lemma{}\Afootnote[nosep]{}}. 
\pend

\pstart
  Aliqui Doctores tenent praedicabilitatem esse obiectum adaequatum huius libri, quae opinio tribuitur \textsc{Scoto}\index[persons]{Ioannes Duns Scotus}. Sed nihilominus contrarium docent  \edtext{ \name{\textsc{Soto}\index[persons]{Dominicus de Soto}} quaestione 4 }{\lemma{}\Afootnote[nosep]{}}; \edtext{ \name{\textsc{Veracrux}\index[persons]{Alphonsus de Vera Cruce}}  ibidem }{\lemma{}\Afootnote[nosep]{}}; \edtext{ \name{\textsc{Mercadus}\index[persons]{Thomas de Mercado}}  capitulo ultimo \worktitle{De praedicabilius} }{\lemma{}\Afootnote[nosep]{}}; \edtext{ \name{\textsc{Toletus}\index[persons]{Franciscus Toletus}} quaestione 1 }{\lemma{}\Afootnote[nosep]{}}; \edtext{ \name{\textsc{Oña}\index[persons]{Petrus de Oña}} quaestione 4, articulo 1 }{\lemma{}\Afootnote[nosep]{}}; \edtext{ \name{\textsc{Antonius Rubius}\index[persons]{Antonius Ruvius Rodensis}} quaestione 8 \worktitle{Universalium} }{\lemma{}\Afootnote[nosep]{}} et communiter Thomistae. 
\pend

\pstart
  Sit conclusio: obiectum adaequatum huius libri est universale. Ita \edtext{ \name{\textsc{Angelius Doctor}\index[persons]{Thomas Aquinas}} quaestione 9 huius libri }{\lemma{}\Afootnote[nosep]{}}. Et probatur rationem. Illud est obiectum adaequatum partialis \textnormal{|}\ledsidenote{BNC 132va}   scientiae cui primo et per se communicatur ratio obiecti totalis scientiae, sed commune ens rationis logicum, quod est obiectum adaequatum Dialecticae prius convenit universali et eo mediante praedicabili: igitur universale est obiectum adaequatum huius libri. Probatur minor. De quolibet praedicabili verum est dicere, ideo est praedicabile quia est universale: igitur etiam est verum dicere ideo est ens rationis praedicabile, quia est ens rationis universale. 
\pend

\pstart
  Obiectio 1. Illud est obiectum scientiae \emph{sub [?]} cuius ratione constituantur omnia a tali scientia, sed omnia quae constituantur in hoc libro constituantur sub ratione praedicabilitatis: igitur praedicabilitas est obiectum adaequatum. 
\pend

\pstart
  Respondeo distinguendo minorem. Constituantur sub ratione praedicabilitatis radicaliter, concedo; formaliter, nego. Praedicabilitas radicaliter est sua essentia, scilicet universalitas; formaliter est ipsa passio. Omnia quae tractantur hic, tractatur sub ratione praedicabilitatis radicalis, scilicet universalitatis ob rationem praefatam. 
\pend

\pstart
  Obiectio 2. Alii scientiae tractant de universalibus, sed si universale esset obiectum huius libri, aliae scientiae non agerent de universalibus: igitur universale non obiectum huius libri. Respondeo distinguendo maiorem. Alii scientiae tractant de universalibus per materiali, concedo; per formali, nego. Omnes scientiae tractant de naturis abstractis, quae sunt universale logicum materialiter, sed non tractant de ipsa universalitate, quae est secunda intentio. 
\pend

\pstart
  Obiectio 3. Hic agit Porphyrium? de individuo, quod est singulare et non universale, est tamen praedicabile: igitur sub ratione praedicabilitatis continentur omnia hic tractanda. Respondeo agit de individuo formaliter, nego; terminative, concedo. Cognitio individui conducit ad cognitionem speciei, quapropter de \textnormal{|}\ledsidenote{BNC 132vb} eo agere necesse est. Sic agit de aliis, quae non sunt universali, materialiter et per accidens. 
\pend

\pstart
  Obiectio 4. Porphyrius hic non agit peculiariter de universalibus: igitur universale non est obiectum adaequatum. Respondeo de universali in communi non agit, quia potius supponit illud agitatum. Agit tamen de universali in particulari, id est de praedicabilibus. 
\pend

\pstart
  In fine nota discrimen inter praedicatum praedicabile et praedicamentum. Praedicamentum alicuius summi generis et eorum quae sub ipso sunt naturalis dispositio. De hoc agemus suo loco. 
\pend

\pstart
  Praedicatum est id quod de aliquo affirmatur vel negatur. Praedicabile vero est unum aptum praedicari de pluribus. Trifariam accipitur praedicatum. Primo in tota sua latitudine prout compraehendit non solum univoca, verum etiam aequivoca, complexa, analoga, singulare, individuum et omne quod quomodolibet praedicatur. Secundo, in quantum pertinet ad quaestiones dialecticas et probabiles continentes ea de qua cunque re interroganda, quae sunt quatuor: genus, definitio, proprium et accidens. Tertio quatenus praedicatum est actus praedicabilis et hoc modo praedicatum est praedicabile in actu. 
\pend

\pstart
  Inferes denique praedicatum et praedicabile differre per actum et potentiam, a praedicamento vero quia praedicamentum est series praedicabilium. Hucusque egimus de prologo Porphyrii; quia licet ille non expedite et ex professo has quaestiones tractaverit, nihilominus omnibus Doctores placet eas agitare. Restat nunc specialiter omnia praedicabilia examinare legendo Porphyrii textum et resolvendo difficultates hoc tempore hesitatas iuxta eius interpretem fidum Divum \textsc{Thomam}\index[persons]{Thomas Aquinas} 
\pend

        \addcontentsline{toc}{chapter}{Capitulum primum}
        \pstart
        \eledchapter*{\supplied{Capitulum primum}}
        \pend
      
        \addcontentsline{toc}{section}{Textus capituli primi Porphyrii}
        \pstart
        \eledsection*{Textus capituli primi Porphyrii}
        \pend
      
\pstart
  \textnormal{|}\ledsidenote{BNC 133rb} \edtext{\enquote{ ΚΕΦΑΛΑΙΟΝ Α Ἔοικε δὲ μήτε τὸ γένος μήτε τὸ εἶδος ἁπλμῶς λέγεσθαι. Γένος γὰρ λέγεται καὶ ἡ τινῶν ἐχόντων τως πρὸς ἕν τι καὶ πρὸσ ἀλλήλους ἄθροισις. }}{\lemma{}\Afootnote[nosep]{ \textsc{Commentarii Collegii Conimbricensis e Societate Iesu}\index[persons]{}, \worktitle{In universam dialecticam Aristotelis Stagirita} (Lugduni: sumpt. Iacobi Cardon et Petri Cavellat, 1622), p. 94. }} 
\pend

\pstart
 \edtext{\enquote{ \textnormal{|}\ledsidenote{BNC 133rc} Caput primum Videtur itaque neque genus, neque species dici simpliciter. Dicitur enim genus, et aliquorum hominum aggregatio qui quidem ad unum qui propriam, et inter se habent respectum. Etc. }}{\lemma{}\Afootnote[nosep]{ \textsc{Commentarii Collegii Conimbricensis e Societate Iesu}\index[persons]{}, \worktitle{In universam dialecticam Aristotelis Stagirita} (Lugduni: sumpt. Iacobi Cardon et Petri Cavellat, 1622), p. 94. }} 
\pend

        \addcontentsline{toc}{section}{Summa Textus}
        \pstart
        \eledsection*{Summa Textus}
        \pend
      
\pstart
\noindent%
  \textnormal{|}\ledsidenote{BNC 133ra} Duas partes amplectitur hoc caput. In prima numerat Porphyrius tres generis acceptiones. In secunda definit genus et probat suam definitionem esse rectam. Quoad primum genus primo sumitur per collectione multorum, qui inter se sunt cognatione coniuncti trahitur ab uno progenitore vel communi Patria ut genus Heraclidarum ab Hercule; genus Romanorum a Romulo. Secundo sumitur pro principio a quo aliquos trahit originem sive sit Pater, sive Patria; qua ratione dicitur Orestes trahere principium ab Tantalo, qui fuit eius atauus. Sic dicitur Plato Atheniensis et Pindarus Thebanus. Haec secunda acceptio est magis propria, ex ea quae derivatur prima. Tertio sumitur genus qua praedicabile, ideoque appellatur genus Philosophorum, id est logicum. In secunda parte genus in tertia acceptione sumptum definit definitione \edtext{ \name{\textsc{Aristotelis}\index[persons]{Aristoteles}} 1 \worktitle{Topicorum}, capitulo 4 \enquote{genus est quod praedicatur de pluribus differentibus specie in eo, quod quid.} }{\lemma{}\Afootnote[nosep]{}} Probat esse exactam: nam per hic de pluribus distinguitur ab individuo. Per hic differentibus specie distinguitur a specie. Per hic in eo quod quid differt ab differentia proprio et accidenti. 
\pend

        \addcontentsline{toc}{section}{Annotationes circa Litteram Capitis}
        \pstart
        \eledsection*{Annotationes circa Litteram Capitis}
        \pend
      
\pstart
\noindent%
  \textnormal{|}\ledsidenote{BNC 133rb} Circam primam parte capituli venit in duabus cur Porphyrius non numerat omnes generis acceptiones. \textnormal{|}\ledsidenote{BNC 133rc} Siquidem non numerat genus grammaticum divisum in masculino feminino, etc.; nec rhetoricum divisum \textnormal{|}\ledsidenote{BNC 133va}   in demonstrativum, deliberativum et iudiciale; et similiter ceteras acceptiones diverse receptus ab Iurisperitis. Dico igitur Porphyrius solum numerasse acceptiones cum genere logico aliquam habentes similitudinem et conducentes ad illud cognoscendum. Circa secundam partem explicandae veniunt partes definitionis, sed hoc in disputatione. Nunc vero. 
\pend

        \addcontentsline{toc}{section}{Quaestio incidens. An methodus observata a Porphyrio in tradendis praedicabilibus sit optima?}
        \pstart
        \eledsection*{Quaestio incidens. An methodus observata a Porphyrio in tradendis praedicabilibus sit optima?}
        \pend
      
\pstart
  Communiter a Doctores dubitatur de methodo circa quodlibet praedicabile ne autem saepius dubitemus, dubitemus igitur communiter et resolvamus generaliter. Pro quo suppono quod quondam dixi ex \textsc{Aristotele}\index[persons]{Aristoteles}, scilicet methodum compositionis praecipue quando proceditur a simplicioribus ad composita et ab universalioribus ad minus universalia aptiorem esse pro disciplinis tradendis. Eadem methodo utimur quando procedimus a causa ad effectum 
\pend

\pstart
  Sciendum est etiam ex eodem \textsc{Aristotele }\index[persons]{Aristoteles} in \worktitle{Postpraedicamenta} aliquid prius esse alio tripliciter. Primo, prioritate temporis sic Pater prior est filio. Secundo, prioritate naturae quando nempe natura unius prior est natura alterius, sic superiora sunt priora inferioribus et hac etiam causa est prior effectu. Tertio, prioritate dignitatis, qua honorabiliores et praepositi dicuntur priores. Relicta prioritate temporis ac etiam dignitatis de secunda loquemur. 
\pend

\pstart
  Sit conclusio: methodus a \textsc{Porphyrio}\index[persons]{Porphyrius} observata est optima. Sic omnes Doctores concorditer. Probatur ratione. Ex \textsc{Aristotele}\index[persons]{Aristoteles} methodus compos aptior est \textnormal{|}\ledsidenote{BNC 133vb} ad tradendas disciplinas, sed hanc observavit \textsc{Porphyrius}\index[persons]{Porphyrius} in tradendis praedicabilibus: igitur eius methodus est optima. Probatur minor. Substantia est prior natura accidentibus. Ex \name{\textsc{Aristotele}\index[persons]{Aristoteles}} VII \worktitle{Metaphysicae}, capitulo 1: \edtext{\enquote{nam substantia habet respectu accidentium rationem causae}}{\lemma{}\Afootnote[nosep]{}}; igitur contra methodum compositis prius agendum erat de praedicabilibus significatis per modum substantiae, quam de praedicabilibus significatis per modum accidentis. Sed Porphyrius prius egit de significatis per modum substantiae, scilicet de genere, specie et differentia; quam de significatis per modum accidentis, scilicet de proprio et accidenti: igitur observavit methodum compositionis. 
\pend

\pstart
  Rursus in tradendis prioribus significatis per modum substantiae eamdem methodo observat: igitur. Probatur antecedens. Inter praedicabilia essentialia priora sunt quae significantur in quid, id est per modum substantiae; quam quae significantur in quale, id est per modum alteri adiacentis. Sed genus et species se habent in quid, differentia vero in quale: igitur prius agendum erat de genere et specie quam de differentia. Atque \textsc{Porphyrius}\index[persons]{Porphyrius} sic agit: igitur etiam in istis procedit methodo recta. 
\pend

\pstart
  Rursus etiam respectu generis et species procedit hac methodo compositionis: igitur. Probatur antecedens. Quando prius agitur de simpliciori et universaliori, quam de composito et minus universali; agitur proculdubio methodo compositionis. Sed quando \textsc{Porphyrius}\index[persons]{Porphyrius} prius agit de genere quam de specie, prius agit de simpliciori et universaliori, quam de composito et minus universali: igitur tunc agit methodo compositionis. Minor est certissima et non indiget probatione. 
\pend

\pstart
  Denique etiam respectu proprii et accidentis agit hac methodo: igitur. Probatur antecedens. \textnormal{|}\ledsidenote{BNC 134ra}   In accidentibus propria passio immediatius tangit essentiam quam accidens commune: igitur prius agendum erat contra methodum compositis de proprio quam de accidenti. Sed sic agit \textsc{Porphyrius}\index[persons]{Porphyrius}: igitur 
\pend

\pstart
  Obiectio 1. \textsc{Porphyrius}\index[persons]{Porphyrius} promisit in proemio se acturum prius de differentia quam de specie, sed hoc non servat: igitur male procedit. 
\pend

\pstart
  Ut argumento satisfaciam noto: quando duae rationes se impediunt intra idem genus methodi, utendum est de methodo arbitraria, sic docentibus Doctoribus. Ex una parte methodo compositionis prius agendum erat de differentia quam de specie, scilicet quia differentia est simplicior specie et eius pars. Ex alia parte eadem methodo prius agendum erat de specie quam de differentia, scilicet quia species significatur in quid, differentia autem in quale: igitur stantibus his se impedientibus intra idem genus methodi, scilicet compositionis, debuit Porphyrium uti methodo arbitraria. Quapropter promisit se acturum methodo compositionis respectu specie et differentiae. Docens hanc methodum sequendam esse in tradendis disciplinis. Postea est invertit illam et utitur arbitraria. Docens etiam primam methodum suadentibus rationibus aliquando relinquendam esse. 
\pend

\pstart
  Instabis: methodus arbitraria est voluntaria: igitur \textsc{Porphyrius}\index[persons]{Porphyrius} non procedit recte. Respondeo negando consequentiam. Methodo arbitraria est perfectissima, non destructiva artis, sed potius perfectiva. Sicut enim virtus epikieus est summa iustitia et non contra iustitiam quia corrigens verba legis contrarium percipientis est iuxta mentem legislatoris quandoquidem si adverteret circumstantias occurrentes perciperet \textnormal{|}\ledsidenote{BNC 134rb} oppositum. Similiter methodus arbitraria corrigit ceteras propter rationes occurrentes: igitur tunc est perfectissima. 
\pend

\pstart
  Cur tamen Porphyrius methodo arbitraria usus prius egit de specie quam de differentia? Primo quia genus et species sunt correlativa, ideoque cognitio unius est conexa cum alterius cognitione: igitur licuit haec praedicabilia coniungi. Sic Beatus \textsc{Albertus}\index[persons]{Albertus Magnus} et \edtext{\name{\textsc{Boethius}\index[persons]{Boethius}} secundo capitulo \worktitle{De genere} fit mentio de specie }{\lemma{}\Afootnote[nosep]{}}, igitur quantocius tractanda est species, ne detur suspensio. Vide alias rationes in Auctoribus aliis. 
\pend

\pstart
  Obiectio 2. Differentia est nobilior genere. Nam actus nobilior est potentia: igitur est prior illo. Nam nobiliora sunt priora. Respondeo negando consequentiam et distinguendo probationem subsumptam. Nobiliora sunt priora prioritate dignitatis, concedo; naturae, nego. Differentia est prior genere prioritate dignitatis, quae non conducit ad doctrinam. At genus est prius natura, quae conducit; vide numero 567. 
\pend

\pstart
  Obiectio 3. Prius agendum est de parte, quam de toto; sed differentia est pars, species vero totum: igitur prius agendum erat de differentia, quam de specie. Respondeo distinguendo maiorem. Methodo compositionis, concedo; methodo arbitraria, nego. Hoc manet iam satis explicatum. 
\pend

\pstart
  Obiectio 4. A facilioribus est incipiendum, sed \secluded{sed} accidentia communia sunt faciliora. Siquidem sunt sensibus nota: igitur. Respondeo negando maiorem. Non semper licet a facilioribus incipere. Nam in scientiis incipimus a Logica, quae est omnibus difficilior. Potest tamen distinguere minorem. Accidentia communia sunt faciliora cognitione sensus, concedo; cognitione intellectus, nego. Cognitione intellectus universaliora sunt faciliora ut docet  \edtext{ \name{\textsc{Aristoteles}\index[persons]{Aristoteles}} primo \worktitle{Physicorum} capitulo 1 }{\lemma{}\Afootnote[nosep]{}}. 
\pend

        \addcontentsline{toc}{section}{Quaestio 8. De natura et proprietatibus generis}
        \pstart
        \eledsection*{Quaestio 8. De natura et proprietatibus generis}
        \pend
      
\pstart
 Plura supponimus omnibus praedicabilibus communia, sed etiam plura singulis peculiaria restant examinari. Primo igitur agendum est de ipsa natura generis, de eius proprietatibus, de definitione et definitio tam generis, quam ceterorum praedicabilium, imo omniumque accidentium. Pro quo: Articulus 1 Utrum, etc. 
\pend

        \addcontentsline{toc}{section}{Articulus 1. Utrum definitio generis Porphyrii sit exacta, sitque essentialis vel descriptiva?}
        \pstart
        \eledsection*{Articulus 1. Utrum definitio generis Porphyrii sit exacta, sitque essentialis vel descriptiva?}
        \pend
      
\pstart
  Haec est definitio generis: \edtext{\enquote{ Genus est quod praedicatur de pluruibus specie differentibus in eo quod quid. }}{\lemma{}\Afootnote[nosep]{}} Hanc definitionem dat Porphyrium et admittunt omnes Doctores. Sed ut bene intelligatur, noto differentiam inter quid et quale. Quid est relativum substantivum seu absolutum: igitur significat per modum substantiae et per se statis. Quale est relativum relativum seu connotativum: igitur significat per modum accidentis seu alteri adiacentis. 
\pend

\pstart
  Unde ad quaestionem quod respondetur nominibus substantivis generis, scilicet et speciei verbi gratia `quod est homo?' Respondeo est animal, est homo. Siquidem genus et species significant per modus substantiae. Ad quaestionem est quale, respondetur nominibus adiectivis, scilicet differentiae proprii et accidentis, verbi gratia `quale animal est homo?' Respondeo est animal rationale, nisi vite, album. Siquidem haec tria significant per modum accidentis et alteri adiacentis. 
\pend

\pstart
  \textnormal{|}\ledsidenote{BNC 134vb} Cum enim differentia pertineat ad substantiam rei non omnino significatur in quale et cum significetur ut alteri adiacens non omnino significatur in quid. Quomodo igitur? In quale quid, hoc est quoad rem significatam in quid; quoad modum significandi in quale. Ceterum proprium et accidens tam quoad rem significatam quam quo ad modum significandi significantur in quale. 
\pend

\pstart
  In summa genus et species praedicantur omnino in quid; proprius et accidens omnino in quale; et differentia in quale quid. Etiam genus non eodem modo praedicatur in quid ac species. Sed praedicatur de pluribus specie differentibus; species est de pluribus differentibus numero. Quod est dicere genus praedicatur in quid incomplete; species vero complete. Quod sit idem ratio sequens suadet: genus est essentialiter perfectibile: igitur continet sub se plures species. Igitur praedicatur de pluribus specie in quid. Sed ratione qua est perfectibile significatur in quid incomplete: igitur qua praedicatur de pluribus specie differentibus praedicatur in quid incomplete. Similiter species est essentialiter perfecta et constituta: igitur sub se non continet plura essentialiter distincta, sed solum materialiter et numero. Igitur praedicatur in quod de pluribus numero distinctis. Sed ratione qua est perfecta significatur in quid complete: igitur qua praedicatur de pluribus numero distinctis praedicatur in quid complete. 
\pend

\pstart
  Sit conclusio: definitio generis est exacta. Eam docet \edtext{ \name{\textsc{Aristoteles}\index[persons]{Aristoteles}} libro 1 \worktitle{Topicorum} capitulo 4 et libro 4, capitulo 1, 2; et 5 \worktitle{Metaphysicae} capitulo 28 }{\lemma{}\Afootnote[nosep]{}}. Quam etiam admittit \edtext{ \name{\textsc{Angelicum Doctorem}\index[persons]{Thomas Aquinas}} \worktitle{Opusculo 18}, tractato 1, capitulo 2 et in 2, distinctione 34, questione 1, articulo 2, ad 1 }{\lemma{}\Afootnote[nosep]{}}; est communis inter Dialecticos utrique scholae. 
\pend

\pstart
  \textnormal{|}\ledsidenote{BNC 135ra}   Probatur ratione. Illa est bona rectaque definitio quae explicat essentiam rei, facit quae illam differre ab aliis; sed huiusmodi est haec definitio: igitur est recta et exacta. Probatur minor. Genus convenit cum ceteris praedicabilibus per hic quod praedicatur; differt a specie per hic differentibus specie; differt a differentia, proprio et accidenti per hic in quid: igitur haec definitio explicat essentiam generis et facit illud differre a reliquis. 
\pend

\pstart
  Circa secundum punctum in titulo attactum prima sententia est \edtext{ \name{\textsc{Scoti}\index[persons]{Ioannes Duns Scotus}} docentis esse essentialem hic quaestione 14 et 15 }{\lemma{}\Afootnote[nosep]{}} quae sequuntur omnes eius discipuli ac etiam \edtext{ \name{\textsc{Hurtado}\index[persons]{}} disputatione 4, sectione 1  }{\lemma{}\Afootnote[nosep]{}}. Secunda est huic opposita et est communis. 
\pend

\pstart
  Sit conclusio: haec definitio non est essentialis, sed descriptiva. Ita \textsc{Porphyrius}\index[persons]{Porphyrius}; \edtext{ Beatus \name{\textsc{Albertus}\index[persons]{Albertus Magnus}} \worktitle{tractato 3, capitulo 34} }{\lemma{}\Afootnote[nosep]{}} et \edtext{ Doctor \name{\textsc{Thomas}\index[persons]{Thomas Aquinas}} \worktitle{Opusculo 48} tractato 1, capitulo 2 }{\lemma{}\Afootnote[nosep]{}}, quae sequuntur omnes eius discipuli sicut testatur \edtext{ \name{\textsc{Gallego}\index[persons]{Baranbas Gallego de Vera}} \worktitle{Controversia 13} }{\lemma{}\Afootnote[nosep]{}}. Probatur ratione. Illa definitio non est essentialis, sed descriptiva quae traditur non per essentiam, sed per proprietatem. Sed talis definitio traditur non per essentiam, sed proprietatem: igitur talis definitio non essentialis, sed descriptiva est appellanda. Minor constat ex dictis. Siquidem essentiam cuiusvis praedicabilis est universalitas, passio vero praedicabilitas. 
\pend

\pstart
  Obiectio 1. Quidquid definitur est species, sed genus non est species: igitur non definitur. Probatur minor. Genus est distinctum praedicabile a specie: igitur genus non est species. Respondeo concedendo maiorem et distinguendo minorem. Genus non est species secundae intentionaliter, concedo; tertiae intentionaliter, nego. Et similiter ad probationem. 
\pend

\pstart
  Instabis: haec definitio convenit generi in quantum generi, sed genus in quantum genus non est species: \textnormal{|}\ledsidenote{BNC 135rb} igitur in quantum convenit definitio non est genus. Respondeo concedendo maiorem et distinguendo minorem. Genus in quantum genus non est species secundae intentionaliter, concedo; tertiae intentionaliter, nego. 
\pend

\pstart
  Urgebis: si genus definiretur ver per se vel per aliud, non per se, nam sic esset definitum et definitionis pars. Igitur per aliud. Hoc vel definitur per se vel per aliud, non per se ob rationem praefatam: igitur per aliud de quo potest etiam idem quaeri. Igitur ne detur processus in infinitum, dicendum est genus non definiri. 
\pend

\pstart
  Respondeo distinguendo maiorem. Per aliud secunda intentionaliter, nego; tertia intentionaliter, concedo. Similiter ad probationem. Genus ut sic definitur per aliud genus particulare, sicut actus intellectus ut si definitur per actum intellectus particularem. Quod genus particulare est universale ut sic tertio intentionaliter. Si quaeras per quod definitur hoc genus particulare vel quaeris de eo qua genus est et sic manes satisfactus conclusione vel quaeris de eo qua particulare est, scilicet universale. Dico definiri per aliud genus particulare et hoc per aliud usquedum veniamus ad supremum genus illius categoriae, quod nequit definiri, cum non habeat aliud genus supra se. 
\pend

\pstart
  Replicabis: genus ut sic definitur per genus particulare: igitur hoc genus particulare est superius et inferius genere ut sic. Probatur sequela. Genus est superius definitio, sed genus ut sic est definitum: igitur genus particulare est superius genere ut sic. Item omne genus particulare est inferius genere ut sic, sed tale genus est particulare: igitur. 
\pend

\pstart
  Respondeo distinguendo consequens. Est superius est inferius respectu eiusdem intentionis, nego; respectu diversarum intentionum, concedo. Diversitas explicatur in solutionibus. \textnormal{|}\ledsidenote{BNC 135va}   Ad probationem sequelam distinguo consequens. Genus particulare est superius genere ut sic tertio intentionaliter, concedo; secundo intentionaliter, nego. Tertio intentionaliter genus de materiali est genus, de formali vero est species. Ad aliam probationem distinguo consequens. Est inferius genere ut sic secundo intentionaliter, concedo; tertio intentionaliter, nego. Solutio manet explicata supra. 
\pend

\pstart
  Obiectio 2. Haec definitio convenit aliis a definitio: igitur non est bona. Probatur antecedens. Convenit definitioni: igitur. Probatur antecedens. Definitio alis, scilicet vivens sensibile praedicatur de pluribus differentibus specie, scilicet de homine et equo et praedicatur in eo quod quid: igitur convenit definitioni. Respondeo negando antecedens. Ad probationem etiam nego antecedens. Ad probationem distinguo antecedens. Definitio praedicatur complexe, concedo; incomplexe et simpliciter, nego. Definitio manet exclusa per hic quod denotans naturam unam simpliciter univocam. 
\pend

\pstart
  Obiectio 3. Genus supremum ut est substantia et genus intermedium ut est vivens praedicantur de differentibus genere ut constat: igitur male definitur per praedicari de differentibus specie. Respondeo concedendo antecedens et negando consequentiam. Verum est genus supremum et intermedium praedicari de differentibus genere et etiam de differentibus specie. Cumque hoc ultimum conveniat omni generi et primum non nisi supremo et intermedio. Ideo definitur per hoc ultimum. Sic aliqui. 
\pend

\pstart
  Sed melius nego antecedens. Pono praedicationes `corpus est substantia', `spiritus est substantia'. Corpus enim (et spiritus) constituatur dupliciter, scilicet quatenus dicit respectum ad sua inferiora et sic dicitur genus intermedium et quatenus dicit ordinem ad suum praedicatum supremum et sic est species intermedia. Primo modo constituitur praedicabile; secundo modo \textnormal{|}\ledsidenote{BNC 135vb} subiscibile. Cumque genus supremum, scilicet substantia non comparetur ad corpus qua praedicabile, sed qua subiicibile non sequitur praedicari de illo qua genere, sed qua specie. 
\pend

\pstart
  Obiectio 4. Etiam genus praedicatur de pluribus numero distinctis. Verbi gratia `Petrus est animal', `Paulus est animal': igitur male definitur per praedicari de pluribus specie differentibus. Respondeo distinguendo antecedens. Praedicatur de pluribus numero distinctis immediate, nego; mediate, concedo. Genus enim praedicatur immediate de speciebus et his mediantibus de individuis. 
\pend

\pstart
  Obiectio 5. Accidens potest esse genus. Verbi gratia color ad albedinem et nigredinem. Sed accidens praedicatur in quale et extra essentiam: igitur genus praedicatur in quale et non in eo quod quid. Respondeo concedendo maiorem et distinguendo minorem. Praedicatur in quale respectu inferiorum quae constituit, nego; respectu inferiorum quae denominat, concedo. Si comparemus genus accidens ad sua inferiora est genus praedicatur qui in eo quod quid, non minus quam animal respectu hominis et bruti. At respectu subiectorum quae denominat non est genus, sed acciens praedicabile et extra essentiam. 
\pend

\pstart
  Obiectio 6. Etiam ens praedicatur de pluribus differentibus specie in eo quod quid praedicatur, namque de homine, equo et ceteris, sed non es genus: igitur genus non ita debet definiri. Item idem argumentum stat per anima respectu vegetative, sensitive et rationalis. 
\pend

\pstart
  Respondeo distinguendo maiorem. ens praedicatur univoce, nego; analogice, concedo. Ad aliud distinguo. Anima praedicatur complete, nego; incomplete, concedo. Igitur hic quod positum in definitione significat naturam unam simpliciter, id est univocam, unde excluduntur analoga. Item completam, unde excluduntur incompleta. Item incomplexam, unde excluduntur complexa, sicut de definitione dictam est. 
\pend

\pstart
  \textnormal{|}\ledsidenote{BNC 136ra}   Obiectio 7. Substantiae secundae significant in quale quid ex \edtext{ \name{\textsc{Philosopho}\index[persons]{Aristoteles}} capitulo \worktitle{de substantia} }{\lemma{}\Afootnote[nosep]{}}. Sed genus est substantia secunda ut ibidem docet \textsc{Philosophus}\index[persons]{Aristoteles}: igitur genus praedicatur in quale quid: igitur non in eo quod quid. 
\pend

\pstart
  Respondeo explicando \textsc{Philosophum}\index[persons]{Aristoteles} cum \edtext{ \name{\textsc{Angelico Doctore}\index[persons]{Thomas Aquinas}} 7 \worktitle{Metaphysicae} lectione 13 }{\lemma{}\Afootnote[nosep]{}}. Quemadmodum omnis forma dicitur qualitas materiae, quia in illa recipitur et sustentatur. Ita etiam praedicatur in proportione dicitur qualitas subiecti habet enim rationem formae respectu illius ei qui adiacet. Unde in hoc generali sensu ait \textsc{Aristoteles}\index[persons]{Aristoteles} primas substantias significare aliquid, secundas vero quale quid. Hoc est primae substantiae (individua seu supposita) semper habent rationem materiae et subiecti respectu secundarum. Secundae vero (genus, species, etc.) semper habent rationem formae et praedicati respectu primarum. Sed hoc non tollit quominus secundae substantiae praedicentur per modum per se stantis seu in eo quod quid: igitur genus etiam hoc posita ita praedicatur. 
\pend

\pstart
  Obiectio 8. Etiam differentia praedicatur in eo quod quid: igitur male definitur. Probatur antecedens. Modus praedicandi in quid est perfectior modo praedicandi in quale: igitur etiam differentia, etc. Probatur consequentia. Perfectiorem modum essendi sequitur perfectior modus praedicandi, sed differentia habet perfectiorem modum essendi, quam genus. Siquidem hoc est potentia et illa actus: igitur habet perfectiorem modum praedicandi. 
\pend

\pstart
  Respondeo negando antecedens. Ad probationem concedo antecedens et nego consequentiam. Ad probationem distinguo maiorem. Perfectiorem modum essendi secundum se sequitur perfectior modus praedicandi, nego; perfectiorem modum essendi in intellectu, etc., concedo. Et distinguo minorem. Differentia habet perfectiorem modum essendi secundum se, concedo; prout concipitur et significatur, nego. Quamvis differentia secundum se habeat perfectiorem modum \textnormal{|}\ledsidenote{BNC 136rb} essendi quam genus ratione adducta in argumento. Tamen quia a nobis concipitur ad modum accidentis seu qualitatis adiacentis. Genus vero ad modum substantiae et subiecti recipientis illam, ideo secundum modum concipiendi imperfectiorem modum essendi habet quam genus. Cum enim modus perfectior praedicandi sequatur modum praedicandi perfectiorem, ideo genus habet perfectiorem modum praedicandi. 
\pend

\pstart
  Obiectio 9. Haec definitio continet aliquid non spectans ad essentiam generis: igitur. Probatur antecedens. Continet illa de quibus praedicatur, sed haec non spectant ad essentiam generis: igitur continet aliquid, etc. Respondeo negando antecedens. Ad probationem distinguo maiorem. Continet illa in obliquo, concedo; in recto, nego. Relativum enim non potest definiri nisi connotando terminum, cumque genus sit relativum. Ideo in eius definitione connotat ea de quibus praedicatur. 
\pend

\pstart
  Obiectio 10. Differentia intermedia ut sentiens praedicatur in quid de differentibus specie, sed non est genus: igitur genus non praedicatur in quid, etc. Respondeo distinguendo maiorem. Praedicatur in quale quid, concedo; praedicatur in eo quod quid, nego. Genus differt a differentia per hic in eo quod et differentia in eo quod praedicatur in quale. Hoc est genus praedicatur essentialiter ut pars materialis, differentia essentialiter ut pars formalis. 
\pend

\pstart
  Contra secundam conclusionem falsum tenet fundamentum contrarii, scilicet obiectum adaequatum huius libri esse potius praedicabile quam universale. Cum contrarium maneat probatum quaestione 5, articulo 12; nec aliud est alicuius momenti ut occurramus ei. Alia argumenta fiunt, sed ex dictis manent soluta solum observandum est eadem argumenta mutatis mutandis fieri in omnibus definitionibus aliorum praedicabilium et servatis servandis resolvuntur dicto modo. 
\pend

        \addcontentsline{toc}{section}{Articulus 2. Traditur ars definitiva accidentium tum concretorum tum abstractorum}
        \pstart
        \eledsection*{Articulus 2. Traditur ars definitiva accidentium tum concretorum tum abstractorum}
        \pend
      
\pstart
  Dictum est concreta et abstracta significare eamdem rem, differreque secundum modum significandi numero 388 usque ad 391. Sequitur nunc prima nomina non significare res absolute et secundum se. Quomodo? Audito  Divum \name{\textsc{Thomam}\index[persons]{Thomas Aquinas}} 7 \worktitle{Metaphysicae} lectione 11: \edtext{\enquote{Nec modus significandi consequitur modum essendi rerum immediate sed mediante modo intelligendi}}{\lemma{}\Afootnote[nosep]{}}. Ergo in nominibus potior est et immediatior modus significandi modo essendi. Sed concretum accidentale quo ad modum significandi est ens per accidens: igitur cum potius attendatur modo significandi, quam essendi agendum est de concreto accidentali ut de ente per accidens. 
\pend

\pstart
  Sequitur secunda. Accidens in concreto non rigorose definiri bene tamen in abstracto. Qua ratione? Hac: quod definitur debet esse ens per se, non solum quoad rem significatam, sed etiam quoad modum significandi (definimus res nominibus significatas; in nominibus potius attenditur modus significandi). Sed accidens in concreto tractatur ut ens per accidens in concreto tractatur ut ens per accidens et non per se: igitur accidens in concreto non definitur rigorose. Sed accidens in abstracto tam quoad rem significatam quam quo ad modum significandi est ens per se: igitur accidens in abstracto definitur stricto et rigorose. 
\pend

\pstart
  Ast licet accientia concreta non ita rigorose definiantur tamen aliquomodo \textnormal{|}\ledsidenote{BNC 136vb} definiuntur. Sed observato quod sicut concreta accidentium ex modo significandi nec sunt entia per se, nec denominatur genera species, etc., nec ponuntur in linea praedicamentali; ita pariter eorum definitiones non sunt proprie definitiones habentes bonae \emph{definitionis [?]} leges, sed tantum reductive ad sua abstracta. Vide numero 402. Relinquitur: igitur quod sicut concretum et abstractum accidentis significant eamdem rem differuntque quoad modum significandi. Sic eorum definitiones conveniunt quoad rem definitam differuntque quoad modum definiendi penes perfectum et imperfectum. 
\pend

\pstart
  Regula generalis definiendi accidentia tum concreta, tum abstracta haec est: ut proprie perfecte et essentialiter definiatur accidens in abstracto debet poni pro genere praedicatum seu gradus superior inventus in eius essentia et loco differentiae eius proprium subiectum in obliquo significatum. Ut a proprie et essentialiter (modo dicto) definiatur accidens in concreto debet poni loco generis eius proprium subiectum in recto, et loco differentiae gradus superior etiam in recto et adiective significatus. Colligitur ex \edtext{ Doctore \name{\textsc{Thoma}\index[persons]{Thomas Aquinas}} 7 \worktitle{Metaphysicae} lectione 1 et 4 et \worktitle{De sensu et sensato}, lectione 6 et \worktitle{Opusculo 46}, capitulo 1, \worktitle{Opusculo 42}, capitulo 19 et quaestio 3 \worktitle{De Veritate}, articulo 7, ad 2 et 12, questione 53, articulo 2 ad 3 et capitulo ultimo \worktitle{De ente et essentia} }{\lemma{}\Afootnote[nosep]{}}. 
\pend

\pstart
  Dubitas. Quid intelligitur per subiectum proprium accidentis? Non substantia, nam haec cum sit communis pro differentia non posset poni. Non aliud correspondens cuilibet accidentium, incredibile est namque quodlibet accidens habere \textnormal{|}\ledsidenote{BNC 137ra}   subiectum proprium a subiecto alius distinctum. Optime dubitas, optime ratiozinaris. Respondeo obiter sicut certum est quodlibet accidens dependere a substantia, sic est certum quodlibet habere suum proprium subiectum principiumve a quo causatur. 
\pend

\pstart
  Probet metaphysica. Ita intime actus et potentia et essentialiter sibi correspondent ut implicet potentia non respiciens per se aliquam actum ipsam constituentem definientemque ex \edtext{ \name{\textsc{Aristotele}\index[persons]{Aristoteles}} 9 \worktitle{Metaphysicae} texto 13 }{\lemma{}\Afootnote[nosep]{}} et \edtext{ \name{\textsc{Caietanus}\index[persons]{Thomas de Vio Caietanu}} tomo 3, \worktitle{Opusculo}, tractato 3, questione 1 }{\lemma{}\Afootnote[nosep]{}}, igitur econtra repugnat actus cui non correspondeat aliqua potentia ordine ad quam cognoscatur et definiatur. Sed quodlibet accidens est actus secundum quid: igitur implicat illi non correspondere suam potentiam, suum subiectum. 
\pend

\pstart
  Confirmatur: non est assignabile accidens in naturalibus ab aliquo substantiae gradu generico vel specifico vel individuali non procedens: igitur accidens quodlibet intra substantiam habet causam adaequatam sui. Sed eadem est causa efficiens et materialis accidentis: igitur sicut habet causam adaequatam a qua emanat. Ita habet causam materialem subiectumve in quo est. Nec obstat ab eodem substantiae gradu plura accidentia oriri, cum ordine et dependentia hoc fiat. Quemadmodum substantia cum primo accidenti habet rationem principii adaequati respectu sequentis, ita habet rationem subiecti. 
\pend

\pstart
  Solum restat notare idem quod \edtext{ Divus \name{\textsc{Thomas}\index[persons]{Thomas Aquinas}} loco citato \worktitle{De ente et essentia} }{\lemma{}\Afootnote[nosep]{}}, scilicet saepe nos definire accidentia per proprietates vel effectus et non observando praefatam regulam, non quia insufficiens, sed defectu capacitatis non \textnormal{|}\ledsidenote{BNC 137rb} attingentis principia propria accidentium, non semper nobis nota 
\pend

\pstart
  Audito nunc furibundum argumentum. Impossibile est in eadem re definitum convenire cum aliis et differre ab illis, sed genus es modo conveniendi, differentia vero differendi: igitur est impossibile tam superiorem gradum quam subiectum poni ut genus in una definitione et ut differentia in alia. Quomodo igitur curvitas esse potest genus simitatis et differentia simi? Similiter quomodo nasus potest esse genus simi et simitatis differentia? 
\pend

\pstart
  Respondeo distinguendo maiorem. Eodem modo, concedo; diversomodo, nego. Et distinguo consequens. Est impossibile gradum superiorem seu subiectum poni ut genus simpliciter in una definitione et ut differentia secundum quid in alia vel econtra homo. Simpliciter ut genus et ut differentia in utraque, concedo. Supponimus accidens in abstracto recte definiri et non in concreto: igitur in definitione abstracti quod ponitur genus proprie est genus et similiter differentia. At in definitione concreti sufficit poni quod secundum quid est genus similiter differentia propter dicta numero 608 et 609. 
\pend

\pstart
  Quod in na definitione est simpliciter genus non repugnat quod in alia sit secundum quid differentia et econtra respectu diversarum rerum. Saepe saepius contigit ut superius et divisum unius coordinationis et divisionis sit inferius et membrum dividens in alia coordinatione et divisione memento terminorum exedentium et exesorum. Videto exemplum. Dividatur album in homine et equum divisione accidentis in substantia, item dividatur homo in album et nigrum divisione subiecti in accidentia. \textnormal{|}\ledsidenote{BNC 137va}   In homo in prima divisione ut inferius et membrum dividens et in secunda ut superius et divisum. Ergo quamvis album exedat hominem in prima divisione, tamen ab eo exeditur in secunda: igitur prout sic potest esse differentia hominis albi a nigro. Pariter quamvis homo exedat album in secunda divisione, tamen ab eo exeditur in prima: igitur prout sic album potest esse genus. Silogizetur nunc in nostro casu. 
\pend

\pstart
  Quando enim definitur accidens in abstracto convenientia fit cum rebus eiusdem generis et prout sic ponitur per genere quod proprie et simpliciter est genus, et per differentia quod proprie et simpliciter est differentia. Videto exemplum positum, simitas est curvitas nasi. Simitas enim convenit in curvitate cum figuris eiusdem generis, scilicet cum gibositate et crespitudine; differt tamen quia est curvitas talis partis, scilicet nasi; aliae vero sunt curvitates scapullarum et capillorum. 
\pend

\pstart
  Ceterum quando definitur accidens in concreto convenientia fit cum rebus diversi generis, nempe quia insunt eidem subiecto. Quare ponitur pro genere subiectum in quo conveniunt et pro differentia quod pertinet ad genus uniuscuiusque. Revocetur exemplum. Simum est nasus curvus. Simitas enim sic convenit cum rectitudine, sine cum aquitinitate, quae sunt figurae diversi generis in hoc quod insunt eidem subiecto, scilicet naso. Distinguitur vero per id quod pertinet ad proprium genus, scilicet per rectitudinem aut curvitatem. Relinquitur igitur per nasum bene definiri loco generis et per curvitatem loco differentiae simum. His dictis meo videri facilis redditur via definitiva accidentium. Super sicut examinare. 
\pend

        \addcontentsline{toc}{section}{Articulus 3. An accidens in concreto recte definiatur per concretum superius loco generis?}
        \pstart
        \eledsection*{Articulus 3. An accidens in concreto recte definiatur per concretum superius loco generis?}
        \pend
      
\pstart
  Nonnumquam ignoramus subiectum accidentis et ipsum definire per suum effectum formalem aut passionem aliquando est necesse, tunc \secluded{tunc} enim procedit haec , an sicut haec est exacta definitio albedinis: albedo est color disgregativus visus. Ita haec sit etiam album est coloratum disgregativum visus. Relicta igitur opinione quorundam modernorum affirmativa. 
\pend

\pstart
  Sit conclusio: concretum accidentale in rigore sumptum non recte definitur per concretum superius loco generis. Unde in casu quaestionis ponendum esset per genere subiectis concreti superioris. Verbi gratia suppono subiectum coloris esse perspicuum et proprium subiectum albedinis non esse mihi notum. Ideoque definiendam esse per eius effectum, scilicet disgregare visum. Tunc sic definirem album: album est perspicuum disgregativum visus. 
\pend

\pstart
  Probatur ratione efficaci: subiectum debet intrare definitionem accidentis, sed in accidentis in concreto definitione non ingreditur loco differentiae: igitur loco generis. Probatur minor. Si loco generis ingrederetur iam idem modus esset significandi et definiendi concreti et abstracti, iam utraque definitio esset aeque perfecta. Iam utrumque aeque perfecte collocaretur in praedicamento, sed haec omnia sunt contra Doctorem \textsc{Thomam}\index[persons]{Thomas Aquinas}: igitur non ingreditur loco differentiae: igitur loco generis. 
\pend

\pstart
  Confirmatur: genus significatur in quod, sed concretum superius non significatur in quid: igitur non potest \textnormal{|}\ledsidenote{BNC 138ra}   esse genus. Probatur minor. Concretum superius significatur in quale: igitur non in quid. Probatur antecedens:. Connotativum significatur in quale, sed concretum superius est connotativum: igitur significatur in quale. Nota quod loquor de concreto ut concreto quo ad modum significandi. Nam si loquamur quo ad rem significatam, hoc est confundere omnia et terminis absoluti. 
\pend

\pstart
  Obiectio 1. Connotativum tantum significat formam licet connotet subiectum: igitur in eius definitione servandus est modus definiendi formam in abstracto cum connotatione est subiecti, sed forma in abstracto definitur per suum genus (albedo enim per colorem et non per corpus definitur). Igitur connotativum, scilicet album definiri debet per connotativum genericum, scilicet coloratum. 
\pend

\pstart
  Respondeo. Connotativum tantum significat formam sicut abstractum. Nego antecedens. Diverso modo subdistinguo modo quoad rem significatum, nego; quoad modum significandi, concedo. Cum definitio concreti non solum correspondeat rei significatae, sed potius modo significandi et hic sit longe diversus a modo significandi abstracti. Manifestum est non posse eodem modo utrumque definiri. Et hoc patet ex ipso argumento, scilicet connotativum significat formam et connotet subiectum, sed abstractum non significat formam, licet connotet subiectum: igitur non eodem modo significant: igitur neque eodem modo definiuntur 
\pend

\pstart
  Obiectio 2. Haec est bona definitio: albedo est color disgregativus visus. Igitur est perfecta haec: album est coloratum disgregativum visus. Patet consequentia. Quia sicut se habet genus ad formam in abstracto definitam, ita se habet genus in concreto ad formam in concreto. 
\pend

\pstart
  \textnormal{|}\ledsidenote{BNC 138rb} Respondeo negando consequentiam. Imo cum \textsc{Aristotele}\index[persons]{Aristoteles} et Doctore \textsc{Thoma}\index[persons]{Thomas Aquinas} debemus inferre oppositum, cum modo contrario sit procedendum in utraque definitione. Ad probationem negatur paritas. Res enim prout nominibus significatae definiuntur, sed modus significandi concreti est diversus et oppositus modo significandi abstracti. Igitur quamvis res significata sit eadem modus tamen definiendi est diversus. 
\pend

\pstart
  Obiectio 3. Non semper subiectum est univocum: igitur non semper subiectum est genus. Patet antecedens. Natura est subiectum genereitatis caeterarumque intentionum, sed natura non est subiectum univocum, cum sub ipsa compraehendatur omnia praedicamenta: igitur non semper subiectum est univocum. Respondeo negando consequentiam. Non est necessarium subiectum esse univocum. De hoc postea agemus. Articulus 4. Quid sit, etc. 
\pend

        \addcontentsline{toc}{section}{Articulus 4. Quid sit definitum in definitione generis et definitione generis et generaliter quaeritur de omnibus definitionibus}
        \pstart
        \eledsection*{Articulus 4. Quid sit definitum in definitione generis et definitione generis et generaliter quaeritur de omnibus definitionibus}
        \pend
      
\pstart
  In primis suppono primum non investigare quo nomine appellandum sit, quid definitione generis declaratur, nam esset quaerere \emph{esse quos [?]} fuerit Pater filiorum \textsc{Zebedaei}\index[persons]{Zebedaeus}? Quaerimus igitur cuinam ex significatis nomine generis competat definitio iam tradita et declarata? Nam in genere quinque possumus cognoscere, scilicet natura denominata secundum suam entitatem realem. Genereitas ipsa secundum se. Aggregatum seu resultans ex utraque. Item natura denominata in recto connotando secundam intentionem in obliquo. Tandem secunda intentio denominans in recto connotando naturam in obliquo. 
\pend

\pstart
  \textnormal{|}\ledsidenote{BNC 138va}  Suppono 2. Definitum esse duplex materiale, scilicet et formale. Huic primo et per se competit definitio. Illi vero per accidens, scilicet qua substernitur formae. Item definitum formale subdividitur in quo et quid. Definitum quid es cui primo et per se competit definitio. Quo est ratio ob quam definitio competit definito quid. 
\pend

\pstart
  Suppono 3. Discrimen inter complexa et \secluded{et} concreta accidentalia. Conveniunt tamen in aliquo homo albus et album sit in inclutione diversorum praedicamentorum. Differunt vero in modo significandi. Siquidem complexa utrumque per se significant, at concreta unum tantum per se significant. 
\pend

\pstart
  His notatis aliqua sunt certa in quibus Auctores debent convenire. Primum secunda intentio est definitum formale quo. Ratio clamitat. Ratio propter quam definitio convenit definito quid est definitum quo, sed ratio propter quam genus est praedicabile de pluribus, etc., est secunda intentio, quam includit genus ut formam: igitur secunda intentio est definitum formale quo. 
\pend

\pstart
  Secundum certum, natura non est definitum formalem quid, sed tantum materiale. Ratio est definitum formale quid est cui primo et per se competit definitio, materiali vero cui solum per accidens competit, scilicet inquantum substernitur formae. Sed naturae non competit per se; verbi gratia animali, quod sit praedicabilis de pluribus, etc. Sed solum per accidens: igitur. Probatur minor. Solum praedicata essentialia competunt per se naturae, nam animali conpetit vitalitas sensibilis, etc., at praedicabilitas per accidens intellectus operatione. 
\pend

\pstart
  Probatur amplius. Si naturae per se competeret definitio generis, verbi gratia animali plura sequerentur absurda. Primum, homo, leo et aliae animalis species imo et individua essent praedicabilia de pluribus, etc. \textnormal{|}\ledsidenote{BNC 138vb} Sequela patet. Quod per se competit naturae superiori, competit etiam inferioribus contentis sub illa. Sed naturae animalis per se competeret definitio generis: igitur et homini, leoni, etc., contentis sub illa. Secundum, non solum darentur \emph{ex [?]} naturae quae denominari possunt genera, sed etiam daretur ipsa genereitas: igitur sequerentur absurdas. 
\pend

\pstart
  Tertium certum, secunda intentio secundum se prout praescindit a natura non est definitum quid huius definitionis. Ratio constat. Forma in abstracto significata non potest esse definitum in definitione sui concreti, sed secunda intentio secundum se significatur in abstracto: igitur secunda intentio secundum se non est definitum in definitione generis. Restat igitur difficultas de aliis tribus acceptionibus. 
\pend

\pstart
  Prima sententia asserit definitum esse aggregatum seu compositum ex natura et secunda intentione. Ita \edtext{ \name{\textsc{Cabero}\index[persons]{Chrysostomus Cabero}} hic tractato 3, disputatione 2, dubia 1 }{\lemma{}\Afootnote[nosep]{}}; \edtext{ \name{\textsc{Rubius}\index[persons]{Antonius Ruvius Rodensis}} capitulo \worktitle{De Genero} quaestione 2 }{\lemma{}\Afootnote[nosep]{}}. Secunda asserit naturam esse definitum connotando secundam intentionem. Ita \textsc{Toletus}\index[persons]{Franciscus Toletus} et alii relati ab \edtext{ \name{\textsc{Oña}\index[persons]{Petrus de Oña}}  quaestione 1 \worktitle{Universalium}, articulo 1 }{\lemma{}\Afootnote[nosep]{}} et novissime \edtext{ \name{\textsc{Raphael de Aversa}\index[persons]{Raphael de Aversa}} quaestione 8, sectione 2  }{\lemma{}\Afootnote[nosep]{}}. Tertia sententia docet definitum esse secundam intentionem in concreto, id est connotantem naturam. Sic communiter \edtext{ \name{\textsc{Thomas Caietanus}\index[persons]{Thomas de Vio Caietanu}} \worktitle{De ente et essentia} capitulo 9 }{\lemma{}\Afootnote[nosep]{}}; \edtext{ \name{\textsc{Arauxo}\index[persons]{Franciscus de Arauxo}} libro 3 \worktitle{Metaphysicae} quaestione 4, articulo 3 }{\lemma{}\Afootnote[nosep]{}}; \edtext{ \name{\textsc{Soto}\index[persons]{Dominicus de Soto}} in hoc capitulo quaestione unica, conclusione 1 }{\lemma{}\Afootnote[nosep]{}}; \edtext{ \name{\textsc{Sanchez}\index[persons]{Ioannes Sanchez Sedeno}}  libro 3, questione 3 }{\lemma{}\Afootnote[nosep]{}}; \edtext{ \name{\textsc{Didacus  a Iesu}\index[persons]{Didacus a Iesu, Salablanca}}  \worktitle{distinctione 6, questione 1} }{\lemma{}\Afootnote[nosep]{}}; \edtext{ \name{\textsc{Gallego}\index[persons]{Baranbas Gallego de Vera}} \worktitle{Controversia 12} }{\lemma{}\Afootnote[nosep]{}} et alii quos sequitur \edtext{ \name{\textsc{Masius}\index[persons]{Didacus Masius}}  sectione 1, questione 3 }{\lemma{}\Afootnote[nosep]{}}; \edtext{ \name{\textsc{Cursus carmelitarum}\index[persons]{}} disputatione 5, quaestione 3 }{\lemma{}\Afootnote[nosep]{}}; \edtext{ \name{\textsc{Ioannes a Sancto Thoma}\index[persons]{Ioannes a Sancto Thoma}} questione 7, articulo 2 }{\lemma{}\Afootnote[nosep]{}}. Ut a clare procedamus. 
\pend

\pstart
  Sit prima conclusio: definitum in definitionem est generis, non est constitutum ex subiecti et intentione generis. Ita \edtext{ Doctor \name{\textsc{Thomas}\index[persons]{Thomas Aquinas}} \worktitle{De ente et essentia} capitulo 6 }{\lemma{}\Afootnote[nosep]{}}. Probatur ratione. Constitutum ex subiecti et accidente non potest unica \textnormal{|}\ledsidenote{BNC 139ra}   definitione definiri, sed compositum ex natura et secunda intentione est huiusmodi: igitur. Quod igitur si Auctores oppositi intelligant pro constituto quidditatem aliquam resultantem ex subiecto et accidenti? Displicet haec forma, nam ex subiecto et et accidente non resultat unum per se. 
\pend

\pstart
  Explico doctrinam. Ex substantia et accidente non potest resultare aliqua quidditas quae neque substantia sit nec accidens; non enim datur medium. Illud tertium resultans non est sola substantia cum etiam accidens habeat partem sui, nec solum est accidens cum substantiam habeat partem sui. Dein non datur medium inter substantiam et accidens cum dividantur modis contradictionis, scilicet esse per se et esse in alio: igitur non est aliqua quidditas per se, sed ens per accidens. 
\pend

\pstart
  Dices: quod illa quidditas resultans non est talis quae nec substantia sit, nec accidens, sed est utrumque. Ergo taliter est utrumque ut non sit quidditas ab utroque distinctum, cum non sit possibilis haec quidditas: igitur solum est substantia et accidens accidentali unitione convincta non essentiali, et per se, quod est nostrum intentum. 
\pend

\pstart
  Respondent aliqui ex subiecto et accidente non resultare unum per accidens. Audiamus rationem. Quia veram unionem habent inter se, et aliquid verum componunt. Defiat: licet sit vera unio, non est unio per se, sed secundum quid et accidentalis. Qualis enim fuerit unio extremorum talis erit unitas resultans, sed talis unio est secundum quod: igitur quod resultat est ens secundum quid. 
\pend

\pstart
  Alii admittunt esse ens per acciens definibile tamen; sicut definitur `respublica', `exercitus,' `populus' et similia minorem unitatem habentia. Defecit: nonne definitio solum versatur circa \textnormal{|}\ledsidenote{BNC 139rb} quidditatem? Sane. Ergo in tantum definitio est una inquantum attingit unam quidditatem. Sed ens per accidens non dicit unam quidditatem per se et simpliciter; sed plures per accidens unum: igitur non est una definitione definibile. Nonne definitio constat genere et differentia? Constat. Et ubi sunt plures quidditates sunt plura genera? Sunt. Igitur ubi sunt plures quidditates sunt plures definitiones. Ita \edtext{ \name{\textsc{Aristoteles}\index[persons]{Aristoteles}} VII \worktitle{Metaphysicae} }{\lemma{}\Afootnote[nosep]{}} et \edtext{ Divus \name{\textsc{Thomas}\index[persons]{Thomas Aquinas}} lectione 12 }{\lemma{}\Afootnote[nosep]{}}. 
\pend

\pstart
  Sit secunda conclusio: definitum formale huius definitionis (et ceterorum accidentiam) non potest esse ipsa natura nec prout subicitur intentioni. Ita \edtext{ Divus \name{\textsc{Thomas}\index[persons]{Thomas Aquinas}} \worktitle{De ente et essentia}, capitulo 6 et V \worktitle{Metaphysicae} lectione 9 }{\lemma{}\Afootnote[nosep]{}}. Probatur ratione. Si natura esset definitum quod iam definitio conveniret accidentaliter definitio, sed hoc est impossibile: igitur. Probatur assumptum. Hae praedicationes animal praedicatur de pluribus, etc., Color praedicatur de pluribus, etc.; sunt accidentales et in materia contingenti. Sed in eis praedicatur definitio generis de natura seu prima intentione: igitur si natura esset definitum quod, definitio convenire accidentaliter definitio. 
\pend

\pstart
  Audita hac ratione respondet \textsc{Rubius}\index[persons]{Antonius Ruvius Rodensis}. Posita distinctione subiecti seu substracti proximi adaequati et per se a subiecto remoto inadaequato et per accidens generaliter in omni accidentali concreto: efficacem esse loquendo de subiecto inadaequato, secus autem adaequato. Nam haec praedicatio est essentialis cum sit de subiecto adaequato. Corpus inquantum albedinis substratum album est. Haec vero accidentalis: Petrus est albus siquidem est de subiecto remoto inadaequato et per accidens. 
\pend

\pstart
  Admissa enim distinctione subiectorum. Tamen numquam concretum accidentale praedicatur essentialiter de subiecto sive adaequato sive inadaequato. \textnormal{|}\ledsidenote{BNC 139va}   Unde haec praedicatio est accidentalis: corpus inquantum albedinis est album. Clamitat ratio: quantumcunque subiectum consideretur ut subiectum semper est pars compositi accidentalis: igitur semper altera pars, scilicet forma est extra essentiam subiecti. Igitur accidentaliter tantum potest praedicari de illo. 
\pend

\pstart
  Probatur amplius. Posita doctrina \textsc{Rubii}\index[persons]{Antonius Ruvius Rodensis} iam haec praedicatio esset essentialis: `homo est risibilis', `equus est hinnibilis', sed hoc repugnat Dialectica: igitur ruit doctrina. Probatur maior. Ideo \emph{per se [?]} haec praedicatio `corpus inquantum albedinis substratum album est', est essentialis, quia praedicatur de suo subiecto adaequato. Sed homo est subiectum adaequatum risibilitatis: igitur haec praedicatio `homo inquantum subiectum risibilitatis est risibilis', est essentialis. Sed hoc non: igitur quamvis reduplicetur natura inquantum perfectibilis est genus, semper propio manet accidentalis. 
\pend

\pstart
  Probatur secunda pars conclusionis: quantumcunque subiectum cuiuscunque concreti sumatur formaliter ut subiectum seu ut subicitur formae manet semper extra essentiam formae: igitur quantumcunque natura sumatur formaliter ut subicitur secundae intentionis, semper secunda intentio est extra essentiam naturae. Igitur adhuc posita reduplicatione, haec praedicatio est accidentalis, natura inquantum subiicitur intentioni est genus. 
\pend

\pstart
  Respondent contrarii argumentum concludere de natura secundum se, non vero posita reduplicatione. Bene quaero quid reduplicat per hic inquantum subiecte secundae intentioni generis, inquantum habet secundam intentionem generis? Vel aliquid expectans ad ipsam naturam, scilicet subisectionem, capacitatem et potentiam passivam ad recipiendam formam vel ipsam \textnormal{|}\ledsidenote{BNC 139vb} intentionem? Si hoc ultimum, libenter fateor hanc praedicationem esse essentialem natura inquantum subiicitur secundae intentioni generis praedicatur de pluribus, etc. Qua ratione? Quia commune est praedicationes accidentales fieri essentiales facta reduplicatione supra formam ut `homo inquantum albus est disgregativus unus'. Cumque tunc praedicatum appellet supra formam, ipsa forma erit definitum. 
\pend

\pstart
  At si reduplicant aliquam unionem rationis coniungentem genereitatem naturae. Contra esto esse talem unionem (omnino vanam) talis esset accidentis: igitur vi illius genereitas non est essentialis naturae. Denique si reduplicant aliam secundam intentionem aut rationis formam disponentem naturam aptamque facientem ad genereitatem recipiendam absurdum est. Nam ea ratione, qua requiritur secunda intentio ut natura recipiat genereitatem requirentur etiam alia secunda intentio disponens ad secundam intentionem disponentem et ad hanc alia usque in saeculum saeculi. Ceterum dato quod esset necessaria haec secunda intentio, adhuc genereitas esset accidentalis: igitur non satisfacit argumento. 
\pend

\pstart
  Roboratur: quaelibet dispositio est exta essentiam formae ad quam disponi, sed talis secunda intentio compararetur ad generitatem ut dispositio ad formam: igitur esset extra essentiam generitatis. Igitur genereitas est accidentalis naturae quamtum vis disposita 
\pend

\pstart
  Si dicant primum. Contra prima, quia reduplicationes in quibus de subiecto alicuius formae reduplicatur subiectio seu capacitas ad illam formam; ut corpus inquantum subiectum albedinis est album est inanis. Ratio est: ex vi praedicationis subiectum subiicitur praedicatio et recipit illud ratione capacitatis \textnormal{|}\ledsidenote{BNC 140ra}   seu potentiae passivae, quam habet ad ipsum recipiendum: igitur idem est dicere corpus inquantum subiicitur albedini est album, ac corpus es album: igitur inanis est tales reduplicatio. 
\pend

\pstart
  Secundo, siquidem subiectio capacitas seu potentia passiva est extrinseca et extra essentia praedictae formae: igitur aeque accidentaliter convenit definitio generis (sicut et alia praedicabilium) naturae cum reduplicatione inquantum subiiscitur secundae intentioni generis. Ergo tales reduplicationes undequaque consideratae omnino superfluunt. 
\pend

\pstart
  Sit tertia conclusio: definitum formale et quidditativum huius definitiones est secunda intentio in concreto significata connotans naturam. Ita \edtext{ \name{\textsc{Angelicus Doctor}\index[persons]{Thomas Aquinas}} V \worktitle{Metaphysicae}, lectione 9 et VIII \worktitle{Metaphysicae}, lectione 4 et VII \worktitle{lectio 1} }{\lemma{}\Afootnote[nosep]{}} et \edtext{ \name{\textsc{Caietanus}\index[persons]{Thomas de Vio Caietanu}} in capitulo \worktitle{De genere} et \worktitle{De ente et de essentia} capitulo 4 }{\lemma{}\Afootnote[nosep]{}}; \edtext{ \name{\textsc{Complutenses}\index[persons]{}} Disputatione 5, questione 3 }{\lemma{}\Afootnote[nosep]{}}; \edtext{ \name{\textsc{Iohannes a Sancto Thoma}\index[persons]{Ioannes a Sancto Thoma}} quaestione 7, articulo 2 }{\lemma{}\Afootnote[nosep]{}}; \edtext{ \name{\textsc{Merinero}\index[persons]{}}  capitulo 2 \worktitle{De genere}, disputatione unica, questione 1 }{\lemma{}\Afootnote[nosep]{}} et alii citati. 
\pend

\pstart
  Probatur breviter rationibus Complutensis. Hic definitur aliquid commune entibus realibus et rationis, sed nihil aliud potest esse eius commune nisi secunda intentio: igitur ipsa sola est res quae definitur. Maior constat. Tam in realibus quam rationis entibus reperiuntur genera et species: igitur. 
\pend

\pstart
  Confirmatur 1: definitio generis ceterorumque praedicabilium est proprie solius Logicae: igitur in ea definitur id quod per se pertinet ad logicam, sed hoc nil est aliud quam secunda intentio: igitur haec est res definita. Confirmatur 2: haec definitio datur per subiectum loco generis: igitur definitum est secunda intentio in concreto significata. Probat consequentia. Quia iste est modus proprius definiendi. 
\pend

\pstart
  \textnormal{|}\ledsidenote{BNC 140rb} Sit quarta conclusio: quamvis secunda intentio in concreto definitum formale tantummodo est definitiuum quo. Ita Auctores citati cum \edtext{ \name{\textsc{Angelico Doctore}\index[persons]{Thomas Aquinas}} capitulo 4 \worktitle{De ente et essentia} }{\lemma{}\Afootnote[nosep]{}}. Ratio fundamentalis est. Res definiuntur secundum quod significantur per nomina, sed nomen concretum significat in accidentibus solam formam et connotat seu supponit per subiecto: igitur in definitione accidentis definitum est sola forma licet fiat connotatio et suppositio per subiecto tamquam in quo exercetur definitio. 
\pend

\pstart
  Roboratur id proprie est definitum quod est res significata per nomen definiti, quod autem pertinet ad modum significandi non est proprie definitum, sed solum pertinet ad modum definiendi: igitur proprie sola secunda intentio est res definita, licet ex modo definiendi explicetur naturam esse, id in quo exercetur effectus formalis definitionis. 
\pend

\pstart
  Contra insurgis: definitio debet convenire ut quod suo definitio. Sed definitio generis non convenit ut quod secundae intentioni: igitur secunda intentio non est definitum in hac definitione. Respondeo distinguendo minorem. Definitio generis non convenit ut quod secundae intentioni in abstracto sumptae, concedo; in concreto, nego. 
\pend

\pstart
  Instabis: quamvis haec definitio conveniat secundae intentioni ut significatae in concreto, non tamen convenit ei ratione suae entitatis, sed ex modo significand. Sed res significata est proprie de definitum: igitur secundae intentioni secundum quod est res definita non convenit definito ut quod. 
\pend

\pstart
  Respondeo. Maximum esse discrimen inter definitionem perfectam et imperfectam. In definitione perfecta necessarium est assignare definitum quod. Nam ibi solum explicatur natura definiti et nihil aliud \textnormal{|}\ledsidenote{BNC 140va}   connotatur. Unde res est, quam significat et per qua supponit nomen definiti. At in definitione imperfecta (qualis est accidentalium concretorum) nomen definiti significat unam rem et supponit per alia. Unde in huiusmodi definitionibus praeter rem quae est definitum fit suppositio per re, in qua exercetur definitio; ac per consequens definitio non convenit ut quod, sed tantum ut quo rei definitae. Sic igitur explicatur tota essentia accidentis, scilicet esse principum quo suorum effectum et simul explicatur, tales effectus exerceri in subiecto. 
\pend

\pstart
  Obiectio 1. Logica tractat de secundis intentionibus adiunctis primis ex \textsc{Avicenna}\index[persons]{Avicenna}: igitur coniunctum seu compositum est definitum. Respondeo distinguendo antecedens. Tractat ita ut aeque utrasque tracte aut de aliquo ex utrisque resultante, nego; ita ut de intentionibus tractet formaliter e per se de rebus materialiter et denominative, concedo. Patet solutio. 
\pend

\pstart
  Obiectio 2. Illud est definitum cui per se competit definitio, sed compositum ex natura et secunda intentione est huiusmodi: igitur. Probatur minor. Secundae intentioni convenit essentialiter esse id quo natura praedicatur de pluribus et naturae denominatae essentialiter convenit esse id quod praedicatur: igitur utrumque definitur natura ut quod intentio vero ut quo. 
\pend

\pstart
  Respondeo negando minorem. Ad probationem distinguo consequens. Natura ut quod denominative, concedo; essentialiter, nego. Ut enim definitio conveniret essentialiter et per se illi aggregato ut intendit argumentum erat necessarium quod sicut secundate intentioni convenit essentialiter convenit ratione suae entitatis esse principium quo. Sic naturae ratione seu et non ratione intentionis conveniret essentialiter esse \emph{pro [?]} id quod praedicatur. Cumque naturae essentialiter non competat praedicari de pluribus, ideo non tenet argumentum. 
\pend

\pstart
  \textnormal{|}\ledsidenote{BNC 140vb} Obiectio 3. \edtext{ \name{\textsc{Porphyrius}\index[persons]{Porphyrius}} \worktitle{in hoc capitulo} }{\lemma{}\Afootnote[nosep]{}} et \edtext{ \name{\textsc{Aristoteles}\index[persons]{Aristoteles}} 1 \worktitle{Topicorum}, capitulo 4 }{\lemma{}\Afootnote[nosep]{}} post traditam definitionem generis, statim subiungunt ut animal, etc.: igitur ipsa natura est quae definitur. Respondeo hoc esse ut intelligatur in his definitionibus ex modo definiendi fieri suppositionem per subiecto tamquam in quo exercetur effectus definitionis. 
\pend

\pstart
  Obiectio 4. Natura ut subiicitur intentioni et non ipsa intentio praedicatur de pluribus: igitur natura ut connotat intentionem est definitum. Antecedens probatur: secunda intentio non praedicatur de multis est menim modo ob quam natura praedicatur. 
\pend

\pstart
  Respondeo distinguendo antecedens. Natura prout subiicitur intentioni praedicatur de pluribus essentialiter, nego; accidentaliter, concedo. Etiam prout subiicitur intentioni praedicatur natura accidentaliter: igitur licet subiiciatur non est definitum. Ad probationem distinguo. Secunda intentio non praedicatur de multis ut quod, concedo; ut quo, nego. Cum enim huiusmodi definitiones solum ut quo debeant convenire definito si secundae intentioni ut quo convenit secunda intentio est definitum. Sic dicebamus numero 659. 
\pend

\pstart
  Obiectio 5. Concretum accidentale supponit pro subiecto: igitur genus supponit pro natura: igitur natura est definitum. Antecedens probatur. actiones et passiones non tribuuntur formis, sed suppositus; albedo non disgregat visum, sed album; nec forma generatur, sed compositum: igitur concretum supponit pro subiecto. 
\pend

\pstart
  Respondeo concedendo antecedens et primam consequentiam distinguendo. Genus supponit pro natura ex modo significandi, concedo; quo ad rem significatam, nego. Unde negatur ultima consequentia. Sicut enim ex modo significandi concretum supponat pro subiecto tamen sola forma est res significata, ideoque sola illa proprie definitur. 
\pend

\pstart
  \textnormal{|}\ledsidenote{BNC 141ra}   Obiectio 6. In definitione generis subiectum non importatur ut unum: igitur absolute non importatur. Consequentia probat. Si subiectum generis non importatur ut unum, sed ut plura: igitur non bene definitur per modum unius asserendo id quod praedicatur, sed ea quae praedicantur. Primum antecedens probatur. Quod substernitur intentioni generis est tam ens reale quam rationis, tam substantia quam accidens, sed ens sic acceptum non est unum, nec unico conceptu potest concipi: igitur. 
\pend

\pstart
  Respondeo distinguendo antecedens. In definitione generis non importatur subiectum ut unum unitate naturae vel essentiae, concedo antecedens; ut unum unitate capacitatis vel receptibitatis formae, nego. Quodquod enim habet capacitatem illius formae seu intentionis per modum unius subiecti, hoc est unius receptibilis significatur, licet \emph{sunt [?]} naturae diversissimae. Exemplificatur. Asserunt Theologi in illis verbis: \enquote{Hoc est corpus meum}. Per hic hoc intelligi contentum sub speciebus, id est quodquod sub illis speciebus esse potest sive sit corpus vivum sive mortuum, sive panis sive homo. Similiter nomine substracti generis intelligitur quodquod est capax genereitatis sive reale sit, sive non dummodo conveniat in ratione capacitatis, licet non in quiditate naturae et unitate. 
\pend

\pstart
  Obiectio 7. Si definitum generis esset totum constitutum, totum quod pertinet ad essentiam generis explicaretur: igitur constitutum est definitu. Antecedens probatur. Explicatur res praedicata denominative; explicatur forma, qua redditur praedicabilis: igitur totum. 
\pend

\pstart
  Confirmatur: hoc constitutum subordinatur uni conceptui ultimo: igitur est definibile. Patet consequentia: ideo ens per accidens non est definibile, quia non est unum: igitur si illud resultans est unum \textnormal{|}\ledsidenote{BNC 141rb} puta uno conceptui subordinatum erit definibile. Antecedens probatur primo. Non plura inveniuntur in concreto quam in constituto, sed concretum album est unius conceptus: igitur et constitutum ex albedine et corpore. Secundo: tam concretum quam constitutum es ad unam operationem ordinabile, scilicet album ad disgregandum, calidum ad calefaciendum, etc. Igitur uno conceptui subordinatur. 
\pend

\pstart
  Respondeo distinguendo antecedens. Totum quod est in genere explicaretur ex aequo tamquam pertinens ad essentiam, nego; inaequaliter \del{concedo} sit subiectum ut denominatum et intentio ut forma constituens formaliter genus, concedo. Suppono ut impossibilem tertiam entitatem resultantem ex subiecto et forma. Neque illa definitio quadrat definito, quod non est id quod praedicatur cum intentio cum natura non descendat ad inferiora. Neque est id quo praedicatur cum natura non pertineat ad formam reddentem praedicabile, sed ad rem praedicabilem. Relinquetur igitur ex natura et intentione non fieri quidditatem, quae sit genut ut quod, neque ut quo. Quando enim non sumitur pro constituto explicatur utrumque formaliter pro forma denominative pro subiecto. 
\pend

\pstart
  Ad confirmationem distinguo. Concretum constitutum subordinatur uni conceptui importando subiectum et formam aeque, nego; importando subiectum materialiter et denominative formam vero formaliter et essentialiter, concedo. Cum enim tunc forma sit definitum et subiectum denominatum non se habebit ut ens per accidens, sed definiretur ut subordinatur. At distinguamus antecedens primum: constitutum subordinatur esse aeque essentialiter et ut constituens unum tertium, concedo; inaequaliter ut in concreto, nego. Tunc enim illud constitutum est ens per accidens resultans ex extremis accidentaliter unitis. 
\pend

\pstart
  \textnormal{|}\ledsidenote{BNC 141va}   Obiectio 8. Et est roboratio 4. Intentio generis non est id quod praedicatur, sed id quo natura praedicatur: igitur non convenit ei definitio data. Exemplificatur. Quando dico `Petrus est albus' non praedico de illo albedinem, sed habens albedinem: igitur sicut habent albedinem est significatum, ita est definitum. 
\pend

\pstart
  Respondeo distinguendo antecedens. Intentio generis non est id quod praedicatur ut suppositum intentionis et res praedicata, concedo; ut forma praedicandi, nego. Et distinguo consequens. Non convenit definitio data u supposito definitionis, concedo; ut rationi et formae praedicandi, nego. Dictum est in definitionibus accidentium non inveniri definitum quod essentialiter, sed denominative, quia suppositum redditur definito per aliquid extrinsecum, nec accidens per se fert suppositum. Unde haec definitiones dantur per additione: igitur in eis aliud est quodquod aliud cuius est. 
\pend

\pstart
  Quaeres cur definitio accidentis traditur per id quod denominatur tale, non per id quod est tale? Respondeo primo. Accidens habet esse incompletum non in se subsistens, sed dependens a subiecto denominato. Cum enim melius exprimatur ista dependentia et accidentis imperfectio definitione explicante subiectum quod denominative est tale, quam subiectum per se ipsum significatum ut quo: ideo traditur per tale subiectum. Sic \edtext{ Divus \name{\textsc{Thomas}\index[persons]{Thomas Aquinas}} VII \worktitle{Metaphysicae} lectione 4 et VII \worktitle{De ente et essentia} }{\lemma{}\Afootnote[nosep]{}}. 
\pend

\pstart
  Secundo ut conveniret modo significandi. Quando accidens definitur in concreto debet poni subiectum ut reddens suppositum definito: igitur si in tali definitione non poneretur id quod saltem denominative est tale, sed solum id quo aliquid est tale quia tunc solum significaretur ratio \textnormal{|}\ledsidenote{BNC 141vb} abstracta. Verbi gratia quando dico `Petrus est albus', hic albus significat habens albedinem et non albedinem abstractam; at non ex aequo ut dixi. Sed albedinem de formali et habens de connotato. 
\pend

\pstart
  Obiectio 9. Definitio generis solum verificatur de naturis: igitur istae sunt definitum. Probatur antecedens. Praedicari de pluribus, etc., est definitio generis, sed solum convenit hoc naturis: igitur. Probatur minor. Non praedicamus `hace species est genus', `illa species est genus'; sed `homo est animal', `leo est animal'; igitur. 
\pend

\pstart
  Respondeo negando antecedens. Ad probationem distinguo minorem. Hoc convenit naturis exercite et denominative, concedo; signate et essentialiter, nego. Dupliciter potest aliquid praedicari, scilicet signata et exercite. Illud praedicatur in actu signato, quod est ratio praedicandi et illud praedicatur in actu exercito, quod in actuali exercitio affirmatur. 
\pend

\pstart
  Multum interest inter utrumque. Primo. Praedicatio signata datur semper per dicitur vel praedicatur. Praedicatio vero exercita semper datur per seest. Secundo. Praedicatio signata tribuitur secundis intentionibus eo quid sint rationes praedicandi. At praedicatio exercita tribuitur naturis, cum significant subiectum secundum intentionum. Unde dicitur esse exercere in primis intentionibus quod praedicari signat in secundis et optime. 
\pend

\pstart
 Instabis: ex dictis colligitur in hac definitione non esse definitum quod, sed hoc implicat: igitur. Respondeo distinguendo maiorem. Non esse definitum quod essentiale, concedo; definitum quod denominative, nego. Ut supra. 
\pend

\pstart
 Instabis item: sicut naturae convenit praedicabilitas, praedicatur, sed convenit denominative: igitur \textnormal{|}\ledsidenote{BNC 142ra}   praedicatur denominative. Hoc implicat: igitur et illud. Respondeo negando maiorem. Ergo natura praedicatur essentialiter. Distinguo consequens. Praedicatur essentialiter ex parte rei praedicatae, concedo; ex parte modi praedicandi nego. 
\pend

\pstart
 Variatur appellatio quando dicitur praedicatur essentialiter: igitur convenit essentialiter praedicabilitas. Aliud est quomodo praedicabilitas conveniat naturae aliud quomodo praedicatur. Natura non convenit inferioribus denominative, ipsi tamen naturae denominative convenit praedicabilitas. Articulo 5. Quid loco generis, etc. 
\pend

        \addcontentsline{toc}{section}{Articulus 5. In definitione praedicabiliorum quid ponitur loco generis?}
        \pstart
        \eledsection*{Articulus 5. In definitione praedicabiliorum quid ponitur loco generis?}
        \pend
      
\pstart
 Celebris sententia est inter multos Auctores in huiusmodi definitionibus loco generis non poni naturam -seu subiectum, sed concretum superius. Restat difficultas hoc concretum vel est Universale vel praedicabile? Universale asserunt \edtext{ \name{\textsc{Lovaniense}\index[persons]{}} }{\lemma{}\Afootnote[nosep]{}}; \edtext{ \name{\textsc{Titelmannus}\index[persons]{Franciscus Titelmannus}}  }{\lemma{}\Afootnote[nosep]{}} et alii relati a \edtext{ \name{\textsc{Masio}\index[persons]{Didacus Masius}} hic sectione 4, quaestione 5 }{\lemma{}\Afootnote[nosep]{}}. Praedicabile vero tenent \edtext{ \name{\textsc{Toletum}\index[persons]{Franciscus Toletus}} }{\lemma{}\Afootnote[nosep]{}}; \edtext{ \name{\textsc{Villalpandum}\index[persons]{Gaspar Cardillo de Villalpando}}  }{\lemma{}\Afootnote[nosep]{}}. Utrumque esse doceret \edtext{ \name{\textsc{Niger}\index[persons]{Petrus Georgius Niger}}  --> }{\lemma{}\Afootnote[nosep]{}} et alii. 
\pend

\pstart
 Sit prima conclusio: quando praedicabile definitur in concreto est subiectum seu natura denominata loco generis. Ita \textsc{Aristoteles}\index[persons]{Aristoteles} et \textsc{Angelico Praeceptor}\index[persons]{Thomas Aquinas} ex dictis. Quos sequuntur \edtext{ \name{\textsc{Soncinas}\index[persons]{Paulus Barbus Soncinatus}}  7 \worktitle{Metaphysicae} quaestione 7 }{\lemma{}\Afootnote[nosep]{}}; \edtext{ \name{\textsc{Paulus Ventus}\index[persons]{Paulus Venetus}} }{\lemma{}\Afootnote[nosep]{}} et \edtext{ \name{\textsc{Iavelus}\index[persons]{Chrysostomus Iavelli Canapicii}} in hoc capitulo }{\lemma{}\Afootnote[nosep]{}}. \edtext{ \name{\textsc{Masius}\index[persons]{Didacus Masius}} }{\lemma{}\Afootnote[nosep]{}} ut supra \edtext{ \name{\textsc{Antonium Rubium}\index[persons]{Antonius Ruvius Rodensis}} }{\lemma{}\Afootnote[nosep]{}}. \edtext{ \name{\textsc{Complutensis}\index[persons]{}} disputatione 5, quaestione 6 }{\lemma{}\Afootnote[nosep]{}} et alii. 
\pend

\pstart
 Probatur ratione. In definitione omnium accidentium tam realium quam rationis in concreto subiectum ponitur loco generis et non concretum superius. Sed praedicabilia sunt accidentia rationis: igitur in eorum definitionibus in concreto subiectum ponendum est loco generis. Sed natura \textnormal{|}\ledsidenote{BNC 142rb} denominata est subiectum horum accidentium: igitur subiectum definitionis praedicabilis est loco generis. Minor constat toto articulo praecedenti hoc specialiter suadente. 
\pend

\pstart
 Sit secunda conclusio: quando praedicabile definitur in abstracto genus est universalitas. Est communiter accepta. Probatur breviter. Definitio abstracti accidentalis est propria et perfecta: igitur in ea ponitur quod proprie et vere est genus. Sed universalitas ut vidimus est genus ad praedicabilia: igitur universalitas es genus in tali definitione. 
\pend

\pstart
 Videamus nunc quo pacto praedicabilia definiantur tam in abstracto quam in concreto definitionibus essentialibus. In abstracto. Genereitas est universalitas naturae perfectibilis seu constitutibilis. Specieitas est universalitas naturae perfectae seu constitutae. Differentialitas est universalitas naturae perfectivae seu constitutive. Proprietareitas est universalitas naturae perfluentis ab essentia. Accidentalitas es universalitas naturae extrinsece seu contingentis. 
\pend

\pstart
 In concreto: Genus est natura perfectibilis seu constitutibilis universalis. Species est natura perfecta seu constituta universalis. Differentia est natura perfectiva seu constitutiva universalis. Proprium est natura perfluens ab essentia universalis. Accidens es natura extrinseca seu contingens Universalis. 
\pend

\pstart
 Insurgis fortiter ut olim: Universale est praedicatum superius in quo conveniunt praedicabilia, tale praedicatum non ponitur loco differentiae: igitur male ponitur per differentia in definitione concretiva. Hoc argumentum manet solutum supra numero 616. Nihilominus iterum solvitur. 
\pend

\pstart
 Respondeo. Non inconvenire quod in definitione praedicabilis est proprie et simpliciter genus, esse etiam secundum \textnormal{|}\ledsidenote{BNC 142vb}   quod differentiam. Qua ratione? Quia propter diversum modum significandi concreti et abstracti per diversam lineam fit convenientia. Cum definitur genereitas fit convenientia secundum lineam seu divisionem essentialem, scilicet cum intentionibus eiusdem generis, nempe cum specieitate, etc., qua propter universalitas pro genere ponitur cum sit genus vere et proprie; ac per differentia naturae perfectibilis unde sumitur vera differentia. Cum autem definitur genus convenientia fit secundum lineam seu divisionem subiecti in accidentia, scilicet cum intentionibus diversi generis, nempe cum subiscibilitate aut aliis quae possunt esse cum genereitate in eadem natura perfectibili. Unde ponitur subiectum in qyi conveniunt et pro differentia universalis quia per hoc distinguitur genus ab individuo generico. Sic enim habet rationem differentiae dum quod tantum. 
\pend

\pstart
 Obiectio 1. Definitio essentialis et quaelibet eius pars praedicatur essentialiter de definitio ex \edtext{ \name{\textsc{Aristotele}\index[persons]{Aristoteles}} 1 \worktitle{Posteriorum} capitulo 4 }{\lemma{}\Afootnote[nosep]{}}. Sed subiectum seu natura denominata non praedicatur de praedicabilibus: igitur natura non potest poni pro genere. Patet minor. Ens reale non valet praedicari essentialiter de entibus rationis: igitur natura non praedicatur de praedicabilibus. 
\pend

\pstart
 Confirmatur: definitio et definitum sunt idem ex \edtext{ \name{\textsc{Aristotele}\index[persons]{Aristoteles}} }{\lemma{}\Afootnote[nosep]{}} et \edtext{ Divo \name{\textsc{Thoma}\index[persons]{Thomas Aquinas}} 1 distinctione 25, quaestione 1, articulo 1, ad 2 }{\lemma{}\Afootnote[nosep]{}}. Sed secunda intentio non est idem quod subiectum seu natura posita loco generis: igitur non est idem cum tota definitione. 
\pend

\pstart
 Respondeo distinguendo maiorem explicando Philosophum. Definitio perfecta et quaelibet eius pars, quae vere et secundum rem significatam est pars praedicatur essentialiter de definitio, concedo; definitio imperfecta et quaelibet eius pars quae solum secundum modum significandi est pars praedicatur essentialiter de definitio, nego. Partes enim definitionis \textnormal{|}\ledsidenote{BNC 142vb} perfectae cum constituant essentiam definiti, praedicantur essentialiter de definitio. At subiectum ponitur in definitione concreti in recto, non quia prout sic sit vera pars eius, sed quia ex modo significandi concretum supponit pro subiecto. 
\pend

\pstart
 Ad confirmationem distinguo maiorem. Definitio perfecta et simpliciter et definitum sunt idem, concedo; definitio imperfecta et secundum quid et definitum sunt idem, nego: sic intelliguntur \textsc{Aristoteles}\index[persons]{Aristoteles} et \textsc{Angelicus Doctor}\index[persons]{Thomas Aquinas}. Cum enim definitio quae simpliciter et perfecte est talis nihil aliud praestet quam explicare essentiam definiti; nihil aliud continet nisi partes essentiales definiti: igitur quid mirum esse idem cum definitio. Ceterum definitio imperfecta ex modo signifificandi, ita explicat essentiam definiti ut simul exprimat subiectum in quo exercetur definitio: igitur continet aliud praeter essentiam definiti et si continet no poterit esse idem cum definitio in disparitas efficax. 
\pend

\pstart
 Obiectio 2. Genus debet esse univocum, sed natura seu subiectum non est \del{genus} univocum: igitur non est genus in definitione praedicabilium. Probatur minor. Ad ens reale et rationis, ad substantiam et accidens non datur unus conceptus univocus. Sed natura continet ens reale et rationis, substantiam et accidens: igitur prout sic non est quod univocum: igitur nec genus. 
\pend

\pstart
 Respondeo DISTINGUENDO maiorem. genus debet esse univocum genus proprie et simpliciter dictum, concedo; improprie dictum, nego. Dictum est, in definitione concreti non poni proprie genus, sed id pro quo supponit concretum. Unde non est necesse tale genus supponere universaliter distributive; sit sufficit quod supponat indeterminate et vage pro hac vel altera natura. Ceterum potest esse unum unitate receptibilitatis, non unitate essentiae vel naturae ut dicebam numero 669. 
\pend

\pstart
 \textnormal{|}\ledsidenote{BNC 143ra}   Replico. Subiectum positum in definitione accidentium realium supponit distributive et est univocum ad subiecta inadaequata: igitur pariter es dicendum de subiecto accidentum rationis. Antecedens patet. Quando dico album est perspicuum disgregativum visus vel simum est nasus cursus, hic perspicuum et nasus supponunt pro omnibus perspicuis et nasis: igitur. 
\pend

\pstart
 Respondeo. Pro nunc transeat antecedens. Adhuc enim dato ut vult argumentum nego consequentiam. Est disparitas: subiectum accidentium realium non est solum subiectum, sed habet etiam rationem principii et causae. Cum enim principium accidentis constituatur per aliquam unam rationem in esse principii. Sequentur subiectum in definitione supponere distributive et esse univocum. Ast accidens rationis non causatur ex intrinsecis rerum, sed extrinsece ab intellectu: igitur subiectum accidentium rationis habens solum rationem subiecti et non principii non constituitur in esse talis per aliquam rationem univocam, sed potest intellectus rebus quamvis diversis tribuere ista accidentia rationis. Articulo 6. Genus, etc.  
\pend

        \addcontentsline{toc}{section}{Articulus 6. Genus multifariam acceptum explicat ac compositionem methaphysicam (Logicam melius) illustrat}
        \pstart
        \eledsection*{Articulus 6. Genus multifariam acceptum explicat ac compositionem methaphysicam (Logicam melius) illustrat}
        \pend
      
\pstart
 Loquebamur usque huc de praedicabilibus pro formali, loquamur igitur de ipsis pro materiali. Expresse tetigimus genereitatem, nunc naturam perfectibilem, etc. Pro quo notato 1: naturam genericam, verbi gratia animal, quadrupliciter posse considerari. 
\pend

\pstart
 Primo ut includit sua praedicata essentialia, scilicet genus et differentiam ex quibus constat prout sic \textnormal{|}\ledsidenote{BNC 143rb} enim dicitur totum actuale. Secundo ut actu simul cum differentia ipsum contrahente componit speciem inferiorem sibi vera hominem vel leonem, et sic dicitur pars acutalis. Tertio ut quid contrahibile per differentias ad esse huius vel illius speciei et sic dicitur pars potentialis. Quarto ut quid continens ratione suae superioritatis plures species in potentia subiectiva et sic denique dicitur totum potentiale. 
\pend

\pstart
 Notato 2. Ratione totius et partis in universalibus dupliciter considerari, scilicet ex parte modi conspiciendi et ex parte rei conceptae. Ex parte modi concipiendi omne universale dicit rationem totius et non partis respectu suorum inferiorum. Siquidem debet praedicari de illis, sed ut pars non debet praedicari, nam totum est maius sua parte: igitur ut totum. Quare ratio totius ex modo concipiendi conpetit omni universali et praedicabili. 
\pend

\pstart
 Ex parte autem rei significatae ratio totius et ratio partis considerantur secundum quod constituunt ipsam quidditatem, et hoc modo solum tribus primis praedicabilibus competit ratio totius et partis: nam genus et differentia concurrit ut partes ad constituendam speciem, et species est ipsa quidditas totalis ex illis constituta. Relinquitur. Genus ex modo concipiendi esse totum et parte rei concptae esse partem. 
\pend

\pstart
 Notato 3. Totum potentiale bifariam intelligi, scilicet prout opponitur toti essentiali et prout opponitur toti actuali. Primo modo totum potentiale dividitur in partes potestativas, de quibus non praedicatur; totum essentiale in partes subiectivas, de quibus dicitur. At genus dicitur totum potentiale prout distinguitur a toto actuali, quia non explicat totum actualiter constitutum, sed partem, nempe ulterius \textnormal{|}\ledsidenote{BNC 143va}   actuabilem et contrahibilem. 
\pend

\pstart
 Dubitatur respectu quorum genus dicitur potentiale an, scilicet potentialitas conveniat illi qua toti vel qua parti? Ut resolvam notato 4 discrimen inter esse aliquid, continere aliud et includere aliud. Esse aliquid idem est ac non distingui ab illo. Continere aliud dicitur ac habere illud vel in virtute effectiva sicut causa efficiens continet suos effectus vel in potentia materiali sicut materia prima continet omnes formas substantiales vel in potestate et superioritate sicut superius continet inferiora. Includere aliud idem est ac in suo esse essentiali et quidditativo illud imbibere, quo pacto species includit genus et differentiam. 
\pend

\pstart
 Ex hac doctrina sequitur primum genus quomodolibet \emph{consiatum [?]} non includere differentias. Ita \edtext{ \name{\textsc{Aristoteles}\index[persons]{Aristoteles}} 7 \worktitle{Metaphysicae} capitulo 12 et \worktitle{Topicorum} capitulo 1 et 2 }{\lemma{}\Afootnote[nosep]{}} et \edtext{ \name{\textsc{Porphyrius}\index[persons]{Porphyrius}} hic }{\lemma{}\Afootnote[nosep]{}}. Constat hac ratione: in eodem non possunt esse simul opposita et in composibilia, scilicet contraria; sed differentiae quibus contrahitur genus sunt oppositae et in composibiles: igitur non possunt esse simul in eodem. Praeterea tunc sequeretur quamlibet differentiam essentialem et quidditativam denominari ab alia differentia essentiali. Unde homo diceretur hinnibilis essentialiter, hoc repugnat: igitur et inclusio. Patet sequela. In qualibet specie includitur essentialiter genus in quo includuntur essentialiter omnes differentiae: igitur. 
\pend

\pstart
 Sequitur secundo: genus tam ut totum potentiale quam ut partem potentialem non esse differentias, continere tamen illas ut totum potentiale in potestate et superioritate, veluti continet species, ut partem vero potentialem in potentia quasi materiali, sicut materia prima continet omnes formas substantiales. 
\pend

\pstart
 \textnormal{|}\ledsidenote{BNC 143vb} Assertum constat ratione. Quantum ad primum: genus et differentia sunt idem realiter in tertio, scilicet in specie; sed genus qua totum potentiale similiter et qua pars potentialis non constituatur ut est in tertio simul cum differentia componens speciem, quia potius ut abstractum et separatum ab illa: igitur genus his modis \emph{consiatum [?]} non est identificatum cum differentiis. Urgentur: quae sunt eadem uni tertio sunt idem inter se, sed differentiae sunt idem realiter cum genere his modis expectato: igitur sunt idem inter se. Quod ergo absurdius? 
\pend

\pstart
 Secundum docet \edtext{ \name{\textsc{Porphyrium}\index[persons]{Porphyrius}} hic }{\lemma{}\Afootnote[nosep]{}} et \edtext{ \name{\textsc{Aristoteles}\index[persons]{Aristoteles}} 4 \worktitle{Physicorum} capitulo 3 }{\lemma{}\Afootnote[nosep]{}}. Animadverte discrimen inter genus qua totum potentiale et qua partem potentialem. Cum enim genus qua pars potentialis solum contineat differentias in genere causae materialis non valet praedicari de illis, sicut nec materia prima praedicatur de formis quas continet. Ast cum genus qua totum potentiale contineat differentias in potestate et superioritate, sicut continet species, quarum differentiae sunt partes essentiales, praedicatur non solum de speciebus, quas continet in potestate subiectiva, sed etiam de differentiis constituentibus species et cum ipsis identificatis. Unde bene audiuntur hae praedicationes: rationale est animal, rugibile est animal, et non in sensi formali, tamen in sensu materiali, et identico. 
\pend

\pstart
 Ultimum assertum. Genus qua pars actualis est differentia cum qua actu componit speciem. Unde non distinguitur realiter ab illa, sed potius identificatur. Ratio potest esse quia genus et differentia non sumuntur a diversis rebus, scilicet materia et forma, sed a tota entitate ut modo videbimus: igitur in illa specie, quam constituunt non distinguuntur. 
\pend

\pstart
 \textnormal{|}\ledsidenote{BNC 144ra}   Satisfacio quaesito, genus dicitur potentiale generaliter et specialiter. Explico voces. `Potentiale generaliter' id dicitur quod importat aliquam rationem communem contrahibilem per additionem alicuius differentiae inclusae non actu sed in potentia in ipso. Hoc quidem commune est omni univoco ut, discriminatur ab analogo, quod imbibitur in ipsis differentiis. Cumque genus sit univocum hoc modo dicitur etiam potentiale. 
\pend

\pstart
 Ceterum specialiter genus dicitur potentiale respectu compositis metaphysicae. Nam genus significatur ut pars indeterminata et potentialis ad distinctionem differentiae, quae se habet ut pars determinata et qualificans. Unde generaliter dicitur potentiale ut distinguatur a transcendentibus et specialiter ut distinguatur a differentia qua parte. 
\pend

\pstart
 Ex quo constat genus dici potentiale et qua est totum et qua est pars, sed secundum diversas considerationes. Est pars potentialis comparatione ad differentiam et totum potentiale comparatione ad species inferiores. Demum dubitas de actuali comparatione cuius genus dicitur potentiale. Est igitur ultima differentia specifica reddens quidditatem non amplius contrahibilem, nisi per differentias materiales. 
\pend

\pstart
 Quo ad secundum investigandum Notato 5. Compositionem ex materia et forma esse physicum seu naturalem; materiam quam formam partes physicas seu naturales. Ceterum datur alia compositio non proprie ex partibus, sed ex gradibus eiusdem essentiae quorum aliter significatur ut materia, et potentia et est quod commune; aliter vero ut forma et actus est quod proprium. Verbi gratia in homine praeter compositionem realem ex materia prima et anima rationali ex gradus \textnormal{|}\ledsidenote{BNC 144rb} communi in quo convenit cum aliis, scilicet animali et ex alio proprio per quem distinguitur ab illis, scilicet rationali. Hanc compositionem vocat Auctores vulgus metaphysicae, ego etiam sed per antonomasiam a digniori scientia proprie vero eam voco logicam. 
\pend

\pstart
 Suppono igitur esse compositionem logicam videamus quod sit. Pro quod Notato 6. In compositione metaphysica proprie non esse partes sed gradus. Partes compositi physici sunt separabiles, quare hoc compositum ita ex his partibus compositus ut una pars non includatur in alia, nec de illa, nec de toto praedicetur: non valet `anima est corpus' vel econtra `anima est homo' vel econtra. At partes metaphysicae non sunt \secluded{non sunt} res \emph{distinctae [?]} componentes unum totum, sed diversae designationes eiusdem rei penes magis vel minus determinatum. Ita \edtext{ Divus \name{\textsc{Thomas}\index[persons]{Thomas Aquinas}} \worktitle{De ente et essentia} capitulo 3 }{\lemma{}\Afootnote[nosep]{}}. 
\pend

\pstart
 Cum enim tales gradus differant penes magis vel minus determinatum eiusdem rei proprie non sunt partes realiter distinctae, sed idem totum magis vel minus designatum vel determinantum. Uno verbo dicam, idem totum ut exprimens seu designans rationem minus determinatam. Verbi gratia animal et idem totum ut designans rationem magis determinatam; verbi gratia rationale componunt hominem, quae exprimit intentionem rationalem quidditatis. Sic intelligitur  \edtext{ Divus \name{\textsc{Thomas}\index[persons]{Thomas Aquinas}} in 1 distinctione 25, questione 1, articulo 1, ad secundum }{\lemma{}\Afootnote[nosep]{}}. 
\pend

\pstart
 Unde genus non dicitur pars physica, sed metaphysica, quae simul esse potest totum. Hoc pacto pars in quantum designat et exprimit unum tantum constitutivum quidditatis specificae: totum vero in quantum implicite ex ipso modo significandi et concipiendi dicit totum quod est in inferiori de quo praedicatur. Loquatur exemplum animal ex parte rei signficatae solum dicit explicat et designat sensibilitatem, non vero omne pertinens \textnormal{|}\ledsidenote{BNC 144va}   ad essentiam hominis vel equi, etc., at implicite dicit quidquid est in homine vel in equo, etc. Nam non sic explicat sensibilitatem ut excludat omne valens coniugi ipsi sensibilitati. Hoc enim quod est, non excludere omne valens coniungi sensibilitati est illud implicite continere. Articulo 7. Utrum Genus praedicetur? 
\pend

        \addcontentsline{toc}{section}{Articulus 7. Utrum Genus praedicetur ut pars vel ut totum}
        \pstart
        \eledsection*{Articulus 7. Utrum Genus praedicetur ut pars vel ut totum}
        \pend
      
\pstart
 Solum est advertendum quaestio procedere etiam de universalitate, an scilicet genus sit universale et praedicabile prout est pars vel prout est totum? Prima sententia asserit genus praedicari de speciebus non solum qua est totum, sed etiam qua est pars absque distinctione partis actualis a potentiali. Ita \edtext{ \name{\textsc{Scotus}\index[persons]{Ioannes Duns Scotus}} questione 16 }{\lemma{}\Afootnote[nosep]{}} et nostro  \edtext{ \name{\textsc{Paulus Venetus}\index[persons]{Paulus Venetus}} capitulo \worktitle{de specie} }{\lemma{}\Afootnote[nosep]{}}; \edtext{ \name{\textsc{Oña}\index[persons]{Petrus de Oña}}  articulo ultimo }{\lemma{}\Afootnote[nosep]{}}; \edtext{ \name{\textsc{Sanchez}\index[persons]{Ioannes Sanchez Sedeno}} libro 3, quaestione 6 }{\lemma{}\Afootnote[nosep]{}} et alii. Secunda sententia docet genus praedicari qua partem potentialem. Ita \edtext{ \name{\textsc{Canterus}\index[persons]{Ioannes Cantero}}  in hoc capitulo, dubia 3 }{\lemma{}\Afootnote[nosep]{}}. His relictis. 
\pend

\pstart
 Sit nostra conclusio: genus praedicatur de speciebus prout est totum potentiale. Ita nostro Fundatissimus \edtext{ Doctor \name{\textsc{Aegidius Columnae}\index[persons]{Aegidius Romanus}} Sanctae Romanae Ecclesiae Cardinalis Archiepiscopus Bituricensis in 1 distinctione 25, principio 1, questione 3 et 7 \worktitle{Metaphysicae} questione 27 }{\lemma{}\Afootnote[nosep]{}} citatus a nostro \edtext{ \name{\textsc{Friderico Nicolao Gavardi}\index[persons]{Fridericus Nicolaus Gavardi}} quaestione \worktitle{de genere} articulo 1, sectione 1 }{\lemma{}\Afootnote[nosep]{}}. Ita \edtext{ \name{\textsc{Aristoteles}\index[persons]{Aristoteles}} 2 \worktitle{Posteriorum} capitulo 8 et 5 \worktitle{Metaphysicae} }{\lemma{}\Afootnote[nosep]{}}. Sic \edtext{ \name{\textsc{Porphyrius}\index[persons]{Porphyrius}} }{\lemma{}\Afootnote[nosep]{}} et \edtext{ \name{\textsc{Angelicus Praeceptor}\index[persons]{Thomas Aquinas}} \worktitle{Opusculo 48}, tractato 1, capitulo 4 et in 1 distinctione 25 quaestione 1, articulo 1, ad 2 et \worktitle{De ente et essentia} capitulo 3 }{\lemma{}\Afootnote[nosep]{}}. Quae sequuntur \edtext{ \name{\textsc{Caietanus}\index[persons]{Thomas de Vio Caietanu}} }{\lemma{}\Afootnote[nosep]{}} ibi et alii relati a \edtext{ \name{\textsc{Complutensis}\index[persons]{}} Disputatione 5, questione 5, sectione 3 }{\lemma{}\Afootnote[nosep]{}}. 
\pend

\pstart
 Dubitas. Unde venit tantus \emph{Doctores [?]} \textnormal{|}\ledsidenote{BNC 144vb} hucusque nec nominatus? Dico, haberi \del{habens} \emph{\added{invonisse} [?]}, praedictum \textsc{Gavardinum}\index[persons]{Fridericus Nicolaus Gavardi} \emph{\del{nomine egregius tam, sed professio Theologis.} [?]} Ne autem mihi obstat `Sathirus' omnia videns et ignorans contempenti tam doctrinam non novam, quam authorem ignavum etsi clarum, eum produco et otiose. 
\pend

\pstart
 Probatur 1. Ut aliquid praedicatur de aliquo debet saltae in potentia continere illud, sed genus prout pars non continet species: igitur prout pars non praedicatur de speciebus. Maior apud omnes est nota. Minor manet ut certa ex praecedentibus. Nam genus ut pars significat aliquam perfectionem cum exclusione aliarum: igitur prout sic non praedicatur. 
\pend

\pstart
 Probatur 2. Ex nostro \textsc{Fundatissimo Doctore}\index[persons]{Aegidius Romanus} de ratione generis est quod praedicetur de pluribus, sed genus prout est pars actualis vel potentialis alicuius speciei non praedicatur de pluribus. Non enim valet animalitas hominis est animalitas bruti: igitur genus non praedicatur ut pars actualis vel potentialis. 
\pend

\pstart
 Obiectio 1. Ex \edtext{ \name{\textsc{Philosopho}\index[persons]{Aristoteles}} 1 \worktitle{Posteriorum} capitulo 4 }{\lemma{}\Afootnote[nosep]{}} numerante inter modos per se praedicationem in qua definitio et eius partes praedicantur de definito: igitur iam genus prout est pars praedicatur. Respondeo Philosophum locutum fuisse materialiter. In sensus partes definitionis, id est gradus praedicantur de definito non tamen qua partes, sed qua tota. 
\pend

\pstart
 Obiectio 2. Genus non est proprie pars, sicut est materia: igitur bene potest praedicari. Respondeo concedendo antecedens et negando consequentiam. Sicut enim genus non sit proprie pars ut dictum est, tamen qua gradus designans minus determinatum, non designat totum quod est in specie: igitur prout sic non potest praedicari. Discrimen enim inter partes physicas et logicas manet statutum. \textnormal{|}\ledsidenote{BNC 145ra}    Nam partes physicae cum sunt simpliciter partes non possunt esse simpliciter tota. Unde non praedicantur in recto, verbi gratia `homo est matura et forma'. At gradus possunt concipi cum mutua praecisione et sic sunt partes vel sine tali praecisio et sic sunt tota. 
\pend

\pstart
 Instabis: etiam materia praedicatur de toto, verbi gratia `homo est materia': igitur etiam genus. Respondeo distinguendo. Materia praedicatur ut pars naturae, per modum totius, verbi gratia `homo est materialis', concedo. 
\pend

\pstart
 Obiectio 3. Genus ut pars potentialis non admittit potentialem continentiam qua continebat species: igitur prout sic praedicatur. Respondeo negando antecedens. Genus enim ut pars sive actualis, sive potentialis sumitur cum exclusione alterius compartis: igitur prout sic excludit illam. In quo differat genus pro ut totum potentiale et prout pars potentialis manet explicatum numero 711, 712 et 713. 
\pend

        \addcontentsline{toc}{section}{Articulus 8. Unde sumantur genus et differentia tam in substantiis quam in accidentibus}
        \pstart
        \eledsection*{Articulus 8. Unde sumantur genus et differentia tam in substantiis quam in accidentibus}
        \pend
      
\pstart
 Notandum est quasdam esse naturas compositas ex materia et forma ut homo; quasdam solum formas licet dependentes a materia ut accidentia, quasdam denique nec compositas ex materia et forma, nec dependente a materia ut Angeli. 
\pend

\pstart
 Praesens difficultas solum stat in explicatione axiomatis, scilicet Genus sumitur a materia et differentia a forma. Sed nota genus sumi a materia non esse, genus esse materiam. Siquidem genus dicit totum essentiam: igitur non solum partem, scilicet \emph{materialem [?]}. 
\pend

\pstart
 \textnormal{|}\ledsidenote{BNC 145rb} Sit prima conclusio: in substantiis spiritualibus genus et differentia sumuntur proxime a tota essentia, radicaliter vero genus ab ipsa entitate ut indeterminata et perfectibili; et differentia ab ipsa entitate ut determinata et perfecta. Ita noster Fundatissimus \edtext{ Doctor \name{\textsc{Aegidius}\index[persons]{Aegidius Romanus}} in 2 distinctione, 3 parte, 1 questione, articulo 1, sectione advertendum et 2 \worktitle{Physicorum} \emph{4 [?]}, capitulo 28 }{\lemma{}\Afootnote[nosep]{}} quae sequitur nostra schola. Ita etiam \edtext{ \name{\textsc{Angelicus Doctor}\index[persons]{Thomas Aquinas}} 1 parte, quaestione 50, articulo 2, ad 1, quae sequuntur omnes }{\lemma{}\Afootnote[nosep]{}} eius discipuli et alii citati a \textsc{Complutensibus}\index[persons]{}. 
\pend

\pstart
 Probatur ratione: in substantiis spiritualibus non est alia compositio realis praeter compositionem ex essentia et existentia, sed ab hac compositione non potest sumi genus et differentia: igitur ab ipsa simplici entitate. Minor sic probatur. Implicat differentia sumi seu fundari supra id quod non peritne ad essentiam eius. Sed existentia non pertinet ad essentiam substantiae spiritualis: igitur compositio logica ex genere et differentia nequit fundari in illa reali compositione. 
\pend

\pstart
 Sit secunda conclusio: in substantiis corporeis genus et differentia sumuntur proxime a tota entitate, radicaliter vero genus sumitur a materia et differentia a forma. Ita Auctores citati praecipue \edtext{ Doctor \name{\textsc{Thomas}\index[persons]{Thomas Aquinas}}loco citati et 1 parte et \worktitle{Opusculo 42} }{\lemma{}\Afootnote[nosep]{}}. 
\pend

\pstart
 Probatur ratione: in substantiis materialibus omnis potentialitas oritur a materia et omnis actualitas a forma, sed gradus genericus est potentialis et perfectibilis; gradus quae differenntialis est actualis et perfectivus: igitur nedum genus et differentiam sumuntur proxime a tota entitate, sed radicaliter genus sumitur a materia et differentia. \textnormal{|}\ledsidenote{BNC 145va}    a forma. Probatur maior. Proximum in unoquoque genere est causa esorum, quae sunt postea in illo genere ex \edtext{ \name{\textsc{Philosopho}\index[persons]{Aristoteles}} 2 \worktitle{Metaphysicae} texto 4 }{\lemma{}\Afootnote[nosep]{}}. Sed in substantiis materialibus, materia prima est prima potentia et forma est primus actus: igitur in substantiis materialibus, etc. 
\pend

\pstart
 Sit tertia conclusio: in accidentibus genus et differentia proxime sumuntur a tota entitate, radicaliter vero genus sumitur a proprio modo essendi, scilicet quo quodlibet praedicamentum accidentium constituitur in esse talis entis. Differentia vero sumitur radicaliter ex diversitate principiorum a quibus causantur. Ita Auctores citati. 
\pend

\pstart
 Probatur ratione. In accidentibus non datur compositio ex materia et forma: igitur genus et differentia non sumuntur a materia et forma. Item non est alia compositio realis ex actu et potentia: igitur nec ab his sumuntur. Ergo sumuntur a tota entitate: igitur essentia ut determinabilis et perfectibilis est radix generis, et ut determinata et perfecta est radix differentiae. 
\pend

\pstart
 Obiectio 1. Contra omnem conclusionem. Ex eodem principio non possunt sumi genus et individuatio, sed individuatio sumitur a materia: igitur genus non sumitur a materia. Maior probatur. Genus est indeterminatum individuum determinatum, sed determinatum et indeterminatum non possunt sumi ab eodem principio: igitur genus et individuatio non possunt sumi ab eodem principio. 
\pend

\pstart
 Respondeo distinguendo maiorem. Non possunt sumi genus et individuatio ab eodem principio secundum diversos respectus et considerationes, nego; secundum eundem respectum et considerationem concedo. Igitur \textnormal{|}\ledsidenote{BNC 145vb} quatenus materia respectu formae relinquit indeterminationem et indifferentiam ad resipiendum gradum imperfectum unius formae et ultra alium perfectio rem usque ad ultimum est radix generis. At quatenus resipit formam est subiectum ultimum ita ut non est ulterius communicabile et ratione huius ultimae incommunicabilitatis est radix individuationis. 
\pend

\pstart
 Obiectio 2. Contra secundam. Gradus genericus non solum dicit potentialitatem, sed etiam aliquam actualitatem: igitur prout sic non sumitur genus a materia. Respondeo distinguendo consequens. Non sumitur genus a materia ut praecisa a forma, concedo; ut coniuncta, nego. In hoc stat discrimen inter materiam et formam prout radices generis et differentiae. Et est disparitas argumenti. Cum enim matura nude sumpta sit pure potentia hailla actualitas generis quamvis imperfecta non potest sumi a materia nude accepta, sed a materia habente esse imperfectum a forma. At cum forma sit radix omnium perfectionum ipsa sola est radix differentiae. 
\pend

\pstart
 Instabis: non minus animal dicit potentialitatem et aliquam actualitatem quamvis imperfectam, quam rationale dicit aliquam potentialitem et si dicat actualitatem, sed eo quod genus dicat potentialitatem et aliquam actualitatem connotat materiam et formam: igitur eo quod differentia dicat actualitatem et aliquam potentialitatem connotat etiam formam et materiam. Quod differentia dicat aliquam potentialitatem probat: nam rationale, nempe hominis differentia, non significat quam cunque naturam intellectivam, sed materialem: igitur. 
\pend

\pstart
 Respondeo concedendo totum. Ast est maximum discrimen in hac connotatione. Genus enim est compositum \textnormal{|}\ledsidenote{BNC 146ra}    ex gradu superiori et proprio constitutivo; differentia vero est conceptus simplex absque ulla compositione. Unde genus connotat materiam et formam intrinsece; differentia vero extrinsece et illative cum ex differentia inferatur genus. 
\pend

\pstart
 Obiectio 3. Ex prima et tertia conclusione apparet eodem modo sumi radicaliter genus et differentiam in accidentibus et substantiis spiritualibus. Respondeo non esse eodem modo. Substantiae spirituales sunt entia simpliciter: igitur habent perfectum definitionem. Accidentia enim sunt entia secundum quod: igitur habent definitionem incompletam et per additionem, hoc est cum dependentia a subiecto. \emph{Tamen [?]} discrimen essentia in substantiis spiritualibus ut perfecta sine dependentia ab aliquo extrinseco est radix differentiae. Ceterum essentia accidentis completur in esse radicis. per extrinsecum subiectum. Ideo dixi, differentia in hominis sumi radicaliter a propriis principiis, id est a propriis subiectis a quibus causantur. Hoc non potest expedite percipi per nunc usquedum de singulis disputamus intelligetur. 
\pend

\pstart
 Quaeris 1. Quae natura est capax genereitatis? Dico, quae est capax esse directe in praedicamento sive sit realis, sive rationis, sive sit substantia aut modus sine accidens. Denique omne ens, quod est habet rationem totius constans compositione metaphysica est etiam capax habendi genus et differentiam. Unde excluduntur partes sive physecaephysicae, sive metaphysicae sive integrantes. 
\pend

\pstart
 Ratio excludendi partes physicas et integrantes haec est. Quod non potest habere genus supremum, non potest habere genus intermedium, sed hae \textnormal{|}\ledsidenote{BNC 146rb} partes non possunt habere genus supremum: igitur. Probatur minor. Si haberent genus supremum hoc vel esset substantia completa vel incompleta? Si primum contra: quod vere non praedicatur de aliquo, non praedicatur generice; substantia completa vere no praedicatur de partibus: igitur nec generice. Si secundum contra: substantia incompleta se habet transcendentaliter: igitur non est genus. Antecedens probatur. quod imbibitur in ultimis differentiis transcendentaliter se habet, sed substantia incompleta imbibitur in ultimis differentiis: igitur. Probatur minor. Substantia incompleta imbibitur in omni incompleto, sed ultimae differentiae partium sunt incompletae: igitur. Simile argumentum fit de corpore viventi, etc. Hoc argumentum non militat contra accidentia, nam non habent partes, sed sunt formae simplices. 
\pend

\pstart
 Contra partes metaphysicas haec est ratio. Si differentia haberet genus daretur processus in infinitum: igitur non differentia non habet genus. Probatur sequela. Omne genus est contrahibile per differentiam; sed differentia habet genus: igitur hoc est contrahibile per aliam differentiam. Item haec differentiam habet genus: igitur contrahibile per aliam differentia, etc. Ergo differentia non habet genus, sed reducitur ad latus. 
\pend

\pstart
 Quaeris 2. An haec praedicatio `haec forma est forma', `hoc caput est caput' sit praedicatio speciei de individuo? Dico esse secundum quod et reductive, quia ibi non praedicatur natura, sed aliquod naturae. Unde licet sit praedicatio essentialis est incompleta et sic reductive pertinet ad suum completum. Aliae difficultates solent hic agitari, sed quia conectuntur cum aliis praedicabilibus omittuntur hic ut alibi tractandae. 
\pend

        \addcontentsline{toc}{chapter}{Capitulum secundum}
        \pstart
        \eledchapter*{\supplied{Capitulum secundum}}
        \pend
      
        \addcontentsline{toc}{section}{Textus capituli secundi Porphyrii}
        \pstart
        \eledsection*{Textus capituli secundi Porphyrii}
        \pend
      
\pstart
 \textnormal{|}\ledsidenote{BNC 146vb} \edtext{\enquote{ ΚΕΦΑΛΑΙΟΝ Β Τὸ δὲ εἶδος λέγεται μὲν καὶ ὲπὶ τῆσ ἑκάστου μορφῆς, καθὸ ο εἴρηται [...]. Λέγεται δὲ εἶδος καὶ τὸ ὑπὸ τὸ ἀποδοθὲν γένος, καθὸ εἰώθαμεν λέγειν τὸν μὲν, καὶ τὰ ἕτερα. }}{\lemma{}\Afootnote[nosep]{ \textsc{Commentarii Collegii Conimbricensis e Societate Iesu}\index[persons]{}, \worktitle{In universam dialecticam Aristotelis Stagirita} (Lugduni: sumpt. Iacobi Cardon et Petri Cavellat, 1622), p. 105. }} 
\pend

\pstart
 \textnormal{|}\ledsidenote{BNC 146vc} \edtext{\enquote{ Caput secundum Species autem dicitur quidem et de uniuscuiusque forma, qua significatione dici solet praestantissimam formam dignam esse imperio. Dicitur autem et de eo quod sub assignato genere collocatur, etc. }}{\lemma{}\Afootnote[nosep]{ \textsc{Commentarii Collegii Conimbricensis e Societate Iesu}\index[persons]{}, \worktitle{In universam dialecticam Aristotelis Stagirita} (Lugduni: sumpt. Iacobi Cardon et Petri Cavellat, 1622), p. 105. }} 
\pend

        \addcontentsline{toc}{section}{Summa Textus}
        \pstart
        \eledsection*{Summa Textus}
        \pend
      
\pstart
 \textnormal{|}\ledsidenote{BNC 146va} Tres continet partes hoc caput. In prima acceptiones et speciei definitiones enumerantur. In secunda series praedicamenti constituitur adducto exempto in substantia. In tertia definitur individuum. Quoad primam species primo sumitur pro pulchritudine externa. Unde dicitur: \enquote{praestissima species digna est imperio.} Secundo logice et hoc dupliciter primo pro specie subiicibili quatenus subiicitur generi quam sic definit: \enquote{species est id quod subiicitur generi et de quo genus hoc ipso quod est seu in eo quod quid praedicatur.} Secundo pro specie praedicabili, quam sic definit: \enquote{Species est id quod de pluribus differentibus numero hoc ipso quod es praedicatur.} In secunda parte docet in unoquoque praedicamento esse et genus supremum (supra quod aliud non est) et speciem infimam (sub qua non est alia species) et genera intermedia. Haec simul sunt genera et species: genera si comparentur inferioribus et species si superioribus comparentur. Infertus intermedia habere duos respectus, scilicet ad superiora et inferiora: extrema vero unum tamtum, sit genus supremum ad inferiore et speciem infimam ad superiora. Addit genera suprema decem esse, species infimas habere determinatum numerum et individua esse infinita. In tertia parte distinguit tres acceptiones individui, scilicet individuum determinatum ut `Socrates'; demonstrativum ut `hic homo' demonstrato Socrate et circumlocutum seu ex suppositione ut `\textsc{Sophronici}\index[persons]{Sophroniscus} filius', supposito \textsc{Socrates}\index[persons]{Socrates} fuisse unicum filium. Individuum commune sic definit: \enquote{Individuum est quod de uno tantum praedicatur.} Item \enquote{est illud quod ex talibus conditionibus constat quarum item aggregatum numquam in alio fit.} 
\pend

        \addcontentsline{toc}{section}{Annotationes circa Litteram Capitis}
        \pstart
        \eledsection*{Annotationes circa Litteram Capitis}
        \pend
      
\pstart
 \textnormal{|}\ledsidenote{BNC 147rb} Sicut \textsc{Porphyrius}\index[persons]{Porphyrius} capitulo praecedenti non posuit omnes acceptiones generis, sed dumtaxat habentes aliquam similitudinem cum genere; ita pariter hic non numerat omnes acceptiones speciei. Species enim aliter sumitur a Rhetoricis, aliter a Grammaticus et aliter a Iurisprudentibus. Etiam species sumitur per pulchritudine similitudinarie, nam sicut pulchritudo provenit ex actuatione materiae, quae ex se est informis, sic species logica provenit ex actuatione generis, quod ex se est potentiale. 
\pend

\pstart
 Nec obstat \textsc{Porphyrius}\index[persons]{Porphyrius} fecisse circulum in his definitionibus tradendo, scilicet genus per speciem, speciemque per genus. Nam talis circulus non est vitiosus, sed potius necessarius in correlativis. Verbi gratia vero relatio terminetur ad absolutum vel ad relativum relinquitur suo loco examinandum. 
\pend

\pstart
 Quaeris primum. Quomodo praedicata intermedia habeant duos respectus extrema vero unum tantum? Ratio dubitandi est: etiam species infima praedicatur de individuis, sed supponitur subiicibilis: igitur dicit duos respectus, scilicet ad genus, quo constituitur subiicibilis et ad individua, quo constituitur praedicabilis: igitur male asserit \textsc{Porphyrius}\index[persons]{Porphyrius} extrema unum habere tantum respectum. 
\pend

\pstart
 Respondeo. Intentum \textsc{Porphyrii}\index[persons]{Porphyrius} esse praedicata intermedia \del{est} habere respectus praebentes diversas denominationes. Nam genus intermedium \textnormal{|}\ledsidenote{BNC 147rc} tale dicitur comparatione ad species infimas et iterum dicitur species intermedia comparatione ad genus supremum. Ceterum extrema praedicata licet dicant diversos respectus, non tamen ab eis aliter denominantur. Verbi gratia substantia sive comparetur ad ens, sive ad species semper est genus. Similiter species infima sive comparetur ad suum genus sive ad sua individua semper est species. Unde patet solutio ad rationem dubitandi. 
\pend

\pstart
 Quaeris secundum. Quomodo individuum collocandum sit in arbore praedicamentali? Respondeo. Individuum non poni in recta linea ut partem essentialem coordinationis, sed tantum ut fundamentum cui innituntur omnia illa praedicata. 
\pend

\pstart
Quaeris tertium. Quomodo differant ille tres acceptiones individuis, scilicet `\textsc{Socrates}\index[persons]{Socrates}', `hic homo' et `\textsc{Sophronici}\index[persons]{Sophroniscus} filius>'? Dico eamdem rem significare; differre tamen quoad modum significandi. Nam `\textsc{Socrates}\index[persons]{Socrates}' cum significat expraesse et determinate; `hic homo' significat confusse et indeterminate; et denique `\textsc{Sophronici}\index[persons]{Sophroniscus} filius' eum significat per quamdam circunlocutionem. Nota quod licet \textsc{Porphyrius}\index[persons]{Porphyrius} hic faciat mentionem de praedicamentis, nos tamen suo loco, scilicet \emph{cum [?]} \textsc{Aristotele}\index[persons]{Aristoteles} exponamus. Similiter proponam arborem hic, sed infra. Nunc vero de specie logica agamus pro quo. Quaestio septima de natura et proprietatibus secundi praedicabilis, scilicet speciei. 
\pend

        \addcontentsline{toc}{section}{Quaestio 9: De specie subii subicibili et praedicabili}
        \pstart
        \eledsection*{Quaestio 9: De specie subii subicibili et praedicabili}
        \pend
      
\pstart
 Genus enim de specie praedicatur prout sic enim est subiicibilis iterum ipsa de individuis praedicatur prout sic est praedicabilis. Ut enim explicemus secundum praedicabile necesse est examinare subiicibilitatem et praedicabilitatem. Unde Articulus 1. An definitiones, etc. 
\pend

        \addcontentsline{toc}{section}{Articulus 1. An definitiones speciei a Porphyrio traditae sint bonae?}
        \pstart
        \eledsection*{Articulus 1. An definitiones speciei a Porphyrio traditae sint bonae?}
        \pend
      
\pstart
\noindent%
 Articulus 1. An definitiones speciei a Porphyrio traditae sint bonae? 
\pend

\pstart
 Sic definitur species prout subiicibilis: \enquote{Species est id quod subiicitur generi et de quo genus in eo quod quid praedicatur.} Praedicabilis sic: \enquote{Species es id quod de pluribus differentibus numero hoc ipso quod est praedicatur.} Sed nota, definitio essentialis speciei subiicibilis claudi illis verbis; species est id quod subiicitur generi vel in quod est genus; et descriptiva in aliis, scilicet et de quo genus hoc ipso quid est praedicatur. Definitio vero speciei praedicabilis non est essentialis, sed descriptiva, manet hoc probatum in definitione generis 
\pend

\pstart
 Sit conclusio: tales definitiones sunt bonae. Ita omnes Auctores sequentes \edtext{ \name{\textsc{Aristotelem}\index[persons]{Aristoteles}} 2 \worktitle{Topicorum} capitulo 2, libro 4, capitulo 1 et 2 }{\lemma{}\Afootnote[nosep]{}}. Ita \edtext{ \name{\textsc{Angelicus Doctor}\index[persons]{Thomas Aquinas}} \worktitle{Opusculo 18}, tractato 1, capitulo 3 }{\lemma{}\Afootnote[nosep]{}}. Ex dictis manet probata conclusio: siquidem traduntur per subiectum loco generis, item quae sunt descriptive traduntur per probationes: igitur sunt bonae. Argumentis enim magis, magisque corroborabitur. 
\pend

\pstart
 \textnormal{|}\ledsidenote{BNC 147vb} Obiectio 1. Etiam genus praedicatur de individuo: igitur individuum est species. Respondeo distinguendo antecedens. Genus praedicatur de individuo immediate, nego; mediante, concedo. Contra ergo hic immediate \emph{distinguo [?]} definitioni. Nego consequentia. Hic enim agimus de eo quod tamquam universale seu unum ex quinque praedicabilibus subiicitur. Cumque individuum non subiiciatur ut universale, ideo nequem ut species. 
\pend

\pstart
 Instabis: genus praedicatur de individuo generico immediate: igitur individuum genericum est species. Respondeo distinguo antecedens. Praedicatur de individuo generico tamquam de universali, nego; tamquam de individuo, concedo. Et distinguo amplius: praedicatur immediate simpliciter, nego; secundum quod, et quo ad explicationem, concedo. In re non datur tale individuum nisi mediante specie. Unde quando duo `hoc animal' intelligitur aliqua specie mediante singularizatum. 
\pend

\pstart
 Obiectio 2. Deus praedicatur in quid de tribus personis numero differentibus, sed non est species: igitur species, etc. Respondeo negando maiorem. Non praedicatur Deus ut multiplicabilis, nec ut contrahibilis a personis, sed ut natura singularis. Unde natura est universalis, nec personae inferiora numerro differentia. Sunt tria supposita naturae singularis quare multiplicantur supposita non multiplicata natura. 
\pend

\pstart
 Obiectio 3. Entia incompleta praedicantur de numero differentibus: igitur sunt species. Respondeo. Praedicantur de numero differentibus simpliciter et directe, nego; secundum quod et reductive, concedo. Sicut enim entia incompleta directe non ponuntur in \textnormal{|}\ledsidenote{BNC 148ra}   praedicamento, sed reductive; sic non per se, sed reductive sunt genus, species, etc. 
\pend

\pstart
 Obiectio 4: Sol, Luna, Phoenix et natura angelica sunt species, sed nihilominus non habent individua: igitur. Respondeo distinguendo minorem. Non habent individua existentia, concedo; possibilia, nego. Hoc enim sufficit ut sunt praedicabiles. Quod iam si essent impossibilia? Postea resolvamus. 
\pend

\pstart
 Obiectio 5. Numerus est species quantitatis, in substantiis spiritualibus non datur quantitas: igitur nec numero differentia. Respondeo distinguendo maiorem. Numerus praedicamentalis est species quantitatis, concedo; numerus transcendentalis, nego. Hic enim sufficit ut saltem in accidentibus spiritualibus detur praedicatio de numero differentibus. 
\pend

\pstart
 Obiectio 6. Etiam differentia praedicatur in quod de differentibus, nego: igitur mala est definitio speciei. Respondeo distinguendo antecedens. Praedicatur in quod quoad rem praedicatam, concedo; quoad modum praedicanti, nego. Differentia enim praedicatur in quale quid, species vero in quid. 
\pend

\pstart
 Obiectio 7. Unum correlativum non potest definiri per aliud correlativum, sed genus et species \emph{differri [?]} sunt correlativa: igitur non potest species per genus definiri. Respondeo distinguendo maiorem. Unum correlativum per aliud tamquam per partem essentialem non potest definiri, concedo; tamquam per quod extrinsecum et connotatum, nego. Sic enim potius requiritur unum correlativum ad definitionem alterius. 
\pend

\pstart
 Obiectio 8. Genus definitur in ordine ad speciem: igitur si species definiretur per genus daretur circulus. Respondeo primum ut in numero 748. Respondeo secundum si unum correlativum ingrederetur definitionem. \textnormal{|}\ledsidenote{BNC 148rb} et pars essentialis, talis circulus implicaret, et non implicat quia solum connotatur. Articulo 2. 
\pend

        \addcontentsline{toc}{section}{Articulus 2. Utrum species constituatur in esse talis per subiicibilitatem vel praedicabilitatem}
        \pstart
        \eledsection*{Articulus 2. Utrum species constituatur in esse talis per subiicibilitatem vel praedicabilitatem}
        \pend
      
\pstart
 Duos respectus cognoscimus in universali, scilicet esse in multis et praedicari de multis. Primus est essentialis, scilicet universalitas; secundus es propria passio, scilicet praedicabilitas. His correspondent alii duo respectus in subiicibili. Primus dicitur subiicibilitas in qua et secundus subiicibilitas de qua. Primus respectus pertinet ad essentiam subiicibilis et secundus est passio. Unde si species infima est subiicibilis et praedicabilis debet habere quatuor relationes; dicas ut subiicibilis et duas ut praedicabilis. Ut subiicibilis habet subiicibilitatem in qua quatenus est id in quo est genus; et subiicibilitatem de qua quatenus est id de quod genus praedicatur. Similiter ut praedicabilis habet universalitatem quatenus est apta esse in pluribus et praedicabilitatem quatenus est apta praedicari de pluribus. 
\pend

\pstart
 Unde sicut praedicabilitas es propria passio universalis, ita subiicibilitas de qua est proximo subiicibilitatis in qua. Non quaerimus an praedicabilitas sit passio univesalis? An subiicibilitas de qua sequatur necessario ad subiicibilitatem in qua? Quaerimus igitur an utraque subiicibilitas competat naturae ut universali ut sit subiicibilis quia universalis vel an hae relationes se habeant disparate? 
\pend

\pstart
 \textnormal{|}\ledsidenote{BNC 148va}   Prima sententia asserit subiicibilitatem esse essentiam specie ideoque oriri ut proprietates universalitatem et praedicabilitatem. Ita \textsc{Caietanus}\index[persons]{Thomas de Vio Caietanu} et alii Secunda sententia est \edtext{ \name{\textsc{Francisci Toleti}\index[persons]{Franciscus Toletus}} quaestione 1 }{\lemma{}\Afootnote[nosep]{}} asserit specie esse universale et qua est subiicibilis et qua est praedicabilis. 
\pend

\pstart
 Sit prima conclusio: species formaliter quatenus subiicibilis non est universalis. Est communis inter Thomistas et expresse \edtext{ \name{\textsc{Ioannes a Sancto Thoma}\index[persons]{Ioannes a Sancto Thoma}} \worktitle{conclusione} 1 }{\lemma{}\Afootnote[nosep]{}}; \edtext{ \emph{Illustrissimus [?]} \name{\textsc{Rubius}\index[persons]{Antonius Ruvius Rodensis}} capitulo 3, \worktitle{De specie}, quaestione 2, numero 9 }{\lemma{}\Afootnote[nosep]{}} et \edtext{ \name{\textsc{Complutensis}\index[persons]{}} hic eam docent }{\lemma{}\Afootnote[nosep]{}}. 
\pend

\pstart
 Probatur ratione. Constitutivum speciei ut praedicabilis est universalitas: igitur constitutivum speciei ut subiicibilis non est universalitas. Probatur antecedens. Si constitutivuum speciei praedicabilis est universalitas: igitur tale constitutivum non stat in subiicibilitate. Patet consequentia. Universalitas in communi non consistit in subiicibilitate: igitur universalitas in particulari, scilicet secundi praedicabilis non consistit in subiicibilitate. Patet consequentia. Implicat quod relatio in communi tendat in aliquam terminum et relatio particularis quae est eius species tendat in oppositum. Sed relatio universalis in communi tendit in inferiora: igitur implicat quod relatio speciei, quae eius species tendat in superius, scilicet in terminum oppositum. Ergo species ut \emph{prima [?]} universalis debet respicere inferiora, sed ut subiicibilis non respicit inferiora: igitur ut subiicibilis non est universalis. 
\pend

\pstart
 Sit secunda conclusio: subiicibilitas et praedicabilitas sunt diversae relationes ac proinde una non oritur ex alia. Est communis inter Thomistas. Eam tenet \edtext{ \emph{Illustrissimus [?]} \name{\textsc{Rubius}\index[persons]{Antonius Ruvius Rodensis}}, }{\lemma{}\Afootnote[nosep]{}} \edtext{ \name{\textsc{Complutenses}\index[persons]{}} et alii ab }{\lemma{}\Afootnote[nosep]{}} ipsis citati lici supra relatio. 
\pend

\pstart
 \textnormal{|}\ledsidenote{BNC 148vb} Probatur ratione. Subiicibilitas et praedicabilitas adveniunt naturae comparatae ad diversos terminos. Sed comaprationes quibus sic denominatur non dependent  ad invicem, nec ex sese oriuntur: igitur \emph{nec [?]} rationes sit subiicibilitas et praedicabilitas inter se. Probatur minor. Eo quod intellectus comparet naturam ad genus non comparat ad individua: igitur tales comparationes non dependent ad invicem. 
\pend

\pstart
 Obiectio 1. Omne inferius participat rationem sui superioris, sed species subiicibilis est inferior universale: igitur participat universalitatem. Respondeo concedendo totum. Nam species, quae est secundum praedicabile est universalis. Contra. Sed inquantum suiicibilis est universalis: igitur. Probatur minor. In tantum species participat universalitatem inquantum respicit genus, sed inquantum respicit genus subiicitur generi: igitur inquantum subiicitur generi est universalis. 
\pend

\pstart
 Respondeo negando minorem. Ad probationem nego maiorem. Potius valet econtra, In tantum respicit genus, inquantum participat. Prius enim est secundum nostrum modum intelligendi, quod homo sit animal, quam quod respici animal ut genus. Siquidem ex eo quod sit animal sumit fundamentum ut eum consideret respicientem animal ut illi subiectum. Similiter igitur prius consideramus universalitatem in specie, deinde concipimus eam ut subiectam generi. 
\pend

\pstart
 Obiectio 2. Species non constituitur in ratione secundi praedicabilis per respectum ad sua inferiora, sed per respectum ad suum superius: igitur ut subiicibilis est universalis. Probatur antecedens. Tota ratio speciei secundi praedicabilis est esse correlativum generis, sed hanc rationem non habet prout respicit inferiora: igitur. Respondeo negando antecedens. Ad cuius probationem distinguo maiorem. Tota ratio speciei ut subiicibilis \textnormal{|}\ledsidenote{BNC 149ra}   est esse correlativum generis, concedo; ut praedicabilis et universalis, nego. Patet solutio. 
\pend

\pstart
 Obiectio 3. Ab hac et illa specie subiicibili potest abstrahi ratio ut sic, sed haec ratio potest praedicari de illis: igitur prout subiicibilis erit praedicabilis. Respondeo concedendo maiorem et distinguendo minorem. Haec ratio sic abstracta prout praedicari remote, concedo; proxime, nego. Contra igitur saltem dum comparatur erit praedicabilis formaliter. Distinguo retinendo subiicibilitatem formaliter, nego; retinendo materialiter, concedo. Tunc enim compararetur ad illas species ut praedicabilis. 
\pend

\pstart
 Obiectio 4. Species subalterna est subiicibilis et non praedicabilis: igitur tota essentia speciei est subiicibilitas. Responde distinguendo antecedens. Et non praedicabilis quatenus genus, nego; quatenus species, concedo. Et distinguo consequens. Tota essentia speciei subiicibilis est subiicibilitas, concedo; speciei praedicabilis, nego. Species enim subalterna non est universalis qua species, sed qua genus, qua species habet subalternationem. Unde secundum diversas rationes habet esse universalem, scilicet ut genus; et esse subiicibiliem ut species. Ceterum species infima est universalis non qua genus, sed qua species et sic prius est in ea esse universalem quam subiicibilem. 
\pend

\pstart
 Obiectio 5. Id quod immediate subiicitur  generi est universale, sed species qua subiicibilis immediate subiicitur generi: igitur prout subiicibilis est universalis. Respondeo distinguendo maiorem. Id quod immediate subiicitur generi est universale materialiter, concedo; formaliter, nego. Vel aliter est universale specificative, concedo; reduplicative, nego. Vel aliter est universale secundo intentionaliter, concedo; tertio intentionaliter, nego. Igitur species verbi gratia est universalis secundo intentionaliter. \textnormal{|}\ledsidenote{BNC 149rb} Deinde tertio intentionaliter \del{igitur} universaliter est genus ad ipsam. Tunc enim solum constituitur in esse subiicibilis, non in esse  universalis. Sic intelligitur hic materialiter, hic specificative. Hac solutione innumera argumenta facibi negotio fracidi solventur. 
\pend

\pstart
 Obiectio 6. Primus intelligitur natura contrahens genus, quam contrahibilis, sed qua contrahens est subiicibilis: igitur qua subiicibilis supponitur prius. Respondeo distinguendo maiorem. Prius intelligitur natura contrahens genus per constitutionem ex genere et differentia, quam contrahibilis, concedo; per subiicibilitatem, nego. Vel loquimur de specie prima intentionaliter vel secunda intentionaliter. Si prima, dico prius homo constituitur animali et rationali, quam subiiciatur, quam quod subiiciatur. Siquidem primum pertinet ei secundum se et secundum per accidens operatione intellectus. Si secunda, dico prius constituitur species universalitati et specieitate, quam subiiciatur, quam quod subiiciatur. Siquidem primum pertinet ei secundi intentionaliter et secundum per accidens tertii intentionaliter. 
\pend

\pstart
 Obiectio 7. Correlativa sunt sub eodem gener, se genus qua correlativum est universalI: igitur species qua correlativum, scilicet qua subiicibilis, est universalis. Respondeo DISTIGNUENDO maiorem. sunt sub eodem genere remoto, concedo; sub eodem genere proximo, nego. Tam praedicabilitas quam subiicibilitas sunt relationes rationis et in hoc conveniunt. At distinguuntur, quia praedicabilitas sua speciali ratione constituit superius et subiicibilitas sua speciali ratione constituit inferius. 
\pend

\pstart
 Quaeris 1. Quomodo species dicatur de subiicibili et praedicabili? Respondeo. Pure aequivoce. Probatur. Quae conveniunt nomine et \textnormal{|}\ledsidenote{BNC 149va}   non re tantum conveniunt aequivoce. Sed species subiicibilis et praedicabilis praecise ratione speciei solum conveniunt nomine et non re significata: igitur tantum conveniunt aequivoce. 
\pend

\pstart
 Quaeris 2. An praedicabilitas sint nobilior subiicibilitate vel econtra? Respondeo. Praedicabilitatem nobiliorem esse. Probatur. Praedicatum nobilius est subiecto: igitur praedicabilitas nobilior est subiicibilitate. Antecedens constat. Perfectior et nobilior est forma materia; sed praedicatum est forma et subiectum materia: igitur nobilius est praedicatum subiecto. Illa consequentia probatur. Potentia accipit speciem et perfectionem ab actu, sed praedicabilitas est potentia respectu praedicati: igitur si praedicatum est nobilius subiecto, praedicabilitas nobilior est subiicibilitate. 
\pend

\pstart
 Obiectio. Relationes capiunt specificationem et perfectionem a terminis, sed subiicibilitas habet perfectionem terminum termino praedicabilitatis: igitur subiecti perfectior est subiecti. Probatur minor. Perfectius est genus individuis, sed terminus subiecti est genus et terminus praedicabilitatis est individua: igitur. 
\pend

\pstart
 Respondeo negando minorem. Ad cuius probationem nego maiorem. Cum enim genus sit potentiale individuaque includant speciei perfectionem, imperfectius est individuis. Nihilominus demus genus nobilius esse individuis.  tunc distinguo maiorem. Relationes capiunt perfectionem a terminis ut respecti, concedo; absolute, nego. Non tam a termino quam a modo suspiciendi eum relatio perficionatur. Cum enim perfectior sit respicere ut forma et praedicatum, quam ut terminus et subiectum; et species ut praedicabilis respiciat ut forma et praedicatum; et ut subiicibilis respiciat ut subiectum et terminus. Adhuc posito termino nobiliori speciei \textnormal{|}\ledsidenote{BNC 149va} subiicibilis remanet praedicabilitas nobilior subiicibilitate. Sic \textsc{Rubius}\index[persons]{Antonius Ruvius Rodensis} loco citato. 
\pend

        \addcontentsline{toc}{section}{Articulus 3. Utrum possit dari species subiicibilis, quae non sit praedicabilis?}
        \pstart
        \eledsection*{Articulus 3. Utrum possit dari species subiicibilis, quae non sit praedicabilis?}
        \pend
      
\pstart
 Haec quaestio existatur in specie non habente nisi unum tantum individuum, sed adhuc datur differentia inter naturas habentes unum tantum. Nam quaedam natura est quae de facto non habet plura individua potest tamen habere de possibili ut Sol, Luna, etc., quae habent materiam divisibilem et non naturaliter. Alia natura est quae nec de facto, nec de possibili potest habere multiplicationem individualem ex parte divisionis materiae. Siquidem caret omni materia, ut natura angelica. Demum animalia est natura quae nec de facto, nec de possibili habet plura individua. Imo nec potest concipi cum aliqua potentialitate ad individuationem et ut contrahi>bilis per singularitatem. Siquidem est actus purus. Unde non potest esse universalis, licet possit concipi ipsa natura non explicata singularitate, haec natura est divina. 
\pend

\pstart
 Non procedit difficultas de prima natura, nam certum est esse universale. Nec procedit de tertia cum saepe dixerim non esse. Procedit tantum de natura angelica. Quo igitur non est, an \emph{sint [?]} potentiales necne plures Angeli solum nego distincti? Hoc enim pertinet ad \edtext{ theologiam 1 parte, questione 5 }{\lemma{}\Afootnote[nosep]{}}. Nam si sunt potentiales no datur difficultas, dicendum erit quod de natura solis. Dato enim, quod talia individua sunt impossibilia, haec natura est universalis et praedicabilis? 
\pend

\pstart
 Pro quo. Notato 1. Habere plura individua bifariam intelligi, scilicet positive vel \textnormal{|}\ledsidenote{BNC 150ra}   per non repugnantiam saltem ex parte alicuius principii. Quando datur in re positivum fundamentum ut sit praedicatio de multis, scilicet quia in re datur multiplicatio de multis, tunc datur praedicari de multis positive. Quando autem in multis non invenitur tale fundamentum, sed in uno tantum, tunc datur praedicari per non repugnantiam. Quomodo hoc fit? Dico, quia talis natura ita se habet ad illud unum individuum ut possit ab illo abstrahi sicut si abstraheretur a pluribus, scilicet manens indifferens et cum potentia ad illam individuationem. Quare vi talis abstractionis natura no habet repugnantiam ad plura individua, licet illam habeat vi carentiam materialem. 
\pend

\pstart
 Notato 2. Quod licet in natura angelica non concipiatur respectus positive ad plura individua, nec detur unitas abstracta a pluribus individuis, potest tamen concipi natura ut indifferens et potens ad unum individuum taliter, quod ex se sufficeret ad plura si in re ponerentur. Quare talis natura ex se est pluribus communicabilis, licet non possint exerceri defectu eorum quibus facienda est communicatio. 
\pend

\pstart
 Notato 3. Quod licet ratio specifica et ratio individualis sit intrinseca. \textsc{Michaele}\index[persons]{} verbi gratia hoc no obiecte, quam eadem simplicem entitatem uno conceptu concipiamus ut constituit \textsc{Michaelem}\index[persons]{} in specie et habet rationem differentiae essentialis et alio conceptu ut individuat et habet rationem differentiae numericae constituentis eum in esse individui et singularis. Ex quo sequitur, nos posse praescindere unum ab alio et abstrahere naturam specificam \textsc{Michaelis}\index[persons]{} ab eius differentia individali. Nam licet eadem entitas realis sit differentia specifica et \textnormal{|}\ledsidenote{BNC 150rb} individualis munera tamen sunt diversa. Cum singularitas non sit de essentia alicuius creaturae. 
\pend

\pstart
 Partem negativam defendunt \edtext{ \name{\textsc{Molina}\index[persons]{Ludovicus de Molina}} 1 parte, questione 50, articulo 4 }{\lemma{}\Afootnote[nosep]{}}; \edtext{ \name{\textsc{Suarez}\index[persons]{Franciscus Suarez}} \worktitle{Disputatione 5 Metaphysicae} sectione 2, numero 18 }{\lemma{}\Afootnote[nosep]{}}; \edtext{ \name{\textsc{Perez}\index[persons]{Antonius Perez}}  certamine 3 }{\lemma{}\Afootnote[nosep]{}}; \edtext{ \name{\textsc{Rubius}\index[persons]{Antonius Ruvius Rodensis}} questione 4 }{\lemma{}\Afootnote[nosep]{}}; \edtext{ \name{\textsc{Hurtado}\index[persons]{}} disputatione 5, sectione 3 }{\lemma{}\Afootnote[nosep]{}}; \edtext{ \name{\textsc{Cabero}\index[persons]{Chrysostomus Cabero}}  tractato 3, disputatione 3 dubia 3 }{\lemma{}\Afootnote[nosep]{}}; \edtext{ \name{\textsc{Raphael de Aversa}\index[persons]{Raphael de Aversa}} questione 8, sectione 4 }{\lemma{}\Afootnote[nosep]{}} et alii extra scholam Doctoris \textsc{Thomae}\index[persons]{Thomas Aquinas}. Affirmativam partem tenent omnes \textsc{Angelico Praeceptor}\index[persons]{Thomas Aquinas} discipulos, quos citat \edtext{ \name{\textsc{Gallego}\index[persons]{Baranbas Gallego de Vera}} \worktitle{controversia} 15 }{\lemma{}\Afootnote[nosep]{}}, qui addit non esse censendum Thomistam qui oppositum senserit. 
\pend

\pstart
 Sit nostra conclusio: natura angelica inadaequate concepta est universalis et praedicabilis. Ita \edtext{ \emph{\textsc{Angelico Doctor}\index[persons]{Thomas Aquinas} [?]} in questione \worktitle{De spiritualibus creatura} articulo 8, ad 4 et \worktitle{Opusculo 42} capitulo 5 et 1 parte, quaestione 13, articulo 9 et \worktitle{De ente et essentia} capitulo 5 }{\lemma{}\Afootnote[nosep]{}}; quae sequuntur omnes Thomistae. 
\pend

\pstart
 Probatur ratione: datur individuum naturae angelicae positum in praedicamento, sed non immediate sub genere: igitur sub specie athoma. Maior est certa cum sit substantia. Minor probatur. Quod immediate subiicitur generi est universale, ex eius definitione patet. Sed individuum non est universale: igitur non ponitur immediate sub genere. Non restat alius modus ponendi nisi media specie. Tunc sic: species athoma sub qua immediate ponitur tale individuum est superior illo: igitur universalis et praedicabilis. Probatur antecedens. Talis species non praedicatur tamquam aequalis de individuo: igitur est praedicatio superioris de inferiori. Probatur antecedens. Quando individuum indentice de individuo praedicatur, non est praedicatio speciei de individuo; sed praedicata specie angelica ut aequali et non ut superiori est praedicatio individui de individuo: igitur non es praedicatio speciei. 
\pend

\pstart
 \textnormal{|}\ledsidenote{BNC 150va}   Confirmatur: natura una apta ad essendum  in pluribus est universalis, sed talis natura inadaequate concepta est apta, aptitudine quae de se est sufficiens ad plura individua, \emph{ut [?]} de facto non exerceatur multiplicatio, defectu conditionis requisitae , scilicet divisionis ex parte materiae: igitur talis natura es universalis. 
\pend

\pstart
 Confirmatur amplius: natura indifferens ad plura individua vel positive vel per non repugnantiam est universalis, sed talis natura abstracta a suo individuo est indifferens per non repugnantiam: igitur est universalis. Probatur minor. Natura sic abstracta non includit actu illam individuationem, nec pluralitatem: igitur ex vi illius conceptus non excludit unam nec plures individuationes: igitur tendit in plura saltem per non repugnantiam. 
\pend

\pstart
 Roboratur: In tantum natura angelica non esset universalis in quantum naturae angelicae repugnarent individua, sed talia individua non repugnant naturae inadaequate conceptae: igitur natura inadaequate concepta est universalis. Probatur minor. Talia individua repugnant naturae ut incommunicabili, sed inadaequate concepta non est incommunicabilis: igitur talia individua non repugnant naturae inadaequate conceptae. Maior constat. Repugnant naturae quia est individua, est individua propter irreceptibilitatem materiae, sed natura ut natura non dicit irreceptibilitatem: igitur nec incomunicabilitatem. Probatur minor. Natura inadaequate concepta est natura sub conceptu species, sed natura sub conceptu speciei non dicit modum individualem: igitur natura inadaequate concepta non dicit irreceptibilitatem. 
\pend

\pstart
 Probatur denique summatim. Natura specifica \textsc{Michaelis}\index[persons]{} secundum se potest esse in multis \textnormal{|}\ledsidenote{BNC 150vb} individuis: igitur est universalis et praedicabilis. Antecedens est \edtext{ \name{\textsc{Angelici Praeceptoris}\index[persons]{Thomas Aquinas}} loco citato dicentis: \enquote{natura huius Angeli non prohibetur esse in multis ex eo quod pertinet ad rationem speciei, sed ex eo quod pertinet ad rationem individui.}  Thomas Aquinas, \worktitle{Quaestio disputata de spiritualibus creaturis}, ed. Roberto Busa, p.???  }{\lemma{}\Afootnote[nosep]{}} Intellige nostram sententiam clarissima exemplo \emph{scilicet [?]} \edtext{ Illustrissimus \name{\textsc{Ioannes Caramuel}\index[persons]{Ioannes Caramuel Lobkowitz}} }{\lemma{}\Afootnote[nosep]{}} noto colorem esse visibilem a multis: quod si unum tantum visivum existeret in rerum natura et omnia alia visiva repugnarent? Color enim ille quantum esset ex se posset a pluribus vieri quod autem haec potentia non reduceretur ad actum proveniret non ab ipso colorem, sed a visivorum pluralitate repugnante. Nam sicut non est in culpa color, quod expositus non videatur de facto a pluribus cum unus solus aspicit. Sic similiter non esset in culpa, ne a pluribus videri posset si unus tantum haberet potentiam visivam, quae aliis non posset nec divinitus communicari. 
\pend

\pstart
 Obiectio 1. Non datur universale sine praedicatione de multis, sed natura angelica habet repugnantiam ut praedicetur de pluribus: igitur non est universalis. Respondeo distinguendo maiorem. Non datur universale sine praedicatione de multis vel positive vel per non repugnantiam, concedo; positive praecise, nego. Et distinguo minorem. Natura angelica habet repugnantiam ut praedicetur de pluribus adaequate concepta, concedo; innadaequate concepta, nego. Solutio patet ex dictis. 
\pend

\pstart
 Contra cui praedicatio repugnat praedicabilitas repugnat, sed naturae angelicae ab intrinseco repugnat praedicatio de multis: igitur et praedicabilitas. Respondeo distinguendo maiorem. Cui praedicatio repugnat ab intrinseco et praedicabilitas repugnat, concedo; cui praedicatio repugnat ab extrinseco modo vel impedimento, nego. Et distinguo minorem. Naturae angelicae adaequate sumptae ab \textnormal{|}\ledsidenote{BNC 151ra}   intrinseco repugnat praedicatio de multis, concedo; naturae angelicae inadaequate conceptae, nego. Cum enim adaequate concepta dicat modum sic non potest praedicari, at inadaequate est sine illo modo, est qui potens ad ipsum, sic enim est universalis non positive, sed per non repugnantiam. 
\pend

\pstart
 Sicut enim aliquid in re absolute et simplice non possint poni, cum tamen respectu alicuius principii non repugnat, sub conceptu illius principii potest poni per non repugnantiam. Substantia enim in re non ponitur sine aliquo accidenti et tamen non implicat sine accidentibus intelligi. Similiter natura angelica in re ut individuata non potest multiplicari, at vi naturae ut non induit individuationem non repugnat se habere ad plura. 
\pend

\pstart
 Obiectio 2. Fundamentum unitatis formalis et universalis es convenientia individuorum, sed in natura angelica non datur convenientia individuorum cum sint impossibilia: igitur nec fundamentum: igitur non datur. Respondeo distinguendo maiorem. Fundamentum unitatis formalis et universalis est convenientia individuorum in naturis compositis, concedo; in naturis simplicibus, nego. Contra: igitur talis unitas est sine fundamento. Nego consequentiam. In naturis enim simplicibus unitas formalis sumitur ab unitate naturae conceptae sine individuatione et cum indifferentia, quae \emph{deesse [?]} non repugnat communicabilitas quando concipitur ad instar rei corporalitate 
\pend

\pstart
 Obiectio 3. Individua \textsc{Michaelis}\index[persons]{} sunt impossibilia: igitur si natura constituitur universalis per ordinem ad illa tantum est chimerice. Respondeo distinguendo maiorem. Individua \textsc{Michaelis}\index[persons]{} sunt impossibilia vi conceptus naturae, nego; vi irreceptibilitatis, concedo. Contra: igitur \textnormal{|}\ledsidenote{BNC 151rb} ex vi naturae sunt possibilia. Distinguo consequens. Ex vi naturae sunt possibilia absolute et in re, nego; sunt possibilia per  non repugnantiam, concedo. Ut autem aliquid in re sit possibile requiritur quod attentis omnibus concurrentibus sit possibile. Cumque in re obstet irreceptibilitas, talia individua absolute et in re sunt impossibilia. Ceterum ut non sit impossibile non repugnant sufficit quod vi naturae non sint impossibilia. 
\pend

\pstart
 Obiectio 4. Natura angelica non potest abstrahi a sua individuatione: igitur non est universalis. Antecedens probatur. Non potest abstrahi ab eo quod essentiale est, sed irreceptibilitas in materia essentialiter convenit Angelo: igitur non potest abstrahi ab illo modo individuationis, qui est per irreceptibilitatem in materia. 
\pend

\pstart
 Respondeo negando antecedens. Ad probationem concedo maiorem et distinguo minorem. Irreceptio in materia radicaliter essentialiter convenit Angelo, concedo; irreceptio designata, nego. Natura angelica essentialiter est irreceptibilis in materia et carens illa, unde ab ista carentia non potest praescindi. At ista carentia et irreceptibilitas non est ipsa individuatio Angeli, sed radix illius. Individuatio Angeli non solum dicit principium incomunicabilitatis, sed etiam designationem et determinationem certam individui, quae dicit ordinem ad praedicata extrinseca, scilicet ad subexistentia et accidentia. 
\pend

\pstart
 Personalitas Angeli potest separari a natura, ut si natura angelica assumeretur a Deo. Nam personalitas singularitatis est terminus. Quare singularitas et individuatio Angeli ordinatur ad principia extrinseca, nec pertinet ad constitutionem essentialem Angeli. Ut hoc intelligas, nego \textnormal{|}\ledsidenote{BNC 151va}   toto. In rebus corporeis materia habet duplicem considerationem, scilicet quatenus componit et constituit ipsam naturam et quatenus est radix individuationis, secundum se remote, ut signata quantitate proxime. Quando enim consideratur natura in communi, consideratur ratio materiae ut constituens, non tamen ut principium individuationis. 
\pend

\pstart
 Similiter carentia materiae seu irreceptibilitas in ea dupliciter consideratur in spiritualibus, scilicet quatenus constituit ipsam formam simplicem et spiritualem et quatenus pertinet ad modum individuationis reddendo naturam incommunicabilem et immultiplicabilem. Quatenus enim haec irreceptio ut requiritur ad constitutionem est ipsa ratio simplicis formae et prout sic non fundat repugnantiam multiplicationis. At quando ipsa irreceptio fundat subexistentiam et singularitatem completam et probet incommunicabilitatem, tunc praebet individuationem et repugnat multiplicari. 
\pend

\pstart
 Ceterum quando abstrahitur natura specifica Angeli ab eius individuatione non abstrahitur id quod est essentiale quantum ad constitutionem quidditatis, sed abstrahitur id quod pertinet ad modum individuationis. Unde hic modus originatur in ipsa natura specifica; at formaliter in ea non consistit. Siquidem in designatione a principiis extrinsecis. 
\pend

\pstart
 Obiectio 5. Si natura angelica inadaequate conceptu esset universalis, etiam Petrus inadaequate conceptus esset universalis. Siquidem prout sic solum dicit conceptum naturae et individuationem, sed hoc non est dicendum: igitur nec illud. Respondeo negando paritatem. Ratio est quia Petrus non solum radicaliter, sed etiam formaliter explica individuationem, quare nec inadaequate potest concipi sine differentia numerica, nisi destructo conceptu `Petri' et \textnormal{|}\ledsidenote{BNC 151vb} accepto conceptu naturae. At natura angelica non explicat formaliter ipsam individuationem, etiam si eius radicem includat. 
\pend

\pstart
  Obiectio 6. Etiam natura divina prout a nobis concipitur sine individuatione esset universalis, sed hoc non: igitur licet natura angelica concipiatur sine individuatione, non est universalis. Respondeo negando paritatem. Ratio est quia licet natura divina possit concipi sine singularitate, non tamen ut contrahibilis per differentias, cum sit actus purus et non in potentia: igitur nec ut universalis. 
\pend

\pstart
 Dubitas quod discrimen interveniat inter naturam divinam et angelica? Dico: quod etsi Deus possit concipi sine singularitate et consequenter ut communis ex modo significandi. Tamen natura sic concepta non manet cum indifferentia et potentia ad talem individuationem, nam semper est actus purus. Unde solum uno titulo est communis, scilicet ratione nostri conceptus, non vero ratione indifferentiae et potentialitatis ex parte naturae. 
\pend

\pstart
 Triplici titulo natura composita est universalis, scilicet ratione conceptus abstrahendo a singularibus; ex indifferentia naturae secundum se; et ex parte individuo naturam, ad quae terminatur relatio universalis. Natura enim angelica duplici titulo est universalis, scilicet ex modo concipiendi et ex indifferentia secundum se. At defectu multorum individuorum exerceri non potest haec communicabilitas, quare est universale sine exercitio. 
\pend

\pstart
 Ut autem natura divina concipiatur ut communis sufficit quod concipiatur sine singularitate ex parte modi concipiendi. Ut autem foret universalis deberet concipi ut contrahibilis et determinabilis, ut magis et minus determinata. Totum hoc dicit potentialitatem incompossibilem cum actu purissimo. 
\pend

\pstart
 \textnormal{|}\ledsidenote{BNC 152ra}   Obiectio 7. \emph{Esto [?]} concipiamus Angelos ad instar rei corporem, non tamen eis tribuimus proprietates rei corporem, cum potius eas ab eis removeamus. Sed multiplicatio numerica convenit corporibus et repugnat spiritibus: igitur non possumus tribuere eis individuationem. 
\pend

\pstart
  Responde distinguendo maiorem. Non tamen eis attribuimus proprietates rei corporem per modum atributionis et Iudicii, concedo; per modum simplicis repraesentationis, nego. Sicut enim natura angelica eo quod sit materiae expers sit intelligibilis actu et secundum praedicata specifica et secundum praedicata individualia, quia tamen nos eam cognoscimus per species corporum. Ideo natura angelica non cognoscitur a nobis secundum quod in se est intelligibilis et secundum principia individualia eius, sed secundum nostrum modum intelligendi, quae attingit rationem specificam praescissam ab individuali. Unde diverso modo cognoscimus de facto Angelo et intentiones aliter tribuuntur eis, ac si perciperemus secundum quod in se sunt vel intellectu angelico. Unde ex eo quod abstrahamus naturam ab individuatione sequitur universalitas; nam cognoscitur tunc inadaequate natura. 
\pend

\pstart
 Obiectio 8. Natura angelica non solum concepta instar rei corporeae est in praedicamento, sed etiam ut est in se, sed quatenus eam praedicamento est inadaequate universalis: igitur etiam ut est in se est universalis. Respondeo distinguendo maiorem. Quoad secundam partem, sed etiam ut est in se quantum ad coordinationem graduum superiorum et inferiorum, concedo; quantum ad conditionem ponendi, scilicet quantum ad universalitatem et praedicabilitatem, nego. Cum enim natura angelica sit quod creatum et quidditas composita genere et differentia est posita in praedicamento. At quod \textnormal{|}\ledsidenote{BNC 152rb} sit universalis et praedicabilis provenit a cognitione eam attingente inadaequate et ad instar rei corporeae posito tamen fundamento in ipsa natura. 
\pend

\pstart
 Obiectio 9. Ut natura corporea sit universalis debet abstrahere a principio individuationis, sed ipsa natura angelica est radix et principium individuationis ratione irreceptibilitatis: igitur neque inadaequate potest esse universalis. Respondeo distinguendo maiorem. Ut natura corporea sit universalis debet abstrahere a principio proximo individuationis, scilicet a materia signata, concedo; debet abstrahere a principio remoto, scilicet a materia, nego. Et distinguo maiorem. Ipsa natura angelica est principium individuationem ratione irreceptibilitatis constituentis formam simplicem remote et proxime ratione irreceptibilitatis signatae, concedo; aliter, nego. 
\pend

\pstart
 Igitur in natura angelica carentia materiae et irreceptibilitas in ea ut conducit ad ratione spiritus et formae simplicis non est principium proximum, sed remotum. Ut fundat subsistentiam et ordinem ad principia extrinseca designantia ipsam naturam in singulari, sic est principium proximum et ab ista abstrahit natura angelica ut universalis concepta. 
\pend

\pstart
 Obiectio 10. Quando quod convenit alicui per se eius oppositum nec per accidens potest eidem convenire, sed naturae angelicae ratione sui convenit immultiplicabilitas: igitur multiplicabilitas nec per accidens, scilicet inadaequate potest convenire. Respondeo distinguendo consequens. Multiplicabilitas positiva nec per accidens potest convenire, concedo; multiplicabilitas non repugnanter neque per accidens potest convenire, nego. Aliud est non convenire, aliud non repugnare ex aliquo capite. Multiplicatio enim nullo modo positive et simpliciter convenit naturae angelicae. At non repugnat illi ex vi \textnormal{|}\ledsidenote{BNC 152va}   naturae indifferenter conceptae, quare ut sic potest esse universalis et praedicabilis non positive, sed per non repugnantiam. 
\pend

\pstart
 Obiectio 11. Universali logicum est tale per comparationem ad individua, sed natura angelica nequit comparari ad individua: igitur esto sit universalis metaphysice non tamen logice. Respondeo distinguendo minorem. Natura angelica nequit attingere unum positive et plura non repugnant, nego; nequit attingere plura positive et simpliciter, concedo. Attingere unum actu et plura non repugnant sufficit ut detur universale logicum. 
\pend

\pstart
 Obiectio 12. Eo modo quo \emph{naturae [?]} angelicae repugnat materia, eo modo repugnat multiplicabilitas, sed etiam inadaequate conceptae repugnat materia, cum sit spiritus: igitur. Respondeo distinguendo maiorem. Eo modo quo repugnat materia signata, concedo; materia absoluta, nego. Nam natura humana etiam prout universalis dicit materiam et nihilominus est multiplicabilis: igitur ex opposito in natura angelica. Qua caret materia absolute constituitur spiritus, qua caret materia signata redditur individua. 
\pend

        \addcontentsline{toc}{section}{Articulus 4. An genus possit conservari in unica specie et species in unico individuo?}
        \pstart
        \eledsection*{Articulus 4. An genus possit conservari in unica specie et species in unico individuo?}
        \pend
      
\pstart
 Primo sciendum est discrimen inter naturam genericam et specificam. Natura enim generica cum sit divisibilis per differentias formales in se est quod imperfectum et incompletum et informe intra latitudinem perfectionis essentialis in qua perficitur et completur per tales differentias. At natura specifica cum in se sit quid perfectum, completum et actuale intra latitudinem perfectionis essentialis, in hac nullo modo perficitur per differentias numericas, sed solum \textnormal{|}\ledsidenote{BNC 152vb} dividitur et multiplicatur, quae potius est imperfectio. 
\pend

\pstart
 Sciendum est secundo natura tam genericam quam specificam dupliciter considerari. Primo physice seu materialiter, scilicet secundum realem entitatem quam dicunt praedicata essentialia actu inclusa, in quo sensu certum est genus in una specie et speciem in uno individuo conservari. Siquidem solo Petro existente vere existit entitas realis. 
\pend

\pstart
 Secundo metaphysice et formaliter, scilicet quatenus sunt conceptus quidam praecisi et gradus inter se diversi ex quibus genus est quodam totum potentiale et perfectibilem per differentias vero divisibile; species vero est totum actuale et perfectum ultimoque constitutum in essentia et divisibile solum per differentias numericas. In hoc igitur sensu procedit difficultas an, scilicet solo existente Petro daretur vere species, et solo existente homine daretur genus? His positis. 
\pend

\pstart
 Prima sententia docet posse genus conservari in unica specie et speciem in unico individuo. Ita \textsc{Scotus}\index[persons]{Ioannes Duns Scotus} et alii citati a \textsc{Complutensis}\index[persons]{}. Secunda huic ex regione opposita negat genus in unica specie et speciem in uno individuo. Ita \textsc{Oña}\index[persons]{Petrus de Oña} quae sequuntur aliqui moderni. 
\pend

\pstart
 Sit nostra conclusio. Genus nequit conservari unica in specie, bene tamen species in unico individuo. Sic \edtext{ \name{\textsc{Nostro Reverendissimo}\index[persons]{Aegidius Romanus}}  1 \worktitle{Posteriorum} lectione 12 et in 1, distinctione 19, questione 4, articulo 2 }{\lemma{}\Afootnote[nosep]{}}, quem sequuntur eius discipuli citati a \textsc{Complutensis}\index[persons]{} eam tenet \edtext{ \name{\textsc{Aristoteles}\index[persons]{Aristoteles}} 4 \worktitle{Topicorum} capitulo 3 et 3 \worktitle{Metaphysicae}, texto 10 }{\lemma{}\Afootnote[nosep]{}}. Item \edtext{ \name{\textsc{Ioannes a Sancto Thoma}\index[persons]{Ioannes a Sancto Thoma}} hic }{\lemma{}\Afootnote[nosep]{}}. 
\pend

\pstart
 Probatur prima pars 1. Integra perfectio potentiae sumitur ab eius actu adaequato, sed genus est potentia metaphysica respiciens ut adaequatum actum omnes differentias \textnormal{|}\ledsidenote{BNC 153ra}   contrahentes: igitur non habet talem perfectionem usquedum sit actuata et contracta per tales differentias. Confirmatur. Sic se habet genus respectu differentiarum sicut materia prima respectu formarum in corruptibilibus, sed materia non habet omnem perfectionem omnem sub una forma, aliter non appeteret alias: igitur nec genus habet omnem suam perfectionem sub una differentia. 
\pend

\pstart
 Probatur 2. In rerum natura nullum datur genus quod non habeat pluras species, danturque plures species habentes tantum unum individuum: igitur ideo quia genus petit plures species ut perficiatur, non tamen species plura individua. Probat consequenti. Aliter natura superflueret in multitudine illarum specierum vel deficeret in singularitate individuorum, sed utrumque est alienum ab ingenio naturae: igitur. 
\pend

\pstart
 Probatur secunda pars. Natura specifica intra latitudinem perfectionis essentialis non habet rationem \emph{potentiae [?]}, sed potius est totum actuale perfectum et ultimo constitutum: igitur non respicit individua ut actus essentiales: igitur ut natura specifica intellegatur cum omnimoda perfectione essentialis non requiruntur multa individua. Confirmatur: et si natura specifica comparetur materiae esset rerum incorruptibilium, quae sub sua forma intelligitur esse perfecta, nec habet appetitum ad alias: igitur sicut materia incorruptibilium sub una forma est perfecta, sic natura specifica in unico individuo. 
\pend

\pstart
 Obiectio 1. Potentialitas materiae optime salvatur sub una forma: igitur etiam genus sub unica differentia contractum. Respondeo distinguendo antecedens. Potentialitas physica, concedo; potentialitas metaphysica, nego. Sicut enim tota entitas huius potentialitatis physice et materialiter inveniatur sub qualibet forma, non tamen perfecte et ultimo acuatur metaphysice. Similiter de genere. 
\pend

\pstart
 \textnormal{|}\ledsidenote{BNC 153rb} Obiectio 2. Si genus non saluatur in unica specie: igitur neque in duabus cum semper sit totum potentiale. Respondeo distinguendo consequens. Nequi in duabus quoad sufficientum, nego; quoad omnem suam potentialitatem, concedo. Tot exigit species genus quot differentias divisivas sive hae sint finitae sive infinitae. At ut intelligatur genus ut totum potentiale sufficiunt duae differentiae ad minus, una positiva et alia negativa, ut proponitur in arbore \textsc{Porphyrii}\index[persons]{Porphyrius}. 
\pend

\pstart
 Instabis: bene potest materia satiari una forma incorruptibili: igitur et genus. Respondeo. Potest satiari quoad potentialitatem contrarietatis et extensionis, nego; quoad potentialitatem actuationis et informationis, concedo. Et hoc est loquendo de materia, quae est capax unius formae incorruptibilis. Nam de materia, quae capax plurium formarum corruptibilium dixi numero 727. 
\pend

\pstart
 Obiectio 3. Ex dictis sequeretur Deum nec de potentia absoluta posse conservare genus in una species, sed hoc est inconveniens: igitur. Respondeo. Si loquamur metaphysice certum est \emph{omni [?]} posse. Siquidem sequeretur talem differentia esse actum adaequatum et esse non adaequatum, quod Deus nequit facere 
\pend

\pstart
 Obiectio 4. Post mundi finem manet homo, sed talis potest definiri: igitur constabit genere et differentia. Respondeo distinguendo consequens. Constabit genere et parte, concedo; ut toto potentiali, nego. Tunc enim daretur gradus potentialis, potentialitate partis respectu differentiae; non vero potentialite totius respectu partium subiectarum. 
\pend

\pstart
 Instabis: si unum tantum animal esset posibile, scilicet homo, haec praedicatio esset formalis: `homo est animal'. Sed non est spicei aut differentiae: igitur generis. Respondeo. Talem \textnormal{|}\ledsidenote{BNC 153va}   praedicationem non esse formalem, sed identicam ut constat. 
\pend

\pstart
 Obiectio 5: si una tantum esset species in illa potest accipi gradus potentialis et indetermiatur et gradus actualis et determiatur: igitur et genus. Antecedens probat: talis species esset in praedicamento: igitur habet gradum superiorem et inferiorem: igitur habet determiatur et determiabile cum illa praedicata non sint aequalis actualitatis et determinationis. 
\pend

\pstart
 Respondeo concedendo antecedens et negando consequentiam. Et distinguo consequens probationis. Habet gradum superiorem superioritate ad unum, concedo; superioritate ad multa, nego. Et distinguo ultimam consequentiam. Illa praedicata non sunt aequalia in actualitate, concedo; in extensione, nego. Universalitas enim generica petit extensionem ad multa, qua est totum potentiale. 
\pend

\pstart
 Obiectio 6. Sicut genus est in potentia ad plures formas essentiales, ita species ad plures differentias numericas, sed ideo genus non salvatur in una specie: igitur nec species in uno individuo. Respondeo negando paritatem. Disparitas stat in differentia generis a specie iam assignata. Deinde posita specie in uno individuo est communicabilis saltem non repugnant. At posito genere in una specie non est communicabile aliis speciebus impossibilibus. Ceterum licet omnes differentiae individuales non essent in illo individuo, esset tam tota essentia speciei. 
\pend

\pstart
 Ut hoc melius intelligas scito. Iam dixi genus se habere ut materiam et speciem sicut formam ad materialem divisionem. Materia est communicabilis, communicabilis est etiam forma, sed diverso pacto. \emph{Materia [?]} est communicabilis passive (ut intelligar), hoc est potentia recipere et perfici a pluribus: igitur haec \textnormal{|}\ledsidenote{BNC 153vb} communicabilitas pervenit a pluribus: igitur si haec plura sunt impossibilia, communicabilitas materiae est impossibilis. Similiter dicitur de se habente ut materia: igitur de genere. 
\pend

\pstart
 Forma autem est communicabilis active, hoc est ex ipso conceptu formae quantum est ex se quia est actus et perfectio et bonum ut bonum est communicativum sui; licet ex parte materiae recipientis possit impediri communicatio. Ergo licet illa plura sunt impossibilia species est communicabilis quantum est ex se. Ex hoc enim oritur tota ratio, cur genus cum sit communicabile pluribus his impossibilibus non sit communicabile et species adhuc illis impossibilibus sit communicabilis quantum est ex se. 
\pend

\pstart
 Et nota materiam primam ex se esse principium incommunicabilitatis, limitationis et restrictionis, quia est ultimum subiectum ultra quod non detur communicatio. Quare ex possibilitate formarum circa materiam sumitur potentialitas in materia ad plura. At forma non a ratione recipientium est communicabilis, sed id habet ex ratione formae et actus. 
\pend

\pstart
 Obiectio 7. Si species conservaretur in unico individuo, talis species vel haberet repugnantiam ad plura vel non. Si non, tollitur casus. Si habet: igitur nec ex modo concipiendi est communicabilis. Respondeo. Habere repugnantiam adaequate, non vero inadaequate. Ut manet explicatum at antecedenti. 
\pend

\pstart
 Obiectio 8. In illo casu species et individuum non distinguuntur: igitur non datur in specie communicabilitas. Respondeo distinguendo. Non distinguuntur ratione, nego; realiter, concedo. Et distinguo consequens. Non datur in specie communicabilitas formaliter, concedo; fundamentaliter, nego. \textnormal{|}\ledsidenote{BNC 154ra}   Cum enim species sit communicabilis ex se sine dependentia a communicantibus, adhuc his imposibilibus manet fundamentaliter communicabilis et indifferens. At genus non est communicabile a se, sed a communicantibus, ideo si haec sunt impossibilia, nec fundamentaliter genus est communicabile. 
\pend

\pstart
 Obiectio 9. Etiam species esset totum potentiale in casu quo salvetur: igitur licet genus sit totum potentiale, etc. Respondeo. Esset species totum potentiale quo  ad rem significatam, nego; quoad modum significandi subdistinguo respecti illius tantum individui, nego; respectu plurium non repugnant, concedo. Solutiones patent ex dictis. Nunc vero de Individuo. 
\pend

        \addcontentsline{toc}{section}{Quaestio 10: De Individuo}
        \pstart
        \eledsection*{Quaestio 10: De Individuo}
        \pend
      
\pstart
 Aliud est individuum designatum et determinatum ut \textsc{`Socrates'}\index[persons]{Socrates}; aliud vagum seu in communi ut `aliquos homo'. Cum enim individuum sit correlativum speciei, ideo necessarium fuit de hoc agere illa iam scita. Pro quo Articulo 1. Quid sit individuum, etc. 
\pend

        \addcontentsline{toc}{section}{Articulus 1. Quid sit individuum et quotuplex? Ubi etiam examinantur definitiones traditae ab Porphyrio nostro}
        \pstart
        \eledsection*{Articulus 1. Quid sit individuum et quotuplex? Ubi etiam examinantur definitiones traditae ab Porphyrio nostro}
        \pend
      
\pstart
 Aliter loquendum est de individuo generaliter accepto et aliter de individuo in particulari praesertim de individuo substantiae. In omni individuo duo constituatur, scilicet principium individuationis et ipsa individuatio. At in substantiis etiam est intelligendus terminus seu complementum huius individuationis, quo natura \textnormal{|}\ledsidenote{BNC 154rb} illa redditur completa et subsistens. Nam potest inveniri natura individuata sine tali termino et complemento, ut natura humana assumpta a divino Verbo. 
\pend

\pstart
 Individuum dupliciter sumitur, scilicet vel primo intentionaliter vel secundo intentionaliter. Primo intentionaliter dicit singularitatem prout \emph{a parte rei [?]} invenitur. Secundo intentionaliter constat duplici relatione, scilicet subiicibilitate respectu praedicatorum superiorum sive immediate ut ad speciem athomam sive mediate ut ad genus; et praedicabilitate non respectu inferiorum, sed respecti ui ipsius. 
\pend

\pstart
 His notatis triplex definitio traditur. Prima de individuo primo intentionaliter, quae talis est: \enquote{Individuum est illud cuius collectio proprietatum ita est in uno ut non sit in alio.} Et si quaeras quae sunt istae proprietates? Respondet versus. Forma, Figura, Locus, Tempus cum nomine Sanguis, Patria, sunt septem, quae non habet unus et alter. Cum enim hic numerentur proprietates quae realiter inveniuntur satis constat definitionem esse individui primo intentionalis. Ceterum soli substantiae competit habere collectionem harum proprietatum: igitur solum individuum substantiae definitur. Imo non omni individuo substantiae sed completo et subsistenti quod vocatur suppositum seu persona. Et hoc constat, nam nomen sanguis, patria soli personae competunt, nam manus non dicitur habere patriam vel genus suum ducere a tali sanguine, sed totus homo 
\pend

\pstart
 Impossibile est esse duos homines habentes eamdem dispositionem partium, eadem figuram, natos in eodem \textnormal{|}\ledsidenote{BNC 154va}   loco, eodem momento temporis, vocatos eodem nomine, genitos ab iisdem parentibus et in eadem patria. Et licet hae proprietates disiunctive pluribus possint convenire, non tam collective. 
\pend

\pstart
 Quaeres 1. An haec definitio sit descriptiva vel essentialis? Dico esse descriptivam. Ratio est quia traditur per proprietates: igitur supponit essentiam individui. Nec dicas supponere essentiam speciei, nam respectu huius non sunt proprietates, sed accidentia. Alias omnibus convenirent. Ceterum individuum definitur per has proprietates non ut consecutas et posteriores ipsa individuatione, sed ut signantes materiam, ut ex ipsa materia signata nascatur individuatio, sed de hoc in suo loco. 
\pend

\pstart
 Quaeres 2. Quae sit ista individuo? Dico aliud esse principium individuationis, aliud essentiam individuationem. Principium individuationis in rebus compositis est materia signata quantitate et accidentibus vel materia respiciens illam. In rebus forma vero simplicibus principium individuationis est ipsa forma ut incommunicabilis materiae vel ut irreceptibilis in materia. Ast individuo metaphysice est differentia numeralis; physice est unitas numeralis, qua ens taliter est unum ut non sit amplius divisibile seu communicabile per differentias contrahentes; sed redditur natura indivisa in se et divisa ab alia. 
\pend

\pstart
 Sed notato substantiam per seipsam individuari, hoc est sine ordine ad aliquod extra se. Quare in substantiis aliud est singularitas et individuatio, aliud substantia seu suppositalitas. Sic ergo individua substantiae vocatur supposita personae hypostases, prima substantia, res naturae et subsistentia, quae significant individuum completum. 
\pend

\pstart
 \textnormal{|}\ledsidenote{BNC 154vb} Secunda definitio est de individuo secundo intentionaliter ut suiicibili, scilicet: \enquote{Individuum est illud quod de uno solo praedicatur.} Tertia de eodem ut praedicabili, scilicet: \enquote{Individuum est quod subiicitur speciei infimae.} Cum enim principalis ratio individui sit inferioritas ad universale, ideo praecipua relatio in eo est subiicibilitas. Et nota praedicationem individui, quae est de uno tantum aut de seipso, non esse artificialem, sed identicam. 
\pend

\pstart
 Obiectio 1. Individuum non est definibile, nec subiicibile: igitur male definitur. Respondeo. Definiri ipsam individuationem in communi. Et si instes rationem communem non esse individuum, sed universale satisfaciat tibi articulo sequens. Unde individuatio primo intentionaliter accepta potest concipi in communi analogice et sic definiri. At secundo intentionaliter etiam potest concipi et hoc univoce et per consequens definiri. 
\pend

\pstart
 Obiectio 2. Hic cuius collectio proprietatum vel intelligitur de proprietatibus in communi vel in singulari. Respondeo. Non in communi. Siquidem omne accidens in communi potest multis convenire. Sed contra. 1. Petitur principium quia definitur individuum per proprietatem individuam. 2. Nam tunc frustra dicitur cuius collectio, etc. Siquidem non solum collectio, sed nec una sola proprietas individuata et singularis potest esse in alio. 
\pend

\pstart
 Respondeo. Non peti principium. Ad probationem distinguo. Definitur individuum per proprietatem individuatam, ut individuatam, nego; ut designantem materiam, concedo. Hae proprietates ita designant et determinant hoc individuum, quod in alio reperiri non possunt. \textnormal{|}\ledsidenote{BNC 155ra}   \emph{Caeterunt [?]} licet hoc non possint praestare nisi individuatae, non tamen ut individuatae. Quare individua dependet a materia ut a principio et ab his accidentibus ut a conditionibus. Ad secundam \emph{replicam [?]} dicitur, quod aliquae proprietates conveniunt pluribus, scilicet quae denominant extrinsece ut Patria, nomen, sanguis, locus; aliae vero uni, ut haec forma, haec figura, unde necesse fuit de tota loqui collectione. 
\pend

\pstart
 Obiectio 3. Aliqua individua praedicatur de pluribus, nam eadem materia est in Petro et postea in eius cadavere; eadem actio denominat causam primam et secundam; eadem cognitio plura cognita; idem locus plura locata: igitur falsa es definitio individui praedicabilis. 
\pend

\pstart
 Respondeo negando assumptum. Bene potest unum accidens extrinsece denominans convenire pluribus tamquam inadaequatis et partialibus subiectis, sed hoc non est individuum praedicari de inferioribus vel formam esse in pluribus subiectis individuis adaequatis, quod erat necessarium ut individuum praedicaretur adaequate et simpliciter. 
\pend

\pstart
 Obiectio 4. Eadem numero aqua potest dividi in plures aquas, quarum quaelibet est unum individuum: igitur male dicitur individuum esse indivisum in se, etc. Respondeo distinguendo antecedens. Potest dividi in plures aquas per multiplicationem et praedicationem sui, nego; per corruptionem, concedo. Tunc enim definit aqua. 
\pend

        \addcontentsline{toc}{section}{Articulus 2. Quid sit individuum vagum?}
        \pstart
        \eledsection*{Articulus 2. Quid sit individuum vagum?}
        \pend
      
\pstart
 Suppono cum \edtext{ Divo \name{\textsc{Thoma}\index[persons]{Thomas Aquinas}} 1 parte, questione 30, articulo 4 }{\lemma{}\Afootnote[nosep]{}} individuum vagum significare naturam communem cum determinato modo essendi, de hoc tria quaerentur. Primum, an individuum \textnormal{|}\ledsidenote{BNC 155rb} sic conceptum importe aliquam rationem communem quae sit realis vel rationis? Secundum an haec ratio communis sit univoca vel analoga terium ad quae praedicationem ex quinque praedicabilibus pertineat? 
\pend

\pstart
 Circa primum relictis sententiis. Sit prima  conclusio: individuum vagum dicit rationem communem et realem omnibus individuis. Ita \edtext{ Doctor \name{\textsc{Thomas}\index[persons]{Thomas Aquinas}} loco citato et questione 9 \worktitle{De potentia} articulo 2, ad 1; et quaestione 8 \worktitle{De potentia} articulo 3, ad 11; et in 1 distinctione 25, questione 1. articulo 3 }{\lemma{}\Afootnote[nosep]{}}; \edtext{ \name{\textsc{Ioannes a Sancto Thoma}\index[persons]{Ioannes a Sancto Thoma}} questione 9, articulo 2 }{\lemma{}\Afootnote[nosep]{}}; \edtext{ \name{\textsc{Rubius}\index[persons]{Antonius Ruvius Rodensis}} questione 6 }{\lemma{}\Afootnote[nosep]{}} et alii 
\pend

\pstart
 Probatur ratione. Individuatio est modus quo natura redditur singularis et individua, sed hic modus potest significari in communi: igitur non repugnat ratio communis ipsius individuationis. Probatur minor. Audito nomine individuum vel persona, aliquod concipio tali nomine significatum, sed non concipio naturam: igitur ipsam individuationem. Probatur minor. Si conciperem naturam esset vel naturam in communi et universalem vel in particulari? Non primum, quia significat naturae individuationem. Si secundum, solum significat individuationem, sed non determinatam, nam aliud est dicere aliquos homo, ac Petrus. Siquidem ratio individui convenit pluribus et esse Petrum solum uni: igitur significat individuationem in communi. 
\pend

\pstart
 Confirmatur: quando dicitur aliquos homo, hic homo non est singularis, alias idem esset, aliquos homo ac hic homo: igitur est communis, sed hic aliquos homo est individuum vagum: igitur individuum dicit rationem communem. Ut hoc melius explicetur ceteris assertis indiget. 
\pend

\pstart
 Dico primo. Haec ratio communis non est negativa. Probatur breviter. Cuilibet individuo convenit propria differentia numerica positiva, sed per hanc Petrus non est Paulus: igitur potest abstrahi ab his individuis una \textnormal{|}\ledsidenote{BNC 155va}   ratio communis individui. 
\pend

\pstart
 Circa secundum. Dico primo. Individuatio abstracta a differentiis individualibus est conceptus analogus. Probatur ratione. Ratio transcendens ultimas differentias est analoga, sed talis conceptus transcendit ultimas differentias: igitur est analogus. Probatur minor. Si enim transcenderet darentur aliae differentiae, sed ex hoc sequitur processus in infinitum: igitur transcendit. Praeterea tales differentiae sunt primo diversae: igitur non possunt resolvi in duos conceptus: igitur talis conceptus transcendit: igitur est analogus. 
\pend

\pstart
 Dico secundo. Individuum adaequate conceptum in unoquoque genere est univocum reductive et secundum quod, at simpliciter analogum. Probatur ratione. Conceptus qui non est simpliciter unus non est univocus, sed ille conceptus adaequatus individui in unoquoque genere non est simpliciter unus: igitur non est univocus. Maior est certa. Conceptus qui ex una parte est unus et ex alia non praescindat simpliciter a differentiis, sed illas actu includat manet partim unus et partim diversus quod est analogum. Sicut si animal includeret actu rationale et irrationale, non esset genus homini et equo: igitur conceptus qui, etc. 
\pend

\pstart
 Probatur minor. Illa conceptus confiatur ex natura et modo individuationis in communi accepto, sed ille modus continet actu differentias cum sit analogus: igitur licet ex parte naturae sit unus ex parte tamen modi est diversus: igitur non est simpliciter unus: igitur nec univocus simpliciter. Ceterum si hoc individuum sumatur vi complexi, sic nec univocum nec analogum est, cum dicat non unum conceptum, sed duos, scilicet naturae et individuationis. 
\pend

\pstart
 Dico tertio. Individuum secundo intentionaliter est univocum respectu omnium subiicibilium. Probatur ratione. Potest intellectus ab hac, et illa \del{specie} \textnormal{|}\ledsidenote{BNC 155vb} intentione subiicibili abstrahere unam rationem communem, sed talis esset univoca: igitur tale individuum secundo intentionaliter est univocum. Probatur minor. Omnia subiicibilia conveniunt non solum nomine, sed etiam re significata per nomen subiicibile: igitur talis ratio esset univoca. 
\pend

\pstart
 Solum restat, ut sicut dividitur universale in quod et in quale; et rursus universale in quid, \emph{in ut [?]} parte in quod ut pars materialis, et in quale quid, et in quod ut tota essentia: universale vero in quale, in quale necessarium et in quale contingens. Sic similiter subiicibile verbi gratia `Petrus est animal', `Petrus est homo', `Petrus est rationalis', `Petrus est risibilis', `Petrus est albus'. Similiter secatur individuum ut praedicabile de uno. Verbi gratia `Petrus est hoc animal', `Petrus est hoc homo', `Petrus est hoc rationale', `Petrus est hoc risibile', `Petrus est hoc album'. 
\pend

\pstart
 Circa tertium. Dico primo. Individuum non constituere aliud praedicabile. Probatur. Licet ratio individuis communis et univoca, tamen est pars incompleta; sed nullum incompletum ponitur in linea recta praedicabilium: igitur ratio individui non ponitur directe in linea praedicabilium. Ceterum individuum qua ratione est analogum excluditur a praedicabilibus, sicut et cetera analoga. 
\pend

\pstart
 Dico secundo. Individuum reducitur ad praedicabile speciei. Probatur. Modus rei reducitur ad praedicabile rei, cuius est modus; sed individuatio est modus speciei: igitur reducitur ad praedicabile speciei. Probatur minor. Differentiae individuales sunt modi determinantes \emph{speciei [?]}, sed individuatio abstrahitur ab his differentiis: igitur ipsa est modus speciei. 
\pend

\pstart
 Obiectio 1: Communicabilitas repugnat individuo: igitur non datur individuum vagum. Respondeo distinguendo antecedens. Repugnat individuo determinato, concedo; \textnormal{|}\ledsidenote{BNC 156ra}   individuo vago et in communi, nego. Individuum determinatum et signatum ut `Petrus', `hic homo', posse communicari est impossibile, at individuum vagum ut aliquos homo potest communicari. 
\pend

\pstart
 Respondeo secundo. Individuum est quod de uno tantum praedicatur ut quo, concedo; ut quod nego. Individuum est forma praedicabilitatis, at concepta ut res quaedam et in actu signato potest denominative et ut quod praedicari de pluribus. Quare licet talis forma formaliter et exercite non valeat de pluribus praedicari, quia non potest esse forma et ratio praedicandi de pluribus, tamen concepta materialiter et signate est comunicabilis denominative. 
\pend

\pstart
 Obiectio 2. Ex \textsc{Philosopho}\index[persons]{Aristoteles} prima substantia de nullo praedicatur, sed prima substantia est singulare: igitur singulare non potest praedicari. Respondeo distinguendo maiorem. Prima substantia designata et determinata de nullo praedicatur, concedo; prima substantia vaga et in communi, nego. Vel alio modo ut quo, concedo; ut quod, nego. 
\pend

\pstart
 Obiectio 3. Differentiae ultimae sunt primo diversae, sed quae sic se habent in nullo conveniunt: igitur. Respondeo distinguendo minorem. In nullo conveniunt univoce, concedo; analogice, nego. Contra. Si conveniunt debent habere rationem differendi: igitur. Respondeo. In eadem ratione, concedo; in diversa, nego. Nam haec est natura analogorum. 
\pend

\pstart
 Obiectio 4. Omnes differentiae individuales possunt concipi secundum quod conveniunt in officio singularizandi: igitur prout sic potest dari ratio univoca ad illa. Respondeo distinguendo antecedens. Analogice concedo; univoce, nego. Igitur modus singularizandi non potest praescindi actu a differentiis determinatis et individualibus, quia cum \emph{significant [?]} primo diversae, non sunt resolubilis in duos conceptus, \textnormal{|}\ledsidenote{BNC 156rb} quorum aliter sit contrahens et aliter contrahibilis. 
\pend

\pstart
 Obiectio 5. Esto individuatio sit communis, non tamen ut quid reale, sed rationis. Vel aliqua negatio ut incommunicabilitas vel negatio collectionis proprietatum vel aliqua secunda intentio \emph{subiicibilis et praedicabilis [?]}. Respondeo negando assumptum. Persona enim et similiter singulare non est nomen negationis vel intentionis, sed rei. Habet enim positivum effectum, scilicet reddere naturam indivisam in se et ab aliis divisam. Similiter relinquit naturam capacem positive effectus, scilicet existentiae. 
\pend

\pstart
 Quaeres an \emph{a parte rei [?]} detur individuum vagum et an Deus illud valeat producere? Respondeo negative. Cuius ratio est, omne productum a Deo est determinatum ad certum pondus, numerum et mensuram; \emph{equeaeque [?]} singulare determinate cum terminet actionem productivam, sed individuum vagum non sic se habet: igitur non datur \emph{a parte rei. [?]} 
\pend

\pstart
 In fine nota, quod sicut universale est nomen concretum dicens de formali universalitatem et de materiali naturam; sic individuum est nomen concretum dicens de materiali naturam et de formali ipsam individuationem. Similiter sicut definitum in definitionem generis dicimus esse secundam intentionem in concreto concernentem naturam; sic definitum formale in definitione individui dicimus esse ipsam intentionem in concreto concernentem naturam. Caeterum de aliis definitionibus dicendum est sicut de accidentibus realibus. 
\pend

\pstart
 Cum enim tota haec difficultas involvat analogiam, usque hanc explicemus non possemus illam radicitus compraehendere. Et licet \textsc{Porphyrius}\index[persons]{Porphyrius} hic tractaverit de arbore, nos tamen omissam facimus ut tractandam et plantandam in praedicamento substantiae rigente nostro \textsc{Aristotele}\index[persons]{Aristoteles}. 
\pend

        \addcontentsline{toc}{chapter}{Capitulum tertium}
        \pstart
        \eledchapter*{\supplied{Capitulum tertium}}
        \pend
      
        \addcontentsline{toc}{section}{Textus capituli tertii Porphyrii}
        \pstart
        \eledsection*{Textus capituli tertii Porphyrii}
        \pend
      
\pstart
 \textnormal{|}\ledsidenote{BNC 156vb} \edtext{\enquote{ ΚΕΦΑΛΑΙΟΝ Β Διαφορὰ δὲ κοινῶς τε καὶ ἰδίως καὶ ἰδιαίτατα λεγέσθω. Κοινῶς μὲν γὰρ διαφέρειν ἕτερον ἑτέρου λέγεται τὸ ἑτερότητι διαλλάττον ὁπωσοῦν ἤ πρὸς ἄλλο, καὶ τὰ ἕτερα. }}{\lemma{}\Afootnote[nosep]{ \textsc{Commentarii Collegii Conimbricensis e Societate Iesu}\index[persons]{}, \worktitle{In universam dialecticam Aristotelis Stagirita} (Lugduni: sumpt. Iacobi Cardon et Petri Cavellat, 1622), p. 121. }} 
\pend

\pstart
 \textnormal{|}\ledsidenote{BNC 156vc} \edtext{\enquote{ Caput tertium Differentia vero et communiter et proprie et proprissime dicatur. Differre namque qui propriam communiter dicitur cum diversitate aliqua quouis modo ut a seipso aut ab alio differt, etc. }}{\lemma{}\Afootnote[nosep]{ \textsc{Commentarii Collegii Conimbricensis e Societate Iesu}\index[persons]{}, \worktitle{In universam dialecticam Aristotelis Stagirita} (Lugduni: sumpt. Iacobi Cardon et Petri Cavellat, 1622), p. 121. }} 
\pend

        \addcontentsline{toc}{section}{Summa Textus}
        \pstart
        \eledsection*{Summa Textus}
        \pend
      
\pstart
\noindent%
 \textnormal{|}\ledsidenote{BNC 156va} Duas continet partes hoc caput. In prima quinque traduntur divisiones differentiae, in secunda totidem dantur definitiones. Quoad primum, primo dividitur differentia in commune, propriam et proprissimam, verbi gratia, sedere risibile, rationale. Secundo, in differentiam accidentalem ut communis et propria; et in essentialem et proprissima. Aliter in facientum diversum et varium, et in facientem omnino aliud. Tertio, in differentiam separabilem ut moveri, quiescere; et in inseparabilem ut simum, aquilinum, rationale. Quarto, differentia inseparabilis dividitur in inseparabilem per se ut rationale et in inseparabilem per accidens ut aquilinum. Haec potest suscipere magis minusve, secus illa. Quinto, differentia per se dividitur in divisivam generis ut rationale et in constitutivam et sensibili respectu animalis. 
\pend

\pstart
 Quoad secundum, prima definitio est: \enquote{Differentia est id, quod exedit genus.} Secunda: \enquote{Differentia est quae praedicatur de pluribus differentibus specie in quale quid.} Tertia: \enquote{Differentia est id, quod ea, quae sunt sub eodem genere separat.} Quarta: \enquote{Differentia est id quod singula differunt.} Quinta: \enquote{Differentia est qua ita singula differunt, ut contineat ad substantiam et rationem, sit qui pars rei, cuius est differentia.} Contra primam obiecti vel genus continet differentias vel non? Si continet: igitur species per differentiam non excedit genus. Si non: igitur species ex nihilo fit. Respondeo. Genus continere differentias in potentia, speciem vero in actu: ergo hoc pacto species excedit genus: igitur species non fit ex nihilo. Sic respondet sibi ipsi \textsc{Porphyrius}\index[persons]{Porphyrius}. 
\pend

        \addcontentsline{toc}{section}{Annotationes circa Litteram Capitis}
        \pstart
        \eledsection*{Annotationes circa Litteram Capitis}
        \pend
      
\pstart
\noindent%
 \textnormal{|}\ledsidenote{BNC 156vb} Primo igitur notandum est differentiam significare \textnormal{|}\ledsidenote{BNC 156vc} distantiam, distinctionem et diversitatem inter aliqua. \textnormal{|}\ledsidenote{BNC 157ra}   Unde dupliciter sumitur, scilicet late pro quacunque diversitate et stricte. Pro cuius intelligentia notandum est, aliud est differre, aliud diversum esse, aliud igitur est differentia et aliud diversitas. Quae autem se ipsis differunt et no per aliud superadditum, ut sunt primo diversa non differunt, sed sunt diversa. Quae autem in aliquo conveniunt et in alio different dicuntur proprie differre, nam per superadditum differunt. Igitur omnis differentia supponit convenientiam et quae non habent ullam sut diversa ut praedicamenta et ipsae differentiae, quae sunt rationes differendi. 
\pend

\pstart
 Notato 2. Circa primam divisionem, quod differentia communis non dicitur talis quia invenitur in utroque extremo; nam sessio non distinguit sedentem a sedente, sed autem non sedente. Dicitur enim communis quia neutri extremorum est accidens proprium, et converso. Differentia propria aut proprissima non dicitur talis praecise quia non invenitur in altero extremo, sed quia aut est passio aut constitutivum. 
\pend

\pstart
 Notato 3. Circa secundam. Quod haec divisio non est distincta a praecedenti. Siquidem differentia communis et propria coincidunt cum differentia per accidens seu quae facit diversum et varium. Differentia vero proprissima coincidit cum differentia per se seu quae facit aliud. Unde cum Divo \textsc{Thoma}\index[persons]{Thomas Aquinas} est sciendum relativum aliud neutris generis dupliciter sumi. Primo ut significat diversitatem formalem in essentia, secundo ut significat in essentia diversitatem materialem. Unde homo est aliud ab equo et Petrus aliud a panne. Cum enim in essentia divina nulla diversitas inveniatur nullo modo Filius est aliud a Patre, nisi adiective aliud suppositum vel alius. 
\pend

\pstart
 \textnormal{|}\ledsidenote{BNC 157rb} Sicut hoc ita sit tamen hic \emph{hic [?]} aliud sumitur primo modo, scilicet per diversitate formali, quam facit differentia essentialis. Denique differentia faciens aliud dicitur specifica; at faciens diversum et varium simppliciter dicitur differentia, non ratione praestationis dignitatis, sed potius indignitatis, admodum quo bruta appellare solemus sine addito animalia. Tertia et quarta divisio non distinguuntur ab praedictis. 
\pend

\pstart
 Notato 4. Circa ultimam. Quod membra dividentia possunt dupliciter intelligi. Primo ita ut eiusdem generis sed differentia constitutiva et divisiva. Nam animal constituitur sensibilitate et dividitur et dividitur rationalitate. Secundo iuxta diversas rationes eiusdem differentiae. Nam rationale secundum unam rationale unam rationem est constitutivum hominis et secundum aliam est divisivum animalis. Quaestio 9. 
\pend

        \addcontentsline{toc}{section}{Quaestio 11: De Differentia}
        \pstart
        \eledsection*{Quaestio 11: De Differentia}
        \pend
      
\pstart
 Superst examinare de natura constitutiva seu perfectiva, quod intendimus in hac quaestione: et similiter cognoscere respecti cuius differentia sit universalis nominanda, sed prius est agendum de prolixa differentiae divisione. Une Articulo 1, etc. 
\pend

        \addcontentsline{toc}{section}{Articulus 1. An bene dividatur differentia in communem, propriam et proprissimam?}
        \pstart
        \eledsection*{Articulus 1. An bene dividatur differentia in communem, propriam et proprissimam?}
        \pend
      
\pstart
 Notato 1. Hanc divisionem esse adaequatam, cuius ratio est: ista divisio fit per membra exhaurientia et adaequantia totum divisum: igitur est adaequata. Probatur antecedens. Omnis forma faciens differre vel est essentialis, et intrinseca vel accidentalis? Si essentialis: igitur differentia est \textnormal{|}\ledsidenote{BNC 157va}   proprissima. Si accidentalis vel illud accidens est separabile vel inseparabile? Si inseparabile est proprium; si separabile, est commune etiam aliis: igitur. 
\pend

\pstart
 Contra hoc obiectio 1. Propria passio facit differre, sed non essentialiter, cum non sit constitutiva, nec accidentaliter, nam quae habent diversas passiones differunt essentialiter. Respondeo distinguendo maiorem. Propria passio facit differre accidentaliter proprie formaliter et virtualiter seu illative essentialiter, concedo; secus, nego. Passio enim sequitur ad differentiam proprissimam. Unde licet ipsa sit differentia accidentalis proprie, tamen infert illam. 
\pend

\pstart
 Obiectio 2. Differentiae individuales in substantia faciunt differre, sed non essentialiter alioquin differrent specie, nec accidentaliter quia substantitaliter differunt: igitur. Respondeo differentiam hic divisam esse capacem intentionis universalis, cumque individuatio sit huius incapax, ideo non continetur sub hac divisio est. Ex ea parte autem qua facit substantialiter differre reducitur ad differentiam proprissimam. Et ex ea qua provenit a quantitate reducitur ad communem. 
\pend

\pstart
 Instabis: \textsc{Porphyrius}\index[persons]{Porphyrius} tradens exemplum pro differentia communi ponit differentiam inter \textsc{Socratem}\index[persons]{Socrates} et \textsc{Platonem}\index[persons]{Plato}, sed talis est individualis: igitur individualis est communis. Respondeo. \emph{Secundo [?]} individuatione in substantia facere differre proprie, hoc est inseparabiliter. Ad minorem argumenti distinguo secundam partem. Nec accidentaliter accidentalitate praedicamentali, concedo; accidentalitate praedicabili, nego. Bene stat, quod differunt substantialiter et non essentialiter, sed accidentaliter dicto modo. Quare individuatio non dividit formaliter substantiam, sed tantum materialiter. 
\pend

\pstart
 Notato 2. Quod differentia in communi nihil \textnormal{|}\ledsidenote{BNC 157vb} est aliud quam forma faciens differre. Unde duplex oritur ordo in differentia. Primo, ad subiectum seu rem quam facit differre. Secundo, ad extremum seu terminum a quo facit differre. Cum enim sit segregatio dicit subiectum quod segregat et terminum a quos segregat. Primus ordo constituit differentiam praedicabilem. Secundus constituit differentiam metaphysicam.. 
\pend

\pstart
 Suposito divisionem esse adaequatam quaeruntur duo, scilicet quodnam sit divisum et quae membra dividentia? Et an haec divisio sit univoca vel analoga? Circa primum divisum et membra dividentia quattor modis possunt comparari. 
\pend

\pstart
 Primo, si sumantur materialiter quo pacto est quo id quod denominatur a formalitate differentiae in communi, dividatur in commune propriam et proprissimam? Secundo, si divisum sumatur materialiter et membra formaliter. Tertio si divisum sumatur formaliter et dividentia materialiter. Quarto si tam divisum quam dividentia sumantur formaliter. 
\pend

\pstart
 Dico 1. Si divisum et dividentia sumuntur primo intentionaliter per naturis denominatis, haec divisio est analoga. Probatur. Non potest abstrahi ratio univoca ab entitatibus denominatis intentionibus differentiae communis propriem et proprissimem. Verbi gratia ab albedine, nasi curvitate et rationalitate: igitur talis divisio non est univoca. Antecedens patet. A rebus divagantibus per omnia praedicamenta non potest abstrahi ratio univoca; tales sunt differentiae \emph{primo [?]} intentionaliter sumptae: igitur. Eadem ratione diximus universale materialiter non esse univocum quinque praedicabilibus. 
\pend

\pstart
 Dico 2. Si divisum sumatur \textnormal{|}\ledsidenote{BNC 158ra}   materialiter et dividentia formaliter est divisio subiecti in accidentia. Constat, siquidem tunc subiectum dividitur in diversas intentiones formalitates-ve. Dico 3. Si divisum sumatur formaliter et dividentia materialiter divisio est accidentis in substantia. Siquidem tunc illa formalitas differentiae dividitur in diversa substantia. Quare secunda divisio similis est illi, qua \del{\emph{ruborum [?]}} homo dividitur in album et nigrum. Tertio illi qua album in lac et saccharum. 
\pend

\pstart
 Dico 4. Si divisum et dividentia sumantur formaliter divisio est analogi in analogata. Est communis inter Thomistas. Probatur ratione. Sola differentia proprissima participat simpliciter differentialitate et ipsa sola constituit tertium praedicabile: igitur differentiae communis et propria solum analogice conveniunt cum illa. Probatur antecedens. Formalissima ratio differentiaee constituentis tertium praedicabile est universalitas naturae constitutivae et perfectibilis, sed universalitas naturae constitutive seu perfectibilis non potest simpliciter convenire differentiis communi et propriae: igitur solum convenit simpliciter differentiae proprissimae. Probatur minor. Actus seu constitutivum debet esse in eodem genere cum potentia, cuius est actus seu constitutivum, sed differentiae communis et propria non sunt in eodem genere cum genere: igitur universalitas naturae, etc. Maior patet. Alias ex utraque non fieret unum per se. Probatur minor. Differentiae communis et propria sunt extra genus faciuntque cum illo unum per accidens: igitur non sunt in eodem genere. 
\pend

\pstart
 Confirmatur: illa divisio est analoga, quae dicit divisum non simpliciter idem in membris dividentibus, sed solum secundum quod; sed si habet ratio commmunis, quae dividitur hac divisione: igitur est analoga. \textnormal{|}\ledsidenote{BNC 158rb} Minor, in qua est difficultas manet probata praefata ratione: igitur est divisio analogi in analogata. 
\pend

\pstart
 Ex dictis sequitur 1. Divisum in hac divisione non esse differentiam constituentem tertium praedicabile. Siquidem haec potius est membrum dividens. Sequitur 2. Divisum nec esse secundam intentionem ex quinque praedicabilibus ut videnti constabit? Quod igitur est? 
\pend

\pstart
 Dico 5. In formis realibus differentia metaphysica seu formalitas differendi est realis. Notant dixi formalitas, non intentio ut communiter Auctores loquuntur ne involvamus intentiones logicas cum formalitatibus metaphysicis. Haec formalitas non est ipsa forma, sed munus faciendi differre, quod consistit in respectu ad terminum a quo separat, vide numero 891. 
\pend

\pstart
 Antiquam probetur hoc assertum notato et considerato quattuor in ipsa differentia. Primum, forma faciens differre. Secundum, segregatio seu divisio ex ipsa forma resultans. Tertium relatio differentiae seu dissimilitudinis. Quantum, ipsa ratio tertii praedicabilis. Nego loquimur in hoc asserto de ipsa forma, nec de ratione tertii praedicabilis; solum loquimur de formalitate seu munere differendi. 
\pend

\pstart
 Probatur. Effectus illius formalitatis est realis: igitur ipsa realis est. Antecedens probatur. \emph{A parte rei [?]} datur distincto et segregatio \emph{\del{realia} [?]} inter differentia, ceu inter Petrum et aequum alia ratione convenientes, sed talis convenientia et similitudo realis est: igitur et distincto realis est, nam realiter non sunt idem. 
\pend

\pstart
 Contra hoc ultimum assertum. Obiectio 1. Differentia supponit convenientiam, sed convenientia est rationis: igitur et formalitas differendi. Respondeo negando minorem. Convenientia enim et similitudo inter Petrum et Paulum, \textnormal{|}\ledsidenote{BNC 158va}   et inter Petrum et equum est realis. Unde aequivoco laborant Auctores contrarii non discernentes convenientiam ab unitatem. Unitas enim communis non datur \emph{a parte rei [?]}, ast datur convenientia et similitudo. 
\pend

\pstart
 Obiectio 2. Differentia supponit differentiam genericam convenientiam: igitur non realem, sed rationis. Respondeo distinguendo antecedens. Genericam convenientiam fundamentalem, concedo; formalem, nego. Sufficit quod supponat convenientiam, qua fundamentum, non vero qua relationem. 
\pend

\pstart
 Contra tres primas assertiones non invenio difficultatem resolvendam. Quare contra 4. Obiectio 1. Formalitas differendi univoce et simpliciter exercetur a quacunque differentia: igitur est divisio univoca. Respondeo negando antecedens. Differentia proprissima tollit identitatem, quod est oppositum differentiae simpliciter: accidentalis autem secundum quo. Vide numero 896. 
\pend

\pstart
 Obiectio 2. Si haec divisio esset analoga maxime, quia differentia proprissima facit differre essentialiter et aliae accidentaliter, quando hoc non obstat: igitur. Probatur minor. Hoc non obstat ut praedicabilia conveniant univoce, cum tria conveniant essentialiter et alia accidentaliter: igitur non obstat. 
\pend

\pstart
 Respondeo negando minorem. Ad cuius probationem nego consequentiam et paritatem. Disparitas est quia praedicabilitas non consequitur rem ut est in se, sed ut intellectam. Intellectus enim absolute et univoce applicat praediatum ad subiectum sive sit essentiale sive accidentale in qua applicatione stat praedicatio. Similiter hoc modo abstrahit unum a multis sive essentialiter convenientibus sive accidentaliter, in quo universalitas fundatur. At differre non est effectus solius intellectus, cum in re unum ab alio differat; forma a quae essentialiter facit \textnormal{|}\ledsidenote{BNC 158vb} diferre tollit simpliciter identitatem, quae est in essentia: accidentalis non tollit, ut dictum est. 
\pend

\pstart
 Obiectio 3. Convenientia et similitudo essentialis et accidentalis univoce conveniunt in ratione \secluded{in ratione} assimilandi: igitur similiter differentia essentialis et accidentalis univoce conveniunt. Consequentia constat. Respondeo distinguendo consequens. Quantum ad relationem, concedo; Quantum ad formalitatem, nego. Non loquimur hic de relatione praedicamentali. Nam certum est relationem differentiae univoce convenire in praedicamento relationis sive sit relatio differentiae essentialis sive accidentalis. 
\pend

\pstart
 Obiectio 4. Quamvis accidens dependeat a substantia hoc non tollit quominus aeque denominentur ab intentionibus sit generis, etc. Igitur quamvis differentia communis et propria dependeant a proprissima, hoc non tollit quominus aeque conveniant univoce in aliqua secunda intentione. 
\pend

\pstart
 Respondeo negando consequentiam et paritatem. Disparitas est, quia licet accidens dependeat a substantia habet tam in suo genere quidquid requiritur ad fundas intentiones generis, speciei et differentiae. Siquidem habet gradus habentes rationem actus, et potentiae, et compositi intra idem genus ac proinde reperitur perfectibile, perfectivum et perfectum: igitur genus, species et differentia. At differentia communis et propria etiam respectu cuius sunt differentiae sunt actus extranei et alterius generis facientesque cum eo unum per accidens: igitur non habent quidquid requiritur ad constituendam intentionem constitutivam huius praedicabilis. 
\pend

\pstart
 Obiectio 5. Quando intellectus comparat risibilitatem \emph{Primo [?]}, ut quo distinguitur ab aequo, et quando illi comparat albedinem, ut id quo differet ab aethiope, \textnormal{|}\ledsidenote{BNC 159ra}   illi tribuit intentiones universalis, sed tales intentiones non pertinet ad primum et secundum praedicabile, nec etiam ad quartum et quintum: igitur ad tertium. Probatur minor. Per talem actum non cognoscitur risibilitas ut perfluens ab essentiam, nec albedo ut contingens: sed prout sic pertinet ad quartum et quintum praedicabile: igitur. Respondeo distinguendo minorem. Nec  ad quartum et quintum directe, concedo; reductive, nego. Tunc sumitur incomplete illa ratio. Articulo 2. etc. 
\pend

        \addcontentsline{toc}{section}{Articulus 2. Respectu cuius sit differentia universalis, et praedicationis?}
        \pstart
        \eledsection*{Articulus 2. Respectu cuius sit differentia universalis, et praedicationis?}
        \pend
      
\pstart
 Notato 1. Differentias communem et propriam non constituere praedicabilia sub conceptu differentiae, cum non \emph{precentur [?]} in quale essentialiter, sed solum sub conceptibus quarto et quinto praedicabilis. Unde relictis solum restat agere de differentia proprissima. Haec enim sic definitur: \enquote{Differentia est id quod de pluribus et differentibus specie hoc ipso quale quid praedicatur.} Est descriptiva ut diximus de definitione generis. Ceterum alia definitio, scilicet quarta convenit essentialiter differentiae, imo non est alia ab ea, quam dedimus ut essentialem. Ceterum definitiones conpetunt differentiae metaphysice acceptae. 
\pend

\pstart
 Notato 2. In nostra differentia tres esse respectus, scilicet respectus ad genus, quod differentia contrahit et supra quod aliquid addit. Secundus ad speciem, quam constituit contrahendo genus et separando speciem ab aliis. Tertius ad inferiora de quibus praedicatur in quale quid. Iuxta hos respectus insurgunt tres difficultates. Prima, an sit universalis respectu speciei vel respectu inferiorum? Secunda, quomodo in definitione tertii praedicabilis \textnormal{|}\ledsidenote{BNC 159rb} dicatur praedicari de specie differentibus, quod videtur solum convenire differentiae subalternae? Tertia an differentia contineantur sub genere, ita quod differentiae inferiores contineantur in superioribus? Has ultimas resolvam in sequentibus articulis. Prima hic. 
\pend

\pstart
 Notato 3. Differentiam habere inferiora immediata et mediata: verbi gratia reale potest praedicari de Petro et Paulo ut de inferioribus hominis, et potest praedicari de hoc et illo rationali ut de inferioribus immediatis. Inquirimus igitur respectu quorum constituatur universalis? Similiter cum differentia convertatur cum specie et species sit universalis, dubitatur an ratione speciei, quam constituit sit universalis. Loquimur tam de specie athoma, quam de subalterna proportionabiliter. 
\pend

\pstart
 Prima sententia asserit, differentiam constitui per ordinem ad propria inferiora, ut rationale non constituitur in esse universalis per ordinem ad hominem, neque per ordinem ad Petrum et Paulum, sed ordinem ad hoc et illud rationale. Haec opinio non cognoscit Patronos, refertur a \edtext{ \name{\textsc{Sanchez}\index[persons]{Ioannes Sanchez Sedeno}} libro 3, questione 28 }{\lemma{}\Afootnote[nosep]{}}. Secunda tenet constitui per ordinem ad speciem, ita ut haec praedicatio `homo est rationalis' sit tertio praedicabilis. Ita \textsc{Caietanus}\index[persons]{Thomas de Vio Caietanu}, \textsc{Soto}\index[persons]{Dominicus de Soto}, \textsc{Toletus}\index[persons]{Franciscus Toletus} hic. 
\pend

\pstart
 Dico 1. Inferiora propria tertium praedicabilis non sunt haec et illa differentia, sed individua speciei. Ita communiter Dialectici. Probatur efficaciter. Illa sunt inferiora differentiae qua tertii praedicabilis, de quibus praedicatur in quale quid, sed de hac et illa differentia non praedicatur in quale quid: igitur haec et illa differentia non sunt inferiora propria differentiae quo talis. Probatur minor. Praedicari ut tota essentia, non est praedicari in quale quid, sed hac et illa differentia praedicatur ut \textnormal{|}\ledsidenote{BNC 159va}   \secluded{ut} tota essentia: igitur non praedicatur in quale quid. Probatur minor. Nam quando dico `rationale est rationale' non est praedicatio in quale, cum non precetur aliqua differentia, sed tota et integra essentia. 
\pend

\pstart
 Dico 2. \del{Species} Differentia constituitur universalis per ordinem ad individua, non per ordinem ad speciem. Probatur ratione. Universale constituitur per ordinem ad inferiora, sed species non est inferior respectu differentiae: igitur differentia non constituitur universalis per ordinem ad speciem. Probat minor. Nam species est aequalis differentiae: igitur non \emph{inferiore [?]} . 
\pend

\pstart
 Respondent contrarii 1: differentiam esse et aequalem et superiorem. Primum physice et secundum logice. Probant assumptum. Species praedicatione formali constituitur universalis, sed differentia praedicatur formaliter tum de individuis, tum de specie: igitur. Sed contra eo dictis de medio tollitur praedicatio aequalis de aequali ab omnibus concessa aptiorque ad demonstrationem perfectissimam: igitur ruit responsio. 
\pend

\pstart
 Rursus universale tantum est superius respectu eorum a quibus abstrahitur abstractione universali logice loquendo, sed differentia non abstrahitur ab specie abstractione universali, sed formali: igitur differentia non est superior respectu speciei. Probatur minor. Ab illis aliquid abstrahitur abstractione universali, quae immediate unit in una ratione, in quibus qui multiplicatur per contractionem, sed differentia non unit in una ratione speciem cum individuis, nec per contractionem multiplicatur in specie et individuis, sed in his tantum: igitur. 
\pend

\pstart
 Respondent 2: differentiam praedicatam de apecie virtute praedicari de pluribus contentis in specie: igitur praedicari de specie\emph{speciebus [?]} aequivalet praedicari de pluribus. Sed contra prima continentia virtualis non sufficit ad veram universalis praedicationem: igitur ruit doctrina. \textnormal{|}\ledsidenote{BNC 159vb} Probatur antecedens. Non minus universale est simpliciter unum quam multa sunt simpliciter ad quae ordinatur, sed individua contenta in specie non sunt simpliciter multa: igitur differentia non est universalis respectu speciei eo quod contineat individua. Probatur minor. Individua ut contenta in specie sunt unum teste \textsc{Philosopho}\index[persons]{Aristoteles} \textsc{Porphyrium}\index[persons]{Porphyrius}: \enquote{participatione speciei plures homines sunt unus homo}, igitur. 
\pend

\pstart
 Contra secunda. Quando differentia praedicatur de specie, species tunc formaliter nos est universalis, sed subiicibilis: igitur non continet prout sic inferiora. Patet consequentia. Praedicatum qua tale debet potentia continere subiectum: igitur subiectum qua tale similiter debet continere praedicatum actu: igitur continere inferiora potentia non convenit subiecto qua tali, sed quando differentia praedicatur de specie, species est subiectum: igitur qua tale non explicat continentiam praefatam. 
\pend

\pstart
 Obiectio 1. Quodlibet universale habet suum correlativum, sed correlativum differentiae, verbi gratia rationalis non est homo, nec Petrus cum \emph{sint [?]} correlativa generis et speciei: igitur differentiae inferiora non sunt individua. Respondeo distignuendo minorem. Nec Petrus ut subicitur integrae essentiae, concedo; ut subiicitur constitutivo qualificativo, nego. Secundum diversas subiicibilitas individuum constituitur correlativum praedicabilium. 
\pend

\pstart
 Obiectio 2. Differentia formaliter quatenus talis est universalis, sed est talis in ordine ad speciem: igitur est universalis in ordine ad speciem. Probatur minor. Prout ordinatur ad speciem dicit constitutivum esse illius: igitur. Respondeo distinguendo maiorem. Differentia quatenus talis est universalis reduplicando supra entitatem seu formalitatem differendi, nego; reduplicando supra intentionem seu relationem ad inferiora, concedo. Et distinguo minorem. \textnormal{|}\ledsidenote{BNC 160ra}   Est talis in ordine ad speciem, reduplicando supra tendentiam, nego; supra formalitatem, concedo. Ceterum, si \emph{termino [?]} talis reduplicet supra differentiam differentiae nego maiorem. Nam ut postea videbimus differentia non includit genus: homo enim qua rationalis non est animal ut patet: igitur similiter. 
\pend

\pstart
 Obiectio 3. Haec praedicatio `homo est rationalis' est tertii praedicabilis, sed ibi praedicatur differentia de specie: igitur. Confirmatur: universale constituitur in esse universalis per ordinem ad immediatum correlativum, sed species est immediatum et individuum mediatum: igitur. 
\pend

\pstart
 Respondeo distinguendo maiorem. Est tertii praedicabilis materialiter et metaphysice, concedo; formaliter et logice, nego. Praedicatur enim res quae est tertiam praedicabile, non tamen sub formalitate praedicabilis et universalis. Quaeris quomodo appellanda sit haec praedicatio? Appellatur tertii praedicati, idem dices, quando praedicatur propia passio de specie, scilicet esse quarti praedicati. 
\pend

\pstart
 Ad confirmationem distinguo maiorem. Per ordinem ad immediatum correlativum constitutionis, nego; praedicationis, concedo. Et minorem. Species est immediatum correlativum constitutionis, concedo; praedicationis, nego et vice versa ad secundam partem. Igitur species est immediatam constitutam per differentiam individuam vero est immediatam correlativam praedictionis. Praeterea species respicit differentiam metaphysice individuum logice. Unde differentia respicit speciem in ratione constitutivi et individua in ratione praedicabilis. 
\pend

\pstart
 Obiectio 4. Differentia eadem relatione tagit speciem et individua, sed prout tangit individua est universalis: igitur ut tangit speciem. Probatur maior. Respiciendo speciem respicit omnia \textnormal{|}\ledsidenote{BNC 160rb} contenta in specie: igitur. Confirmatur: differentia praedicatur de specie ut aequalis et ad convertentiam, sed solum ut universalis est aequalis: igitur praedicatur ut universalis. 
\pend

\pstart
 Respondeo distinguendo maiorem. Eadem relatione constitutivi, concedo; eadem relatione praedicabilitatis, nego. Et minorem. Prout tangit individua relatione constitutivi, nego; relatione praedicabilitatis, concedo. Ad probationem distinguo antecedens. Respiciendo speciem constitutive, concedo; praedicabiliter, nego. Ad confirmationem distinguo minorem. Ut universalis est aequalis universalitate habita per ordinem ad speciem, nego; per ordinem ad individua, concedo. Articulo 3. etc. 
\pend

        \addcontentsline{toc}{section}{Articulus 3. Quomodo intelligenda sit definitio Porphyrii differentiae praedicabilis?}
        \pstart
        \eledsection*{Articulus 3. Quomodo intelligenda sit definitio Porphyrii differentiae praedicabilis?}
        \pend
      
\pstart
 Dico 1. Contra non nullos dari differentias ultimas et simplices convertibilesque cum speciebus. Ita  Divus \textsc{Thomas}\index[persons]{Thomas Aquinas} 2 \worktitle{Posteriorum} lectione 13 et 7 \worktitle{Metaphysicae} texto 42, lectione 12 et communiter Doctores. Et constat ratione. in rebus materialibus differentia sumitur a forma ut iam vidimus. Sed in re materiali est quaedam forma ultima per quam ultimo constituitur in suo esse: igitur et differentia ultima  Item talis forma est una et simplex, non compota ex pluribus formis: igitur etiam differentia ultima est una et simplex, non compota ex pluribus differentiis. 
\pend

\pstart
 Confirmatur 1. Differentia subalterna est simplex: igitur etiam infima. Consequentia tenet. Nam si a gradu imperfecto et potentiali, quae tribuit forma sumitur differentia simplex: igitur a fortiori a gradu perfecto et actuali. Confirmatur 2: passiones consequtae gradus infimos et specificos sunt simplices et aliis perfectiores: igitur et differentiae a quibus dimanant radicaliter. 
\pend

\pstart
 \textnormal{|}\ledsidenote{BNC 160va}   Obiectio 1. Ex \edtext{ \name{\textsc{Aristotele}\index[persons]{Aristoteles}} 2 \worktitle{Posteriorum} capitulo 14 \enquote{talia eo usque sumenda sunt (loquitur de differentiis) quousque tot fuerit sumpta: ut unumquodque illorum quidem pluribus competat cuncta autem simul non plura sese extendant} }{\lemma{}\Afootnote[nosep]{}}. Igitur non admittit tot differentias ultimas quot species. Nam hoc videtur deduci ex textu: igitur non dantur. Respondeo. \edtext{ Divus \name{\textsc{Thomas}\index[persons]{Thomas Aquinas}} 2 \worktitle{Posteriorum} lectione 13; \name{\textsc{Aristotelem }\index[persons]{Aristoteles}}ibi loqui de differentiis accidentalibus convincitis divisim latius patentibus, sed simul adaequantibus }{\lemma{}\Afootnote[nosep]{}}. 
\pend

\pstart
 Obiectio 2. Ex \edtext{ \name{\textsc{Porphyrio}\index[persons]{Porphyrius}} capitulo \worktitle{De communitatibus generis et differentiae asserente} }{\lemma{}\Afootnote[nosep]{}} idem cum \textsc{Aristotele}\index[persons]{Aristoteles} citato: igitur. Respondeo ut supra. Auctoritas \textsc{Porphyrii}\index[persons]{Porphyrius} et explanatio est resolutio praesentis , unde infra. 
\pend

\pstart
 Obiectio 3. Per differentias subalternas simul sumptas sufficienter constituitur species ab aliisque distinguitur: igitur superfluunt ultimae. Probatur antecedens. Haec est optima definitio trinitatis ex \edtext{ \name{\textsc{Philosopho}\index[persons]{Aristoteles}}: \enquote{trinitas est numerus impar primus} }{\lemma{}\Afootnote[nosep]{}} in qua singulae particulae latius patent definitio, scilicet trinitate. Siquidem numerus impar etiam competit numero quinario, septenario, etc., similiter numerus primus competit binario. At utrumque simul, scilicet esse imparem primum solum competit trinario: igitur hae differentiae subalternae sufficienter constituunt trinitatem. 
\pend

\pstart
 Respondeo. Differentias essentiales subalternas non posse constituere speciem ultimo; accidentales quantumcunque convinctas solum significare propriam et specificam differentiam. Hoc enim quando ignoramus ultimam differentiam licete facere, sicut \textsc{Philosophus}\index[persons]{Aristoteles} fecit definiendo trinitatem. Ceterum illae differentiae assignatae non sunt essentiales. Siquidem cum idem genus \textnormal{|}\ledsidenote{BNC 160vb} semel sumptum tantum possit in species dividi: numerus una divisione adaequate dividitur in parem  et imparem; alia in primum et non primum et non primum: igitur necessario aliqua differentia est accidentalis. Sic lucet \textsc{Aristoteles}\index[persons]{Aristoteles}, sic \textsc{Porphyrius}\index[persons]{}. 
\pend

\pstart
 Supposito enim dari differentias ultimas proadit difficultas, quomodo differentiae athomae possit competere definitio data, scilicet praedicari de differentibus specie? Ratio dubitandi est quia hoc solum differentiis subalternis convenit. Ergo non infimis et si hoc: igitur diminute procedit \textsc{Porphyrius}\index[persons]{Porphyrius}. 
\pend

\pstart
 Prima sententia docet perdictam definitionem convenire etiam differentiis infimis. Videamus quo pacto. Aliqui sic explicantur: differentia infima praedicatur de pluribus differentibus specie non in se, sed ab aliis; verbi gratia rationale praedicatur de Petro et Paulo, quae inter se non differunt specie, at specie differunt ab equo et leone. 
\pend

\pstart
 Haec subtilitas nimis crassa est (loquor cum \textsc{Complutensibus}\index[persons]{}) unde reiicitur primo sensus huius propositis, Petrus et Paulus differunt specie, hic est Petrus est unus specie similiter Paulus; at ille non est eiusdem speciei ac est iste, hoc est falsum: igitur. Secundo non minus Petrus et Paulus differunt specie ab aliquibus quam genere ab aliis, sed propter id, differentia praedicatur de differentibus specie: igitur propter hoc differentia praedicatur de differentibus genere. 
\pend

\pstart
  Alii aliter praecedunt, scilicet differentiam praedicari de pluribus specie differentibus per non repugnantiam: est antiquissima haec explicatio defenditur a \textsc{Caietano}\index[persons]{Thomas de Vio Caietanu} et ab aliis. Probatur haec sententia: plures definitiones explicantur per non repugnantiam: igitur haec. Probatur \textnormal{|}\ledsidenote{BNC 161ra}   antecedens. Definitio quanti ab \edtext{ \name{\textsc{Aristotele}\index[persons]{Aristoteles}} data 5 \worktitle{Metaphysicae}, scilicet \enquote{quantum est, quod est divisibile in ea, quae insunt, etc. Et positive conveniat pluribus ut ligno, lapidi, aliis convenit per non repugnantiam ratione communi ut Caelo.} }{\lemma{}\Afootnote[nosep]{}} Item minimum naturale qua tale non possit dividi, tam qua quantum potest dividi non repugnanter: igitur quamvis differentia ultima qua talis non praedicetur de differentibus specie, quia tamen ei non repugnat qua differentia est, dicetur per non repugnantiam praedicari de illis. 
\pend

\pstart
 Contra nam si semel admittatur definitio per non repugnantiam ratione communi in plura tamen invenimus absurda. Tunc enim haec esset bona definitio aequus est substantia sensibilis rationalis: patet, licet equo qua equo repugnet rationalitas, non tam repugnat qua animali: igitur per non repugnantiam ratione communi est rationalis similiter de omnibus. Patet sequela. Non minus rationale et irrationalem sunt differentiae oppositae quam esse subalternum et infimum, sed hoc non obsta quominus definitio differentiae subalternae competat per non repugnantiam differentiae infimae: igitur ne illud obstat quominus definitio rationalis per non repugnantiam competat hinnibili. His relictis. 
\pend

\pstart
 Dico 2. Definitio tradita non competit differentiae athomae. Probatur definitio tradita competit differentiae quae praedicatur de pluribus specie differentibus, sed differentia athoma non praedicatur de pluribus specie differentibus alioqui non esset athoma: igitur praefata definitio non competit differentiae athomae. 
\pend

\pstart
 Dubitas cur \textsc{Porphyrius}\index[persons]{Porphyrius} solam differentiam subalternam definit? Respondeo primo cum \textsc{Burleo}\index[persons]{Gualterius Burlaeus}.  Sicut capitulo praecedenti plures adducit definitiones speciei, quarum aliquae conveniunt specie infimae, \textnormal{|}\ledsidenote{BNC 161rb} et subalternae aliae vero soli differentiae infimae. Ita hic plures adducit definitiones convenientes utrique differentiae ut 1 et 3 et 4 et 5 et aliam, quae solum convenit subalternae sit secundam. 
\pend

\pstart
 Secundo. Respondeo cum communi Doctoribus. Id fecisse \textsc{Porphyrium}\index[persons]{Porphyrius} quia cum differentia athoma convertatur cum specie tractata specie consequenter tractata fuit illa. Item differentiae subalternae sunt communiores et notiores, athomae vero multiplice et vix cognitae. Item quia definita differentia subalterna facili negotio definitur infima. Unde. 
\pend

\pstart
 Dico 3. Differentia ex suo conceptu est quod praedicatur de pluribus et specie differentibus in quale quid, sunt expressa verba \textsc{Porphyrii}\index[persons]{Porphyrius}. Dico 4. Differentia subalterna est quae de pluribus differentibus specie hoc ipso quod quale quid praedicatur. Dico 5. Differentia athoma est quae de pluribus nego differentibus hoc ipso quod quale quid praedicatur. Sic manet clara meo videri tota controversia. 
\pend

\pstart
 Ast argumenta \textsc{Caietani}\index[persons]{Thomas de Vio Caietanu} resolvamus. Nego antecedens. Ad probationem nego assumptum. Quantum qua tale divisibile est positive. At in Caelo divisio est impedita, quod autem non exerciatur in Caelo et in minimo naturali provenit non ratione quantitatis, sed ratione subiecti in quo est. 
\pend

\pstart
 Respondeo secundo cum Doctore \textsc{Thoma}\index[persons]{Thomas Aquinas} propriam passionem quantitatis non esse divisibilitatem realem, naturalemve, sed divisibilitatem per designationem seu mathematicam: haec autem potest covenire Caelo, et minimo naturali. Praeterea quod definitio divini per non repugnantiam competere possit uni membro dividenti, non infert huic convenire definitionem alterius membri per non re \textnormal{|}\ledsidenote{BNC 161va}    repugnantiam. 
\pend

\pstart
 Infertur ex dictis hanc esse definitionem rigurosam tertium praedicabilis: quod praedicatur de pluribus in quale quid. Nam praedicari de differentibus specie vel differentibus numero est per accidens differentiae et competit ratione naturae. Qua propter divisio in differentiam infimam et subalternam est accidentis in substantia. Articulo 4. 
\pend

        \addcontentsline{toc}{section}{Articulus 4. An differentia athoma includat subalternas?}
        \pstart
        \eledsection*{Articulus 4. An differentia athoma includat subalternas?}
        \pend
      
\pstart
 Tria sub unico quaeruntur. Primo, an differentiae subalternae includantur in athomis: verbi gratia sensibilitas in rationali. Secundo, an genus includatur in differentia athoma: verbi gratia animal in rationali. Tertio, an tam differentiae subalternae quam genus precedentur de differentia athoma. 
\pend

\pstart
 Suppono ut certum differentiam athomam per materiali dicere speciem, quam constituit, verbi gratia rationale hominem: per formali vero dicere ullum gradum, qui simul cum differentia constituit speciem, verbi gratia rationale rationalitatem. In primo sensu non est dubium, nam realiter rationale est homo, est animal, est sensibile cum gradus metaphysici non distinguantur realiter. Solum est difficultas in sensu formali; an, scilicet differentia athoma secundum quod exprimit et prout distinguitur ratione a genere et differentiis subalternis includat tam genus quam differentias genericas. 
\pend

\pstart
 Prima sententia affirmat. Ita \edtext{ \name{\textsc{Soncinas}\index[persons]{Paulus Barbus Soncinatus}} 7 \worktitle{Metaphysicae} questione 35; }{\lemma{}\Afootnote[nosep]{}} Beatus \textsc{Albertus}\index[persons]{Albertus Magnus} et \textsc{Boethius}\index[persons]{Boethius} et nominalium turba et alii. Secunda negat. Ita \edtext{ \name{\textsc{Aristoteles}\index[persons]{Aristoteles}} 6 \worktitle{Topicorum}, capitulo 6 et 3 \worktitle{Physicae}, textu 10 }{\lemma{}\Afootnote[nosep]{}}; \textsc{Scotus}\index[persons]{Ioannes Duns Scotus}, \textsc{Suarez}\index[persons]{Franciscus Suarez}, \textsc{Oña}\index[persons]{Petrus de Oña}, \textsc{Masius}\index[persons]{Didacus Masius}, \textsc{Antonius Andreas}\index[persons]{Antonius Andreas}, \textsc{Niphus}\index[persons]{Augustinus Niphus} \textsc{Villalpandus}\index[persons]{Gaspar Cardillo de Villalpando}, \textsc{Avicenna}\index[persons]{Avicenna}, \textsc{Ferrara}\index[persons]{Franciscus Silvester Ferrariensis} et omnes Thomistae. 
\pend

\pstart
  Dico 1. Genus non includitur in differentia athoma. Ita \edtext{ \name{\textsc{Divus Thomas}\index[persons]{Thomas Aquinas}} 3 \worktitle{Methaphysicae} lectione 8 et 1 \worktitle{Contra gentes} capitulo 17 et \worktitle{De ente et essentia} capitulo 7. }{\lemma{}\Afootnote[nosep]{}} Probatur. Genus et differentia formaliter comparantur ut actus et potentia, sed actus est extra ratione potentiae et econtra: igitur extra rationem differentiae est genus et econtra. Secundo, genus est contrahibile et differentia contractiva, sed contrahibile est extra conceptum contrahentis: igitur. Tertio, genus et differentia sunt partes compositis metaphysicae, sed de ratione partis est excludere aliam compartem: igitur. Quarto, differentiae sunt omnino diversae: igitur non includunt genus. Probatur consequentia. Si in aliquo convenirent non essent primo diversae; sed si includerent genus in illo convenirent: igitur. Quinto, si differentiae includerent genus, hoc dupliciter conveniret speciebus: semel ratione seu et iterum ratione differentiae, hoc est absurdum, alioqui in definitionibus daretur nugatio haec sit, homo est animal, animal rationale ut videnti patebit: igitur. Sexto, si includeretur in differentiis: esset analogum ut patet: igitur. 
\pend

\pstart
 Dico 2. Differentiae genericae seu subalternae non includuntur in athomis. Eisdem rationibus probatur. Eadem est ratio de differentiis subalternis, ac de genere, sed genus non includitur in athomis: igitur. Maior patet. Siquidem differentia subalterna est pars quidditativa generis: igitur. Secundo, differentia contrahens genus, contrahit etiam differentiam genericam, sed differentia contrahens genus est extra rationem generis: igitur. Tertio, si differentia subalterna includeretur in athoma esset ut integra essentia: igitur quae conveniunt subalterne differunt athome sub eadem ratione. Tenet consequentia. Differentia athoma est simplex, alioqui esset species composita ex contrahibili et contrahente: igitur in una, eademque simplici entitate convenirent et different. 
\pend

\pstart
 Dico 3. Nec genus, nec differentia subalternae \textnormal{|}\ledsidenote{BNC 166ra}   praedicantur de athomis. Probatur faciliter. Omnis praedicatio supponit identitatem, sed formaliter genus et differentiae subalternae non identificantur athomis: igitur non praedicantur formaliter. Dixi formaliter: siquidem materialiter et identice praedicantur propter identitatem materialem et in tertio. 
\pend

\pstart
 Obiectio 1. Ex \edtext{ \name{\textsc{Aristotele}\index[persons]{Aristoteles}} 7 \name{\textsc{Methaphysicae}\index[persons]{}} textu 43 dicente tria }{\lemma{}\Afootnote[nosep]{}}. Primum, differentiae communes seu superiores clauduntur in infimis, unde nihil refert si in definitione ponantur illae omnes sive ista sola. Secundum, differentia infima est tota rei substantia quare ea posita in definitione non licet superiorem adiungere; alioqui daretur nugatio, scilicet homo est animal rationale sensibile. Tertium, ut inveniatur ultima differentia debet procedi per divisionem differentiae superioris in inferiores, scilicet eius per se divisivas: verbi gratia habere pedes dividitur per se in habere pedes fissos vel non fissos, sed verum est dicera fissio pedum aliqua pedalitas est: igitur de differentia infima praedicatur subalterna. Item \edtext{ 10 \worktitle{Methaphysicae} textu 23 }{\lemma{}\Afootnote[nosep]{}} ostendi in generibus, sub quibus inveniuntur contraria, differentias medias componi differentiis contrariis: igitur. 
\pend

\pstart
 Respondeo \textsc{Aristotelem}\index[persons]{Aristoteles} in primo et secundo loqui materialiter et identice \edtext{ \name{\textsc{Complutenses}\index[persons]{}} Distinctione 7, quaestione 4, numero 43 }{\lemma{}\Afootnote[nosep]{}}. Ad tertium dico: differentias communes dividi per alias non formaliter hoc ratione sui, sed materialiter, hoc est ratione generis constituti per differentiam superiorem. \textsc{Aristoteles}\index[persons]{Aristoteles} maluit dicere differentias superiores dividi non genut ut ostenderat differentias tam per se quam per accidens alicuius generis. Quando genus est in potentia ex intrinseca ratione ad differentias dividitur per se differentiis; sic dividitur animal bipes in habere pedes fissos vel no fissos. Quando vero est in potentia non ratione suae differentiae constitutivae, sed aliunde, \emph{tund [?]} dividitur \textnormal{|}\ledsidenote{BNC 166rb} differentiis per accidens; sic dividitur animal bipes in album et nigrum: igitur non solum differentia habendi pedes, sed etiam ipsum animal dividitur. Unde fissio est pedalitas reddit hunc sensum fissio continetur sub pedalitate, sicut differentiae continentur sub genere, cuius pars est differentia generica. Sic \textsc{Ioannes a Sancto Thoma}\index[persons]{Ioannes a Sancto Thoma} hic. Ad alium locum dicetur ibi non loqui seu sumere differentias medias per differentiis, quae sunt tertium praedicabile, sed pro speciebus oppositis. 
\pend

\pstart
 Obiectio 2. Rationale,verbi gratia est substantia: igitur vel corporea vel incorporea, animata vel inanimata: ergo iam habet aliquod praedicatum superius. Respondeo distinguendo antecedens. Rationale est substantia in quid, nego; in quale, concedo. Dupliciter enim sumitur substantia (idem indicium est de corporeo vel incorporeo) proprie, scilicet et late. Proprie secundum quod substantia est primum genus corporeum et incorporeum differentiae divisive et in hoc sensu rationale neque est substantia, nec corporeum aut incorporeum. Hoc enim est esse substantiam in quod seu completam et in recta linea. Late secundum habitudinem ad suum principium et reductive et hoc modo rationale dicitur substantia. 
\pend

\pstart
 Obiectio 3. In hoc syllogismo: `Omnis homo est sensibilis, sed omnis homo est rationalis: igitur aliquod rationale est sensibile;' praemissae sunt verae: igitur et conclusio. Respondeo distinguendo consequens et conclusio in sensu formali, nego; materialiter et identice, concedo. Non requiritur quod ex praemissis verificatis in sensu formali sequatur conclusio etiam vera in sensu formali, cum sufficiat in sensu indentico, quod saepe accidit in tertia figura: verbi gratia omne lac est album; omne lac est dulce: igitur aliquod dulce est album. 
\pend

\pstart
 Instabis: tota ratio ruit si addatur particula formaliter dicendo: omnis homo est sensibilis \textnormal{|}\ledsidenote{BNC 166va}   formaliter; omnis homo est rationalis formaliter: igitur aliquod rationale est formaliter sensibile. Responde falsam esse consequentiam quia variatur appellatio hic formaliter. Siquidem in praemissis appellat supra res formaliter se includentes; at in conclusione supra res de quarum ratione est, unam excludere aliam. Vide hoc diffuse in \edtext{ \name{\textsc{Ioannes a Sancto Thoma}\index[persons]{Ioannes a Sancto Thoma}} quaestione 10, articulo 3 }{\lemma{}\Afootnote[nosep]{}}. 
\pend

\pstart
 Obiectio 4. Differentia infima non est actus purus: igitur aliquam habet compositionem: igitur potest resolvi in conceptum communem seu genericum et specialem seu differentialem. Respondeo distinguendo consequens. Aliquam habet compositionem ex his, nego; huic, concedo. Differentia enim est pars, pars enim non est simplex sicut actus purus simplicitate perfectionis, sed simplicitate negationis quia non est plene perfecta. Et nego ultimam consequentiam. Est enim alteri componibilis a quo perficiatur, non tamen composita per resolutionem in plures conceptus, nec subordinatur directe et formaliter superiori, sed praesupponit illam veluti genus compositum ex illa. 
\pend

\pstart
 Obiectio 5. Intellectivum ut quattuor est differentia specifica alicuius Angeli, sed intellectivum ut quattuor includit intellectivum: igitur genus includitur in differentiis. Respondeo distinguendo minorem. Materialiter et identice, concedo; formaliter, nego. Differentia enim expressatur in hic quattuor. Sic \edtext{ Divus \name{\textsc{Thomas}\index[persons]{Thomas Aquinas}} 1 parte, questione 50, articulo 4, 1 }{\lemma{}\Afootnote[nosep]{}}. 
\pend

\pstart
 Obiectio 6. Haec est vera praedicatio `homo est realis': igitur et haec, `rationale est animal'. Probat consequentia. Quidquid convenit subiecto convenit praedicato, sed subiecto convenit animal: igitur. Respondeo distinguendo consequens. Identice, concedo; formaliter, nego. Ad probationem distinguo maiorem. Eodem modo, nego; diverso, scilicet dicto, concedo. Et hoc si enim conveniat subiecto propter praedicatum ut contingit in praesenti. Cetera argumenta omissa facio vel quia ex dictis solvuntur vel quia ex non dictis vic intelliguntur. Articulo 5. Etc. 
\pend

        \addcontentsline{toc}{section}{Articulus 5. An quaelibet differentia sit perfectior genere?}
        \pstart
        \eledsection*{Articulus 5. An quaelibet differentia sit perfectior genere?}
        \pend
      
\pstart
 Non comparamus differentiam generi simpliciter, sed tantum secundum quid. Loquatur \edtext{ Divum \name{\textsc{Thomas}\index[persons]{Thomas Aquinas}} loco citato \worktitle{De piritualibus creaturis}: \enquote{sicut determinatum (loquitur de comparatione differentiae cum genere) indeterminato et proprium communi, non autem sicut animalia et animalia natura.} }{\lemma{}\Afootnote[nosep]{}} In hoc sensu quaerimus an semper differentia determinet genus addendo aliquam perfectionem? Negat \edtext{ \name{\textsc{Ona}\index[persons]{Thomas Aquinas}} 1 quaestione, articulo 5 }{\lemma{}\Afootnote[nosep]{}} quem sequuntur aliqui Magistris. Affirmativa est communis. 
\pend

\pstart
 Dico igitur omnis differentia est perfectior suo genere nullaque dari, quae non addat aliquam perfectionem. Ita \edtext{ \name{\textsc{Aristoteles}\index[persons]{Aristoteles}} 8 \worktitle{Metaphysicae} textu 10 }{\lemma{}\Afootnote[nosep]{}}. Et \edtext{ Doctor \name{\textsc{Thomas}\index[persons]{Thomas Aquinas}} \worktitle{De spiritualibus creaturis} articulo 8, ad 9 et 1 parte, quaesitone 50 articulo 4, ad 1 et questione 57, articulo 2, ad 2 }{\lemma{}\Afootnote[nosep]{}} Quem sequuntur eius amantissimi Doctores. 
\pend

\pstart
 Probatur ratione. Genus est perfectibile ulteriori perfectione: igitur omnis differentia addit aliquam perfectionem. Probatur consequentia. Illa ulterior perfectio ad quam genus est in potentia est differentia, sed in quacunque specie actuatur et perficitur potentialitas generis: igitur in quacunque specie differentia addit generi aliquam perfectionem. Patet minorem. Nulla datur species, quae non sit totum actuale in essentia ultimo actuatum et constitutum. 
\pend

\pstart
 Confirmatur: omnis actus seu forma est nobilior potentia seu materia, sed omnis differentia habet rationem actus seu formae respectu generis: igitur omnis differentia est nobilior genere eique superaddit aliquam perfectionem. Maiorem constat. Omnis actus aliquam tribuit potentiae perfectionem omnisque forma tribuit etiam effectum formalem: igitur. 
\pend

\pstart
 \textnormal{|}\ledsidenote{BNC 167ra}   Obiectio 1. Volitio prava est imperfectior volitione in communi: igitur. Confirmatur: sapientia creata etiam est imperfectior sapientia ut sic: igitur. Respondeo distinguendo antecedens. Volitio prava est imperfectior in eo moris formaliterque, concedo; in esse rei et physice, nego. Peccatum in esse moris sive consistat in positivo, sive in negatio est omnimoda imperfectio. At in esse physico superaddit aliquam perfectionem generi volitionis. Ad confirmationem negatur assumptum. Cum sapientia non sit genus, sed ratio analoga ad creatam et in creatam. In analogis enim maiorem perfectionem dicit analogum, quam analogata. 
\pend

\pstart
 Obiectio 2. Ex his sequeretur unamquamque speciem habere distinctam operationem ab aliis speciebus eiusdem generis, hoc no potest esse: igitur. Probatur minor. Probabile est omnes species angelicas habere easdem operationes, scilicet intelligere et velle. Item quaedam animalia non solum non habent operationem distinctam, verum etiam carent aliquibus convenientibus generi. Nam talpa caret visu, olfactu pisces, similiter de aliis. 
\pend

\pstart
 Respondeo. \edtext{ Divus \name{\textsc{Thomas}\index[persons]{Thomas Aquinas}} loco citato per haec verba: \enquote{Illud quod constituit in specie est nobilius eo quod constituit in genere, sicut determinatum indeterminato: habent enim se determinatum ad indeterminatum sicut actus ad potentiam. Non autem ita, quod semper illud, quod constituit in specie ad nobiliorem naturam pertineat, ut patet in speciebus alium irrationalium. Non enim constituuntur huiusmodi species per additionem alicuius naturae nobilioris supra naturam sensitivam, quae est nobilissima in eis, sed per determinationem ad diversos gradus in illa natura et similiter dicendum est de intellectuali, quod est commune in angelis.} }{\lemma{}\Afootnote[nosep]{}} 
\pend

\pstart
 Unde negatur sequela. Tunc enim differentia est radix novae operationis, quando elevat genus ad altiorem naturam. Etiam in praefatis animalibus propria differentia uniuscuiusque est perfectior genere. Nam saltem determinat aliquam operationem indeterminate et confuse convenientem generi. 
\pend

\pstart
 \textnormal{|}\ledsidenote{BNC 167rb} Obiectio 3. Differentia non elevans genus ad altiorem naturam significat perfectionem generis magis conttractam et limitatam: igitur potius perfectionem auferunt. Consequentia patet. Contractio et limitatio intra idem genus attestatur imperfectioni rei limitatae, nam tollit aliquid generis et non addit alterius gradus perfectionem. 
\pend

\pstart
 Respondeo negando consequentiam. Ad probationem distinguo antecedens. Si contractum sit ex se illimitatum in genere causae materialis, nego; si sit illimitatum in alio genere, concedo. Illimitatio in genere causae materialis est maxima imperfectio, nam eam causat privatio perfectionis, scilicet formae et actus, quare materia prima est ens imperfectissimum ac prope nihil, quia ex se est omnino illimitata in genere causae materialis, est enim pira potentia. Similiter genus proportionabiliter in genere causae materialis est illimitatum: igitur quod determinetur a differentiis arguit potius perfectionem quam imperfectionem. 
\pend

\pstart
 Nota genus dividi per differentias in species, non in differentias. Hic solet agitari compositio metaphysica an sit realis vel rationis? Ex dictis facile respondetur esse rationis ratiosinatae. Unde talis compositio supponit in re fundamentum, quod non est aliud quam exuberantia rei et deficientia intellectus, sic \edtext{ Fundatissimo Doctor \name{\textsc{Aegidius}\index[persons]{Aegidius Romanus}} }{\lemma{}\Afootnote[nosep]{}}. Aliae quaestiones nimius prolixe omittuntur, non tamen necessariae. Nunc videamus alia praedicabilia, scilicet quae sunt extra essentiam et primo de proprio quarto praedicabili. 
\pend

        \addcontentsline{toc}{chapter}{Capitulum quartum}
        \pstart
        \eledchapter*{\supplied{Capitulum quartum}}
        \pend
      
        \addcontentsline{toc}{section}{Textus capituli quarti Porphyrii}
        \pstart
        \eledsection*{Textus capituli quarti Porphyrii}
        \pend
      
\pstart
\noindent%
 \textnormal{|}\ledsidenote{BNC 163va}   Caput quartum \textsc{Porphirii}\index[persons]{Porphyrius} de Proprio quarto Praedicabili Textus 
\pend

\pstart
 \textnormal{|}\ledsidenote{BNC 163vb} \edtext{\enquote{ \worktitle{ΚΕΦΑΛΑΙΟΝ Δ} Τὸ δὲ ἴδιον διαιροῦσι τετραχῶς · καὶ γὰρ ὃ μόνῳ τινὶ εἴδει συμβέβηκεν, εἰ καὶ μὴ παντί, ὡς ἀνθρώπῳ τὸ ἰατρεύειν ἢ τὸ γεωμετρεῖν · καὶ ὃ παντί, καὶ τὰ ἕτερα. }}{\lemma{}\Afootnote[nosep]{ \textsc{Commentarii Collegii Conimbricensis e Societate Iesu}\index[persons]{}, \worktitle{In universam dialecticam Aristotelis Stagirita} (Lugduni: sumpt. Iacobi Cardon et Petri Cavellat, 1622), p. 134. }} 
\pend

\pstart
 \textnormal{|}\ledsidenote{BNC 163vc} \edtext{\enquote{ \worktitle{Caput quartum} Proprium autem quatuor dividitur modis. Nam proprium est id quod soli cuipiam accidit speciei et se non omni ut homini accidit mederi aut metiri, etc. }}{\lemma{}\Afootnote[nosep]{ \textsc{Commentarii Collegii Conimbricensis e Societate Iesu}\index[persons]{}, \worktitle{In universam dialecticam Aristotelis Stagirita} (Lugduni: sumpt. Iacobi Cardon et Petri Cavellat, 1622), p. 134. }} 
\pend

        \addcontentsline{toc}{section}{Summa Textus}
        \pstart
        \eledsection*{Summa Textus}
        \pend
      
\pstart
\noindent%
 \textnormal{|}\ledsidenote{BNC 163va} Quattuor in hoc capite adducit \textsc{Porphyrius}\index[persons]{Porphyrius} acceptiones `proprii'. Primo proprium dicitur quod convenit soli alicui speciei, sed non omni; ut esse Medicum vel Geometram respectu hominis. Secundo, quod convenit omni, sed non soli; ut esse bipedem in homine. Tertio, quod convenit omni, soli, sed non semper; ut hominem canescere in senectute. Quarto, quod convenit omni, soli et semper; ut risibilitas. Hoc praefert omnibus \textsc{Porphyrius}\index[persons]{Porphyrius} est enim proprie proprium. 
\pend

        \addcontentsline{toc}{section}{Annotationes circa Litteram Capitis}
        \pstart
        \eledsection*{Annotationes circa Litteram Capitis}
        \pend
      
\pstart
\noindent%
 \textnormal{|}\ledsidenote{BNC 163vb} Acceptiones praefatae in re non sunt aliae ab illis, quas tradidit \edtext{ \name{\textsc{Aristoteles}\index[persons]{Aristoteles}} 1 \worktitle{Topicorum} capitulo 4 }{\lemma{}\Afootnote[nosep]{}}, ibi dividit proprium in prorium ad aliquid, in proprium quando et in proprium simpliciter. Proprium ad aliquid coincidit cum proprio primo et secundo modo dictum. Dicitur ad aliquid quia debet res cuius sunt propria ad aliud comparari: verbi gratia homo dicitur Geometra comparatione ad Leonem qui Geometra non dicitur. Similiter ut dicatur bipes. Proprium vero quando coincidit cum proprio tertio modo sumpto. Dicitur quando, quia non semper convenit rebus, sed aliquando. Proprium denique simpliciter coincidit cum proprie dicto. 
\pend

\pstart
 \textnormal{|}\ledsidenote{BNC 163vc} \del{\added{tradi}} Proprium quarto modo acceptum est proprium denominabile a relatione universalis seu quarti praedicabilis. Unde est de quo in praesenti est agendum. Alia dicuntur propria analogice. Notato nunc proprium dicitur dupliciter vel prout distinguitur ab improprio vel prout distinguitur a communi. In hoc ultimo modo sumitur hic. 
\pend

\pstart
 Circa primam acceptionem. Notato esse Medicum vel Geometram in exemplo alato non summi per capacitate seu aptitudine acquirendi scientias, alioqui proprium quarto modo feret dicendum. Unde \textnormal{|}\ledsidenote{BNC 164ra}    sumitur per habitu vel actu Medicinae vel Geometriae ut explicat \textsc{Villalpandum}\index[persons]{Gaspar Cardillo de Villalpando}. 
\pend

\pstart
  Circa secundam acceptionem venit in dubium, si \textsc{Aristoteles}\index[persons]{Aristoteles} asserit esse bipedem esse differentiam, quo modo nunc appellatur proprium? Respondeo \textsc{Aristotelem}\index[persons]{Aristoteles} ibi loqui non rigorose vel loquitur de differentia propria. Ceterum esse bipedem est accidens commune secundum quod tantum dicitur proprium et differentis. 
\pend

\pstart
  Circa tertiam notato canitie dici proprium hominibus, quia frequentior et magis nota est anities in hominibus quam in aliis animalibus. Sic omittimus difficultatem aliquorum dicentium etiam aliis animalibus convenire. Nunc Quaestio 10. etc. 
\pend

        \addcontentsline{toc}{section}{Quaestio 12. De proprio quarto praedicabili}
        \pstart
        \eledsection*{Quaestio 12. De proprio quarto praedicabili}
        \pend
      
\pstart
 Multa quae circa hoc caput tractari possunt manent ex dictis indagata, quae restant faciliter, breviterque videntur in praesenti questtione. Unde Articulo 1. Explicatur definitio, etc. 
\pend

        \addcontentsline{toc}{section}{Articulus 1. Explicat naturam Proprii eiusque divisiones expendit?}
        \pstart
        \eledsection*{Articulus 1. Explicat naturam Proprii eiusque divisiones expendit?}
        \pend
      
\pstart
  Proprium quadrupliciter dicitur (necessaria est repetitio). Primo, quod competit soli specie, at non omni individuo, ut esse Geometram. Secundo, econtra omni individuo, at non soli speciei, ut esse hominem bipedem. Tertio, omni speciei et omni individuo, at non semper ut in senectute canescee. Quarto, soli speciei et omni individuo et \textnormal{|}\ledsidenote{BNC 164rb} semper, ut hominem esse risibilem: hoc est maximum. 
\pend

\pstart
  Pro horum intelligentia est notandum proprium a \textsc{Porphyrio}\index[persons]{Porphyrius} per ordinem ad speciem explicari. Unde perfecta ratio proprii sumitur ex perfecta convenientia cum specie. Duo nobis obiiciuntur cognoscenda. Primum, quodnam sit divisum, et quaenam sit divisio? Secundum, quomodo intelligatur definitio proprii in quarto modo? 
\pend

\pstart
  Dico 1. Divisum huius divisionis est ratio proprii communis consistens in tendentia convenientiae ad unum cum exclusione communicationis alterius. Ratio est quia ratio quarto praedicabilis est ut precetur in quale necessario accidentaliter. Nam per hic in quale differt a genere et differentia specie; per hic accidentaliter a differentia; per hic necessario ab accidente communi; sed hae conditiones non conveniuntur divisio, sed tantum uni membro, scilicet proprio in quarto modo: igitur quartum praedicabile non est divisum: igitur proprium communiter dictum. 
\pend

\pstart
  Dico 2. Proprium communiter dictum divisione analoga dividitur in quattuor modos proprii. Est communis inter Auctores et vocamus proprium commune proprium metaphysicum. Et ratio est: quotuscunque aliqua res dividitur inaequaliter dividitur analogice, sed sic dividitur proprium in communi: igitur. Probatur minor. Solum proprium in quarto modo est simpliciter proprium, cetera vero secundum quid: igitur. 
\pend

\pstart
  Ut hoc melius percipiatur nota duas requiri conditiones ad proprium. Prima quod conveniat uni inseparabiliter quoad unitatem. Secunda ut non conveniat alteri extraneo. Prima conditio servatur quando convenit rei ubicunque inventae \textnormal{|}\ledsidenote{BNC 164va}   semperque, hoc est semper et firmiter et inseparabiliter. Ut autem aliquid firmiter conveniat alicui debet habere radicem et principium in illo, alioqui transeunter et separabiliter conveniet vi conexionis, nec obstat quod possit permanere, tunc erit ab extrinseco conservate quoad existam et non quoad convenientiam. 
\pend

\pstart
  Secunda servatur, quando convenit soli, non debet convenire alii extra ipsum. Quapropter si deficiat aliqua ex his conditionibus non datur simpliciter integreque proprium. Siquidem caret ratione proprii et conditionibus requisitis. Hae conditiones deficiunt in tribus primis modis. Nam primum proprium convenit soli, sed non omni nec semper: igitur est separabile: igitur non proprium simpliciter. Secundum convenit omni, sed non soli: igitur convenit extraneis et est commune. Tertium convenit omni soli, sed non semper: igitur non permanenter, sed separabiliter. 
\pend

\pstart
  Dubitas. Cur dicantur propria? Respondeo. Esse propria secundum quid. Tum quia habent aliquas conditiones proprii, tum quia comparative sunt propria. Nam esse bipedem licet absolute non sit proprium homini, comparative tamen ad quadrupedia proprium est illi. 
\pend

\pstart
  Dico 3. Talem divisionem esse adaequatum. Probatur tres conditiones requisitae, scilicet omni, soli, et semper vel omnes conveniunt vel deest aliqua. Sicut conveniunt omnes est proprium simpliciter et in quarto modo. Si deest aliqua vel est prima, scilicet omni et sic est proprium in primo modo vel deest secundo, scilicet soli et sic est proprium secundo modo vel deest tertio, scilicet semper et sic est in tertio modo proprium. Si desunt omnes non est proprium. 
\pend

\pstart
  Circa secundum, sic definitur proprium in quarto modo: \enquote{\textsc{Porphyrius}\index[persons]{Porphyrius} est quod convenit omni, soli \textnormal{|}\ledsidenote{BNC 164vb} et senper.} Unde sic formatur definitio quarti praedicabilis: \enquote{Proprium est quod praedicatur de pluribus in quale necessario accidentaliter.} Sed haec definitio articulo sequente examinabitur. 
\pend

\pstart
  Dubitas. Quod est consequi seu dimanare ab essentia, habereque illam pro radice? Dico pro nunc dimanationem non esse causalitatem effectivam, hoc est agentem actionem, habentemque modum efficientiae. Ratio est substantia sine accidentibus non efficit nec operatur: igitur accidentia quibus operatur dicuntur dimanare a substantia, quin ab ipsa producantur: a quonam? Ab agente simul cum substantia, dependent a substantia cuius sunt passiones: igitur dimanatio nihil est aliud quam dependentia, quia ab ipsa tamquam a principio participatur. Unde habet rationem activam reductive non foraliter. Sic \edtext{ Divus \name{\textsc{Thomas}\index[persons]{Thomas Aquinas}} 1 parte, quaestione 77, articulo 63 et articulo 7, 1 et \worktitle{Opusculo 48} capitulo \worktitle{De proprio} }{\lemma{}\Afootnote[nosep]{}} 
\pend

\pstart
  Obiectio 1. Proprium in re est singulare: igitur vel est proprium secundum se vel prout individuatum. Si \supplied{s}ecundum se: igitur est commune et prout sic non invenitur in re. Si ut singularizatum: igitur non dimanat ab essentia. Confirmatur: proprium quod est divisum dicitur de pluribus: igitur proprium divisum non est prout est in re, sed prout est in intellectu: igitur prout est logicum. 
\pend

\pstart
  Respondeo est proprium ratione naturae, non ratione singularitatis. Et distinguo consequens. Est commune analogice, concedo; univoce, nego. Et aliam compartem: non invenitur in re formaliter, concedo; fundamentaliter, nego. Sicut enim praedicata essentialia inveniuntur in re singularizata quin obstet, ut non sint talia ratione singularitatis, sic proprium invenitur in re singularizatum, quin sit talem ratione singularitatis. Ad \textnormal{|}\ledsidenote{BNC 165ra}   confirmationem distinguo consequens. Proprium divisum non est in re primo intentionaliter, nego; secundo intentionaliter, concedo. Sicut enim genus quoad naturam substrata est in re et non quoad praedicabilitatem, nisi fundamnetaliter. Sic proprium est in re quoad naturam substratam et non quoad praedicabilitatem. 
\pend

\pstart
  Obiectio 2. Divisum huius divisionis est accidens vel quod abstrahit a quarto et quinto praedicabilis vel determinate quarto vel quinto. Si primum: igitur divisum non est proprium, sed abstrahens. Si secundum: igitur divisum est quarto praedicabile. Si tertium, quomodo quarto praedicabile oppositum quinto est membrum huius divisionis? Confirmatur: Proprium secundo modo est accidens commune: igitur non proprium. 
\pend

\pstart
  Respondeo distinguendo. Est accidens valens funddare conditiones proprii et convenientiam relativam ad aliquid sub formalitate proprietatis, concedo; est accidens praesicce fundans praedicabilitates quarto et quinto praedicabili, nego. Sicut enim differentia divisa in communem propriam et proprissimam non est sub conceptu praedicabilis, sed sub conceptu facientis differre. Sic Proprium in illos modos divisum non est sub conceptu praedicabilis, sed sub conceptu convenientiae ad unum cum exclusione consortii. 
\pend

\pstart
  Ad confirmationem distinguo antecedens. Est accidens commune sub ratione universalitatis et praedicabilitatis, concedo; sub conceptu convenientiae ad unum, nego. Simpliciter est accidens commune, at secundum quid et comparative est proprium; unde reductive pertinet etiam ad quartum. Ceterum non est inconveniens, quod simpliciter pertineat ad unum praedicabile, secundum quid reductive ad aliud, nam rationale secundo intentionaliter pertinet ad tertium et tertio intentionaliter ad secundum. 
\pend

\pstart
  Obiectio 3: Illa particula semper non convenit omnibus passionibus: igitur ruit. Probatur antecedens. Primo, potest Deus \textnormal{|}\ledsidenote{BNC 165rb} separare propriam passionem a subiecto, sicut separatur quantitas a corpore in eucharistia vel sine illa subiectum conservare, ut solem sine luce. Ceterum naturaliter potest impediri passio, ut motus deorsum in lapide, frigus in aqua, potentiae sensitivae in animali: igitur. Secundo, existentia contingenter convenit creaturae, nec potest esse passio: alioqui idem se ipsum efficeret, sed tamen per se convenit formae sicut rotunditas circulo teste \edtext{ Divo \name{\textsc{Thoma}\index[persons]{Thomas Aquinas}} 2 \worktitle{Contra Gentes} 55 }{\lemma{}\Afootnote[nosep]{}} Igitur non stat passio in emanatione et perseitate a subiecto. 
\pend

\pstart
  Respondeo negando antecedens. Ad probationem distinguo. Potest Deus separare proprias passiones reales, concedo; modales, nego. At hic convenit positum in definitione idem est ac debetur, conectitur et postulatur non; dicit praesicce existere seu adesse subiecto. Quare licet separetur passio, cum tamen debentur naturae, verificatur quod convenit semper. Pro speciali doctrina scito, quod passio non potest negari subiecto per secundam operationem, videlicet per compositionem et divisionem; licet possit per primam. At accidens commune etiam per secundam operationem potest negari subiecto, quin destruatur. Sic \edtext{ Doctor \name{\textsc{Thomas}\index[persons]{Thomas Aquinas}} \worktitle{Opusculo 48 De accidente} et in 1, distinctione 26 }{\lemma{}\Afootnote[nosep]{}}. 
\pend

\pstart
  Ad aliam probationem dico omnes passiones dimanare a subiecto eique conecti, sed omnes non eodem modo. Quaedam dimanant absolute, sine ulla alia dependentia quam a subiecto, ut risibilitas ab homine, diaphanitas a Caelo, lux a Sole. Quaedam dimanant supposito aliquo statu connaturali, ut lapidi quiescere non absolute, sed in centro; moveri non absolute, sed extra centrum, \emph{gratiae [?]} in statu consummationis lumen gloriae in statu \emph{causae [?]} fides, spes; aque in suo connaturali statu frigus, animali potentiae sensitivae supposito organi temperamento, demum gravi moveri per medium liberum non impeditum. \textnormal{|}\ledsidenote{BNC 165va}   Primae passiones non possunt impediri, ultimae vero possunt ab extrinseco. 
\pend

\pstart
  Ad secundum distinguo minorem. Per se convenit formae perseitate logica, nego; perseitate physica, concedo. Perseitas logica fundatur in habitudine convenientiae intrinsece; physica fundatur in re ut producta et ut subiecte influxui physico agentis. Unde existentia non est propria passio. Sumit enim determinationem et specificationem a forma cuius est; et in forma dependente a materia solvitur existentia per segregationem formae a materia; si autem non dependet a materia, non tollitur a forma nisi per suspensionem actionis agentis et hoc est per annihilationem. Articulo 2. Quod sit Proprium, etc. 
\pend

        \addcontentsline{toc}{section}{Articulus 2. Quid sit proprium quartum praedicabile?}
        \pstart
        \eledsection*{Articulus 2. Quid sit proprium quartum praedicabile?}
        \pend
      
\pstart
  Quaerimus an definitio quarto praedicabilis coveniar omnibus propriis, an solum proprio quarto modo? Prima sententia affirmat. Sic \textsc{Lovanienses}\index[persons]{}, \textsc{Canterus}\index[persons]{Ioannes Cantero} , \textsc{Murcia}\index[persons]{Franciscus Murcia de la Llana} , \textsc{Hurtado}\index[persons]{}. Secunda sententia asserit salte aliquod ex tribus pertinere ad hoc praedicabile. De secundo tenet \textsc{Fonseca}\index[persons]{Petrus Fonsecae}. . De tertio aliqui Magistri ut refert Magistro \textsc{Rubio}\index[persons]{Antonius Ruvius Rodensis}. 
\pend

\pstart
  Dico igitur solum proprium quarto modo pertinet simpliciter ad hoc praedicabile, ceterum vero simpliciter ad quintum, reductive vero et secundum quod ad hoc. Sic \edtext{ \name{\textsc{Aristoteles}\index[persons]{Aristoteles}} 1 \worktitle{Topicorum} 2 }{\lemma{}\Afootnote[nosep]{}}; \edtext{ Doctor \name{\textsc{Thomas}\index[persons]{Thomas Aquinas}} \worktitle{Opusculo 48}, capitulo 6 et 7 }{\lemma{}\Afootnote[nosep]{}}; \textsc{Caietanus}\index[persons]{Thomas de Vio Caietanu}; \textsc{Iabellus}\index[persons]{Chrysostomus Iavelli Canapicii}; \textsc{Soto}\index[persons]{Dominicus de Soto}; \textsc{Masius}\index[persons]{Didacus Masius}; \textsc{Arauxo}\index[persons]{Franciscus de Arauxo}; \textsc{Sanchez}\index[persons]{Ioannes Sanchez Sedeno}; \textsc{Toletus}\index[persons]{Franciscus Toletus}; \textsc{Oña}\index[persons]{Petrus de Oña}; \textsc{Rubius}\index[persons]{Antonius Ruvius Rodensis}; \textsc{Conimbricensis}\index[persons]{} et alii \edtext{ \name{\textsc{Gallego}\index[persons]{Baranbas Gallego de Vera}} \worktitle{Controversia 18} }{\lemma{}\Afootnote[nosep]{}}. Demum \textsc{Complutensis}\index[persons]{}, \textsc{Ioannes a Sancto Thoma}\index[persons]{}. 
\pend

\pstart
  Probatur ratione. Illud est vere et proprie et simpliciter quod dimanat a principiis essentialibus, cum eis \textnormal{|}\ledsidenote{BNC 165vb} qui habet naturalem necessariamque convexionem, sed tantum proprium conveniens omni, soli, et semper est huiusmodi: igitur. Probatur minor. Quod habet necessariam conexionem cum praedicatis essentialibus invenitur in omnibus, in quibus sunt talia principia essentialia: igitur convenit omni habenti talia principia. Item ergo ubi non fuerint talia principia essentialia neque erit tale proprium: igitur convenit solum habentibus illa. Tandem: igitur quandiu habuerint talia principia tamdiu habebunt proprietatem, sed quae participant aliquam essentiam, semper habent illam: igitur et proprietatem: igitur tale proprium convenit omni, soli, semperque. 
\pend

\pstart
  Confirmatur: tria prima propria vere non dimanant a principiis essentialibus: igitur non habent necessariam conexionem: igitur non constituunt hoc praedicabile. Antecedens probatur. Esse Geometram si sumatur per habitu vel actu Geometriae, porro non fluit ab his essentia. Si vero sumatur per inclinatione ad hanc scientiam potius quam ad aliam provenit hoc ex principiis individualibus, a diverso temperamento, a vario \emph{sumderum [?]} aspectu, sub quo quosque natus est, vi cuius homines quibus prae est Iupiter sunt ad scientias proclives quibus Mercurius eloquentes, etc. Similiter esse bipedem aut anescere consequuntur ex dispositione materiae, sicut esse masculum vel feminam, viventem vel senem, etc. Igitur talia dimanant ex principiis materialibus et individualibus non vero ex essentialibus. 
\pend

\pstart
  Obiectio 1. Tria propria speciali modo respiciunt inferiora: igitur non sunt accidens commune. Probatur antecedens. Proprium primo modo respicit ratione solius, secundo modo ratione omnis, tertio ratione omnis et solius; sed accidens commune solum respicit ratione contingentis: ergo. 
\pend

\pstart
  Respondeo distinguendo antecedens. Speciali modo, qui sufficit ad proprium simpliciter, nego; qui sufficit ad \textnormal{|}\ledsidenote{BNC 166ra}   proprium secundum quod et reductive, concedo. Dictum est proprium simpliciter constitui per illas tres conditiones, ita ut si non inveniantur aut aliqua tantum; vel non datur proprium vel datur tantum secundum quid. 
\pend

\pstart
  Obiectio 2. Illa propria sumpta per aptitudine dimanant ab essentia: igitur pertinent ad quarto praedicabile. Antecedens probatur. Primo ex \edtext{ Divo \name{\textsc{Thoma}\index[persons]{Thomas Aquinas}} \worktitle{De ente et essentia} 7 }{\lemma{}\Afootnote[nosep]{}} affirmante omne accidens causari a principiis substantiae: igitur. Secundo ab inductione, aptitudo ad esse Geometram oritur ex principiis essentialibus naturae humanae naturaliter scire desiderantis: igitur. 
\pend

\pstart
  Respondeo concedendo antecedens. Sed tunc sunt proprium quarto modo ut dictum est, nec sic sumuntur in praesenti. At si sumantur pro speciali habilitate, dimanant a principiis materialibus et individualibus, quapropter sunt accidentia communia. Ad probationem prima distinguo omne accidens causatur a substantia eodem modo, nego; diverso, concedo. Quaedam causantur a substantia secundum principia formalia et haec sunt passiones, quia habent conexionem necessitatis formalis. Quaedam secundum praedicata materialia et sunt accidentia communia seu accidentia individuorum. Ad secundam respondeo sicut ad argumentum. 
\pend

\pstart
  Obiectio 3. Quod convenit omni et semper convenit necessario: igitur proprium secundo modo convenit necessario, verbi gratia esse bipedem. Antecedens probatur. Non requiretur formalitas solius; alioqui proprium subalternum non esset proprium. 
\pend

\pstart
  Respondeo negando antecedens. Albedo convenit cygno omni semperque et tamen est accidens commune. Ultra requiritur quod conveniat firmiter simpliciter, non per accidens ut dictum est. Ad probationem negatur antecedens. Proprium subalternum soli naturae competit: igitur nihil deest. 
\pend

\pstart
  Obiectio 4. Non requiritur formalitas semper: igitur. \textnormal{|}\ledsidenote{BNC 166rb} Probatur antecedens. Quod dimanat ab essentia est passio, sed quod dimanat ab essentia non tollitur eo quod non semper coveniat, sed tempore statuto, quod patet in potentia visia canis possit nonum diem. 
\pend

\pstart
  Respondeo negando antecedens. Ad probationem distinguo minorem. Non tollitur eo quod non semper conveniat quoad unitatem, nego; quoad existentiam, concedo. Hoc manet explicatum numero 993. Ad exemplum, ibi non datur variatio secundum tempus in ipsa potentia, sed in dispositione organica ut eliciatur visio. Articulo 3. Etc. 
\pend

        \addcontentsline{toc}{section}{Articulus 3. In ordine ad quae proprium constituatur praedicabile?}
        \pstart
        \eledsection*{Articulus 3. In ordine ad quae proprium constituatur praedicabile?}
        \pend
      
\pstart
  Dico 1. Inferiora propria quarto praedicabilis qua talis non sunt propria, sed individua speciei, a qua dimanat. Probatur. Illa sunt inferiora quarto praedicabilis de quibus praedicatur in quale necessario accidentaliter, sed de hoc et illo proprio non sic praedicatur: igitur propria non sunt inferiora. Probatur minor. Praedicari in quod complete non est praedicari in quale necessario accidentaliter, sed de propriis praedicatur in quid complete: igitur. Probat minor. Praedicatur ut species: igitur in quid complete. 
\pend

\pstart
  Dico 2. Proprium non constituitur universale et praedicabile per ordinem ad speciem, sed per ordinem ad individua. Probatur. Universale constituitur per ordinem ad inferiora; sed species non est inferior proprio: igitur. Probatur minor. Species est aequalis proprio: igitur non inferior. Antecedens patet. Nam convertitur cum illa: igitur. Intelligito conclusiones etiam de proprio subalterno procedimus enim sicut de differentia. 
\pend

\pstart
  Obiectio 1. Ideo Petrus est risibilis quia est homo: \textnormal{|}\ledsidenote{BNC 166va}   prius praedicatur de specie quam de individuis. Antecedens constat. In hoc distinguitur accidens a proprio. Siquidem homo est albus quia Petrus est albus, at Petrus est risibilis quia homo est risibilis. Respondeo distinguendo antecedens. metaphysice, concedo; Logice nego. Ad probationem distinguo. In hoc distinguuntur ratione metaphysica, concedo; logica, nego. Entitas prius convenit speciei, at praedicabilitas solum individuis. Eodem prorsus modo dicitur eadem relatione tangere speciem et individua metaphysice, non logice. 
\pend

\pstart
  Obiectio 2. Idem accidens respectu illius a quo dimanat est propria passio, respectu vero inferiorum est accidens commune: igitur aut ruit quarto praedicabile aut per ordinem ad speciem constituitur. Antecedens probatur. Omne accidens habet proprium subiectum a quo dimanat et per quod definitur: igitur omne accidens respectu subiecti est propria passio. At si respectu individuorum etiam esset proprium rueret quinto praedicabile ut patet: igitur. 
\pend

\pstart
  Respondeo negando antecedens. Ad probationem distinguo antecedens. Omne accidens habet proprium subiectum ut causam materialem et receptivam, concedo; ut causam non solum receptivam, sed etiam reductive effectivam,nego. Accidens commune habet habitudinem ad subiectum ut ad causam materialem et receptivam. Causa autem sic non fundat necessariam conexionem cum accidentali forma. Ast proprium respicit subiectum non solum ut causam materialem et receptivam, verum etiam ut principium suae emanationis, unde habet causam conexionis. Ita \edtext{ Divus \name{\textsc{Thomas}\index[persons]{Thomas Aquinas}} \worktitle{Opusculo 48 De Proprio}. }{\lemma{}\Afootnote[nosep]{}} Accidens commune licet dicat subiectum non tam necessariam conexionem requisitam ad rationem proprii. 
\pend

\pstart
  Obiectio 3. Sola praedicatio quarti praedicabilis potest a priori demonstrari ex \edtext{ \name{\textsc{Aristotele}\index[persons]{Aristoteles}} 1 \worktitle{Posteriorum} capitulo 4 }{\lemma{}\Afootnote[nosep]{}}. Sed demonstratur de specie: igitur de specie praedicatur qua \textnormal{|}\ledsidenote{BNC 166vb} quartum praedicabile. Maior constat. Praedicatio quinto est contingens, ceterorum essentialis: igitur. Respondeo. Potest demonstrari qua est proprium prioristicum (hoc est universale et praedicabile), nego; qua est posterioristicum (hoc est quoad convertentia), concedo. Demonstratio de proprio dicitur de quarto praedicato, non de quarto praedicabili. 
\pend

\pstart
  Obiectio 4. Proprium quarto modo no convenit individuis necessario neque convertibiliter: igitur respecti illorum non est praedicabile. Secundo proprium non convenit soli individuo, sed aliis: igitur non habet necessariam conexionem, neque convertentiam; nam si valeat est Petrus: igitur risibilis, non tamen est risibilis: igitur Petrus. 
\pend

\pstart
  Respondeo distinguendo antecedens. Ratione naturae, nego; ratione individuationis, concedo. Unde mediante convenit individuis, non immediate. Et negatur consequentia. Aliud est proprium convenire individuis metaphysice, aliud logice. Ad secundam probationem distinguo antecedens. Sed aliis extraneis, nego; propriis, concedo. Communitas non tollit rationem proprii si sit ad propria individua; tollit si sit ad extranea. Et distinguo consequens. Non habet convertentiam ratione individuationis, concedo; ratione naturae, nego. Sufficit enim ad rationem proprii convertentia cum natura, non amplius requiritur cum omni individuo. 
\pend

\pstart
  Obiectio 5. Proprium genericum respectu inferiorum non est quarto modo, sed secundo: igitur non est quarto praedicabile respectu inferiorum. Antecedens probatur. Bipes est proprium animalis qua distincti a quattorpepedibus est enim genus innominatur ad omnia bipedia; sed respecti hominum est proprium secundo modo: igitur. Secundo: accidentia communia sunt proprium quarto modo respectu alicuius subiecti ut albedo, nigredo et similia, sed respectu singularium sunt accidentia communia: ergo. Tertio, haec praedicatio `homo est \textnormal{|}\ledsidenote{BNC 167ra}   risibilis' est quarto praedicabilis, sed ibi praedicatur proprium de specie: igitur respectu speciei est quartum praedicabile. 
\pend

\pstart
  Respondeo negando antecedens. Sicut enim proprium specificum competit individuis, sic proprium genericum competit speciebus mediante, non immediate: sicut tactivum. Ad probationem. Negatur assumptum siquidem esse bipedem est proprium secundo modo. Ceterum si ab aliquibus accipiatur ut essentialis differentia ideo est quia saepe utimur accidentalibus differentiis nobis notioribus pro essentialibus. Ad secundum dico talia accidentia non esse proprias passiones, nam non oriuntur ex principiis essentialibus, sed ex materialibus habent tamen ab extrinseco inseparabilitatem ut nigredo in corvo, albedo in cygno. Ad ultimum dico, talem praedicationem esse quarto praedicati, non quarto praedicabilis. Vide argumenta pro differentia, quae servatis servandis resoluuntur in nostro casu. Supponimus enim dari proprium subalternum nempe quod dimanat a gradu generico; et proprium infimum, scilicet quod emanat a gradu differentiali: hoc notato. 
\pend

\pstart
  Dico 3. Tam proprium infimum quam subalternum pertinet ad hoc praedicabile. Probatur. Plura datur accidentia, quae dimanant a principiis genericis et cum eis habent necessariam conexionem: igitur vere dantur propria generica seu subalterna. Antecedens constat. De potentiis sensitivis respectu animalis, de nutritiva respectu viventis: ergo. 
\pend

\pstart
  Dico 4. Proprium stricte est quod praedicatur de pluribus in quale necessario accidentaliter. Dico 5. Proprium subalternum est quod praedicatur de pluribus specie differentibus, etc. Dico 6. Proprium infimum seu athomum est quod praedicatur de pluribus numero differentibus, etc. Unde haec divisio \textnormal{|}\ledsidenote{BNC 167rb} est accidentalis. 
\pend

\pstart
  Obiectio 1. \textsc{Porphyrius}\index[persons]{Porphyrius} hic et \edtext{ \name{\textsc{Aristoteles}\index[persons]{Aristoteles}} \emph{1 [?]} \worktitle{Topicorum} capitulo 4 }{\lemma{}\Afootnote[nosep]{}}. solum definiunt proprium infimum, igitur illud tantum pertinet ad hoc praedicabile. Respondeo. Hoc factum fuisse; primum, quia proprietatibus infimis frequentius utimur quam subalternis vice versa de differentia; secundum, definito proprio infimo, faciliter definitur subalternum. 
\pend

\pstart
  Obiectio 2. Emanatio proprie passionis reducitur ad genus causae efficientis ut diximus, et ut docet \edtext{ Divus \name{\textsc{Thomas}\index[persons]{Thomas Aquinas}} 1 parte, quaestione 77, articulo 7 }{\lemma{}\Afootnote[nosep]{}}. Sed nullum accidens potest sic dimanare a gradu generico: igitur non datur proprium genericum. Probatur minor. Nihil agit, quin sit in actu et determinatum, sed genus est potentiale et indeterminatum: igitur nullum accidens potest dimanare a gradu generico. 
\pend

\pstart
  Respondeo negando minorem. Ad probationem distinguo minorem. Genus est potentiale et indeterminatum qua totum potentiale, concedo; qua totum actuale, nego. Proprium genericum non dimanat a genere ut toto potentiali, qua comparatur ad inferiora illaque continet superioritate, sed ut a toto actuali qua est contitutum. Mane hoc sufficienter explanatum. Cum enim proprium sit accidens essentiae tribuens ei esse secundum quid, necessario supponit eam constitutum simpliciter. 
\pend

\pstart
  In calce notato aliquando propriam passionem non esse accidens determinatum, sed disuinctum; verbi gratia par et impar respectu numeris, masculinum et femininum respecti animalis. Tunc quodlibet est accidens \emph{determinata [?]}, disuinctum convenit per se; sic \edtext{ Divus \name{\textsc{Thomas}\index[persons]{Thomas Aquinas}} 1 \worktitle{Posteriorum} lectione 10 }{\lemma{}\Afootnote[nosep]{}}. Haec de quarto praedicabili. Videamus quintum. 
\pend

        \addcontentsline{toc}{chapter}{Capitulum quintum}
        \pstart
        \eledchapter*{\supplied{Capitulum quintum}}
        \pend
      
        \addcontentsline{toc}{section}{Textus capituli quinti Porphyrii}
        \pstart
        \eledsection*{Textus capituli quinti Porphyrii}
        \pend
      
\pstart
\noindent%
 \textnormal{|}\ledsidenote{BNC 167va}   Caput quintum \textsc{Porphyrii}\index[persons]{Porphyrius} de Accidente quinto Praedicabili Textus 
\pend

\pstart
 \textnormal{|}\ledsidenote{BNC 167vb} \edtext{\enquote{ \worktitle{ΚΕΦΑΛΑΙΟΝ Ε} Συμβεβηκὸς δέ ἐστιν ὃ γίνεται καὶ ἀπογίνεται χωρὶς τῆς τοῦ ὑποκειμένου φθορᾶς. Διαιρεῖται δὲ εἰς δύο · τὸ μὲν γὰρ αὐτοῦ χωριστόν ἐστιν, τὸ δὲ ἀχώριστως, καὶ τὰ ἕτερα. }}{\lemma{}\Afootnote[nosep]{ \textsc{Commentarii Collegii Conimbricensis e Societate Iesu}\index[persons]{}, \worktitle{In universam dialecticam Aristotelis Stagirita} (Lugduni: sumpt. Iacobi Cardon et Petri Cavellat, 1622), p. 144. }} 
\pend

\pstart
  \textnormal{|}\ledsidenote{BNC 167vc} \edtext{\enquote{ \worktitle{Caput quintus} Accidens autem est quod adest atque abest sine subiecti corruptione. Dividitur autem enim duo. Aliud est separabile, aliud inseparabile. Atque dormire quidem accidens separabile est, etc. }}{\lemma{}\Afootnote[nosep]{ \textsc{Commentarii Collegii Conimbricensis e Societate Iesu}\index[persons]{}, \worktitle{In universam dialecticam Aristotelis Stagirita} (Lugduni: sumpt. Iacobi Cardon et Petri Cavellat, 1622), p. 144. }} 
\pend

        \addcontentsline{toc}{section}{Summa Textus}
        \pstart
        \eledsection*{Summa Textus}
        \pend
      
\pstart
\noindent%
  \textnormal{|}\ledsidenote{BNC 167va} Tres definitiones, unaque divisionem contine hoc caput. Prima definitio est: \enquote{Accidens est quod adest et abest sine subiecti corruptione.} Secunda definitio est: \enquote{Accidens est quod inesse et non inesse idem potest.} Tertia: \enquote{Accidens est quod neque genus, neque species, neque differentia, neque proprium est; semper autem est in subiecto.} Divisio est in separabile ut dormire, et inseparabile ut nigredo in corvo vel Aethiope. Obiicitur, si datur accidens inseparabile: igitur accidens non est, quod abest et adest. Respondeo. Cogitatione et intellectu corvum album vel Aethiopae non nigrum sine subiecti corruptione concipi posse. Inexplicata etiam definitio. 
\pend

        \addcontentsline{toc}{section}{Annotationes circa Litteram et Sensum huius Capitis}
        \pstart
        \eledsection*{Annotationes circa Litteram et Sensum huius Capitis}
        \pend
      
\pstart
\noindent%
  \textnormal{|}\ledsidenote{BNC 167vb} Minime obstant \textsc{Avicenna}\index[persons]{Avicenna} et alii Arabes Porphyrium reprehendentes prius accidens definientem, quam dividentem. Siquidem accidens hic definiendum non est aequivocum: igitur non prius debuit dividi, quam definiri, sic Beatus \textsc{Albertus Magnus}\index[persons]{Albertus Magnus}. 
\pend

\pstart
  Notandum est primo accidens bifariam sumi, scilicet physice et logice seu primo et secundo intentionaliter. Accidens physice significat id, quod raro contingit (de hoc loquitur \edtext{ \name{\textsc{Philosophus}\index[persons]{Aristoteles}} \worktitle{de demonstratione} et 6 \worktitle{Metaphysicae}, 2) }{\lemma{}\Afootnote[nosep]{}} et quod enti advenit \textnormal{|}\ledsidenote{BNC 167vc} \emph{a parte rei [?]} post eius esse completum. Sic sumptum ab eodem \edtext{ V \worktitle{Metaphysicae} }{\lemma{}\Afootnote[nosep]{}} divisum fuit in novem praedicamenta. Accidens logicum est aequivocum ad quattuor. Primo, significat idem quod extraneum, de quo loquitur \edtext{ \name{\textsc{Philosophus}\index[persons]{Aristoteles}} in \worktitle{Elenchis} de fallacia accidentis }{\lemma{}\Afootnote[nosep]{}}. Secundo, per quacunque secunda intentione, imo per omni rationis ente unde omnia praedicabilia accidentia sunt. Tertio, per praedicato non quidditativo et essentiali condistincto contra praedicatum quidditativum, quo pacto competit his ultimis praedicabilibus. Quarto, per praedicato nec quidditativo, neque \textnormal{|}\ledsidenote{BNC 168ra}   perfluente ab essentia, neque cum illa convertibili; et in hac acceptione pertinet ad nostrum institutum. 
\pend

\pstart
  Notandum est 2. Duas primas definitiones non esse in re diversas differre solum penes hoc quod prima datur per actum et secunda per potentiam. Cetera causa definitiones traditas notanda tota questione explicanda sunt. Unde Quaestio 11. De Accidente, etc. 
\pend

        \addcontentsline{toc}{section}{Quaestio 13: De accidente quinto praedicabili}
        \pstart
        \eledsection*{Quaestio 13: De accidente quinto praedicabili}
        \pend
      
\pstart
 Relinquenda sunt multa pertinentia ad accidens praedicamentale quamquam ab aliquibus hic examinata, solo accidente logico respecto. Constat ex dictis esse unum ex praedicabilibus, ut quae desunt praestringantur, sit Articulo 1. Utrum definitiones, etc. 
\pend

        \addcontentsline{toc}{section}{Articulus 1. Utrum definitiones divisionesque Accidentis ab Porphyrio traditae exactae rectaeque sint?}
        \pstart
        \eledsection*{Articulus 1. Utrum definitiones divisionesque Accidentis ab Porphyrio traditae exactae rectaeque sint?}
        \pend
      
\pstart
  Discriminatur accidens praedicabile ab accidente praedicamentali in hoc, scilicet quod accidens praedicamentale contraponitur substantiae. Unde solum stat pro accidente inhaerente. Praedicabile vero contraponitur praedicato essentiali, unde quidquid non convenit essentialiter vel necessario dicitur accidens praedicabile sive sit reale sive rationis, sive inhaerens, sive subsistens. 
\pend

\pstart
  Duo nobis obiiciuntur in articulo disputanda. Primum, an hic definiatur accidens primo intentionaliter vel pro secunda intentione quinti praedicabilis? Secundum quomodo intelligantur illae \textnormal{|}\ledsidenote{BNC 168rb} particulae definitionis, quod adest et abest sine subiecti corruptione? Relictis sententiis circa primum. 
\pend

\pstart
  Dico 1. Definitio tradita convenit substrato seu primae intentioni huius praedicabilis sive formaliter sive materialiter sumptae. At si aliquid addatur convenit etiam secundae intentioni. Ita expresse \edtext{ \name{\textsc{Ioannes a Sancto Thoma}\index[persons]{Ioannes a Sancto Thoma}} quaestione 12, articulo 1 }{\lemma{}\Afootnote[nosep]{}}. 
\pend

\pstart
  Probatur ratione. Quidquid invenitur in singularibus et ad illa descendit non pertinet solum ad secundam intentionem praedicabilitatis, sed quod explicant hae definitiones invenitur etiam in singulari et individualiter accepto: igitur hic definitur prima intentio sive formalis, sive materialis. Maior constat cum universalitas et praedicabilitas non descendant ad singularia. Minor probatur. Adesse et abesse subiecto etiam convenit accidenti singulari: igitur. Probat antecedens. Accidens separabile aliquando adest, aliquando abest subiecto singulari sine eius corruptione: igitur. Hic accidens inseparabile individualiter acceptum non est a quo subiectum seu natura cui inest dependet, licet individualis conservatio aliquando dependeat: igitur. 
\pend

\pstart
  Si autem vis praedictam definitionem sumere secundo intentionaliter, ita ut competat secundae intentionis debes sic eam transmutare; accidens est quod praedicatur de multis ut praedicatum, quod adest et abest praeter subiecti corruptionem. Puto Auctores contrarios hoc sensu loqui. 
\pend

\pstart
  Obiectio 1. \textsc{Martinez}\index[persons]{Ioannes Martinez de Prado}  hic definitur aliquid conveniens accidenti per intellectum, non in re: igitur. Probatur antecedens. Licet adesse uni et abesse alii realiter conveniat accidenti. Tamen adesse et abesse eidem subiecto non convenit in re, sed per rationem praesertim accidenti inseparabili: igitur. 
\pend

\pstart
  \textnormal{|}\ledsidenote{BNC 168va}   Respondeo negando antecedens. Ad probationem negatur assumptum. Adesse et abesse uno formalem qui subiecto in re covenit accidenti sive separabili, sive inseparabili, cum nihil aliud significetur quam contingentia convenientiae. 
\pend

\pstart
  Obiectio 2. Secundae intentiones non sunt primae, sed etiam secundis intentionibus competit haec definitio: igitur. Respondeo distinguendo minorem. Materialiter acceptis, concedo; formaliter acceptis, nego. Etiam in entibus rationis invenitur prima intentio ut negatio, privatio, denominatio extrinseca; imo una secunda intentio materialiter sustemitur denominaturque ab alia secunda intentione: igitur includit quasi modum primae intentionis. 
\pend

\pstart
  Obiectio 3. Abesse et adesse sumitur secundo intentionaliter pro affirmari et negari sine corruptione subiectio: igitur. Respondeo transeat antecedens. Nam etiam concesso non convincit hanc definitionem tradi per secundam intentionem universalitatis; cum universale non fiat comparatione compositiva et attributiva, sed simplici. Ceterum affirmari et negari utroque modo accipitur, scilicet fundamentaliter quod est primae intentionis et invenitur in omni accidente singulari et formaliter per secunda intentione affirmandi et negandi statim explicanda. 
\pend

\pstart
  Secunda difficultas inter Auctores est, an illae particulae adest et abest intelligantur de reali praesentia et absentia; an vero de affirmatione et negatione? Ratio dubitandi est: non potest intelligi de reali praesentia et absentia: igitur. Probatur antecedens. Accidens inseparabile non potest abesse et adesse; alioqui esset separabile: igitur. Etiam non potest intelligi secundum intentionaliter pro affirmari et negari: igitur. Probatur. Tunc sequeretur omne accidens esse separabili: igitur. Probatur sequaela. Nullum est accidens quod per intellectum no possit separari; cum etiam subiectum concipiatur sine proprio: igitur. Item \textnormal{|}\ledsidenote{BNC 168vb} tum admitteretur, consistere in reali praesit et absit quod probat: talis affirmatio vel negatio aut est vera aut falsa? Si falsa: igitur falsum est accidens consistere in affirmatione et negatione. Si vera: igitur est \emph{a parte rei [?]}, quod adest et abest in res sine subiecti corruptione. 
\pend

\pstart
  Si dicas, non intelligi de affirmatione et negatione per modum compositis, sed per modum simplicis praecisionis. Contra primam, quia est contra \edtext{ Doctorem \name{\textsc{Thomam}\index[persons]{Thomas Aquinas}} \worktitle{De spiritualibus creaturis} articulo ultimo, ad 7 et \worktitle{Opusculo 48} capitulo de accidente }{\lemma{}\Afootnote[nosep]{}}. Secundam, quia etiam proprium poset abesse et adesse praescindendo per simplicem cognitionem: igitur. 
\pend

\pstart
  Prius modus dicendi tenetur ut propria sententia ab \textsc{Avicenna}\index[persons]{Avicenna}, \textsc{Caietanus}\index[persons]{Thomas de Vio Caietanu}, \textsc{Rubius}\index[persons]{Antonius Ruvius Rodensis} et aliis. Secundus, etiam cognoscit Patronos, scilicet, \textsc{Soto}\index[persons]{Dominicus de Soto}, \textsc{Sanchez}\index[persons]{Ioannes Sanchez Sedeno} et alios, quo eferunt et sequuntur \textsc{Complutensibus}\index[persons]{}. Alii singularem viam sunt secuti ut \textsc{Martinez}\index[persons]{Ioannes Martinez de Prado}, quam opinionem refutat \edtext{ \name{\textsc{Ioannes a Sancto Thoma}\index[persons]{Ioannes a Sancto Thoma}} quaestione 12, articulo 1 }{\lemma{}\Afootnote[nosep]{}}. 
\pend

\pstart
  Dico 2. Abest et adest sine subiecti corruptione intelligitur de contingentia accidentis et indemnitate subiecti sua quidditate ex convenientia vel disconvenientia talis accidentis. Hanc sententiam debent admittere omnes ore pleno dicentes, accidens est quod praedicatur de pluribus in quale contingenter. Est expressa mens \edtext{ \name{\textsc{Angelici Praeceptoris}\index[persons]{Thomas Aquinas}} quatione 2, \worktitle{De anima}, articulo 12, ad 7 }{\lemma{}\Afootnote[nosep]{}}. Ita \textsc{Ioannes a Sancto Thoma}\index[persons]{Ioannes a Sancto Thoma} hic. Manet probata rationibus antecedentibus. Unde Solum restat eam explicare. 
\pend

\pstart
   Adest et abest sine subiecti corruptione idem ac dicere, convenit subiecto sine dissolutione quidditatis et naturae in praedicatis essentialibus. Ceterum respectu individui non naturae dicitur accidens separabile vel inseparabile, cum habeant in illis aliquando causam permanentem, aliquando trans \textnormal{|}\ledsidenote{BNC 169ra}    transeuntem. Quapropter accidens potest convenire vel abesse subiecto corrupto individuo, non tamen corrupta specie, sed cum illius indemnitate. Hac ratione distinguitur etiam a proprio, quod si abest vel non convenit speciei dissolvitur haec. Unde individuum potest dependere in conservatione et existentia, imo in individuationem ab accidente, minime species. Relinquitur negatione, affirmationem convenientiam, disconvenientiam accidentis ad subiectum esse cum indemnitate quidditatis. 
\pend

\pstart
  Ex dictis sequitur 1. Nos non loqui hic de inhaerere vel non inhaerere, consequentiam solum competat accidentia praedicamentali; 2. Adesse et abesse significare etiam affirmari vel negari, sed non solum, quin etiam fundamentaliter sit, convenire vel non convenire naturae sine dissolutione essentialis conexionis praedicatorum. 
\pend

\pstart
  Nunc videamus fundamenta opposita. Ad primum respondeo distinguendo. Accidens inseparabile non potest adesse et abesse secundum realem praesentem vel absentem respectu individui in ordine ad quod sumitur divisio separabilis et inseparabilis, concedo; respectu quidditatis, nego. Solutio constat ex dictis. 
\pend

\pstart
  Ad secundum distinguo. Omne accidens esse separabile respectu naturae, concedo; respectu individui, nego. Separabile et inseparabile dicitur in accidentibus respectu subiecti ex parte individuationis, abesse autem et adesse sine subiecti corruptione ex parte quidditatis. 
\pend

\pstart
  Instabis: convenientia vel separatio respectu speciei ut distinguitur ab individuo est secunda intentio rationis: igitur intelligitur haec definitio secundo intentionaliter. Respondeo. Respectu speciei formaliter, concedo; fundamentaliter, concedo. Unde potius utimur nomine. \textnormal{|}\ledsidenote{BNC 169rb} quidditatis, ut non intelligatur loqui nos ita formaliter. 
\pend

\pstart
  Obiectio 1. Etiam proprium potest intelligi et separari ab essentia, sine huius corruptione: igitur non distinguitur ab accidente. Respondeo distinguendo. Et intelligi operatione componente et dividente, nego; operatione simplici, concedo. Est expressa solutio \edtext{ \name{\textsc{Angelici Doctori}\index[persons]{Thomas Aquinas}} \worktitle{De spiritualibus creaturis} articulo ultimo, ad 7 }{\lemma{}\Afootnote[nosep]{}}. 
\pend

\pstart
  Instabis: non minus possum dicere cygnus non est albus, quin homo non est risibilis; sed utraque est falsa: igitur. Respondeo negando paritatem, nam licet utraque sit falsa quia in re non ita est, tamen falsitas unius plus destruit, quin alterius. Eo quod negetur albedo cygno, non destruitur eius quidditas; at si negetur passio destruitur essentia. Nec obstat hoc quominus possit Deus, separare passiones ab essentia non quantum ad convenientiam, sed quantum ad existentiam: ut dictum est. 
\pend

\pstart
  Obiectio 2. Mors est praedicatum accidentale, sed non solum corrumpit naturam, sed etiam individuum: igitur. Respondeo. Mors in facto esse, concedo; in \emph{fieri [?]}, nego. Contra non abest et adest sine corruptione: igitur. Distinguo. Sine corruptione physica, concedo; logica, nego. In facto esse praedicatur essentialiter de cadavere, nam homini repugnat, tunc enim ampliatur propio homo olim, modo est mortuus. Si vero non ampliatur est praedicatum repugnans, unde non est quintus praedicabilis. Vide \edtext{ \name{\textsc{Complutensis}\index[persons]{}} distinctione 9, quaestione 1, numero 7 }{\lemma{}\Afootnote[nosep]{}}. 
\pend

\pstart
  Obiectio 3. Quando praedicatur existentia est praedicatio quinti praedicabilis, sed non potest dici, quod absit sine corruptione subiecti: igitur non in hoc stat essentia accidentis. Respondeo concedendo maiorem et distinguendo minorem. Non potest dici, quod absit sine corruptione subiecti respectu individuationis, concedo; respectu quidditatis, nego. Semper manet incolumis et immunis quidditas. 
\pend

        \addcontentsline{toc}{section}{Articulus 2. Quomodo accidens sit quintum praedicabile et respectu quorum inferiorum sit tale?}
        \pstart
        \eledsection*{Articulus 2. Quomodo accidens sit quintum praedicabile et respectu quorum inferiorum sit tale?}
        \pend
      
\pstart
  Tria quaeritur titulus. Primum. Quomodo verificetur accidens de subiecto ratione nominis vel ratione rei? Secundum. Respectu quorum dicatur quintum praedicabile, scilicet respectu suorum inferiorum, an respectu subiectorum? Tertium. An praecedetur respectu subiectorum existentium solum, an etiam respectu possibilium? 
\pend

\pstart
  Circa primum notandum denominationem dupliciter fieri. Primo ex parte nominis ut album ab albedine. Secundo ex parte formae significatae, ut officiosus ab officio. Quando dicitur Petrus est albus, non verificatur ratione nominis, nam sensus esset Petrus est vox album, sed verificatur ratione formae significatae. Omitto sententiam negativam. 
\pend

\pstart
  Dico 1. In praedicationibus accidentalibus non praedicatur nomen, sed res significata per nomen denominative tamen. Probatur ratione. Si accidens praedicatur quantum quod nomen, non esset unum ex quinque praedicabilibus, sed hoc nemo audet asserere: igitur. Probatur assumptum. Non esset univocum, sed omne praedicabile est univocum: igitur. Probatur maior. Esset aequivocum: igitur non univocum. Probatur antecedens. Quod praedicatur de pluribus secundum nomen, et non secundum rem est aequivocum. Sed accidens praedicatur de pluribus secundum nomen et non secundum rem: igitur. 
\pend

\pstart
  Obiectio 1. Ex \edtext{ \name{\textsc{Philosophum}\index[persons]{Aristoteles}} capitulo 5 \worktitle{in antepraedicamenta} dicente: \enquote{quod eorum quae in subiecto sunt nomen praedicari de subiecto nihil prohibet, rationem vero impossibile est praedicari} }{\lemma{}\Afootnote[nosep]{}},  igitur. Respondeo. Communi expressione \textsc{Aristotelis}\index[persons]{Aristoteles} sensum hunc esse: \textnormal{|}\ledsidenote{BNC 169vb} accidentia dicuntur secundum nomen et non secundum rationem, id est non quidditative, sed denominative. 
\pend

\pstart
  Obiectio 2. Praedicatio per verbum substantivum est dicet identitatem praedicati cum subiecto, sed inter accidens et subiectum non datur haec identitas: igitur neque praedicatio. Respondeo distinguo minorem. Non datur haec identitas essentialis et intrinseca, concedo; denominativa et concretiva, nego: est solutio expressa \edtext{ \name{\textsc{Angelico Praeceptor}\index[persons]{Thomas Aquinas}} 1 parte, quaestione 13, articulo 12 }{\lemma{}\Afootnote[nosep]{}}. Accidens in concreto, ut album cum dicat habens albenidem, ratione \emph{totali [?]} habens est idem cum subiecto. Sufficit ab praedicationem identitas denominativa et non solum necessaria et essentialis ex eodem \edtext{ \name{\textsc{Angelicum Doctorem}\index[persons]{Thomas Aquinas}} in 3 distinctione 5, in expositione litterae }{\lemma{}\Afootnote[nosep]{}}. 
\pend

\pstart
  Circa secundum notandum est accidens posse comparari ad substantiam quibus inhaeret et quae denominat; verbi gratia album ad hominem, equum, lapidem, etc., et potest etiam comparari ad inferiora propria; verbi gratia album ad hoc et illud album, coloratum ad album et nigrum. Per ordinem ad propria inferiora constitui quintum praedicabile affirmat \edtext{ \name{\textsc{Torrejon}\index[persons]{Petrus Fernandez de Torrejon}}  disputatione 5, in hoc capitulo }{\lemma{}\Afootnote[nosep]{}}. 
\pend

\pstart
  Dico 2. Quintum praedicabile constituitur per ordinem ad substantiam quae denominat. Est communi. Probatur ratione: Illa sunt inferiora quinto praedicabilis, de quibus praedicatur in quale contingenter, sed de solis subiectis praedicatur in quale contingenter: igitur. Secundo praedicari in quale contingenter, non est praedicari in quid integre, sed de propriis constitutis praedicatur in quid integre vel in quid potentiale: igitur. Tertio haec praedicatio `homo est albus', est praedicatio alicuius universalis; non generis nec speciei, neque differentiae, neque proprii: igitur accidentis. Quarto si ratione constituti esset praedicabile posset et abesse salva eius essentia; assigna mihi album sine albedine? Ergo. 
\pend

\pstart
  \textnormal{|}\ledsidenote{BNC 170ra}   Obiectio 1. Illa sunt inferiora accidentis autem quibus abstrahit, non abstrahit a subiectis, sed ab albis verbi gratia: igitur. Probatur minor. Primo, abstractio universalis est totalis, sed abstractio accidentis a subiecto est formalis: igitur. Secundo, accidens debet abstrahere a convenientibus in aliquo accidentaliter, sed conveniunt accidentaliter in albo, verbi gratia: igitur. Confirmatur: praedicabilia sunt universalia per se, sed album verbi gratia respectu Petri et Pauli praedicatur per accidens, cum per accidens homini identificentur in albo: igitur. 
\pend

\pstart
  Respondeo distinguendo maiorem. A quibus abstrahit abstractione formale, nego; abstractione universali, concedo. Et negatur minor. Dupliciter abstrahi potest album a subiectis. Primo, ut sententia physica a subiecto vel ut abstractum a concreto. Secundo, ut denominativum a denominato. Primo modo est abstractio formalis, nec abstrahitur album, sed albedo, et tunc non praedicatur cum significetur ut pars sicut dictum est de praedicationibus. Secundo modo est abstractio universalis constitutiva quinto praedicabilis, nec abstrahitur albedo, sed album. Ceterum quod abest et adest in hae sive est albedo; quod abest et adest denominative est album. 
\pend

\pstart
  Ad primam probationem distinguo minorem. Abstractio accidentis a subiecto ut forma, concedo; ut denominativum, nego. Ad secundam distinguo maiorem. Accidens debet abstrahere a convenientibus accidentaliter et denominative, concedo; a convenientibus formaliter subdistinguo ut sit quinto praedicabile, nego: ut sit primum vel secundum΄, concedo. Petrus et Paulus sunt inferiora huius praedicabilis non ut albi, sed denominabiles ab albo; tunc enim continetur in illis album denominative non formaliter. 
\pend

\pstart
  Ad confirmationem distinguo maiorem. Sunt universalia per se ex parte rei, nego; ex parte intentionis, concedo. Et distinguo minorem. Ex parte rei, concedo; ex parte subiicibilitatis, nego. \textnormal{|}\ledsidenote{BNC 170rb} Inferiora sunt subiecta per accidens ex parte rei subiectae, sed intentio subiicibilitatis per se est vera intentio, sicut requiritur ad universalitatem. Clarius sunt per accidens physice, non logice. 
\pend

\pstart
  Obiectio 2. Ex dictis sequitur quinto praedicabile non esse essentialiter, sed accidentaliter universale, sed hoc non: igitur. Respondeo negando maiorem. Aliud est secundam intentionem facere, nam praedicari accidentaliter de subiectis, aliud secundam intentionem huius praedicabilis accidentaliter esse universale. 
\pend

\pstart
  Dico 3. Circa tertium licet non existant subiecta quintum praedicabile vere est universale. Probatur. Universalitas consistit in aptitudine ad essendum in pluribus, sed licet non ponatur existentia subiectorum datur talis aptitudo: igitur. Probatur minor. Quod opponitur actui seu exercitio non definiet potentiam seu aptitudinem; cum opposito ad actum non sit opposito ad potentiam; sed carentia existentiae opponitur actui et exercitio praedicandi seu essendi: igitur non opponitur potentiae seu aptitudini essendi et praedicandi. Secunda a paritate: ex eo quod accidens physicum separetur ab subiecto non amittit aptitudinem inhaerendi: igitur similiter accidens logicum non amittit aptitudine ex eo quod exercite non praedicetur. 
\pend

\pstart
  Obiectio 1. Si non existerent subiecta no daretur concretum: igitur solum daretur abstractum. Respondeo distinguendo antecedens. Non daretur concretum singulare, concedo; in communi, nego. Si non daretur existentia subiectorum non darentur subiecta concreta singularia. At cum concretum communio non est subiectum singulare, sed commune, ideo non existentibus subiectis daretur quintum praedicabile. 
\pend

\pstart
  Obiectio 2. Subiecta non possent respici ab accidente ut existentia: igitur ut possibilia, sed existentia possibilis non praedicatur accidentaliter, sed essentialiter verbi gratia. \textnormal{|}\ledsidenote{BNC 170va}   \textsc{\emph{Adam [?]}}\index[persons]{} possibiliter ambulat, possibiliter est albus: igitur non respicit possibilia quintum praedicabile. 
\pend

\pstart
  Respondeo distinguendo consequens. Ut possibilia possibilitate praedicata, nego; terminata, concedo. Et distinguo probationem. Existentia possibilis ut praedicata facit propositionem essentialem, concedo; ut conditio ex part subiecti nego. Accidens non praedicat possibilitatem, sed ea indiget ut conditione ex parte tamen. 
\pend

\pstart
  Obiectio 3. Posita praedicatione impossibili, universale est impossibile, sed existentibus subiectis, praedicatio est impossibilis, cum sit contingens et postulet existentiam ut conditionem: ergo. Respondeo distinguendo maiorem. Posita impossibili ex parte accidentis secundum se, concedo; defectu applicationis, nego. Sicut accidens separatum retinet possibilitatem inhaerendi, licet actu non inhaereat, nec applicetur. Articulo 3. 
\pend

        \addcontentsline{toc}{section}{Articulus 3. In quo differant propria passio et accidens commune?}
        \pstart
        \eledsection*{Articulus 3. In quo differant propria passio et accidens commune?}
        \pend
      
\pstart
\noindent%
 Articulus 3. In quo differant propria passio et accidens commune? 
\pend

\pstart
  Difficultas an omne accidens sit propria passio respectu alicuius subiecti? Ex resolutione patebit discrimen inter haec praedicabilia. \textsc{Lovanienses}\index[persons]{} et alii tenent affirmativam, postea referam eorum fundamenta et satisfaciam. 
\pend

\pstart
  Dico 1. Non omnia accidentia sunt propriae passiones. Unde accidentia quae oriuntur ab essentia ratione formae habentque necessariam conexionem cum ea sunt passiones, quae vero oriuntur ratione materiae conveniuntque primo individuis sunt accidentia communia. Est communis inter Dialecticos, stat per ea \edtext{ \name{\textsc{Aristoteles}\index[persons]{Thomas Aquinas}} 10 \worktitle{Metaphysicae}, textu 25 }{\lemma{}\Afootnote[nosep]{}}. 
\pend

\pstart
  Dico 2. Primum discrimen quod versatur inter proprium et accidens commune est, quod proprium sequitur compositum ratione formae, accidens commune ratio est \textnormal{|}\ledsidenote{BNC 170vb} materiae. Ita \edtext{ Divus \name{\textsc{Thomas}\index[persons]{Thomas Aquinas}} \worktitle{De ente et essentia} 7 et 1, distinctione 17, quaestione 4, articulo 2, 2 et 1 \worktitle{Posteriorum} lectione 14 }{\lemma{}\Afootnote[nosep]{}}. 
\pend

\pstart
  Probatur ratione utraque conclusio. Illae sunt passiones proprie et simpliciter, quae demanat ab essentia, ut essentia, sed essentia constituitur in ratione essentiae per formam, et non per materiam: igitur propriae passiones sunt quae dimanant ratione formae, et non ratione materiae. Secundum, passio sequitur essentiam ratione differentiae specificae et non differentiae numeri causae, sed forma est radix differentiae specificae et materia differentiae numericae: igitur accidentia quae sequuntur formam sunt passiones, et quae sequuntur materiam accidentia communia. Maior constat. Siquidem passio competit omnibus, quibus competit differentia specifica: enim probata etiam prima conclusio. Tertium, forma et non materia est prima radix operationem aut constitutionis essentiae in determinato gradu, sed passiones sunt attributae naturae aut ad aliquas operationes, aut sequuntur essentiam ut constitutam in aliquo gradu entis: igitur solum accidentia, quae sequuntur formam sunt passiones. 
\pend

\pstart
  Dico 3. Secundum discrimen inter proprium et accidens commune est, quod passio prius convenit speciei, deinde individuis, Petrus est risibilis, quia homo est risibilis: at accidens commune prius convenit individuis, deinde speciei, homo est albus quia Petrus est albus. Ita \edtext{ Divus \name{\textsc{Thomas}\index[persons]{Thomas Aquinas}} ibidem lectione 11 }{\lemma{}\Afootnote[nosep]{}}. Probatur. Forma est prima radix unde sumitur species; materia prima radix unde sumitur individuatio. Igitur accidentia provenientia a forma per prius conveniunt speciei; et provenientia a materia per prius conveniunt individuis. Sed passiones oriuntur a forma et accidentia communia a materia: igitur passiones per prius conveniunt speciei, et accidentia communia per prius individuis. 
\pend

\pstart
  Obiectio 1. Contrarii: illud primum discrimen statutum inter passionem et accidens commune non evacuat \textnormal{|}\ledsidenote{BNC 171ra}   difficultatem: igitur potius dicendum est omne accidens esse propriam passionem. Probatur antecedens. Primo, difficile est indagare, quod accidens sequatur materiam, quod formam maxime in sententia \edtext{ \name{\textsc{Angelici Doctori}\index[persons]{Thomas Aquinas}} \worktitle{De ente et essentia} 7 }{\lemma{}\Afootnote[nosep]{}} asserentis nullum accidens dimanare a sola materia. Secundo, quia quantitas sequitur materia ex ipso \edtext{ \name{\textsc{Doctore Angelico}\index[persons]{Thomas Aquinas}} 3 parte, quaestione 90, articulo 2 et \worktitle{De potentia} quaestione 6, articulo 5 }{\lemma{}\Afootnote[nosep]{}}. Sed nihilominus quantitas est propria corporis passio: igitur ruit fundamentum talis discriminis. 
\pend

\pstart
  Respondeo negando antecedens. Ad primam probationem distinguo difficile est defectu nostrae capacitatis, concedo; ex parte rei, nego. Defectu nostro non possumus illam regulam applicare et explicare in particulari, quod accidens sequatur materiam et quod formam; at non ideo est falsa. Et distinguo subsumptum ex \textsc{Doctore Thoma}\index[persons]{Thomas Aquinas}: nullum accidens dimanat a sola matura nude sumpta, concedo; simul cum forma, nego. Accidens commune oritur a toto composito ratione materiae; proprium ab ipso composito ratione formae. 
\pend

\pstart
  Ad secundam probationem negatur minor. In riore loquendo quantitas non est passio, sed accidens commune teste \edtext{ \name{\textsc{Caietanus}\index[persons]{Thomas de Vio Caietanu}} \worktitle{De ente et essentia} 7, quaestione 17 }{\lemma{}\Afootnote[nosep]{}}; unde pertinet ad individuationem. Nec requiritur unum individum habere unum accidens et aliud \secluded{aliud}, sed satis est utrumque habere idem diverso modo. Quod autem quantitas non sit passio probat hac ratione: propria passio vel invenitur in omnibus habentibus eius essentiam eodem modo vel saltem ex se non habet inaequalitatem. Sed quantitas non est eodem modo in individuis, cum individua differant per quantitatem, imo non inveniuntur duo quanta omnino similia cum saltem propriis sitibus differant: igitur. 
\pend

\pstart
  Obiectio 2. Color respectu hominis et lapidis est accidens commune, sed respectu mixti est propria passio: igitur idem accidens et est commune, et \textnormal{|}\ledsidenote{BNC 171rb} proprium quamvis respectu diversorum. Probatur minor. Per veram demonstrationem potest probari, mixtum esse coloratum: igitur est necessaria: igitur non contingens est talis proprio; et si non est contingens: igitur praedicatum non est accidens. 
\pend

\pstart
  Respondeo negando maiorem. Quomodolibet sumatur color non passio mixti: si enim est de genere quantitatis iuxta Pythagoricos non est passio constat enim quantitatem non esse passionem. Si vero est de genere luminis iuxta Platonicos, no est passio, cum sit accidens ab extrinseco adveniens. Si tandem est de genere qualitatis iuxta Peripateticos supponentis vel lumen in ipso perspicuo vel temperamentum primarum qualitatum in tali vel tali proportione, quod nascitur ex principiis individualibus, \emph{est qui [?]} dispositio accidentalis, non est passio ut patet. 
\pend

\pstart
  Ad minoris probationem distinguo antecedens. Haec propositio mixtum est coloratum potest demonstrari, sumpto hic mixtum pro gradu essentiali, nego; si sumatur pro speciali temperamento seu accidentali mixtione potest demonstrari subdistinguo ex parte rei in qua est color, nego; ex parte temperamenti seu mixtionis, concedo. \emph{Iterum [?]} subdistinguo. Est passio simpliciter, nego; secundum quid et reductive, concedo. Tunc enim habet rationem quarto praedicati, non praedicabilis. 
\pend

\pstart
  Obiectio 3. Accidentia, quae causantur ex varia dispositione materiae ut masculinum et femininum sunt propriae passiones: igitur. Antecedens patet. Primum, ex \edtext{ \name{\textsc{Aristotele}\index[persons]{Aristoteles}} 13 \worktitle{Metaphysicae} textu 25, dicente: \enquote{Masculus autem et femina propriae quidem alis passiones sunt} }{\lemma{}\Afootnote[nosep]{}}. Secundum, quia talia praedicantur per se de illa dispositione et sunt eius effectus: igitur. 
\pend

\pstart
  Respondeo distinguendo antecedens. Sunt propriae passiones simpliciter, nego; secundum quid et reductive, concedo. Cum enim dispositio illa sit accidens commune respectu alis. \textnormal{|}\ledsidenote{BNC 171va}   Unde fit ut talia accidentalia sunt communia, licet respectu sui adaequati subiecti habeant rationem quarto praedicati. Nec aliud intendit \textsc{Aristoteles}\index[persons]{Aristoteles} loco citato interpraete Divo \textsc{Thoma}\index[persons]{Thomas Aquinas} ibidem. 
\pend

\pstart
  Nunc refutatur haec opinio et simul nostra. Probatur. Primo, plura sunt accidentia, quae quibuscunque comparentur sunt communia puta contingenter convenientia; ut sunt actus vitatis et intellectus: igitur. Patet antecedens. Alioqui haec praedicationes homo intelligit, amat possent perfecte demonstrari, hoc est absurdum: igitur. Secundo, tunc ex antecedenti necessario sequeretur consequens contingens: igitur. Sequela patet. Iuxta contrarios haec propositio est contingens: `homo est coloratus'; haec vero necessaria: `mixtum est coloratum', sed prima sequitur ex secunda in hoc syllogismo: omne mixtum est coloratum, homo est mixtum: igitur est coloratus: igitur ex antecedenti necessari sequeretur consequens contingens. 
\pend

\pstart
  Colligitur 1. Definitionem tertiam accidentis non esse contemnendam. Primam, quia non datur per solas negationes. Secundam, quia quando ut cognitae res, per quarum negationes aliquid explicatur, ut accidit hic, non materiale definitur per tales negationes loco differentiae. Quare praedicta definitio reddit hunc sensum: accidens est quod est in subiecto (non inhaesive, sed praedicative), id est quod praedicatur de pluribus non sunt genus, species, differentia, vel proprium. Unde illis cognitis manet clarum quinto praedicabile. 
\pend

\pstart
  Colligitur 2. Hucusque solum sex \textsc{Porphyrii}\index[persons]{Porphyrius} capita me notavisse, restant alia annotanda, inscripta de communitatibus quinque praedicabilium. Hoc enim est tertia pars tractatus \textsc{Porphyrii}\index[persons]{Porphyrius} iuxta eius partitionem in Prooemio positam. In hac prolixe singula \textnormal{|}\ledsidenote{BNC 171vb} praedicabilia inter se comparat, et in quibus conveniant et in quibus differant determinat. Ast haec omnia sufficienter constat, tam in universalibus incommuni, quam ex dictis ex unoquoque in sua specie. Unde cum communi \textsc{Doctore Angelico}\index[persons]{Thomas Aquinas} hanc partem omissam facimus. 
\pend

\pstart
  Dubitas. Communiter dicitur quod convenit per accidens reduci debet ad id, quod convenit per se, quomodo igitur accidens potest reduci ad id, quod convenit per se? Respondeo. Quod reducitur ad id, quod est per se non quantum ad necessariam conexionem cum aliquo subiecto a quo dimanat, sed quantum ad causalitatem efficientem vel finalem. Ratio est quia non per accidens, sed per se ab aliqua causatur. Non requiritur quod omne quod convenit per accidens reducatur ad omne modum per se, sed satis est ad aliquem, ut ad causalitatem per se, licet non ad conexionem necessariam cum subiecto. Aliqui ut Illustrissimus \textsc{Merinero}\index[persons]{} hic resolvit difficultatem illam, an inhaerentia sit de essentia accidentis praedicamentalis? Nos vero eam relinquimus suo proprio loco tractandam. Nam hic solum agimus de accidente praedicabili, non praedicamentali, nec intendo in vitium hoc \emph{tonto [?]} Doctori imputare; cum ipse dicat, non invenire aptiorem locum. 
\pend

\pstart
 Hucusque de \worktitle{Isagoge} \textsc{Porphyrii}\index[persons]{Porphyrius} feliciter, hoc est de introductione ad \textsc{Aristotelis}\index[persons]{Aristoteles} praedicamenta. Nunc vero faelitius ingenio et affectu opus est ad cathegoremata volitare: Auspice Deo uno trinoque. Cui sit laus et gloria semper. Amen. 
\pend


\endnumbering
\end{otherlanguage*}


% Bibliography ───────────────────────────────────────┐
\cleardoublepage
\nocite{*}
\begin{otherlanguage*}{spanish}
\printbibliography
\end{otherlanguage*}
% ────────────────────────────────────────────────────┘


% Indices ────────────────────────────────────────────┐
\cleardoublepage
\printindex[persons]
% \printindex[works]
% ────────────────────────────────────────────────────┘


\end{document}
% ════════════════════════════════════════════════════════╝
